 \documentclass[12pt,letterpaper]{article}

\usepackage{multicol}
\usepackage[x11names,table]{xcolor}
\usepackage{pstricks}
\usepackage{marginnote}
\usepackage[shortlabels]{enumitem}

\usepackage{parskip}
\usepackage{times}

%Configuracion del la hoja

\usepackage{geometry} %Paquete de margenes
\geometry{left=4cm, right=3cm, top=3cm, bottom=3cm}% Tamaño del área de escritura de la páginas
\usepackage{lscape}


%Paquetes para el entorno de escritura
\usepackage[spanish]{babel}
\usepackage[utf8]{inputenc} %Reconoce tildes y otros simbolos propios del español
\setlength\parindent{12pt}
\usepackage[breaklinks=true, hidelinks]{hyperref}


%Paquetes necesarios en el entorno cientifico
\usepackage{amsmath} %Paquete de smbología matemática de la American Mathematical Society.
\usepackage{amsfonts}%Paquete de smbología matemática de la American Mathematical Society.
\usepackage{amssymb}
\usepackage{latexsym}
\usepackage{graphicx} % Required for the inclusion of images
\usepackage{subfigure} % subfiguras
%\usepackage{circuitikz} % Requerido para dibujar circuitos
%\usepackage{tikz}
%\usepackage{siunitx} % Provides the \SI{}{} and \si{} command for typesetting SI units

%Paquetes para herramientas útiles en el desarrollo del texto
\usepackage{float}%Requerido para obligar a los elementos a colocarse donde uno quiera
\usepackage[final]{pdfpages} % util para agregar pdf al final del documento

%Paquetes para la bibliografia
\usepackage{natbib} % Requerido para cambiar bibliografia a formato APA
%\usepackage{cite}	
\usepackage{listings}


\renewcommand{\labelenumi}{\alph{enumi}.} % Make numbering in the enumerate environment by letter rather than number (e.g. section 6)

\newcommand{\newfig}[4]{%
\begin{figure}[H]%
\centering%
\includegraphics[width=#1\linewidth]{#2}%
\caption{\emph{\small{#3}}}%
\label{fig:#4}
\end{figure}%
}

\usepackage{pgfgantt} %Cronograma de actividades

%----------------------------------------------------------------------------------------
%	PRESENTACION DEL DOCUMENTO
%----------------------------------------------------------------------------------------



\author{} % Author name

\date{FECHA} % Date for the report

\begin{document}

\renewcommand{\listfigurename}{Lista de Figuras}
\renewcommand{\listtablename}{Lista de Tablas}
\renewcommand{\contentsname}{Lista de Contenidos}
\renewcommand{\figurename}{Figura}
\renewcommand{\tablename}{Tabla}


% Insert the title, author and date
\begin{center}

	\vspace{3cm} UNIVERSIDAD CENTRAL DE VENEZUELA \\
	FACULTAD DE INGENIERÍA \\
	ESCUELA DE INGENIERÍA ELÉCTRICA \\
	DEPARTAMENTO DE ELECTRÓNICA, COMPUTACIÓN Y CONTROL\\
	ANTEPROYECTO DE TRABAJO DE GRADO \\

	\vspace{7cm} DISEÑO DE UN SENSOR INTELIGENTE PARA APLICACIONES DE MONITOREO DE SALUD ESTRUCTURAL
\end{center}


\vspace{6cm}

\begin{flushleft}
	TUTOR ACADÉMICO: MSc José Romero \\
\end{flushleft}


\begin{flushright}

	Presentado ante la ilustre\\
	Universidad Central de Venezuela\\
	por el BR. Jose Alejandro Tovar Briceño \\
	para optar por el título de \\
	Ingeniero Electricista   \\

\end{flushright}

\begin{center}
	Caracas, agosto de 2023
\end{center}


\vspace{2cm}
\thispagestyle{empty}
\newpage


\begin{center}
	\section*{ INTRODUCCIÓN}
\end{center}

La seguridad de las infraestructuras es un tema de gran importancia en la actualidad, especialmente cuando se trata de estructuras como edificios o puentes.

Los primeros indicios del monitoreo del estado de las infraestructuras data de nuestros comienzos como especie sedentaria.  En la antigüedad, los especialistas utilizaban técnicas de inspección visual y auditiva para detectar posibles problemas en las estructuras, como grietas o ruidos inusuales. Con el tiempo, se desarrollaron técnicas más avanzadas para el monitoreo de estructuras, como la utilización de medidores de deformación, inclinación, sensores de vibración, entre otros.

La integración de la instrumentación con el análisis estructural comenzó a desarrollarse en la década de 1960 con el advenimiento de la informática y la disponibilidad de computadoras capaces de realizar cálculos estructurales complejos. En esa época, se comenzaron a utilizar sistemas de adquisición de datos para recopilar información sobre el comportamiento de las estructuras en tiempo real y utilizarla para calibrar y validar los modelos estructurales.

Actualmente, las normas sismo-resistentes apuntan a estructuras que sean capaces de mantener su integridad ante un evento de cierta magnitud. Además, el monitoreo continuo de ciertos indicadores en la estructura permiten determinar un índice de la salud estructural y ajustar el modelo a las condiciones actuales de la misma para evaluar el cumplimiento de la normativa sismorresistente. Para el monitoreo a largo plazo, el resultado de este proceso es información actualizada periódicamente sobre la capacidad de la estructura para desempeñar su función prevista a la luz del inevitable envejecimiento y degradación resultantes de los entornos operativos.

Según (Balageas, 2010), el monitoreo de la salud estructural (SHM) tiene por objeto proporcionar, en cada momento de la vida de una estructura, un diagnóstico del estado de los materiales constitutivos, de las diferentes
partes, y del conjunto de estas partes que constituyen la estructura en su totalidad. El estado de la estructura debe permanecer en el ámbito especificado en el diseño, aunque este puede verse alterado por el envejecimiento normal debido al uso, por la acción del medio ambiente y por sucesos accidentales. Gracias a la dimensión temporal de la supervisión, que permite tener en cuenta toda la base de datos histórica de la estructura, y con la ayuda del monitoreo del funcionamiento. También puede proporcionar un pronóstico (evolución de los daños, vida residual, entre otros).

Si consideramos solo la primera función, el diagnóstico, podríamos estimar que el monitoreo de la salud estructural es una forma nueva y mejorada de realizar una evaluación no destructiva. Esto es parcialmente cierto, pero SHM es mucho más. Implica la integración de sensores, posiblemente materiales inteligentes, transmisión de datos, potencia computacional y capacidad de procesamiento en el interior de las estructuras. Permite reconsiderar el diseño de la estructura y la gestión completa de la propia estructura y de la estructura considerada como parte de sistemas más amplios.


%Incluir conjunto de elementos que influyen en el deterioro de la estructura. Envejecimiento, mantenimiento

En este sentido, el monitoreo de las estructuras se ha convertido en una herramienta esencial para garantizar la seguridad de las personas durante la vida útil de la misma, incluyendo la ocurrencia de eventos de cierta magnitud. Además, el monitoreo de las estructuras puede ayudar a mejorar la eficacia de las normas sismorresistentes, ya que permite validar y mejorar los modelos estructurales utilizados en la normativa.

Según (Nagayama, 2007), dado que las edificaciones suelen ser grandes y complejas, la información
de unos pocos sensores es inadecuada para evaluar con precisión el estado estructural. El
comportamiento dinámico de estas estructuras es complejo tanto a escala espacial como temporal. Además,
los daños y/o el deterioro es intrínsecamente un fenómeno local. Por lo tanto, para comprender el
comportamiento dinámico, el movimiento de las estructuras debe ser supervisado por sensores
con una frecuencia de muestreo suficiente para captar las características dinámicas más destacadas. Esta información combinada con el registro del comportamiento estático de la estructura permiten tener una visión más amplia del estado actual de la estructura.


El primer paso, además de un mantenimiento adecuado, para garantizar la seguridad de estas estructuras, es contar con sistemas de monitoreo que permitan detectar posibles daños o fallas en su funcionamiento y tomar medidas preventivas. Por tanto, los sistemas de adquisición de datos y monitoreo son herramientas esenciales en la prevención de accidentes y daños.

A su vez, según (Nagayama, 2007), un dispositivo inteligente, es decir, con capacidad de procesamiento de datos en el caso de los sensores, es una característica esencial que permite incrementar el potencial de los sensores al ser estos inalámbricos. Los sensores inteligentes pueden procesar localmente los datos medidos y trasmitir solo la información importante a través de comunicaciones inalámbricas. Cuando estos son configurados como una red, se extienden las capacidades de los mismos.

Los sensores inteligentes, con sus capacidades de cómputo y de comunicación integradas, ofrecen nuevas oportunidades para la SHM. Sin necesidad de cables de alimentación o comunicación, los costes de instalación pueden reducirse drásticamente. Los sensores inteligentes ayudarán a que el monitoreo de las estructuras con un denso conjunto de sensores sea económicamente práctico. Se espera que los sensores inteligentes instalados en masa sean fuentes de información muy valiosa para la SHM.

En este trabajo de grado se abordará el diseño para una futura implementación de un sistema de adquisición de datos de bajo costo basado dispositivos programables con capacidad de interconexión para el monitoreo y procesamiento de variables como aceleración, inclinación, humedad y temperatura en estructuras críticas, con el objetivo de prevenir daños y accidentes.

\newpage



\begin{center}
	\section*{ PLANTEAMIENTO DEL PROBLEMA}
\end{center}


\vspace{0.3cm}

La seguridad en las estructuras es un tema crítico en la ingeniería, especialmente en lo que respecta a estructuras críticas como edificios y puentes. A medida que el tiempo pasa, estas estructuras pueden deteriorarse y presentar fallas que pueden poner en riesgo la vida de las personas que las utilizan. Por lo tanto, es fundamental contar con sistemas de monitoreo que permitan detectar posibles problemas en las estructuras y tomar medidas preventivas.

Sin embargo, la mayoría de los sistemas de monitoreo de estructuras disponibles en el mercado son costosos y no están diseñados específicamente para aplicaciones de salud estructural. Además, muchos de estos sistemas no tienen capacidad para admitir comunicación inalámbrica, lo que limita su aplicación en estructuras de gran tamaño o en áreas de difícil acceso, además de los costos asociados a un sistema de cableado fiable que no perturbe las mediciones.



\newpage


\begin{center}
	\section*{JUSTIFICACIÓN}
\end{center}

\vspace{1cm}


Un sensor inteligente que conste de un sistema de adquisición de datos basado en un microcontrolador que recolecte y procese la información adquirida en conjunto con varios sensores dispuestos en un solo dispositivo permite realizar la medición de variables ambientales y mecánicas de interés en aplicaciones de salud estructural a un bajo costo; además, un sistema de este tipo es flexible en cuanto a las variables a medir, puesto que es compatible con distintos tipos de sensores y métodos de comunicación a utilizarse, permitiendo crear soluciones canalizadas a proyectos en específico, logrando disminuir costos y contribuir de una forma más eficaz al proceso de toma de decisiones estructurales.

Existe variedad en cuanto al hardware de bajo costo que puede utilizarse para la implementación de un prototipo del sistema, lo que ofrece una alternativa atractiva a los sistemas de adquisición actuales, además de la capacidad de transmisión que ofrecen las redes de larga distancia disponibles en conjunto con las capacidades de los microcontroladores, permitiendo que la estación base pueda ubicarse lejos de la estructura.

Para lograr medir todas estas variables en tiempo real es conveniente contar con uno o más microcontroladores capaces de gestionar toda esta información, sin dejar de lado la confiabilidad en la adquisición de estos datos y que a su vez sea capaz de emplear las herramientas necesarias para comunicar estos datos de forma inalámbrica.

Sabiendo esto, existe una necesidad clara de desarrollar un sistema de adquisición de datos y monitoreo inalámbrico de bajo costo para aplicaciones de salud estructural que permita reducir la acción humana, minimizando el error humano y los recursos necesarios, aumentando a su vez la confiabilidad y la seguridad; con este desarrollo se busca verificar que el desempeño de la estructura se encuentra entre los rangos establecidos, y en caso contrario notificarlas con el objetivo de garantizar la seguridad de las personas, mediante la realización de un plan de mantenimiento preventivo y correctivo.


\newpage


\begin{center}
	\section*{ANTECEDENTES}
\end{center}

\vspace{1cm}

La implementación de microcontroladores para sistemas de adquisición de datos es un tema que se ha desarrollado en múltiples aplicaciones. Muchas veces, se logran desarrollar soluciones que permiten disminuir costos sin que eso afecte la calidad de las mediciones. Las soluciones que se tomarán como referencia lograron: Desarrollar sistemas de adquisición de datos basados en microcontroladores para la medición de variables físicas. Además, se han logrado crear redes de sensores inteligentes que permiten facilitar el envío de los datos de forma inalámbrica.


El trabajo de \cite{federici2014design} en \textit{"Design of Wireless Sensor Nodes for Structural Health Monitoring applications"} publicado en Procedia Engineering, aborda el diseño de redes de sensores inalámbricos para el monitoreo del estado de estructuras civiles, centrándose específicamente en el diseño de nodos en relación con los requisitos de diferentes clases de aplicaciones de monitoreo estructural.
Los problemas de diseño se analizan con referencia específica a una configuración experimental a gran escala (el monitoreo estructural a largo plazo de la Basílica S. Maria di Collemaggio, L'Aquila, Italia). Se destacaron las principales limitaciones que surgieron y se esbozan las estrategias de solución adoptadas, tanto en el caso de la plataforma de detección comercial como de las soluciones totalmente personalizadas. Se revisan las opciones para el diseño y despliegue del sistema de monitoreo, tanto en el caso de la selección de plataformas comerciales como en el caso del desarrollo de plataformas a la medida. El análisis de los registros llevó a importantes consideraciones en la integridad estructural y seguridad, y permitió la identificación de los parámetros modales de la estructura.

En cuanto al desarrollo realizado por \cite{komarizadehasl2022development} en \textit{"Development of Low-Cost Sensors for Structural Health Monitoring Applications"}, el ingeniero Komarizadehasl propone cuatro sistemas de monitorización de alta precisión y bajo coste.
En primer lugar, para medir correctamente las respuestas estructurales, se desarrolla el \textit{Cost Hyper-Efficient Arduino Product} (CHEAP). CHEAP es un sistema compuesto por cinco acelerómetros sincronizados de bajo coste conectados a un microcontrolador Arduino que hace el papel de dispositivo de recogida de datos. Para validar su rendimiento, se efectuaron unos experimentos de laboratorio y sus resultados se compararon con los de dos acelerómetros de alta precisión (PCB393A03 y PCB 356B18). Se concluye que CHEAP puede usarse para Monitoreo de Salud Estructural en estructuras convencionales con frecuencias naturales bajas cuando se cuente con presupuestos de monitoreo escasos. 

En segundo lugar, se presenta un inclinómetro de bajo coste, un \textit{Low-cost Adaptable Reliable Angle-meter} (LARA), que mide la inclinación mediante la fusión de distintos sensores: cinco giroscopios y cinco acelerómetros. LARA combina un microcontrolador basado en la tecnología del Internet de las Cosas (\textit{NODEMCU}), que permite la transmisión inalámbrica de datos, y un software comercial gratuito para la recogida de datos (\textit{SerialPlot}). Para confirmar la precisión y resolución de este dispositivo, se compararon sus mediciones en condiciones de laboratorio con las teóricas y con las de un inclinómetro comercial (\textit{HI-INC}). Los resultados de laboratorio de una prueba de carga en una viga demuestran la notable precisión de LARA. Se concluye que la precisión de LARA es suficiente para su aplicación en la detección de daños en puentes.

En tercer lugar, también se dilucida el efecto de la combinación de sensores de rango similar para investigar el aumento de la precisión y la mitigación de los ruidos ambientales. Para investigar la teoría de la combinación de sensores, el ingeniero Komarizadehasl, propone un equipo de medición compuesto por 75 sensores para la medición de distancias acoplados a dos microcontroladores de Arduino. Los 75 sensores son 25 HC-SR04 (analógicos), 25 VL53L0X (digitales) y 25 VL53L1X (digitales). Los resultados muestran que promediando la salida de varios sensores sin calibrar, la precisión en la estimación de distancia aumenta considerablemente.

El último sistema propuesto por Komarizadehasl presenta un novedoso y versátil sistema de adquisición de datos a distancia que permite el registro del tiempo con una resolución de microsegundos para la sincronización posterior de las lecturas de los sensores inalámbricos situados en diversos puntos de una estructura. Esta funcionalidad es lo que permitiría su aplicación a pruebas de carga estáticas, quasi-estaticas o al análisis modal de las estructuras.

\cite{muttillo2019structural} en su trabajo \textit{"Structural health continuous monitoring of buildings - A modal parameters identification system"} propone en su tesis el diseño de un Sistema de Monitoreo de la Salud Estructural (\textit{Structural Health Monitoring}) para monitorear y comprobar continuamente el comportamiento estructural a lo largo de la vida útil del edificio. El sistema, compuesto por un datalogger personalizado y dispositivos esclavos, permite el monitoreo continuo de la aceleración de la estructura gracias a su facilidad de instalación y bajo coste. El sistema propuesto se basa principalmente en un microcontrolador que: 

\begin{itemize}
	\item Se comunica con los nodos a través del bus RS485, 
	\item Sincroniza las muestras de adquisición
	\item Adquiere los datos medidos por los nodos.
\end{itemize} 

El sistema fue probado en una estructura de aluminio en voladizo, a través de tres campañas experimentales diferentes y los datos medidos, recogidos en una memoria interna del \textit{datalogger}, fueron post-procesados a través de \textit{Matlab}. Los resultados permitieron evaluar con éxito los parámetros modales (frecuencias, amortiguamiento y formas modales) de la estructura analizada y su estado de salud.

%\cite{QCGJ2018} en su trabajo ”Sistema De Monitoreo de Variables Medioambientales Usando Una Red de Sensores Inalámbricos y Plataformas De Internet De Las Cosas” se pensó en un sistema para la recolección de datos meteorológicos usando una red de sensores inalámbricos (RSI), capaz de transmitir datos en tiempo real. El sistema logró automatizar procesos de obtención de datos de manera continua y a largo plazo, por medio de un módulo de abasteclmiento de energía solar que permite autonomía para su funcionamiento. Se propuso la utilización de dos sistemas: DigiMesh y WiFi. El procesamiento se realizó en un Arduino Uno, para la comunicación se utilizó un XBee PRO 900HP (DigiMesh) y un moduló Electric Imp.01 (WiFi). Adicionalmente se evaluó la transmisión de los datos hacia plataformas de Internet de las cosas (IoT), en donde se gestionará y visualizará los datos obtenidos por los nodos. Este sistema fue pensado como alternativa de bajo costo para sistemas meteorológicos y está basado en componentes de hardware y software libre. Al realizarse la validación de los datos obtenidos mediante un análisis estadístico con los datos registrados por una estación meteorológica se obtuvo un error relativo promedio máximo de 4,93 \%.\\


\newpage


\begin{center}
	\section*{OBJETIVOS}
\end{center}

\vspace{1cm}

\subsection*{OBJETIVO GENERAL}

Diseñar un sensor inteligente para aplicaciones de monitoreo de salud estructural.

\subsection*{OBJETIVOS ESPECÍFICOS}


\begin{enumerate}[1.]


	\item Documentar las principales características y la importancia del monitoreo para aplicaciones de salud estructural.

	\item Documentar los principales métodos de recolección de datos soportados por los sensores necesarios para la medición de las variables de interés.

	\item Evaluar y proponer el protocolo de comunicación que permita el envío fiable de los datos recolectados por el sensor.

	\item Seleccionar el hardware adecuado tanto sensores como microcontroladores para la implementación futura del sistema.

	\item Diseñar el módulo de programa encargado de la recolección y almacenamiento de los datos provenientes de los sensores.

	\item Desarrollar el módulo de programa encargado de la comunicación de los datos usando la plataforma escogida.
	
	\item Desarrollar una aplicación para el monitoreo y ajuste del sensor inteligente.

	\item Validar el funcionamiento del sistema haciendo uso de un prototipo de pruebas.

\end{enumerate}
\newpage



\begin{center}

	\section*{MARCO METODOLÓGICO}

\end{center}

\vspace{1cm}

Con la finalidad de realizar efectivamente cada objetivo del proyecto, se llevará a cabo la siguiente metodología:
\bigskip

\begin{enumerate}[1.]



	\item Documentar las principales características y la importancia del monitoreo para aplicaciones de salud estructural.

	      \begin{enumerate}

		      \item Investigación documental acerca de las principales características y aplicaciones de los sistemas de adquisición de datos en salud estructural.


	      \end{enumerate}

	\item  Documentar los principales métodos de recolección de datos soportados por los sensores necesarios para la medición de las variables de interés.


	      \begin{enumerate}

			  \item Comparación de los métodos para extracción de datos desde un sensor según las ventajas que ofrecen al usar un microcontrolador.

		      \item Selección del método de extracción de datos a ser utilizado en el sistema con miras a su posterior procesamiento y envío.

	      \end{enumerate}


	\item Evaluar y proponer el protocolo de comunicación que permita el envío fiable de los datos recolectados por el sensor inteligente.

	      \begin{enumerate}

		      \item Investigación documental sobre los protocolos de comunicación disponibles según el microcontrolador y el hardware de comunicaciones a utilizarse.

		      \item Comparación entre las distintas opciones.

		      \item Selección del método de comunicación y del hardware necesario para el envío y recepción de datos.

	      \end{enumerate}


	\item Seleccionar el hardware adecuado tanto sensores como microcontrolador para la implementación futura del sistema.


	      \begin{enumerate}

			\item Investigación documental sobre el hardware disponible en el mercado capaz de realizar las actividades previstas.

			\item Comparación entre sensores para cada una de las variables de interés.

			\item Comparación entre microcontroladores con base en la naturaleza de los sensores, su método de comunicación y su adaptabilidad al sistema.
			
		 	\item Selección del hardware a utilizarse en el prototipo.

	      \end{enumerate}

	\item Diseñar el módulo de programa encargado de la recolección y almacenamiento de los datos provenientes de los sensores.


	      \begin{enumerate}
			
			\item Programación para el manejo del hardware externo al microcontrolador, capaz de recibir datos de señales analógicas y digitales.
			\item Adquisición de datos de variables físicas.
			\item Almacenamiento de los datos en memoria no volátil.

	      \end{enumerate}



	\item Desarrollar el módulo de programa encargado de la comunicación de los datos usando la plataforma escogida.


	      \begin{enumerate}

			\item Programación para el manejo del hardware capaz de enviar los datos recolectados y manejo de la interfaz con uno o más microcontroladores.

	      \end{enumerate}

	\item Desarrollar una aplicación para el monitoreo y ajuste del sensor inteligente.


	      \begin{enumerate}

			\item Programación de una aplicación capaz de recolectar los datos, pre-procesarlos, verificar que ciertos valores estén en un rango de operación y notificarlo.

	      \end{enumerate}

	\item Validar el funcionamiento del sistema haciendo uso de un prototipo de pruebas.


	      \begin{enumerate}

			  \item Montaje de prototipo tomando en cuenta consideraciones de ruido y acondicionamiento de señales.
		      \item Pruebas para comprobar el funcionamiento del sistema de adquisición.
		      \item Pruebas para comprobar la comunicación hacia la estación base.
		      \item Comparar el desempeño del sensor en comparación con un equipo comercial.

	      \end{enumerate}





\end{enumerate}




\newpage


\begin{center}

	\section*{ALCANCE Y LIMITACIONES}
\end{center}

Los principales equipos de medición estructural cuentan con salidas digitales o en su defecto salidas analógicas que pueden ser convertidas y soportan distintos protocolos de comunicación, lo que representa una ventaja al trabajar con microcontroladores, pues estos son adaptables a la mayoría de los protocolos lo que facilita la adquisición de los datos a partir del medidor.

Los desafíos al usar el convertidor analógico-digital (ADC) de un microcontrolador en mediciones estructurales pueden ser varios. Uno de los principales es la precisión de la medición, ya que el ADC puede presentar errores en la conversión analógico-digital. Además, la resolución del ADC puede limitar la precisión de las mediciones, lo que puede afectar la calidad de los datos obtenidos, esto se verificará comparando los resultados obtenidos con datos.

Otro desafío importante es la velocidad de muestreo, ya que en aplicaciones estructurales puede ser necesario tomar mediciones en tiempo real para detectar posibles fallas o cambios en el comportamiento de la estructura. En este sentido, es importante que el ADC del microcontrolador tenga una velocidad de muestreo adecuada para las necesidades de la aplicación.

Para validar el funcionamiento del sistema diseñado se propondrá la realización un prototipo de pruebas para demostrar el desempeño del diseño.



% \begin{figure}[H]
% 	\centering
% 	\includegraphics[width=0.7\linewidth]{EsquemaTEG.png}
% 	\caption{Representación gráfica de la red descrita en el alcance.}
% 	\label{fig:esquemateg}
% \end{figure}

%\newfig{0.7}{imagenes/EsquemaTEG.png}{Representación gráfica de la red descrita en el alcance. PRUEBA}{esquemateg}

\newpage

\begin{center}
	\section*{ RECURSOS Y FACTIBILIDAD}
\end{center}

Para el desarrollo e implementación del sistema se requiere de microcontroladores, sensores de aceleración, humedad, inclinación y temperatura, módulos de radiofrecuencia y un computador para la instalación de la Interfaz de Desarrollo (IDE) necesario para el desarrollo de aplicaciones en el microcontrolador.

Tanto los microcontroladores como los módulos y sensores necesarios serán suministrados por el Instituto de Materiales y Modelos Estructurales (IMME). El IDE a utilizarse, en conjunto con los frameworks necesarios, se escogerán dependiendo del microcontrolador (MCU) escogido con base en la investigación a realizarse. El código estará disponible de forma libre en el controlador de versiones Github y contará con la documentación necesaria, escrita por los desarrolladores, para el correcto funcionamiento del sistema.

\newpage





\begin{center}

	\section*{CRONOGRAMA DE ACTIVIDADES}



	\vspace{0.5cm}
	\resizebox{\textwidth}{!}{
		\begin{ganttchart}[
				canvas/.append style={fill=none, draw=black!5, line width=.75pt},
				hgrid style/.style={draw=black!5, line width=.75pt},
				vgrid={*1{draw=black!5, line width=.75pt}},
				today label font=\small\bfseries,
				title/.style={draw=none, fill=none},
				title label font=\bfseries\footnotesize,
				title label node/.append style={below=6pt},
				include title in canvas=false,
				bar label font=\mdseries\small\color{black!70},
				bar label node/.append style={left=.3cm},
				bar/.append style={draw=none, fill=black!63},
				bar incomplete/.append style={fill=blue},
				bar progress label font=\mdseries\footnotesize\color{black!70},
				group incomplete/.append style={fill=blue},
				group left shift=0,
				group right shift=0,
				group height=.5,
				group peaks tip position=0,
				group label node/.append style={left=.3cm},
				group progress label font=\bfseries\small,
				link/.style={-latex, line width=1.5pt, red},
				link label font=\scriptsize\bfseries,
				link label node/.append style={below left=-2pt and 0pt},
			]{1}{28}
			\gantttitle{Diseñar un sensor inteligente para aplicaciones de monitoreo de salud estructural}{28} \\
			%\gantttitle{ }{28} \\[grid]
			\gantttitle[
				title label node/.append style={below left=7pt and -3pt}
			]{Semana:\quad1}{1}
			\gantttitlelist{2,...,28}{1} \\
			\ganttgroup{Documentar las aplicaciones de salud estructural}{1}{3} \\
			\ganttbar{\textbf{Investigación Documental}}{1}{1} \\
			\ganttbar{\textbf{Principales características de la salud estructural}}{2}{2} \\
			\ganttbar{\textbf{Importancia del monitoreo}}{3}{3} \\
			\ganttgroup{Documentar métodos de extracción}{4}{8} \\
			\ganttbar{\textbf{Investigación sobre sensores}}{4}{5} \\
			\ganttbar{\textbf{Compatibilidad con microcontroladores}}{5}{6} \\
			\ganttbar{\textbf{Características del Hardware}}{6}{8} \\
			\ganttgroup{Documentar y proponer el protocolo de comunicación}{9}{12} \\
			\ganttbar{\textbf{Investigación documental}}{9}{10} \\
			\ganttbar{\textbf{Compatibilidad con microcontroladores}}{10}{11} \\
			\ganttbar{\textbf{Manejo de buses}}{11}{12} \\
			\ganttgroup{Proponer el hardware necesario}{13}{16} \\
			\ganttbar{\textbf{Estudiar compatibilidad con la aplicación a desarrollar}}{13}{15} \\
			\ganttbar{\textbf{Escoger microcontrolador}}{13}{15} \\
			\ganttbar{\textbf{Escoger sensores y hardware de comunicación}}{15}{16} \\
			\ganttgroup{Desarrollar módulo encargado de recolección de datos}{17}{20} \\
			\ganttbar{\textbf{Extracción fiable de las variables}}{17}{19} \\
			\ganttbar{\textbf{Almacenamiento en memoria}}{19}{20} \\
			\ganttgroup{Desarrollar módulo encargado de la comunicación}{21}{22} \\
			\ganttbar{\textbf{Comunicación interna y externa del sistema}}{21}{22} \\
			%\ganttbar{\textbf{Comunicación interna y externa de la red}}{23}{24} \\
			\ganttgroup{Desarrollar una aplicación para el monitoreo y ajuste del sensor inteligente}{16}{23} \\
			\ganttbar{\textbf{Investigación documental sobre características de la aplicación}}{16}{18} \\			
			\ganttbar{\textbf{Desarrollo de aplicación de monitoreo y ajuste}}{18}{23} \\
			\ganttgroup{Validar funcionamiento}{20}{27} \\
			\ganttbar{\textbf{Montaje de prototipo}}{20}{25} \\
			\ganttbar{\textbf{Comparación con equipo comercial}}{25}{27} \\
			\ganttgroup{Redacción del Trabajo Especial de Grado}{3}{28} \\
		\end{ganttchart}
	}

\end{center}

\newpage

\bibliographystyle{apalike}

\bibliography{JT_Anteproyecto}
\nocite{*}

\end{document}

