La seguridad de las infraestructuras es un tema de gran importancia en la actualidad, especialmente cuando se trata de estructuras como edificios o puentes.

Los primeros indicios del monitoreo del estado de las infraestructuras data de nuestros comienzos como especie sedentaria.  En la antigüedad, los especialistas utilizaban técnicas de inspección visual y auditiva para detectar posibles problemas en las estructuras, como grietas o ruidos inusuales. Con el tiempo, se desarrollaron técnicas más avanzadas para el monitoreo de estructuras, como la utilización de medidores de deformación, inclinación, sensores de vibración, entre otros.

La integración de la instrumentación con el análisis estructural comenzó a desarrollarse en la década de 1960 con el advenimiento de la informática y la disponibilidad de computadoras capaces de realizar cálculos estructurales complejos. En esa época, se comenzaron a utilizar sistemas de adquisición de datos para recopilar información sobre el comportamiento de las estructuras en tiempo real y utilizarla para calibrar y validar los modelos estructurales.

Actualmente, las normas sismo-resistentes apuntan a estructuras que sean capaces de mantener su integridad ante un evento de cierta magnitud. Además, el monitoreo continuo de ciertos indicadores en la estructura permiten determinar un índice de la salud estructural y ajustar el modelo a las condiciones actuales de la misma para evaluar el cumplimiento de la normativa sismorresistente. Para el monitoreo a largo plazo, el resultado de este proceso es información actualizada periódicamente sobre la capacidad de la estructura para desempeñar su función prevista a la luz del inevitable envejecimiento y degradación resultantes de los entornos operativos.

Según \citep{balageas2010structural} (Balageas, 2010), el monitoreo de la salud estructural (SHM) tiene por objeto proporcionar, en cada momento de la vida de una estructura, un diagnóstico del estado de los materiales constitutivos, de las diferentes
partes, y del conjunto de estas partes que constituyen la estructura en su totalidad. El estado de la estructura debe permanecer en el ámbito especificado en el diseño, aunque este puede verse alterado por el envejecimiento normal debido al uso, por la acción del medio ambiente y por sucesos accidentales. Gracias a la dimensión temporal de la supervisión, que permite tener en cuenta toda la base de datos histórica de la estructura, y con la ayuda del monitoreo del funcionamiento. También puede proporcionar un pronóstico (evolución de los daños, vida residual, entre otros).

Si consideramos solo la primera función, el diagnóstico, podríamos estimar que el monitoreo de la salud estructural es una forma nueva y mejorada de realizar una evaluación no destructiva. Esto es parcialmente cierto, pero SHM es mucho más. Implica la integración de sensores, posiblemente materiales inteligentes, transmisión de datos, potencia computacional y capacidad de procesamiento en el interior de las estructuras. Permite reconsiderar el diseño de la estructura y la gestión completa de la propia estructura y de la estructura considerada como parte de sistemas más amplios.


En este archivo debe escribir su introducción.

De acuerdo a Brea  la transformada de Laplace debe estudiarse como
una función definida en el campo de los números complejos
\cite{brea5}.

Otro mode de referencial es \citep{brea5}

El resto del reporte consta de: en el Capítulo \ref{CAP:hist} se
describe...

En el trabajo se emplea el enfoque de \cite{brigham1}

De acuerdo a la ecuación
