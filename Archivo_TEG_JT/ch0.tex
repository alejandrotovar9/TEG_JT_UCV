La seguridad de las infraestructuras es un tema de gran importancia en la actualidad, especialmente cuando se trata de estructuras como edificios o puentes.

Los primeros indicios del monitoreo del estado de las infraestructuras data de nuestros comienzos como especie sedentaria.  En la antigüedad, los especialistas utilizaban técnicas de inspección visual y auditiva para detectar posibles problemas en las estructuras, como grietas o ruidos inusuales. Con el tiempo, se desarrollaron técnicas más avanzadas para el monitoreo de estructuras, como la utilización de medidores de deformación, inclinación, sensores de vibración, entre otros.

La integración de la instrumentación con el análisis estructural comenzó a desarrollarse en la década de 1960 con el advenimiento de la informática y la disponibilidad de computadoras capaces de realizar cálculos estructurales complejos. En esa época, se comenzaron a utilizar sistemas de adquisición de datos para recopilar información sobre el comportamiento de las estructuras en tiempo real y utilizarla para calibrar y validar los modelos estructurales.

Actualmente, las normas sismo-resistentes apuntan a estructuras que sean capaces de mantener su integridad ante un evento de cierta magnitud. Además, el monitoreo continuo de ciertos indicadores en la estructura permiten determinar un índice de la salud estructural y ajustar el modelo a las condiciones actuales de la misma para evaluar el cumplimiento de la normativa sismorresistente. Para el monitoreo a largo plazo, el resultado de este proceso es información actualizada periódicamente sobre la capacidad de la estructura para desempeñar su función prevista a la luz del inevitable envejecimiento y degradación resultantes de los entornos operativos.

Según \citep{balageas2010structural}, el monitoreo de la salud estructural (SHM) tiene por objeto proporcionar, en cada momento de la vida de una estructura, un diagnóstico del estado de los materiales constitutivos, de las diferentes
partes, y del conjunto de estas partes que constituyen la estructura en su totalidad. El estado de la estructura debe permanecer en el ámbito especificado en el diseño, aunque este puede verse alterado por el envejecimiento normal debido al uso, por la acción del medio ambiente y por sucesos accidentales. Gracias a la dimensión temporal de la supervisión, que permite tener en cuenta toda la base de datos histórica de la estructura, y con la ayuda del monitoreo del funcionamiento. También puede proporcionar un pronóstico (evolución de los daños, vida residual, entre otros).

Si consideramos solo la primera función, el diagnóstico, podríamos estimar que el monitoreo de la salud estructural es una forma nueva y mejorada de realizar una evaluación no destructiva. Esto es parcialmente cierto, pero SHM es mucho más. Implica la integración de sensores, posiblemente materiales inteligentes, transmisión de datos, potencia computacional y capacidad de procesamiento en el interior de las estructuras. Permite reconsiderar el diseño de la estructura y la gestión completa de la propia estructura y de la estructura considerada como parte de sistemas más amplios.


%Incluir conjunto de elementos que influyen en el deterioro de la estructura. Envejecimiento, mantenimiento

En este sentido, el monitoreo de las estructuras se ha convertido en una herramienta esencial para garantizar la seguridad de las personas durante la vida útil de la misma, incluyendo la ocurrencia de eventos de cierta magnitud. Además, el monitoreo de las estructuras puede ayudar a mejorar la eficacia de las normas sismorresistentes, ya que permite validar y mejorar los modelos estructurales utilizados en la normativa.

Según \citep{nagayama2007structural}, dado que las edificaciones suelen ser grandes y complejas, la información
de unos pocos sensores es inadecuada para evaluar con precisión el estado estructural. El
comportamiento dinámico de estas estructuras es complejo tanto a escala espacial como temporal. Además,
los daños y/o el deterioro es intrínsecamente un fenómeno local. Por lo tanto, para comprender el
comportamiento dinámico, el movimiento de las estructuras debe ser supervisado por sensores
con una frecuencia de muestreo suficiente para captar las características dinámicas más destacadas. Esta información combinada con el registro del comportamiento estático de la estructura permiten tener una visión más amplia del estado actual de la estructura.


El primer paso, además de un mantenimiento adecuado, para garantizar la seguridad de estas estructuras, es contar con sistemas de monitoreo que permitan detectar posibles daños o fallas en su funcionamiento y tomar medidas preventivas. Por tanto, los sistemas de adquisición de datos y monitoreo son herramientas esenciales en la prevención de accidentes y daños.

A su vez, según \citep{nagayama2007structural}, un dispositivo inteligente, es decir, con capacidad de procesamiento de datos en el caso de los sensores, es una característica esencial que permite incrementar el potencial de los sensores al ser estos inalámbricos. Los sensores inteligentes pueden procesar localmente los datos medidos y trasmitir solo la información importante a través de comunicaciones inalámbricas. Cuando estos son configurados como una red, se extienden las capacidades de los mismos.

Los sensores inteligentes, con sus capacidades de cómputo y de comunicación integradas, ofrecen nuevas oportunidades para la SHM. Sin necesidad de cables de alimentación o comunicación, los costes de instalación pueden reducirse drásticamente. Los sensores inteligentes ayudarán a que el monitoreo de las estructuras con un denso conjunto de sensores sea económicamente práctico. Se espera que los sensores inteligentes instalados en masa sean fuentes de información muy valiosa para la SHM.

En este trabajo de grado se abordará el diseño para una futura implementación de un sistema de adquisición de datos de bajo costo basado dispositivos programables con capacidad de interconexión para el monitoreo y procesamiento de variables como aceleración, inclinación, humedad y temperatura en estructuras críticas, con el objetivo de prevenir daños y accidentes.



De acuerdo a Brea  la transformada de Laplace debe estudiarse como
una función definida en el campo de los números complejos
\cite{brea5}.

Otro modo de referencial es \citep{brea5}

El resto del reporte consta de: en el Capítulo \ref{CAP:hist} se
describe...

En el trabajo se emplea el enfoque de \cite{brigham1}

De acuerdo a la ecuación
