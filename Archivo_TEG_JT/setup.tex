%%%%%%%%%%%%%%%%%%%%%%%%%%%% LIBRERIAS %%%%%%%%%%%%%%%%%%%%%%%%%%%%%%%%%%%%%%%%%%%%

\usepackage[spanish]{babel}
\usepackage[utf8]{inputenc}
%\usepackage[latin1]{inputenc}
\usepackage[T1]{fontenc}  %Estilo de fuente times new roman

\usepackage{amssymb}
\usepackage{amsfonts}
\usepackage{amsmath}
\usepackage{latexsym}
\usepackage[letterpaper]{geometry}

\usepackage{float}
\usepackage{makeidx}
\usepackage{color}

\usepackage{tocbibind}
\usepackage{acronym}
\usepackage{caption2}
\usepackage{epsfig}
\usepackage{graphicx}
%\usepackage{slashbox}
\usepackage{setspace}
\usepackage{multicol}
\usepackage{longtable}
%\usepackage{doublespace}

\usepackage{fancyhdr}
%\usepackage{fancyheadings}

\usepackage{booktabs}

%========= Define el estilo de referencias ===============
%\usepackage[round,authoryear]{natbib}%\usepackage[square,numbers]{natbib}%
%\usepackage[comma,authoryear]{natbib} esto está abajo

%========= Define el estilo de referencias APA ===============
\usepackage[natbibapa]{apacite}%natbibapa
%\usepackage[apaciteclassic]{apacite}%natbibapa
\usepackage[compact]{titlesec} %modificar espaciado

\usepackage{url}
\usepackage{hyperref}
%\usepackage[dvips,colorlinks=true,urlcolor=red,citecolor=black,anchorcolor=black,linkcolor=black]{hyperref}

%%%%%%%%%%%%%%%%%%%%%%%%%%%%%%%%%%%%%%%%%%%%%%%%%%%%%%%%%%%%%%%%%%
%       Definición del Documento PDF, (PDFLaTeX)        %
%%%%%%%%%%%%%%%%%%%%%%%%%%%%%%%%%%%%%%%%%%%%%%%%%%%%%%%%%%%%%%%%%%

\hypersetup{pdfauthor=Nombre}

\hypersetup{pdftitle=Título}%

\hypersetup{pdfkeywords=Palabras clave}

\pdfstringdef{\Produce}{Escuela de Ingeniería Eléctrica, Facultad de Ingeniería, UCV}%

\pdfstringdef{\area}{Área del trabajo}

\hypersetup{pdfproducer=\Produce}

\hypersetup{pdfsubject=\area}

\hypersetup{bookmarksnumbered=true}

%%%%%%%%%%%%%%%%%%%%%%%%%%%%%%%%%%%%%%%%%%%%%%%%%%%%%%%%%%%%%%%%%%

%\setcounter{MaxMatrixCols}{10}

%===================== Re-definición de Ambientes =================
\newtheorem{theorem}{Teorema}
\newtheorem{acknowledgement}[theorem]{Acknowledgement}
\newtheorem{algoritmo}[theorem]{Algorithm}
\newtheorem{supuestos}[theorem]{Supuestos}
\newtheorem{hipotesis}[theorem]{Hipótesis}
\newtheorem{axiom}[theorem]{Axiom}
\newtheorem{case}[theorem]{Case}
\newtheorem{claim}[theorem]{Claim}
\newtheorem{conclusion}[theorem]{Conclusión}
\newtheorem{condition}{Condición}
\newtheorem{conjecture}{Conjecture}
\newtheorem{corollary}{Corollary}
\newtheorem{criterion}{Criterion}
\newtheorem{definition}{Definición}  %{Definition}
\newtheorem{example}[theorem]{Ejemplo}%{Example}
\newtheorem{exercise}[theorem]{Exercise}
\newtheorem{lemma}{Lemma}
\newtheorem{notation}[theorem]{Notation}
\newtheorem{problem}{Problem}
\newtheorem{property}{Property}
\newtheorem{proposition}{Proposition}
\newtheorem{remark}[theorem]{Remark}
\newtheorem{solution}{Solution}
\newtheorem{summary}[theorem]{Summary}
\newenvironment{proof}[1][Proof]{\noindent\textbf{#1.} }{\ \rule{0.5em}{0.5em}}%

\numberwithin{equation}{chapter}%
\numberwithin{figure}{chapter}%
\numberwithin{table}{chapter}%
\numberwithin{definition}{chapter}%
\numberwithin{lemma}{chapter}%
\numberwithin{theorem}{chapter}%
\numberwithin{corollary}{chapter}%
\numberwithin{condition}{chapter}%
\numberwithin{criterion}{chapter}%
 \numberwithin{problem}{chapter}%
\numberwithin{property}{chapter}%
\numberwithin{proposition}{chapter}%
\numberwithin{solution}{chapter}%
\numberwithin{conjecture}{chapter}%

%==================== Separación en sílabas ========================
\hyphenpenalty=6800%

%A
\hyphenation{a-pro-xi-ma-do}


%B
\hyphenation{ba-lan-ce}%

%C
\hyphenation{co-la-bo-ra-do-res}%
\hyphenation{co-rres-pon-dien-tes}%
\hyphenation{co-rres-pon-dien-te}%
\hyphenation{con-ti-nua-men-te}%
\hyphenation{con-si-de-ra-cio-nes}%
\hyphenation{cons-tru-ir}%
\hyphenation{con-si-de-ra-do}%


%D
\hyphenation{di-fe-ren-cia}%
\hyphenation{des-cri-tos}%
\hyphenation{dis-mi-nu-ye}%
\hyphenation{des-cri-to}%
\hyphenation{de-pen-dien-tes}%


%E
\hyphenation{ex-pe-ri-men-to}
\hyphenation{ex-pe-ri-men-ta-cion} %


%P
\hyphenation{pro-ba-bi-li-da-des}%
\hyphenation{pro-ba-bi-li-dad}%
\hyphenation{par-ti-cu-lar}%

%M
\hyphenation{mo-da-li-da-des}%
\hyphenation{mo-de-lo} %
\hyphenation{me-dian-te}%
 \hyphenation{man-te-ni-mien-tos}%
%N

%O
\hyphenation{ope-ra-cio-nal}%
\hyphenation{o-pe-ra-cion}%
\hyphenation{o-pe-ra-cio-nes} %
\hyphenation{o-pe-ra-do-ra}%


%==================== Diseño de Página =============================
%\pagestyle{headings}
%\setlength{\headheight}{0.2cm}
\setlength{\textwidth}{14.52cm}%
%\pagestyle{fancy}
\renewcommand{\chaptermark}[1]{\markboth{#1}{}}
%\renewcommand{\sectionmark}[1]{\markright{\thesection\ #1}}
%\rhead[\fancyplain{}{\bfseries\thepage}]{\fancyplain{}{\bfseries\rightmark}}%\thepage
%\lhead[\fancyplain{}{\bfseries\leftmark}]{\fancyplain{}{\bfseries}} \cfoot{}%

%\fancyhead[R]{}

\rfoot[\fancyplain{}{\textit{E. Brea}}] {\fancyplain{}{}}
\lfoot[\fancyplain{}{}] {\fancyplain{}{\textit{}}}    %%%%%%%%%%%%%%%%%%% OJO ACA %%%%%%%%%%
\cfoot[\fancyplain{}{}] {\fancyplain{}{\bfseries\thepage}}
%\setlength{\footrulewidth}{0.0pt}%
%\setlength{\headrulewidth}{0.1pt}%

%===================================================================

%================== Diseño de Párrafo y delimitador ================
\renewcommand{\baselinestretch}{1.5}% Espaciado entre linea
\geometry{left=4cm,right=3cm,top=3cm,bottom=3cm}
\frenchspacing %
%\raggedright % Sólo para justificar el texto a la izquierda
\renewcommand{\captionlabeldelim}{.}%
\setlength{\parindent}{0.7cm}% Espacio de la sangría
\setlength{\parskip}{14pt plus 1pt minus 1pt}% Separación entre párrafos

%\setlength{\parskip}{1ex plus 0.5ex minus 0.2ex}%

%===================================================================

%==========================  Español venezolano =====================
%%Personalización de caption
\addto\captionsspanish{%
  \def\prefacename{Prefacio}%
  \def\refname{REFERENCIAS}%
  \def\abstractname{Resumen}%
  \def\bibname{REFERENCIAS}%{Bibliografía}%
  \def\chaptername{CAPÍTULO}%
  \def\appendixname{Apéndice}%{Anexo}
  \def\contentsname{ÍNDICE GENERAL}
  \def\listfigurename{LISTA DE FIGURAS}%Índice de Figuras\hspace*{10em}
  \def\listfigurenameTofC{LISTA DE FIGURAS}%Índice de Figuras
  \def\listtablename{LISTA DE TABLAS}%Índice de Tablas
  \def\indexname{Índice alfabético}%
  \def\figurename{Figura}%
  \def\tablename{Tabla}%
  \def\partname{Parte}%
  \def\enclname{Adjunto}%
  \def\ccname{Copia a}%
  \def\headtoname{A}%
  \def\pagename{Página}%
  \def\seename{véase}%
  \def\alsoname{véase también}%
  \def\proofname{Demostración}%
  \def\glossaryname{Glosario}
  }%

%==================================================================

%\setcounter{secnumdepth}{1}
%\setcounter{page}{4}
%\addtocounter{page}{4}%

\pagenumbering{roman}

\makeindex

%%%%%%%%%%%%%%%%%%%%%%%%%%%%%%%%%%%%%%%%%%%%%%%%%%%%%%%%%%%%%%%%%