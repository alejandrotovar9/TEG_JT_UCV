\documentclass[letterpaper,titlepage,12pt,oneside,spanish,final]{report_eie}

%PARA QUE APAREZCAN LAS SUBSUBSECTIONS EN EL INDICE
%\setcounter{tocdepth}{3}

%\documentclass[letterpaper,titlepage,12pt,twoside,openright,spanish,final]{report_eie}

%%%%%%%%%%%%%%%%%%%%%%%%%%%% LIBRERIAS %%%%%%%%%%%%%%%%%%%%%%%%%%%%%%%%%%%%%%%%%%%%

\usepackage[spanish]{babel}
\usepackage[utf8]{inputenc}
%\usepackage[latin1]{inputenc}
\usepackage[T1]{fontenc}  %Estilo de fuente times new roman

\usepackage{subfig}

\usepackage{amssymb}
\usepackage{amsfonts}
\usepackage{amsmath}
\usepackage{latexsym}
\usepackage[letterpaper]{geometry}

\usepackage{float}
\usepackage{makeidx}
\usepackage{color}

\usepackage{tocbibind}
\usepackage{acronym}
%OJO CAMBIE A CAPTION ENVEZ DE CAPTION2
\usepackage{caption} %EL CENTER CENTRA TODOS LOS 
\captionsetup[figure]{format=plain,justification=centering}%CAPTIONS DE TABLAS Y FIGURAS OJO
\usepackage{epsfig}
\usepackage{graphicx}
%\usepackage{slashbox}
\usepackage{setspace}
\usepackage{multicol}
\usepackage{longtable}
%\usepackage{doublespace}
\usepackage{array}
\newcolumntype{P}[1]{>{\centering\arraybackslash}p{#1}}


\usepackage{fancyhdr}
%\usepackage{fancyheadings}

\usepackage{booktabs}

%NUMEROS DE LINEA
% \usepackage[pagewise]{lineno}
% \linenumbers

%========= Define el estilo de referencias ===============
%\usepackage[round,authoryear]{natbib}%\usepackage[square,numbers]{natbib}%
%\usepackage[comma,authoryear]{natbib} esto está abajo

%========= Define el estilo de referencias APA ===============
\usepackage[natbibapa]{apacite}%natbibapa
%\usepackage[square, numbers]{natbib}
%\usepackage[numbers]{natbib} %PARA QUE SEAN NUMEROS EN CORCHETE
%\usepackage[apaciteclassic]{apacite}%natbibapa
%\usepackage[compact]{titlesec} %modificar espaciado

\usepackage{url}
\usepackage{hyperref}
%\usepackage[dvips,colorlinks=true,urlcolor=red,citecolor=black,anchorcolor=black,linkcolor=black]{hyperref}

%%%%%%%%%%%%%%%%CODIGO%%%%%%%%%%%%%%%%

\usepackage{listings}
\usepackage{xcolor}

\definecolor{codegreen}{rgb}{0,0.6,0}
\definecolor{codegray}{rgb}{0.5,0.5,0.5}
\definecolor{codepurple}{rgb}{0.58,0,0.82}
\definecolor{backcolour}{rgb}{0.95,0.95,0.92}

\lstdefinestyle{mystyle}{
    backgroundcolor=\color{backcolour},   
    commentstyle=\color{codegreen},
    keywordstyle=\color{blue},
    numberstyle=\tiny\color{codegray},
    stringstyle=\color{codepurple},
    basicstyle=\ttfamily\footnotesize,
    breakatwhitespace=false,         
    breaklines=true,                 
    captionpos=b,                    
    keepspaces=true,                 
    numbers=left,                    
    numbersep=5pt,                  
    showspaces=false,                
    showstringspaces=false,
    showtabs=false,                  
    tabsize=2
}

\lstset{style=mystyle}

\def\lstlistingname{Código}


%%%%%%%%%%%%%%%%%%%%%%%%%%%%%%%%%%%%%%%%%%%%%%%%%%%%%%%%%%%%%%%%%%
%       Definición del Documento PDF, (PDFLaTeX)        %
%%%%%%%%%%%%%%%%%%%%%%%%%%%%%%%%%%%%%%%%%%%%%%%%%%%%%%%%%%%%%%%%%%

\hypersetup{pdfauthor=Nombre}

\hypersetup{pdftitle=Título}%

\hypersetup{pdfkeywords=Palabras clave}

\pdfstringdef{\Produce}{Escuela de Ingeniería Eléctrica, Facultad de Ingeniería, UCV}%

\pdfstringdef{\area}{Área del trabajo}

\hypersetup{pdfproducer=\Produce}

\hypersetup{pdfsubject=\area}

\hypersetup{bookmarksnumbered=true}

%%%%%%%%%%%%%%%%%%%%%%%%%%%%%%%%%%%%%%%%%%%%%%%%%%%%%%%%%%%%%%%%%%

%\setcounter{MaxMatrixCols}{10}

%===================== Re-definición de Ambientes =================
\newtheorem{theorem}{Teorema}
\newtheorem{acknowledgement}[theorem]{Acknowledgement}
\newtheorem{algoritmo}[theorem]{Algorithm}
\newtheorem{supuestos}[theorem]{Supuestos}
\newtheorem{hipotesis}[theorem]{Hipótesis}
\newtheorem{axiom}[theorem]{Axiom}
\newtheorem{case}[theorem]{Case}
\newtheorem{claim}[theorem]{Claim}
\newtheorem{conclusion}[theorem]{Conclusión}
\newtheorem{condition}{Condición}
\newtheorem{conjecture}{Conjecture}
\newtheorem{corollary}{Corollary}
\newtheorem{criterion}{Criterion}
\newtheorem{definition}{Definición}  %{Definition}
\newtheorem{example}[theorem]{Ejemplo}%{Example}
\newtheorem{exercise}[theorem]{Exercise}
\newtheorem{lemma}{Lemma}
\newtheorem{notation}[theorem]{Notation}
\newtheorem{problem}{Problem}
\newtheorem{property}{Property}
\newtheorem{proposition}{Proposition}
\newtheorem{remark}[theorem]{Remark}
\newtheorem{solution}{Solution}
\newtheorem{summary}[theorem]{Summary}
\newenvironment{proof}[1][Proof]{\noindent\textbf{#1.} }{\ \rule{0.5em}{0.5em}}%

\numberwithin{equation}{chapter}%
\numberwithin{figure}{chapter}%
\numberwithin{table}{chapter}%
\numberwithin{definition}{chapter}%
\numberwithin{lemma}{chapter}%
\numberwithin{theorem}{chapter}%
\numberwithin{corollary}{chapter}%
\numberwithin{condition}{chapter}%
\numberwithin{criterion}{chapter}%
 \numberwithin{problem}{chapter}%
\numberwithin{property}{chapter}%
\numberwithin{proposition}{chapter}%
\numberwithin{solution}{chapter}%
\numberwithin{conjecture}{chapter}%

%==================== Separación en sílabas ========================
\hyphenpenalty=6800%
\input{silabear.tex}

%==================== Diseño de Página =============================
%\pagestyle{headings}
%\setlength{\headheight}{0.2cm}
\setlength{\textwidth}{14.52cm}%
%\pagestyle{fancy}
\renewcommand{\chaptermark}[1]{\markboth{#1}{}}
%\renewcommand{\sectionmark}[1]{\markright{\thesection\ #1}}
%\rhead[\fancyplain{}{\bfseries\thepage}]{\fancyplain{}{\bfseries\rightmark}}%\thepage
%\lhead[\fancyplain{}{\bfseries\leftmark}]{\fancyplain{}{\bfseries}} \cfoot{}%

%\fancyhead[R]{}

\rfoot[\fancyplain{}{\textit{E. Brea}}] {\fancyplain{}{}}
\lfoot[\fancyplain{}{}] {\fancyplain{}{\textit{}}}    %%%%%%%%%%%%%%%%%%% OJO ACA %%%%%%%%%%
\cfoot[\fancyplain{}{}] {\fancyplain{}{\bfseries\thepage}}
%\setlength{\footrulewidth}{0.0pt}%
%\setlength{\headrulewidth}{0.1pt}%

%===================================================================

%================== Diseño de Párrafo y delimitador ================
\renewcommand{\baselinestretch}{1.5}% Espaciado entre linea
\geometry{left=4cm,right=3cm,top=3cm,bottom=3cm}
\frenchspacing %
%\raggedright % Sólo para justificar el texto a la izquierda
%\renewcommand{\captionlabeldelim}{.}%
\setlength{\parindent}{0.7cm}% Espacio de la sangría
\setlength{\parskip}{14pt plus 1pt minus 1pt}% Separación entre párrafos

%\setlength{\parskip}{1ex plus 0.5ex minus 0.2ex}%

%===================================================================

%==========================  Español venezolano =====================
%%Personalización de caption
\addto\captionsspanish{%
  \def\prefacename{Prefacio}%
  \def\refname{REFERENCIAS}%
  \def\abstractname{Resumen}%
  \def\bibname{REFERENCIAS}%{Bibliografía}%
  \def\chaptername{CAPÍTULO}%
  \def\appendixname{Apéndice}%{Anexo}
  \def\contentsname{ÍNDICE GENERAL}
  \def\listfigurename{LISTA DE FIGURAS}%Índice de Figuras\hspace*{10em}
  \def\listfigurenameTofC{LISTA DE FIGURAS}%Índice de Figuras
  \def\listtablename{LISTA DE TABLAS}%Índice de Tablas
  \def\indexname{Índice alfabético}%
  \def\figurename{Figura}%
  \def\tablename{Tabla}%
  \def\partname{Parte}%
  \def\enclname{Adjunto}%
  \def\ccname{Copia a}%
  \def\headtoname{A}%
  \def\pagename{Página}%
  \def\seename{véase}%
  \def\alsoname{véase también}%
  \def\proofname{Demostración}%
  \def\glossaryname{Glosario}
  }%

%==================================================================

%\setcounter{secnumdepth}{1}
%\setcounter{page}{4}
%\addtocounter{page}{4}%

\pagenumbering{roman}

\makeindex

%%%%%%%%%%%%%%%%%%%%%%%%%%%%%%%%%%%%%%%%%%%%%%%%%%%%%%%%%%%%%%%%%

\begin{document}
%\frontmatter

%%%%%%%%%%%%%%%%%%%%%%%%%%%%%%%%%%%%%%%%%%%%%%%%%%%

%               Primera Página
%================================== Portada ========================
\input{portada.tex}

%======================= Constancia de Aprobación ===================
%\newpage
\begin{figure}
        \begin{center}
        %\centering
        %\includegraphics[height=23cm]{aprobacion.eps}

        \vspace{0.5mm}
        \label{Fig.aprobacion}
        \end{center}
        \end{figure}
\thispagestyle{empty}

%======================= Mención Honorífica =========================
\newpage
%\thispagestyle{empty}

\begin{figure}
        \begin{center}
        %\centering
        %\includegraphics[height=24cm]{mencion.eps}
        \vspace{0.5mm}
        \label{Fig.mencion}
        \end{center}
\end{figure}
\thispagestyle{empty}

%======================= Página de Dedicatoria ======================
\newpage%
\newenvironment{dedication}%
{\cleardoublepage \thispagestyle{empty} \vspace*{\stretch{1}}%
\begin{center} \em} {\end{center} \vspace*{\stretch{3}} }%
\begin{dedication}%
A quien desees dedicar este trabajo
\end{dedication}%

%=============================== RECONOCIMIENTOS Y AGRADECIMIENTOS ===================================
\chapter*{RECONOCIMIENTOS Y AGRADECIMIENTOS}
%\markboth{Reconocimientos}{Reconocimientos}%
\addcontentsline{toc}{chapter}{RECONOCIMIENTOS Y AGRADECIMIENTOS}%
%\setlength{\parskip}{0.2cm}%
%\input{agradecimientos.tex}%

%======================= Página de Resumen (Abstract) ==========================
\newpage
\renewcommand*{\abstract}{\begin{center}\end{center}}
%\begin{abstract}
\begin{spacing}{1}
\begin{center}%

\textbf{Autor del Trabajo de Grado}

\begin{large}
\textbf{Título del Trabajo de Grado}
\end{large}
\end{center}

\noindent%
\textbf{Tutor Académico: MSc Jose Romero. Tesis.
Caracas, Universidad Central de Venezuela. Facultad de Ingeniería.
Escuela de Ingeniería Eléctrica. Mención Electrónica y Control. Año 2023,
xvii, 144 pp.}

\noindent
\textbf{Palabras Claves:} Palabras clave. \\[1ex]

\noindent \textbf{Resumen.-} Escribe acá tu resumen

\end{spacing}

%\underline{RESUMEN}
%
\thispagestyle{empty}%
%\input{resumen.tex}%
%\end{abstract}
%====================== Páginas de Contenidos =====================
\renewcommand{\baselinestretch}{1.5}% Espaciado entre linea
\addtocounter{page}{3}%
\setlength{\parskip}{3pt}% Separación entre párrafos

\tableofcontents%

\listoffigures%

\listoftables%


%==================================================================
\chapter*{LISTA DE ACRÓNIMOS}%
%\markboth{Lista de Acrónimos}{Lista de Acrónimos}%
\addcontentsline{toc}{chapter}{LISTA DE ACRÓNIMOS}%
%

\begin{acronym}
\acro{SHM}{Structural Health Monitoring}
\acro{ADC}{Analog-to-Digital Converter}
\acro{NASA}{National Aeronautics and Space Administration}
\acro{SMIS}{Shuttle Modal Inspection System}
\acro{DOF}{Degrees of Freedom}
\acro{VLSI}{Very Large Scale Integration}
\acro{RAM}{Random Access Memory}
\acro{ROM}{Read-Only Memory}
\acro{DAC}{Digital-to-Analog Converter}
\acro{SoC}{System on a Chip}
\acro{MCU}{Microcontroller Unit}
\acro{SBC}{Single Board Computer}
\acro{UART}{Universal Asynchronous Receiver/Transmitter}
\acro{I2C}{Inter-Integrated Circuit}
\acro{SDA}{Serial Data Line}
\acro{SCL}{Serial Clock Line}
\acro{SPI}{Serial Peripheral Interface}
\acro{USB}{Universal Serial Bus}
\acro{RTOS}{Real-Time Operating System}
\acro{SMP}{Symmetric Multiprocessing}
\acro{MIT}{Massachusetts Institute of Technology}
\acro{FIFO}{First In, First Out}
\acro{MEMS}{Micro-Electro-Mechanical Systems}
\acro{DSP}{Digital Signal Processing}
\acro{IEEE}{Institute of Electrical and Electronics Engineers}
\acro{DAQ}{Data Acquisition}
\acro{WiFi}{Wireless Fidelity}
\acro{MQTT}{Message Queuing Telemetry Transport}
\acro{IoT}{Internet of Things}
\acro{M2M}{Machine to Machine}
\acro{CSS}{Chirp Spread Spectrum}
\acro{CRC}{Cyclic Redundancy Check}
\acro{FFT}{Fast Fourier Transform}
\acro{NTP}{Network Time Protocol}
\acro{RTC}{Real-Time Clock}
\acro{IMU}{Inertial Measurement Unit}
\acro{MARG}{Magnetic, Angular Rate, and Gravity}
\acro{JSON}{JavaScript Object Notation}
\acro{CSV}{Comma-Separated Values}
\acro{GUI}{Graphical User Interface}
\acro{ISR}{Interrupt Service Routine}
\end{acronym}
%


%==================================================================
\chapter*{INTRODUCCIÓN}\label{CAP:intro}
\setlength{\parskip}{14pt}% Separación entre párrafos
\addcontentsline{toc}{chapter}{INTRODUCCIÓN}%
%\markboth{Introducción}{Introducción}%

\pagenumbering{arabic}%
La seguridad de las infraestructuras es un tema de gran importancia en la actualidad, especialmente cuando se trata de estructuras como edificios o puentes.

Los primeros indicios del monitoreo del estado de las infraestructuras data de nuestros comienzos como especie sedentaria.  En la antigüedad, los especialistas utilizaban técnicas de inspección visual y auditiva para detectar posibles problemas en las estructuras, como grietas o ruidos inusuales. Con el tiempo, se desarrollaron técnicas más avanzadas para el monitoreo de estructuras, como la utilización de medidores de deformación, inclinación, sensores de vibración, entre otros.

La integración de la instrumentación con el análisis estructural comenzó a desarrollarse en la década de 1960 con el advenimiento de la informática y la disponibilidad de computadoras capaces de realizar cálculos estructurales complejos. En esa época, se comenzaron a utilizar sistemas de adquisición de datos para recopilar información sobre el comportamiento de las estructuras en tiempo real y utilizarla para calibrar y validar los modelos estructurales.

Actualmente, las normas sismo-resistentes apuntan a estructuras que sean capaces de mantener su integridad ante un evento de cierta magnitud. Además, el monitoreo continuo de ciertos indicadores en la estructura permiten determinar un índice de la salud estructural y ajustar el modelo a las condiciones actuales de la misma para evaluar el cumplimiento de la normativa sismorresistente. Para el monitoreo a largo plazo, el resultado de este proceso es información actualizada periódicamente sobre la capacidad de la estructura para desempeñar su función prevista a la luz del inevitable envejecimiento y degradación resultantes de los entornos operativos.

Según \citep{balageas2010structural}, el monitoreo de la salud estructural (SHM) tiene por objeto proporcionar, en cada momento de la vida de una estructura, un diagnóstico del estado de los materiales constitutivos, de las diferentes
partes, y del conjunto de estas partes que constituyen la estructura en su totalidad. El estado de la estructura debe permanecer en el ámbito especificado en el diseño, aunque este puede verse alterado por el envejecimiento normal debido al uso, por la acción del medio ambiente y por sucesos accidentales. Gracias a la dimensión temporal de la supervisión, que permite tener en cuenta toda la base de datos histórica de la estructura, y con la ayuda del monitoreo del funcionamiento. También puede proporcionar un pronóstico (evolución de los daños, vida residual, entre otros).

Si consideramos solo la primera función, el diagnóstico, podríamos estimar que el monitoreo de la salud estructural es una forma nueva y mejorada de realizar una evaluación no destructiva. Esto es parcialmente cierto, pero SHM es mucho más. Implica la integración de sensores, posiblemente materiales inteligentes, transmisión de datos, potencia computacional y capacidad de procesamiento en el interior de las estructuras. Permite reconsiderar el diseño de la estructura y la gestión completa de la propia estructura y de la estructura considerada como parte de sistemas más amplios.


%Incluir conjunto de elementos que influyen en el deterioro de la estructura. Envejecimiento, mantenimiento

En este sentido, el monitoreo de las estructuras se ha convertido en una herramienta esencial para garantizar la seguridad de las personas durante la vida útil de la misma, incluyendo la ocurrencia de eventos de cierta magnitud. Además, el monitoreo de las estructuras puede ayudar a mejorar la eficacia de las normas sismorresistentes, ya que permite validar y mejorar los modelos estructurales utilizados en la normativa.

Según \citep{nagayama2007structural}, dado que las edificaciones suelen ser grandes y complejas, la información
de unos pocos sensores es inadecuada para evaluar con precisión el estado estructural. El
comportamiento dinámico de estas estructuras es complejo tanto a escala espacial como temporal. Además,
los daños y/o el deterioro es intrínsecamente un fenómeno local. Por lo tanto, para comprender el
comportamiento dinámico, el movimiento de las estructuras debe ser supervisado por sensores
con una frecuencia de muestreo suficiente para captar las características dinámicas más destacadas. Esta información combinada con el registro del comportamiento estático de la estructura permiten tener una visión más amplia del estado actual de la estructura.


El primer paso, además de un mantenimiento adecuado, para garantizar la seguridad de estas estructuras, es contar con sistemas de monitoreo que permitan detectar posibles daños o fallas en su funcionamiento y tomar medidas preventivas. Por tanto, los sistemas de adquisición de datos y monitoreo son herramientas esenciales en la prevención de accidentes y daños.

A su vez, según \citep{nagayama2007structural}, un dispositivo inteligente, es decir, con capacidad de procesamiento de datos en el caso de los sensores, es una característica esencial que permite incrementar el potencial de los sensores al ser estos inalámbricos. Los sensores inteligentes pueden procesar localmente los datos medidos y trasmitir solo la información importante a través de comunicaciones inalámbricas. Cuando estos son configurados como una red, se extienden las capacidades de los mismos.

Los sensores inteligentes, con sus capacidades de cómputo y de comunicación integradas, ofrecen nuevas oportunidades para la SHM. Sin necesidad de cables de alimentación o comunicación, los costes de instalación pueden reducirse drásticamente. Los sensores inteligentes ayudarán a que el monitoreo de las estructuras con un denso conjunto de sensores sea económicamente práctico. Se espera que los sensores inteligentes instalados en masa sean fuentes de información muy valiosa para la SHM.

En este trabajo de grado se abordará el diseño para una futura implementación de un sistema de adquisición de datos de bajo costo basado dispositivos programables con capacidad de interconexión para el monitoreo y procesamiento de variables como aceleración, inclinación, humedad y temperatura en estructuras críticas, con el objetivo de prevenir daños y accidentes.



De acuerdo a Brea  la transformada de Laplace debe estudiarse como
una función definida en el campo de los números complejos
\cite{brea5}.

Otro modo de referencial es \citep{brea5}

El resto del reporte consta de: en el Capítulo \ref{CAP:referencial} se
describe...

En el trabajo se emplea el enfoque de \cite{brigham1}

De acuerdo a la ecuación
%

%==================================================================
\chapter{MARCO REFERENCIAL}\label{CAP:referencial}
\section{Planteamiento del problema}

\section{Justificación}

\section{Objetivos}

\subsection{Objetivo general}

\subsection{Objetivos específicos}

\section{Antecedentes}

%

%==================================================================
\chapter{MARCO TEÓRICO}\label{CAP:marco_teor}
%\markboth{Tu Primer Capítulo}{Tu Primer Capítulo}%

En este capítulo se definirán los conceptos  o fundamentos de instrumentación estructural, sensores inteligentes y adquisición de datos, necesarios para llevar a cabo esta investigación.

\section{Estructuras civiles}

\subsection{Características generales}

Una estructura se refiere a un sistema de partes o elementos que se interconectan para cumplir una función es específico. En el caso de la ingeniería civil, suelen ser miembros que se utilizan para soportar una carga. Algunos ejemplos importantes son los edificios, los puentes y las torres; y en otras ramas de la ingeniería, son importantes las corazas de barcos y aviones, los sistemas mecánicos y las estructuras que soportan las líneas de transmisión eléctrica \citep{hibbeler1997structural}.

\subsection{Tipos de estructuras}

Según \citet{hibbeler1997structural}, cada sistema está formado por uno o varios de los cuatro tipos básicos de estructuras: 

\begin{itemize}
    \item Celosías.
    \item Cables y arcos.
    \item Armazones.
    \item Estructuras de superficie.
\end{itemize}

En general, estos elementos suelen soportar cargas, pueden ser estacionarios y también estar restringidos. Sus diferencias suelen basarse en la cantidad de fuerzas a las que están sujetos estos elementos en un instante dado.

La combinación de estos elementos y los materiales que los componen es lo que se denomina un sistema estructural. Estos sistemas, aunque sean pasados por alto, son utilizados diariamente por industrias y personas, siendo elementos claves en el desarrollo y progreso de la civilización actual.

\subsection{Comportamiento de las estructuras civiles}

La gran mayoría de los sistemas cuentan con una respuesta dinámica y estática. Ambas respuestas permiten conocer el comportaiento completo del sistema en estudio ante distintas entradas o en diferentes situaciones. Al estudiar el comportamiento estructural se encuentra una extensa literatura tanto para el estudio dinámico como para el régimen estático, recopilándose lo siguiente:

\begin{itemize}
    \item{Respuesta estática}: En la ingeniería civil toda estructura se diseña para que se encuentre en reposo cuando actúan sobre esta fuerzas externas, es decir, la estructura en conjunto debe cumplir con las condiciones de equilibrio, siendo la fuerza y el momento resultanto sobre esta igual a cero en todo momento. Para describir estas condiciones de equilibrio se cuentan con herramientas matemáticas que proporcionan las condiciones necesarias para su cumplimiento. Estas ecuaciones permiten la resolución estática de la estructura, la cual permite determinar el valor de todas las incógnitas estáticas de interés \citep{basset2014analisis}.
    
    \indent Cuando las fuerzas que actúan sobre la estructura pueden calcularse a partir de las ecuaciones de equilibrio, se tiene una estructura en equilibrio y se denonima estructura estáticamente determinada. En caso de tenerse más fuerzas desconocidas que ecuaciones de equilibrio se habla de una estructura estáticamente indeterminada.

        \begin{itemize}
            \item Rigidez: Uno de los parámetros más importantes dentro de la respuesta estática es la rigidez. Esta se define como la propiedad que tiene un elemento estructural de soportar la deformación o deflección al estar bajo la acción de una fuerza o carga. Una medida de la rigidez viene dada por el Módulo de Young; esta es una constante del material y es independiente de la cantidad de material.
        \end{itemize}

    \item{Respuesta dinámica}: La dinámica estructural se encarga de estudiar el efecto que tienen cargas dinámicas sobre el sistema. La respuesta ante estos eventos, como pueden ser sismos, vientos, equipos mecánicos, paso de vehículos o personas, se denomina respuesta dinámica \citep{hurtado2000}. Además, la respuesta dinámica permite caracterizar algunos parámetros de gran interés para estudiar su comportamiento conocidos como parámetros modales. Estos parámetros surgen al estudiar las ecuaciones diferenciales que describen el movimiento de la estructura, partiendo de un modelo idealizado simple de masa concentrada como el de la Figura \ref{fig:masa_estructural}.
    
    \begin{figure}[H]
        \centering
        \includegraphics[width = 0.25\textwidth]{imagenes/cap1_marcoteo/modelo_masa_simple.png}
        \caption{Modelo de masa concentrada de 1 grado de libertad \citet{hurtado2000}.}
        \label{fig:masa_estructural}
    \end{figure}

    La dinámica de este modelo puede describirse utilizando la ecuación diferencial de movimiento:

    \begin{equation} \label{eq:vib_lib}
        m\ddot{u} + f_R(t) = p(t)
    \end{equation}

    La ecuación \ref{eq:vib_lib} se conoce como ecuación de vibración libre sin amortiguamiento. Donde $p(t)$ representa las cargas dinámicas y $f_R(t)$ la fuerza de restitución propia de un material elástico.  Esta ecuación es una ecuación diferencial de coeficientes constantes, que consta de una solución homogénea más una solución particular. La solución homogénea será la respuesta de la estructura a la vibración libre, es decir, si la masa de la Figura \ref{fig:masa_estructural} se deja oscilar libremente.

    Se sabe que una ecuación de este tipo tendrá una solución como:

    \begin{equation} \label{eq:sol_equ_dif}
        u = A.sin\omega t + B.cos\omega t
    \end{equation}

    La ecuación \ref{eq:sol_equ_dif} contiene información relevante para la caracterización dinámica de la estructura. Esta caracterización parte del estudio de los parámetros modales de la misma.

    Entre estos parámetros modales se encuentran: 
        \begin{itemize}
            \item Frecuencia natural: Toda estructura física tiene asociada una frecuencia de vibración natural. Las máquinas, los puentes, los edificios; todas estas estructuras vibran u oscilan al ser perturbadas o removidas de su estado de reposo inicial. Es una propiedad es intrínseca del sistema y depende de su masa, rigidez y amortiguamiento. Todas tienen al menos una frecuencia natural y es posible que tengan múltiples frecuencias de resonancia \citep{irvine2000introduction}. 
            
                Se suele calcular la frecuencia natural de resonancia de un sistema libre usando:

                \begin{equation}
                    f =  \frac{1}{\sqrt{\frac{k}{m}}}
                \end{equation}

            \item Amortiguamiento: Toda estructura comienza a oscilar una vez es removida de su estado de reposo o equilibrio, sin embargo, ese movimiento no es perpetuo. El amortiguamiento se define como la capacidad de disipación de energía que posee la estructura bajo excitaciones externas. Las soluciones a la ecuación \ref{eq:vib_lib}, al añadir el amortiguamiento de tipo viscoso, arrojan 3 posibles casos:
                \begin{enumerate}
                    \item Sistema críticamente amortiguado: El sistema no vibra.
                    \item Subamortiguado o amortiguado subcrítico: Caso más común por la naturaleza de los materiales utilizados en las estructuras. La respuesta del sistema decae con el tiempo de forma exponencial, como se puede ver en la Figura \ref{fig:resp_subamorti}. 
                    
                    \begin{figure}[H]
                        \centering
                        \includegraphics[width = 0.7\textwidth]{imagenes/cap1_marcoteo/respuesta_sist_subamorti.png}
                        \caption{Respuesta ante vibración libre en sistema Subamortiguado \citep{hurtado2000}.}
                        \label{fig:resp_subamorti}
                    \end{figure}

                    \item Sobreamortiguado: Nunca se encuentra esta respuesta en sistemas estructurales por los materiales utilizados.
                \end{enumerate}
            
        \end{itemize}        
\end{itemize}

\subsection{Respuesta en frecuencia}

    Incluir?

\subsection{Daño en estructuras}

El daño a una estructura civil o mecánica puede definirse como todo cambio en las propiedades materiales o geométricas del material que llegan a afectar de forma adversa la confiabilidad y el desempeño actual o futuro del sistema. Por tanto, el daño es una comparación entre el sistema en cuestión en 2 instantes de tiempo distintos \citep{farrar2007introduction}. Estos efectos adversos pueden ser, en el caso estructural, desplazamientos, estrés indeseado en un elemento o vibraciones estructurales indeseadas \citep{chen2018}.

Toda estructura civil, como puentes y edificios, acumulan daño de forma continua a medida que están en servicio y transcurre su vida útil. Este daño puede manifestarse como fracturas, fatiga, socavaciones o desprendimiento del concreto. El daño que no sea detectado puede conducir a una falla estrcutural que a su vez ocasione pérdidas humanas. Por tanto, es imperativo y necesario detectar el daño en una estrcutrura tan pronto como sea posible, \citep{chen2018}.

Entre algunos de los factores que influyen del deterioro de una estructura se encuentran:
    
        \begin{itemize}
            \item Proceso de degradación natural de los materiales.
            \item Corrosión del acero de refuerzo.
            \item Evento sísmico, incendios o condiciones de guerra.
            \item Carga por encima del límite de diseño.
        \end{itemize}
    
Las escalas de tiempo y de extensión del daño son diversas. Por ejemplo, el deterioro por el paso del tiempo bajo ciertas condiciones climáticas es muy lento comparado al daño causado por un evento catastrófico.

\subsection{Principios de la Sismoresistencia}

Una edificación sismorresistente es aquella que está diseñada y construida para soportar las fuerzas causadas por eventos sísmicos. Sin embargo, incluso las edificaciones diseñadas y construidas según las normas sismorresistentes pueden sufrir daños en caso de un terremoto muy fuerte, sin embargo, las normas establecen los requisitos mínimos para proteger la vida de
las personas que ocupan la edificación

Algunas de las características de una estructura sismoresistente son:

        \begin{itemize}
            \item Forma regular.
            \item Bajo peso.
            \item Mayor rigidez.
            \item Buena estabilidad.
            \item Suelo firme y buena cimentación.
            \item Materiales competentes.
            \item Capacidad de disipación de energía.
            \item Fijación de acabados e instalaciones.
        \end{itemize}

En Venezuela las estructuras deben cumplir con la Norma Venezolana COVENIN 1756:2001 (Edificaciones Sismorresistentes).

Se ha observado que al estudiar el comportamiento de las estructuras luego de un evento sísmico, es evidente que cuando se toman en cuenta las normas de diseño sismorresistente dispuestas en la ley y la construcción es debidamente supervisada, los daños estructurales resultan ser considerablemente menores que en las edificaciones en las cuales no se cumplen los requerimientos mínimos indispensables estipulados en la norma, \citep{blanco2012criterios}.

\subsubsection{Importancia de la instrumentación} La instrumentación estructural permite medir y monitorear las acciones y respuestas estructurales ante distintos eventos. Esto proporciona datos en tiempo real sobre el comportamiento dinámico y estático de la estructura, como deformaciones, aceleraciones y desplazamientos, que son fundamentales para evaluar y verificar si la estructura cumple con los criterios de diseño sismoresistente establecidos en la norma.

La instrumentación estructural ayuda a validar los modelos y suposiciones utilizados en el diseño estructural inicial. Al comparar los datos recopilados por la instrumentación durante un evento sísimco con las predicciones del modelo, es posible verificar si la estructura se comporta de acuerdo con las expectativas y si cumple con los criterios de seguridad establecidos en la normas.

Además, el monitoreo continuo de la estructura permite conocer el estado actual de la misma, tema que representa la idea principal del Monitoreo de Salud Estructural, permitiendo a los ingenieros evaluar si se sigue cumpliendo con la norma para luego tomar decisiones y actuar en pro de la seguridad de la edificación.


\section{Salud estructural}

\subsection{Definición}


El proceso de implementar una estrategia de identificación de daño para estructuras civiles, mecánicas o aeroespaciales se conoce como Monitoreo de Salud Estructural (SHM por sus siglas en inglés). Esta estrategia requiere medir las condiciones y el ambiente en el que opera la estructura, además de la respuesta de la misma durante un período de tiempo tomando muestras periódicamente espaciadas, \citep{farrar2007introduction}.


La estrategia del SHM requiere de equipos multidisciplinarios de ingeniería, ya que necesita de una red de sensores que midan las variables de interés, el procesamiento y análisis de los datos obtenidos y posteriormente una prognosis del daño para una eventual toma de decisiones. El objetivo del SHM es proveer, en toda la vida útil de la estructura, un diagnóstico del estado de sus materiales constitutivos, de los diferentes elementos que la componen y de la estructura en sí como el conjunto de todas estas partes. Esto para garantizar que la misma se comporte dentro de los parámetros iniciales de diseño, aunque estos cambien por la acción natural del tiempo, el ambiente y accidentes, \citep{balageas2010structural}.

El resultado de este proceso es información actualizada sobre el estado de la estructura y sobre su capacidad actual para seguir desempeñado la función para la cual fue diseñada.

Según \citet{enckell2006structural}, el SHM se ha convertido en una herramienta muy conocida y utilizada en ingeniería estructural en los últimos años en diferentes países.

\subsection{Reseña histórica}

Las técnicas de detección de daño basadas en vibración tienen sus primeras aplicaciones desde hace cientos de años. En la antigüedad, los constructores golpeaban las estructuras para encontrar espacios vacíos o grietas en elementos de arcilla. La utilidad de estas inspecciones tan simples indicaban que la sofisticación de estos métodos podía proveer información muy valiosa sobre el elemento de interés, sin embargo, esto requiere de instrumentos y herramientas matemáticas que se han desarrollado con el pasar de los años. El auge en el uso de SHM en años recientes es consecuencia de la evolución y miniaturización del hardware computacional actual.


El uso más exitoso del SHM ha sido el monitoreo de la condición de máquinas rotativas, las cuales actualmente han adoptado un enfoque de indetificación de daño sin basarse en un modelo de forma casi exclusiva, \citep{farrar2007introduction}.

En los años 70 la industria petrolera consideró el uso de técnicas basadas en vibración para identificar daños en plataformas costa-afuera, este enfoque se diferenció de las máquinas rotativas al estudiar un sistema en donde la ubicación del daño es desconocida y difícil de instrumentar.

En esa misma época, la comunidad aeroespacial y la \textit{National Eeronautics Space Agency} (NASA), comenzaron a estudiar esta técnica de identificación de daño en los comienzos de la era de lanzamientos espaciales. Este trabajo continúa hoy en día y el \textit{Shuttle Modal Inspection System} (SMIS) se desarrolló para identificar fatiga en distintos componentes de cohetes espaciales reusables, los cuales representan el futuro de esta industria.


Usualmente, los enfoques de estas industrias se basan en comparar modelos analíticos de estructuras sin daño con las mediciones de estructuras con daño, observando principalmente las propiedades modales de las mismas. Se ha observado que cambios en la rigidez en ambos modelos han permitido localizar y cuantificar el daño, \citep{farrar2007introduction}.

Inicialmente, las técnicas no destructivas fueron introducidas en la ingeniería civil a mediados de los años 40, \citep{mohamed2014}. La necesidad principal surgió en determinar propiedades del concreto fresco \textit{in-situ}. Estas técnicas, que buscaban evaluar la homogeneidad y la resistencia del concreto eran en su mayoría pruebas con martillo y pruebas de \textit{pull-out}. A medida que las estructuras envejecieron, los ingenieros necesitaban idear maneras de medir o estimar las propiedades mecánicas de los elementos que consituyen las estructuras, además de detectar daños que no eran fáciles de observar por la envergadura de las estructuras civiles que se han desarrollado en los últimos 150 años. Es ahí, en los años 70, donde surgen nuevas estrategias no destructivas tales como:

    \begin{itemize}
        \item Emisión acústica.
        \item Métodos de ultrasonido y radar.
        \item Termografía.
        \item Métodos basados en vibración
    \end{itemize}

La comunidad de ingeniería civil ha estudiado la identificación de daño basada en vibración en puentes y edificios desde comienzos de los años 80. Las propiedades modales han sido estudiadas por diferentes autores y son las principales características que se analizan al identificar daño. El auge del SHM es tal, que algunos países asiáticos han implementado regulaciones en donde las compañías constructoras deben verificar la salud estructural de los puentes periódicamente. Estas regulaciones han provocado que la investigación e inversión en esta área siga aumentando de forma considerable, \citep{chen2018}. 

\subsection{Línea de trabajo del Monitoreo de Salud Estructural}

Los sistemas de SHM consisten de varios elementos que permiten a los ingenieros tener información sobre el estado de una estructura, entre esos elementos se encuentran:

\begin{itemize}
    \item Sensores.
    \item Sistemas de adquisición de datos.
    \item Sistema de transmisión de datos.
    \item Sistema de procesamiento de datos.
    \item Sistema de manejo y almacenamiento de datos.
    \item Equipo de análisis y toma de decisiones.
\end{itemize}

\begin{figure}[H]
    \centering
    \includegraphics[width = 0.9\textwidth]{imagenes/cap1_marcoteo/Schematics-of-an-on-line-structural-health-monitoring-system-and-technical-challenges.png}
    \caption{Esquema de un sistema de SHM \citep{lijianfoto2015}.}
    \label{fig:esquema_gral_SHM}
\end{figure}


Autores como \citet{rytter1993vibration} y \citet{farrar2007introduction} han esquematizado la estrategia del SHM categorizando el daño en una estructura por niveles de la siguiente forma:

\begin{enumerate}
    \item Nivel I (detección del daño) ¿Presenta daño el sistema? Es una indicación cualitativa de que puede haber daño presente en la estructura.
    \item Nivel II (localización o ubicación del daño) ¿Dónde está presente el daño? Indica la posible localización del mismo. 
    \item Nivel III (clasificación del daño) ¿Qué tipo de daño está presente? Da información sobre el tipo de daño.
    \item Nivel IV (alcance/grado/extensión del daño) ¿Cuál es el alcance del daño? ¿Qué tan grave es? Da un estimado del alcance.
    \item Nivel V (prognosis del daño) ¿Cuánta vida útil le queda a la estructura? Da un estimado de la seguridad de la estructura.
\end{enumerate}

En la mayoría de los casos, para alcanzar el nivel final es necesario obtener información sobre los niveles previos. Esto indica que a medida que se sube de nivel se tiene un mayor conocimiento sobre el estado de la estructura.

De acuerdo a \citet{chen2018}, los primeros dos niveles, detección y localización, generalmente pueden alcanzarse usando métodos de detección basados en vibración para obtener mediciones sobre la respuesta dinámica de la estructura.

Por su parte, \citet{chen2018} describe el proceso de SHM en general como:

\begin{enumerate}
    \item Observación.
    \item Evaluación.
    \item Calificación.
    \item Gestión.
\end{enumerate}

La estrategia de Monitoreo de Salud Estructural podría resumirse en el siguiente diagrama:

\begin{figure}[H]
    \centering
    \includegraphics[width = \textwidth]{imagenes/cap1_marcoteo/Diagrama SHM timeline.png}
    \caption{Diagrama general del proceso de SHM.}
    \label{fig:diag_SHM}
\end{figure}

\subsection{Criterios de evaluación}

Como se definió anteriormente, el daño estructural puede tener distintas causas y formas. Lo que se sabe con certeza es que una estructura, una vez entra en funcionamiento, análogo a los seres humanos al nacer, estará sujeta a envejecimiento natural y a condiciones adversas. Ahora bien, en el caso del SHM, surge la siguiente pregunta ¿Qué se debe medir para poder detectar este daño?. 

Numerosos autores concluyen que uno de los indicativos de daño de una estructura viene dado por los parámetros modales definidos anteriormente, frecuencia y amortiguamiento. Esta relación entre los parámetros modales y el daño viene dada por la premisa de que todo daño presente en la estructura se reflejará en un cambio en las propiedades dinámicas de la misma. \citet{worden2009modal}, comprobó la relación entre el cambio progresivo en todas las frecuencias naturales de 15 vigas estudiadas a las cuales se les introdujo un daño relacionado con un cambio del Módulo de Young, el cual, como fue mencionado anteriormente, provee un indicativo de la rigidez de un elemento.

Anterioremente en la ecuación \ref{eq:vib_lib} se definió un sistema con un grado de libertad (DOF por sus siglas en ingles), sin embargo, en la realidad las estructuras tienen múltiples grados de libertad, por lo que es conveniente modelarlas de estar forma para obtener resultados más precisos. En el caso de los sistemas \textit{n-dregrees of freedom} (DOF por sus siglas en ingles), la ecuacion de movimiento que describe la dinámica del sistema en vibración libre vendrá dada por:

\begin{equation} \label{eq:ecu_movimiento}
    M\ddot{u} + C\dot{u} + Ku = 0
\end{equation}

Donde M, C y K representan las matrices de masa, amortiguamiento y rigidez de la estructura, respectivamente.

Si se asume un sistema sin amortiguamiento, a fines de estudiar el efecto que tiene sobre la rigidez un cambio en los parámetros modales,  de la ecuación \ref{eq:ecu_movimiento} se obtiene: 

\begin{equation} \label{eq:ecu_movimiento_sindamp}
    M\ddot{u} + Ku = 0
\end{equation}

Si se asume una solución oscilatoria pura, por ser un sistema sin amortiguamiento:

\begin{equation} \label{eq:sol_ecu_motion}
    u = ve^{jwt}
\end{equation}

Al derivar, sustituir y despejar en la ecuación \ref{eq:ecu_movimiento_sindamp} se obtiene:

\begin{equation} \label{eq:eig_problem}
    (K - \lambda M)\phi = 0
\end{equation}

Esta ecuación \ref{eq:eig_problem} representa claramente un problema de autovalores, donde $\lambda$ representa los autovalores asociados a las frecuencias naturales del sistema y $\phi$ representa el autovector de desplazamiento.

Si se introduce un pequeño cambio $\Delta K$ con perturbaciones similares en los otros parámetros:

\begin{equation}
    [(K - \Delta K) - (\lambda - \Delta\lambda)(M - \Delta M)](\phi - \Delta\phi) = 0
\end{equation}

El daño estructural suele venir asociado a un cambio en la rigidez, mas no a cambios en la masa de la estructura, por lo que se asume $\Delta M = 0$, \citep{hearn1991modal}. A su vez, se tiene que $(K - \lambda M)\phi = 0$. \citet{mohamed2014}, \citet{shi1998structural} y \citet{hearn1991modal} desarrollan estas ecuaciones obteniendo la siguiente relación:

\begin{equation} \label{eq:relacion_final}
    \Delta\lambda = \phi^T \Delta K \phi
\end{equation}

De la ecuación \ref{eq:relacion_final} se observa que cambios en los autovalores $\lambda$ que representan las frecuencias naturales, y en los autovectores $\phi$ (formas modales) están directamente relacionados con cambios en la matriz de rigidez (K) del sistema. De aquí surge el interés en monitorear los parámetros modales como indicadores de daño estructural. Es importante recalcar que estos cambios son indicativos de daño global, mas no de la localización del mismo, para lo que se necesitan otras técnicas, \citep{mohamed2014}.

\subsection{Variables de interés}

Tomando en cuenta la relación entre los parámetros modales y el daño presente en una estructura, es preciso definir las variables de interés para el monitoreo de la salud estructural de una estructura. Si bien existen distintas varibales que permiten obtener información valiosa sobre la estructura en estudio, algunas de estas no proporcionan información global del daño, como es el caso de las formas modales y la deflección local \citep{rytter1993vibration}. Sin embargo, estas mediciones proveen indicativos de la ubicación del daño, por lo que pueden constituir parte del sistema de monitoreo en una etapa más avanzada, es decir, una vez el daño fue detectado. A continuación se presentan las más relevantes para el daño global:

    \begin{itemize}
        \item Frecuencias naturales y amortiguamiento: Los parametros modales de la estructura están ligados de forma directa al estado de la misma. El deterioro en una edificiación induce cambios en la rigidez estructural, como se observa claramente en la ecuación \ref{eq:relacion_final}. El daño puede tener efectos distintos en cada modo o cada frecuencia de vibración, por lo que es importante no ubicar los sensores sobre los nodos modales, ya que experimentos han demostrado la ineficacia en las mediciones. Usualmente, el daño se refleja como una disminución en las frecuencias naturales afectadas, aunque se han observado casos de aumento en las frecuencias de vibración en estructuras de concreto pretensado, \citep{rytter1993vibration}.
        
        A su vez, el amortiguamiento varía al introducir daño en la estructura, puesto que su capacidad de disipar energía se ve afectada. Usualmente, los investigadores observan un aumento en el amortiguamiento a medida que el daño aumenta, como se ha demostrado experimentalmente por autores como \citet{hearn1991modal} y \citet{rytter1993vibration}.

        \item Temperatura y humedad: En los sistemas de monitoreo, la detección del daño estructural puede tomar períodos de tiempo considerables, durante los cuales las caracterpisticas sujetas a temperatura y humedad, sufren cambios que afectan la respuesta estructural.
                
        Es evidente que las condiciones climáticas contribuyen con el deterioro de las edificaciones. A pesar de esta conclusión, relacionar las condiciones climáticas con el daño introducido usando mediciones ambientales es difícil. La medición de estas variables suele tomarse en cuenta para poder cuantificar el cambio que producen estas condiciones en los demás indicadores de daño. \citet{rytter1993vibration} observó que la humedad y temperatura afectaban las mediciones de amortiguamiento. Por su parte, \citet{mohamed2014}, observó que las frecuencias naturales de barras y vigas disminuían a medida que aumentaba la temperatura. A su vez, \citet{sohn2007effects} determinó que cuando hay humedad, los puentes de hormigón absorben una cantidad considerable de humedad, lo que aumenta sus masas y altera sus frecuencias naturales.
        
        \item Inclinación y desplazamiento: n practical applications of structural monitoring, the most common is the measurement of linear displacements, which reflect in a very good and direct way the behavior of the structural element / structure or a part of the structure.
        
        Inclinometers are used to measure inclination (tilt) of structural components due to distress in the system. For example, they are often utilised to assess fixity of bridge girders at supports and to monitor longterm movements of bridge piers, abutments and girders.

        Deflection is a very important index for bridge structures, because it not only affects driving comfort, but also reflects the overall response of the bridge. Various factors could contribute to deflection increase during bridge service life, such as concrete creep, steel corrosion, prestress loss, and crack growth. The increase of structural deflection, however, will in turn accelerate the damage accumulation process. Therefore, monitoring bridge deflection is of great significance in the field of structural health monitoring (SHM) to provide early warnings of possible structural changes, damage, or deterioration.

        In several industries during the last few decades, inclinometer sensors have been employed extensively. In fact, in the civil engineering sector, inclinometers were initially used for geotechnical purposes [80]. Improvements in sensor accuracy over time have made it possible to use inclinometers in other areas of civil engineering, such as monitoring the structural health of bridges [79].
        


    \end{itemize}
    

\subsection{Consideraciones y desafíos}

\section{Sensores}

\subsection{Definición y tipos de sensores}

\subsection{Sensores de interés para el Monitoreo de Salud Estructural}

    \begin{itemize}
        \item Acelerómetros e Inclinómetros:
        \item Desplazamiento:
        \item Temperatura:
        \item Humedad:
    \end{itemize}

\subsection{Sensores inteligentes}

\section{Microcontroladores}%

%==================================================================
\chapter{MARCO METODOLÓGICO}\label{CAP:marco_met}
%\markboth{Tu Segundo Capítulo}{Tu Segundo Capítulo}%
%\input{ch3.tex}%

%==================================================================
\chapter{DESARROLLO}\label{CAP:mod}
%\markboth{Tu Segundo Capítulo}{Tu Segundo Capítulo}%
%\input{ch4.tex}

%==================================================================
\chapter{PRUEBAS EXPERIMENTALES}\label{CAP:exp}
%\markboth{Tu Segundo Capítulo}{Tu Segundo Capítulo}%
%\input{ch5.tex}

%==================================================================
\chapter{RESULTADOS}\label{CAP:resultados}
%\markboth{Tu Segundo Capítulo}{Tu Segundo Capítulo}%
%\input{resultados.tex}

%==================================================================
\chapter{CONCLUSIONES}\label{CAP:conclu}
%\markboth{Tu Segundo Capítulo}{Tu Segundo Capítulo}%
%\input{conclusiones.tex}

%==================================================================
\chapter{RECOMENDACIONES}\label{CAP:recomendaciones}
%\markboth{Tu Segundo Capítulo}{Tu Segundo Capítulo}%
%\input{recomendaciones.tex}

\appendix

\renewcommand \thechapter{\Roman{chapter}}
%==================================================================
\chapter{TÍTULO DEL ANEXO}\label{CAP:anexo0}
%\markboth{Tu anexo}{Tu anexo}%
%\input{apendice0.tex}%

%==================================================================
\chapter{TÍTULO DEL ANEXO}\label{CAP:anexo1}
%\markboth{Tu anexo}{Tu anexo}%
%\input{apendice1.tex}%

%\backmatter
%==================================================================
\chapter{TÍTULO DEL ANEXO}\label{CAP:anexo2}
%\markboth{Tu anexo}{Tu anexo}%
%\input{apendice2.tex}%



%\backmatter

%==================================================================
\newpage
%\markboth{Referencias}{Referencias}%
%\addcontentsline{toc}{chapter}{Referencias}%

% References here (outcomment the appropriate case)
% CASE 1: BiBTeX used to constantly update the references (while the paper is being written).
%\bibliographystyle{IEEEtran}%{IEEEtranS}%%% outcomment this and next line in Case 1 siam
\bibliographystyle{apacite}%
\renewcommand{\bibname}{REFERENCIAS}
\let\oldbibsection\bibsection
\bibliography{biblioteca} % if more than one, comma separated and without extension bib


% CASE 2: BiBTeX used to generate EIETdeG.bbl (to be further fine tuned)
%\documentclass[letterpaper,titlepage,12pt,oneside,spanish,final]{report_eie}

%PARA QUE APAREZCAN LAS SUBSUBSECTIONS EN EL INDICE
%\setcounter{tocdepth}{3}

%\documentclass[letterpaper,titlepage,12pt,twoside,openright,spanish,final]{report_eie}

%%%%%%%%%%%%%%%%%%%%%%%%%%%% LIBRERIAS %%%%%%%%%%%%%%%%%%%%%%%%%%%%%%%%%%%%%%%%%%%%

\usepackage[spanish]{babel}
\usepackage[utf8]{inputenc}
%\usepackage[latin1]{inputenc}
\usepackage[T1]{fontenc}  %Estilo de fuente times new roman

\usepackage{subfig}

\usepackage{amssymb}
\usepackage{amsfonts}
\usepackage{amsmath}
\usepackage{latexsym}
\usepackage[letterpaper]{geometry}

\usepackage{float}
\usepackage{makeidx}
\usepackage{color}

\usepackage{tocbibind}
\usepackage{acronym}
%OJO CAMBIE A CAPTION ENVEZ DE CAPTION2
\usepackage{caption} %EL CENTER CENTRA TODOS LOS 
\captionsetup[figure]{format=plain,justification=centering}%CAPTIONS DE TABLAS Y FIGURAS OJO
\usepackage{epsfig}
\usepackage{graphicx}
%\usepackage{slashbox}
\usepackage{setspace}
\usepackage{multicol}
\usepackage{longtable}
%\usepackage{doublespace}
\usepackage{array}
\newcolumntype{P}[1]{>{\centering\arraybackslash}p{#1}}


\usepackage{fancyhdr}
%\usepackage{fancyheadings}

\usepackage{booktabs}

%NUMEROS DE LINEA
% \usepackage[pagewise]{lineno}
% \linenumbers

%========= Define el estilo de referencias ===============
%\usepackage[round,authoryear]{natbib}%\usepackage[square,numbers]{natbib}%
%\usepackage[comma,authoryear]{natbib} esto está abajo

%========= Define el estilo de referencias APA ===============
\usepackage[natbibapa]{apacite}%natbibapa
%\usepackage[square, numbers]{natbib}
%\usepackage[numbers]{natbib} %PARA QUE SEAN NUMEROS EN CORCHETE
%\usepackage[apaciteclassic]{apacite}%natbibapa
%\usepackage[compact]{titlesec} %modificar espaciado

\usepackage{url}
\usepackage{hyperref}
%\usepackage[dvips,colorlinks=true,urlcolor=red,citecolor=black,anchorcolor=black,linkcolor=black]{hyperref}

%%%%%%%%%%%%%%%%CODIGO%%%%%%%%%%%%%%%%

\usepackage{listings}
\usepackage{xcolor}

\definecolor{codegreen}{rgb}{0,0.6,0}
\definecolor{codegray}{rgb}{0.5,0.5,0.5}
\definecolor{codepurple}{rgb}{0.58,0,0.82}
\definecolor{backcolour}{rgb}{0.95,0.95,0.92}

\lstdefinestyle{mystyle}{
    backgroundcolor=\color{backcolour},   
    commentstyle=\color{codegreen},
    keywordstyle=\color{blue},
    numberstyle=\tiny\color{codegray},
    stringstyle=\color{codepurple},
    basicstyle=\ttfamily\footnotesize,
    breakatwhitespace=false,         
    breaklines=true,                 
    captionpos=b,                    
    keepspaces=true,                 
    numbers=left,                    
    numbersep=5pt,                  
    showspaces=false,                
    showstringspaces=false,
    showtabs=false,                  
    tabsize=2
}

\lstset{style=mystyle}

\def\lstlistingname{Código}


%%%%%%%%%%%%%%%%%%%%%%%%%%%%%%%%%%%%%%%%%%%%%%%%%%%%%%%%%%%%%%%%%%
%       Definición del Documento PDF, (PDFLaTeX)        %
%%%%%%%%%%%%%%%%%%%%%%%%%%%%%%%%%%%%%%%%%%%%%%%%%%%%%%%%%%%%%%%%%%

\hypersetup{pdfauthor=Nombre}

\hypersetup{pdftitle=Título}%

\hypersetup{pdfkeywords=Palabras clave}

\pdfstringdef{\Produce}{Escuela de Ingeniería Eléctrica, Facultad de Ingeniería, UCV}%

\pdfstringdef{\area}{Área del trabajo}

\hypersetup{pdfproducer=\Produce}

\hypersetup{pdfsubject=\area}

\hypersetup{bookmarksnumbered=true}

%%%%%%%%%%%%%%%%%%%%%%%%%%%%%%%%%%%%%%%%%%%%%%%%%%%%%%%%%%%%%%%%%%

%\setcounter{MaxMatrixCols}{10}

%===================== Re-definición de Ambientes =================
\newtheorem{theorem}{Teorema}
\newtheorem{acknowledgement}[theorem]{Acknowledgement}
\newtheorem{algoritmo}[theorem]{Algorithm}
\newtheorem{supuestos}[theorem]{Supuestos}
\newtheorem{hipotesis}[theorem]{Hipótesis}
\newtheorem{axiom}[theorem]{Axiom}
\newtheorem{case}[theorem]{Case}
\newtheorem{claim}[theorem]{Claim}
\newtheorem{conclusion}[theorem]{Conclusión}
\newtheorem{condition}{Condición}
\newtheorem{conjecture}{Conjecture}
\newtheorem{corollary}{Corollary}
\newtheorem{criterion}{Criterion}
\newtheorem{definition}{Definición}  %{Definition}
\newtheorem{example}[theorem]{Ejemplo}%{Example}
\newtheorem{exercise}[theorem]{Exercise}
\newtheorem{lemma}{Lemma}
\newtheorem{notation}[theorem]{Notation}
\newtheorem{problem}{Problem}
\newtheorem{property}{Property}
\newtheorem{proposition}{Proposition}
\newtheorem{remark}[theorem]{Remark}
\newtheorem{solution}{Solution}
\newtheorem{summary}[theorem]{Summary}
\newenvironment{proof}[1][Proof]{\noindent\textbf{#1.} }{\ \rule{0.5em}{0.5em}}%

\numberwithin{equation}{chapter}%
\numberwithin{figure}{chapter}%
\numberwithin{table}{chapter}%
\numberwithin{definition}{chapter}%
\numberwithin{lemma}{chapter}%
\numberwithin{theorem}{chapter}%
\numberwithin{corollary}{chapter}%
\numberwithin{condition}{chapter}%
\numberwithin{criterion}{chapter}%
 \numberwithin{problem}{chapter}%
\numberwithin{property}{chapter}%
\numberwithin{proposition}{chapter}%
\numberwithin{solution}{chapter}%
\numberwithin{conjecture}{chapter}%

%==================== Separación en sílabas ========================
\hyphenpenalty=6800%
\input{silabear.tex}

%==================== Diseño de Página =============================
%\pagestyle{headings}
%\setlength{\headheight}{0.2cm}
\setlength{\textwidth}{14.52cm}%
%\pagestyle{fancy}
\renewcommand{\chaptermark}[1]{\markboth{#1}{}}
%\renewcommand{\sectionmark}[1]{\markright{\thesection\ #1}}
%\rhead[\fancyplain{}{\bfseries\thepage}]{\fancyplain{}{\bfseries\rightmark}}%\thepage
%\lhead[\fancyplain{}{\bfseries\leftmark}]{\fancyplain{}{\bfseries}} \cfoot{}%

%\fancyhead[R]{}

\rfoot[\fancyplain{}{\textit{E. Brea}}] {\fancyplain{}{}}
\lfoot[\fancyplain{}{}] {\fancyplain{}{\textit{}}}    %%%%%%%%%%%%%%%%%%% OJO ACA %%%%%%%%%%
\cfoot[\fancyplain{}{}] {\fancyplain{}{\bfseries\thepage}}
%\setlength{\footrulewidth}{0.0pt}%
%\setlength{\headrulewidth}{0.1pt}%

%===================================================================

%================== Diseño de Párrafo y delimitador ================
\renewcommand{\baselinestretch}{1.5}% Espaciado entre linea
\geometry{left=4cm,right=3cm,top=3cm,bottom=3cm}
\frenchspacing %
%\raggedright % Sólo para justificar el texto a la izquierda
%\renewcommand{\captionlabeldelim}{.}%
\setlength{\parindent}{0.7cm}% Espacio de la sangría
\setlength{\parskip}{14pt plus 1pt minus 1pt}% Separación entre párrafos

%\setlength{\parskip}{1ex plus 0.5ex minus 0.2ex}%

%===================================================================

%==========================  Español venezolano =====================
%%Personalización de caption
\addto\captionsspanish{%
  \def\prefacename{Prefacio}%
  \def\refname{REFERENCIAS}%
  \def\abstractname{Resumen}%
  \def\bibname{REFERENCIAS}%{Bibliografía}%
  \def\chaptername{CAPÍTULO}%
  \def\appendixname{Apéndice}%{Anexo}
  \def\contentsname{ÍNDICE GENERAL}
  \def\listfigurename{LISTA DE FIGURAS}%Índice de Figuras\hspace*{10em}
  \def\listfigurenameTofC{LISTA DE FIGURAS}%Índice de Figuras
  \def\listtablename{LISTA DE TABLAS}%Índice de Tablas
  \def\indexname{Índice alfabético}%
  \def\figurename{Figura}%
  \def\tablename{Tabla}%
  \def\partname{Parte}%
  \def\enclname{Adjunto}%
  \def\ccname{Copia a}%
  \def\headtoname{A}%
  \def\pagename{Página}%
  \def\seename{véase}%
  \def\alsoname{véase también}%
  \def\proofname{Demostración}%
  \def\glossaryname{Glosario}
  }%

%==================================================================

%\setcounter{secnumdepth}{1}
%\setcounter{page}{4}
%\addtocounter{page}{4}%

\pagenumbering{roman}

\makeindex

%%%%%%%%%%%%%%%%%%%%%%%%%%%%%%%%%%%%%%%%%%%%%%%%%%%%%%%%%%%%%%%%%

\begin{document}
%\frontmatter

%%%%%%%%%%%%%%%%%%%%%%%%%%%%%%%%%%%%%%%%%%%%%%%%%%%

%               Primera Página
%================================== Portada ========================
\input{portada.tex}

%======================= Constancia de Aprobación ===================
%\newpage
\begin{figure}
        \begin{center}
        %\centering
        %\includegraphics[height=23cm]{aprobacion.eps}

        \vspace{0.5mm}
        \label{Fig.aprobacion}
        \end{center}
        \end{figure}
\thispagestyle{empty}

%======================= Mención Honorífica =========================
\newpage
%\thispagestyle{empty}

\begin{figure}
        \begin{center}
        %\centering
        %\includegraphics[height=24cm]{mencion.eps}
        \vspace{0.5mm}
        \label{Fig.mencion}
        \end{center}
\end{figure}
\thispagestyle{empty}

%======================= Página de Dedicatoria ======================
\newpage%
\newenvironment{dedication}%
{\cleardoublepage \thispagestyle{empty} \vspace*{\stretch{1}}%
\begin{center} \em} {\end{center} \vspace*{\stretch{3}} }%
\begin{dedication}%
Dedicado a Gregoria del Valle Briceño Aray y José Edmundo Tovar Silva.
\end{dedication}%

%=============================== RECONOCIMIENTOS Y AGRADECIMIENTOS ===================================
\chapter*{RECONOCIMIENTOS Y AGRADECIMIENTOS}
%\markboth{Reconocimientos}{Reconocimientos}%
\addcontentsline{toc}{chapter}{RECONOCIMIENTOS Y AGRADECIMIENTOS}%
%\setlength{\parskip}{0.2cm}%
%\input{agradecimientos.tex}%

%======================= Página de Resumen (Abstract) ==========================
\newpage
\renewcommand*{\abstract}{\begin{center}\end{center}}
%\begin{abstract}
\begin{spacing}{1}
\begin{center}%

\textbf{Tovar B., José A.}

\begin{large}
DISEÑO DE UN SENSOR INTELIGENTE PARA APLICACIONES DE MONITOREO DE SALUD ESTRUCTURAL
\end{large}
\end{center}

\noindent%
\textbf{Tutor Académico: Prof. Jose Romero. Tesis.
Caracas, Universidad Central de Venezuela. Facultad de Ingeniería.
Escuela de Ingeniería Eléctrica. Ingeniero Electricista. Opción: Electrónica y Control. Institución: IMME Año 2024,
xvii, 153 h. + anexos}

\noindent
\textbf{Palabras Claves:} Salud estructural, Sensor inteligente,  Respuesta dinámica, Frecuencia natural, Microcontrolador, Acelerómetro, LoRa, ESP32, MQTT, Espectro en frecuencia. \\[1ex]

\noindent \textbf{Resumen.-} En el siguiente trabajo se plantea el diseño de un sensor inteligente para ser utilizado en aplicaciones de salud estructural. En primer lugar, se llevó a cabo la investigación documental necesaria para identificar las variables de interés al evaluar la respuesta dinámica de los sistemas estructurales y su relación con la instrumentación electrónica. Se escogió el hardware necesario para la implementación de un prototipo de pruebas capaz de verificar el funcionamiento del diseño. Se escogieron sensores de tecnología MEMS como el MPU6250, MPU9250 y BME280, en conjunto con el microcontrolador ESP32 y el módulo Ra-02 de Ai-Thinker para las comunicaciones inalámbricas. Se diseñó el software necesario para controlar el sistema haciendo uso de FreeRTOS y se implementaron con éxito las tareas tanto en el sensor inteligente como en la estación base para obtener los registros de vibración y de las variables cuasi-estáticas (temperatura, humedad relativa e inclinación). Se programó una interfaz gráfica de monitoreo y control para observar los registros y enviar comandos al sensor inteligente. Las pruebas realizadas en el IMME demostraron el funcionamiento satisfactorio del sensor inteligente al comparar los registros obtenidos con el equipo de vibración comercial basado en la tarjeta PCI-6221 de National Instruments, obteniendo resultados muy similares que permitieron caracterizar el comportamiento dinámico de una estructura de acero.

\end{spacing}

%\underline{RESUMEN}
%
\thispagestyle{empty}%
%\input{resumen.tex}%
%\end{abstract}
%====================== Páginas de Contenidos =====================
\renewcommand{\baselinestretch}{1.5}% Espaciado entre linea
\addtocounter{page}{3}%
\setlength{\parskip}{3pt}% Separación entre párrafos

\tableofcontents%

\listoffigures%

\listoftables%


%==================================================================
\chapter*{LISTA DE ACRÓNIMOS}%
%\markboth{Lista de Acrónimos}{Lista de Acrónimos}%
\addcontentsline{toc}{chapter}{LISTA DE ACRÓNIMOS}%


\begin{acronym}
\acro{SHM}{Structural Health Monitoring}
\acro{ADC}{Analog-to-Digital Converter}
\acro{NASA}{National Aeronautics and Space Administration}
\acro{SMIS}{Shuttle Modal Inspection System}
\acro{DOF}{Degrees of Freedom}
\acro{VLSI}{Very Large Scale Integration}
\acro{RAM}{Random Access Memory}
\acro{ROM}{Read-Only Memory}
\acro{DAC}{Digital-to-Analog Converter}
\acro{SoC}{System on a Chip}
\acro{MCU}{Microcontroller Unit}
\acro{SBC}{Single Board Computer}
\acro{UART}{Universal Asynchronous Receiver/Transmitter}
\acro{I2C}{Inter-Integrated Circuit}
\acro{SDA}{Serial Data Line}
\acro{SCL}{Serial Clock Line}
\acro{SPI}{Serial Peripheral Interface}
\acro{USB}{Universal Serial Bus}
\acro{RTOS}{Real-Time Operating System}
\acro{SMP}{Symmetric Multiprocessing}
\acro{MIT}{Massachusetts Institute of Technology}
\acro{FIFO}{First In, First Out}
\acro{MEMS}{Micro-Electro-Mechanical Systems}
\acro{DSP}{Digital Signal Processing}
\acro{IEEE}{Institute of Electrical and Electronics Engineers}
\acro{DAQ}{Data Acquisition}
\acro{WiFi}{Wireless Fidelity}
\acro{MQTT}{Message Queuing Telemetry Transport}
\acro{IoT}{Internet of Things}
\acro{M2M}{Machine to Machine}
\acro{CSS}{Chirp Spread Spectrum}
\acro{CRC}{Cyclic Redundancy Check}
\acro{FFT}{Fast Fourier Transform}
\acro{NTP}{Network Time Protocol}
\acro{RTC}{Real-Time Clock}
\acro{IMU}{Inertial Measurement Unit}
\acro{MARG}{Magnetic, Angular Rate, and Gravity}
\acro{JSON}{JavaScript Object Notation}
\acro{CSV}{Comma-Separated Values}
\acro{GUI}{Graphical User Interface}
\acro{ISR}{Interrupt Service Routine}
\end{acronym}
%


%==================================================================
\chapter*{INTRODUCCIÓN}\label{CAP:intro}
\setlength{\parskip}{14pt}% Separación entre párrafos
\addcontentsline{toc}{chapter}{INTRODUCCIÓN}%
%\markboth{Introducción}{Introducción}%

\pagenumbering{arabic}%
La seguridad de las infraestructuras es un tema de gran importancia en la actualidad, especialmente cuando se trata de estructuras como edificios o puentes.

Los primeros indicios del monitoreo del estado de las infraestructuras data de nuestros comienzos como especie sedentaria.  En la antigüedad, los especialistas utilizaban técnicas de inspección visual y auditiva para detectar posibles problemas en las estructuras, como grietas o ruidos inusuales. Con el tiempo, se desarrollaron técnicas más avanzadas para el monitoreo de estructuras, como la utilización de medidores de deformación, inclinación, sensores de vibración, entre otros.

La integración de la instrumentación con el análisis estructural comenzó a desarrollarse en la década de 1960 con el advenimiento de la informática y la disponibilidad de computadoras capaces de realizar cálculos estructurales complejos. En esa época, se comenzaron a utilizar sistemas de adquisición de datos para recopilar información sobre el comportamiento de las estructuras en tiempo real y utilizarla para calibrar y validar los modelos estructurales.

Actualmente, las normas sismo-resistentes apuntan a estructuras que sean capaces de mantener su integridad ante un evento de cierta magnitud. Además, el monitoreo continuo de ciertos indicadores en la estructura permiten determinar un índice de la salud estructural y ajustar el modelo a las condiciones actuales de la misma para evaluar el cumplimiento de la normativa sismorresistente. Para el monitoreo a largo plazo, el resultado de este proceso es información actualizada periódicamente sobre la capacidad de la estructura para desempeñar su función prevista a la luz del inevitable envejecimiento y degradación resultantes de los entornos operativos.

Según \citep{balageas2010structural}, el monitoreo de la salud estructural (SHM) tiene por objeto proporcionar, en cada momento de la vida de una estructura, un diagnóstico del estado de los materiales constitutivos, de las diferentes
partes, y del conjunto de estas partes que constituyen la estructura en su totalidad. El estado de la estructura debe permanecer en el ámbito especificado en el diseño, aunque este puede verse alterado por el envejecimiento normal debido al uso, por la acción del medio ambiente y por sucesos accidentales. Gracias a la dimensión temporal de la supervisión, que permite tener en cuenta toda la base de datos histórica de la estructura, y con la ayuda del monitoreo del funcionamiento. También puede proporcionar un pronóstico (evolución de los daños, vida residual, entre otros).

Si consideramos solo la primera función, el diagnóstico, podríamos estimar que el monitoreo de la salud estructural es una forma nueva y mejorada de realizar una evaluación no destructiva. Esto es parcialmente cierto, pero SHM es mucho más. Implica la integración de sensores, posiblemente materiales inteligentes, transmisión de datos, potencia computacional y capacidad de procesamiento en el interior de las estructuras. Permite reconsiderar el diseño de la estructura y la gestión completa de la propia estructura y de la estructura considerada como parte de sistemas más amplios.


%Incluir conjunto de elementos que influyen en el deterioro de la estructura. Envejecimiento, mantenimiento

En este sentido, el monitoreo de las estructuras se ha convertido en una herramienta esencial para garantizar la seguridad de las personas durante la vida útil de la misma, incluyendo la ocurrencia de eventos de cierta magnitud. Además, el monitoreo de las estructuras puede ayudar a mejorar la eficacia de las normas sismorresistentes, ya que permite validar y mejorar los modelos estructurales utilizados en la normativa.

Según \citep{nagayama2007structural}, dado que las edificaciones suelen ser grandes y complejas, la información
de unos pocos sensores es inadecuada para evaluar con precisión el estado estructural. El
comportamiento dinámico de estas estructuras es complejo tanto a escala espacial como temporal. Además,
los daños y/o el deterioro es intrínsecamente un fenómeno local. Por lo tanto, para comprender el
comportamiento dinámico, el movimiento de las estructuras debe ser supervisado por sensores
con una frecuencia de muestreo suficiente para captar las características dinámicas más destacadas. Esta información combinada con el registro del comportamiento estático de la estructura permiten tener una visión más amplia del estado actual de la estructura.


El primer paso, además de un mantenimiento adecuado, para garantizar la seguridad de estas estructuras, es contar con sistemas de monitoreo que permitan detectar posibles daños o fallas en su funcionamiento y tomar medidas preventivas. Por tanto, los sistemas de adquisición de datos y monitoreo son herramientas esenciales en la prevención de accidentes y daños.

A su vez, según \citep{nagayama2007structural}, un dispositivo inteligente, es decir, con capacidad de procesamiento de datos en el caso de los sensores, es una característica esencial que permite incrementar el potencial de los sensores al ser estos inalámbricos. Los sensores inteligentes pueden procesar localmente los datos medidos y trasmitir solo la información importante a través de comunicaciones inalámbricas. Cuando estos son configurados como una red, se extienden las capacidades de los mismos.

Los sensores inteligentes, con sus capacidades de cómputo y de comunicación integradas, ofrecen nuevas oportunidades para la SHM. Sin necesidad de cables de alimentación o comunicación, los costes de instalación pueden reducirse drásticamente. Los sensores inteligentes ayudarán a que el monitoreo de las estructuras con un denso conjunto de sensores sea económicamente práctico. Se espera que los sensores inteligentes instalados en masa sean fuentes de información muy valiosa para la SHM.

En este trabajo de grado se abordará el diseño para una futura implementación de un sistema de adquisición de datos de bajo costo basado dispositivos programables con capacidad de interconexión para el monitoreo y procesamiento de variables como aceleración, inclinación, humedad y temperatura en estructuras críticas, con el objetivo de prevenir daños y accidentes.



De acuerdo a Brea  la transformada de Laplace debe estudiarse como
una función definida en el campo de los números complejos
\cite{brea5}.

Otro modo de referencial es \citep{brea5}

El resto del reporte consta de: en el Capítulo \ref{CAP:referencial} se
describe...

En el trabajo se emplea el enfoque de \cite{brigham1}

De acuerdo a la ecuación
%

%==================================================================
\chapter{MARCO REFERENCIAL}\label{CAP:referencial}
\section{Planteamiento del problema}

\section{Justificación}

\section{Objetivos}

\subsection{Objetivo general}

\subsection{Objetivos específicos}

\section{Antecedentes}

%

%==================================================================
\chapter{MARCO TEÓRICO}\label{CAP:marco_teor}
%\markboth{Tu Primer Capítulo}{Tu Primer Capítulo}%

En este capítulo se definirán los conceptos  o fundamentos de instrumentación estructural, sensores inteligentes y adquisición de datos, necesarios para llevar a cabo esta investigación.

\section{Estructuras civiles}

\subsection{Características generales}

Una estructura se refiere a un sistema de partes o elementos que se interconectan para cumplir una función es específico. En el caso de la ingeniería civil, suelen ser miembros que se utilizan para soportar una carga. Algunos ejemplos importantes son los edificios, los puentes y las torres; y en otras ramas de la ingeniería, son importantes las corazas de barcos y aviones, los sistemas mecánicos y las estructuras que soportan las líneas de transmisión eléctrica \citep{hibbeler1997structural}.

\subsection{Tipos de estructuras}

Según \citet{hibbeler1997structural}, cada sistema está formado por uno o varios de los cuatro tipos básicos de estructuras: 

\begin{itemize}
    \item Celosías.
    \item Cables y arcos.
    \item Armazones.
    \item Estructuras de superficie.
\end{itemize}

En general, estos elementos suelen soportar cargas, pueden ser estacionarios y también estar restringidos. Sus diferencias suelen basarse en la cantidad de fuerzas a las que están sujetos estos elementos en un instante dado.

La combinación de estos elementos y los materiales que los componen es lo que se denomina un sistema estructural. Estos sistemas, aunque sean pasados por alto, son utilizados diariamente por industrias y personas, siendo elementos claves en el desarrollo y progreso de la civilización actual.

\subsection{Comportamiento de las estructuras civiles}

La gran mayoría de los sistemas cuentan con una respuesta dinámica y estática. Ambas respuestas permiten conocer el comportaiento completo del sistema en estudio ante distintas entradas o en diferentes situaciones. Al estudiar el comportamiento estructural se encuentra una extensa literatura tanto para el estudio dinámico como para el régimen estático, recopilándose lo siguiente:

\begin{itemize}
    \item{Respuesta estática}: En la ingeniería civil toda estructura se diseña para que se encuentre en reposo cuando actúan sobre esta fuerzas externas, es decir, la estructura en conjunto debe cumplir con las condiciones de equilibrio, siendo la fuerza y el momento resultanto sobre esta igual a cero en todo momento. Para describir estas condiciones de equilibrio se cuentan con herramientas matemáticas que proporcionan las condiciones necesarias para su cumplimiento. Estas ecuaciones permiten la resolución estática de la estructura, la cual permite determinar el valor de todas las incógnitas estáticas de interés \citep{basset2014analisis}.
    
    \indent Cuando las fuerzas que actúan sobre la estructura pueden calcularse a partir de las ecuaciones de equilibrio, se tiene una estructura en equilibrio y se denonima estructura estáticamente determinada. En caso de tenerse más fuerzas desconocidas que ecuaciones de equilibrio se habla de una estructura estáticamente indeterminada.

        \begin{itemize}
            \item Rigidez: Uno de los parámetros más importantes dentro de la respuesta estática es la rigidez. Esta se define como la propiedad que tiene un elemento estructural de soportar la deformación o deflección al estar bajo la acción de una fuerza o carga. Una medida de la rigidez viene dada por el Módulo de Young; esta es una constante del material y es independiente de la cantidad de material.
        \end{itemize}

    \item{Respuesta dinámica}: La dinámica estructural se encarga de estudiar el efecto que tienen cargas dinámicas sobre el sistema. La respuesta ante estos eventos, como pueden ser sismos, vientos, equipos mecánicos, paso de vehículos o personas, se denomina respuesta dinámica \citep{hurtado2000}. Además, la respuesta dinámica permite caracterizar algunos parámetros de gran interés para estudiar su comportamiento conocidos como parámetros modales. Estos parámetros surgen al estudiar las ecuaciones diferenciales que describen el movimiento de la estructura, partiendo de un modelo idealizado simple de masa concentrada como el de la Figura \ref{fig:masa_estructural}.
    
    \begin{figure}[H]
        \centering
        \includegraphics[width = 0.25\textwidth]{imagenes/cap1_marcoteo/modelo_masa_simple.png}
        \caption{Modelo de masa concentrada de 1 grado de libertad \citet{hurtado2000}.}
        \label{fig:masa_estructural}
    \end{figure}

    La dinámica de este modelo puede describirse utilizando la ecuación diferencial de movimiento:

    \begin{equation} \label{eq:vib_lib}
        m\ddot{u} + f_R(t) = p(t)
    \end{equation}

    La ecuación \ref{eq:vib_lib} se conoce como ecuación de vibración libre sin amortiguamiento. Donde $p(t)$ representa las cargas dinámicas y $f_R(t)$ la fuerza de restitución propia de un material elástico.  Esta ecuación es una ecuación diferencial de coeficientes constantes, que consta de una solución homogénea más una solución particular. La solución homogénea será la respuesta de la estructura a la vibración libre, es decir, si la masa de la Figura \ref{fig:masa_estructural} se deja oscilar libremente.

    Se sabe que una ecuación de este tipo tendrá una solución como:

    \begin{equation} \label{eq:sol_equ_dif}
        u = A.sin\omega t + B.cos\omega t
    \end{equation}

    La ecuación \ref{eq:sol_equ_dif} contiene información relevante para la caracterización dinámica de la estructura. Esta caracterización parte del estudio de los parámetros modales de la misma.

    Entre estos parámetros modales se encuentran: 
        \begin{itemize}
            \item Frecuencia natural: Toda estructura física tiene asociada una frecuencia de vibración natural. Las máquinas, los puentes, los edificios; todas estas estructuras vibran u oscilan al ser perturbadas o removidas de su estado de reposo inicial. Es una propiedad es intrínseca del sistema y depende de su masa, rigidez y amortiguamiento. Todas tienen al menos una frecuencia natural y es posible que tengan múltiples frecuencias de resonancia \citep{irvine2000introduction}. 
            
                Se suele calcular la frecuencia natural de resonancia de un sistema libre usando:

                \begin{equation}
                    f =  \frac{1}{\sqrt{\frac{k}{m}}}
                \end{equation}

            \item Amortiguamiento: Toda estructura comienza a oscilar una vez es removida de su estado de reposo o equilibrio, sin embargo, ese movimiento no es perpetuo. El amortiguamiento se define como la capacidad de disipación de energía que posee la estructura bajo excitaciones externas. Las soluciones a la ecuación \ref{eq:vib_lib}, al añadir el amortiguamiento de tipo viscoso, arrojan 3 posibles casos:
                \begin{enumerate}
                    \item Sistema críticamente amortiguado: El sistema no vibra.
                    \item Subamortiguado o amortiguado subcrítico: Caso más común por la naturaleza de los materiales utilizados en las estructuras. La respuesta del sistema decae con el tiempo de forma exponencial, como se puede ver en la Figura \ref{fig:resp_subamorti}. 
                    
                    \begin{figure}[H]
                        \centering
                        \includegraphics[width = 0.7\textwidth]{imagenes/cap1_marcoteo/respuesta_sist_subamorti.png}
                        \caption{Respuesta ante vibración libre en sistema Subamortiguado \citep{hurtado2000}.}
                        \label{fig:resp_subamorti}
                    \end{figure}

                    \item Sobreamortiguado: Nunca se encuentra esta respuesta en sistemas estructurales por los materiales utilizados.
                \end{enumerate}
            
        \end{itemize}        
\end{itemize}

\subsection{Respuesta en frecuencia}

    Incluir?

\subsection{Daño en estructuras}

El daño a una estructura civil o mecánica puede definirse como todo cambio en las propiedades materiales o geométricas del material que llegan a afectar de forma adversa la confiabilidad y el desempeño actual o futuro del sistema. Por tanto, el daño es una comparación entre el sistema en cuestión en 2 instantes de tiempo distintos \citep{farrar2007introduction}. Estos efectos adversos pueden ser, en el caso estructural, desplazamientos, estrés indeseado en un elemento o vibraciones estructurales indeseadas \citep{chen2018}.

Toda estructura civil, como puentes y edificios, acumulan daño de forma continua a medida que están en servicio y transcurre su vida útil. Este daño puede manifestarse como fracturas, fatiga, socavaciones o desprendimiento del concreto. El daño que no sea detectado puede conducir a una falla estrcutural que a su vez ocasione pérdidas humanas. Por tanto, es imperativo y necesario detectar el daño en una estrcutrura tan pronto como sea posible, \citep{chen2018}.

Entre algunos de los factores que influyen del deterioro de una estructura se encuentran:
    
        \begin{itemize}
            \item Proceso de degradación natural de los materiales.
            \item Corrosión del acero de refuerzo.
            \item Evento sísmico, incendios o condiciones de guerra.
            \item Carga por encima del límite de diseño.
        \end{itemize}
    
Las escalas de tiempo y de extensión del daño son diversas. Por ejemplo, el deterioro por el paso del tiempo bajo ciertas condiciones climáticas es muy lento comparado al daño causado por un evento catastrófico.

\subsection{Principios de la Sismoresistencia}

Una edificación sismorresistente es aquella que está diseñada y construida para soportar las fuerzas causadas por eventos sísmicos. Sin embargo, incluso las edificaciones diseñadas y construidas según las normas sismorresistentes pueden sufrir daños en caso de un terremoto muy fuerte, sin embargo, las normas establecen los requisitos mínimos para proteger la vida de
las personas que ocupan la edificación

Algunas de las características de una estructura sismoresistente son:

        \begin{itemize}
            \item Forma regular.
            \item Bajo peso.
            \item Mayor rigidez.
            \item Buena estabilidad.
            \item Suelo firme y buena cimentación.
            \item Materiales competentes.
            \item Capacidad de disipación de energía.
            \item Fijación de acabados e instalaciones.
        \end{itemize}

En Venezuela las estructuras deben cumplir con la Norma Venezolana COVENIN 1756:2001 (Edificaciones Sismorresistentes).

Se ha observado que al estudiar el comportamiento de las estructuras luego de un evento sísmico, es evidente que cuando se toman en cuenta las normas de diseño sismorresistente dispuestas en la ley y la construcción es debidamente supervisada, los daños estructurales resultan ser considerablemente menores que en las edificaciones en las cuales no se cumplen los requerimientos mínimos indispensables estipulados en la norma, \citep{blanco2012criterios}.

\subsubsection{Importancia de la instrumentación} La instrumentación estructural permite medir y monitorear las acciones y respuestas estructurales ante distintos eventos. Esto proporciona datos en tiempo real sobre el comportamiento dinámico y estático de la estructura, como deformaciones, aceleraciones y desplazamientos, que son fundamentales para evaluar y verificar si la estructura cumple con los criterios de diseño sismoresistente establecidos en la norma.

La instrumentación estructural ayuda a validar los modelos y suposiciones utilizados en el diseño estructural inicial. Al comparar los datos recopilados por la instrumentación durante un evento sísimco con las predicciones del modelo, es posible verificar si la estructura se comporta de acuerdo con las expectativas y si cumple con los criterios de seguridad establecidos en la normas.

Además, el monitoreo continuo de la estructura permite conocer el estado actual de la misma, tema que representa la idea principal del Monitoreo de Salud Estructural, permitiendo a los ingenieros evaluar si se sigue cumpliendo con la norma para luego tomar decisiones y actuar en pro de la seguridad de la edificación.


\section{Salud estructural}

\subsection{Definición}


El proceso de implementar una estrategia de identificación de daño para estructuras civiles, mecánicas o aeroespaciales se conoce como Monitoreo de Salud Estructural (SHM por sus siglas en inglés). Esta estrategia requiere medir las condiciones y el ambiente en el que opera la estructura, además de la respuesta de la misma durante un período de tiempo tomando muestras periódicamente espaciadas, \citep{farrar2007introduction}.


La estrategia del SHM requiere de equipos multidisciplinarios de ingeniería, ya que necesita de una red de sensores que midan las variables de interés, el procesamiento y análisis de los datos obtenidos y posteriormente una prognosis del daño para una eventual toma de decisiones. El objetivo del SHM es proveer, en toda la vida útil de la estructura, un diagnóstico del estado de sus materiales constitutivos, de los diferentes elementos que la componen y de la estructura en sí como el conjunto de todas estas partes. Esto para garantizar que la misma se comporte dentro de los parámetros iniciales de diseño, aunque estos cambien por la acción natural del tiempo, el ambiente y accidentes, \citep{balageas2010structural}.

El resultado de este proceso es información actualizada sobre el estado de la estructura y sobre su capacidad actual para seguir desempeñado la función para la cual fue diseñada.

Según \citet{enckell2006structural}, el SHM se ha convertido en una herramienta muy conocida y utilizada en ingeniería estructural en los últimos años en diferentes países.

\subsection{Reseña histórica}

Las técnicas de detección de daño basadas en vibración tienen sus primeras aplicaciones desde hace cientos de años. En la antigüedad, los constructores golpeaban las estructuras para encontrar espacios vacíos o grietas en elementos de arcilla. La utilidad de estas inspecciones tan simples indicaban que la sofisticación de estos métodos podía proveer información muy valiosa sobre el elemento de interés, sin embargo, esto requiere de instrumentos y herramientas matemáticas que se han desarrollado con el pasar de los años. El auge en el uso de SHM en años recientes es consecuencia de la evolución y miniaturización del hardware computacional actual.


El uso más exitoso del SHM ha sido el monitoreo de la condición de máquinas rotativas, las cuales actualmente han adoptado un enfoque de indetificación de daño sin basarse en un modelo de forma casi exclusiva, \citep{farrar2007introduction}.

En los años 70 la industria petrolera consideró el uso de técnicas basadas en vibración para identificar daños en plataformas costa-afuera, este enfoque se diferenció de las máquinas rotativas al estudiar un sistema en donde la ubicación del daño es desconocida y difícil de instrumentar.

En esa misma época, la comunidad aeroespacial y la \textit{National Eeronautics Space Agency} (NASA), comenzaron a estudiar esta técnica de identificación de daño en los comienzos de la era de lanzamientos espaciales. Este trabajo continúa hoy en día y el \textit{Shuttle Modal Inspection System} (SMIS) se desarrolló para identificar fatiga en distintos componentes de cohetes espaciales reusables, los cuales representan el futuro de esta industria.


Usualmente, los enfoques de estas industrias se basan en comparar modelos analíticos de estructuras sin daño con las mediciones de estructuras con daño, observando principalmente las propiedades modales de las mismas. Se ha observado que cambios en la rigidez en ambos modelos han permitido localizar y cuantificar el daño, \citep{farrar2007introduction}.

Inicialmente, las técnicas no destructivas fueron introducidas en la ingeniería civil a mediados de los años 40, \citep{mohamed2014}. La necesidad principal surgió en determinar propiedades del concreto fresco \textit{in-situ}. Estas técnicas, que buscaban evaluar la homogeneidad y la resistencia del concreto eran en su mayoría pruebas con martillo y pruebas de \textit{pull-out}. A medida que las estructuras envejecieron, los ingenieros necesitaban idear maneras de medir o estimar las propiedades mecánicas de los elementos que consituyen las estructuras, además de detectar daños que no eran fáciles de observar por la envergadura de las estructuras civiles que se han desarrollado en los últimos 150 años. Es ahí, en los años 70, donde surgen nuevas estrategias no destructivas tales como:

    \begin{itemize}
        \item Emisión acústica.
        \item Métodos de ultrasonido y radar.
        \item Termografía.
        \item Métodos basados en vibración
    \end{itemize}

La comunidad de ingeniería civil ha estudiado la identificación de daño basada en vibración en puentes y edificios desde comienzos de los años 80. Las propiedades modales han sido estudiadas por diferentes autores y son las principales características que se analizan al identificar daño. El auge del SHM es tal, que algunos países asiáticos han implementado regulaciones en donde las compañías constructoras deben verificar la salud estructural de los puentes periódicamente. Estas regulaciones han provocado que la investigación e inversión en esta área siga aumentando de forma considerable, \citep{chen2018}. 

\subsection{Línea de trabajo del Monitoreo de Salud Estructural}

Los sistemas de SHM consisten de varios elementos que permiten a los ingenieros tener información sobre el estado de una estructura, entre esos elementos se encuentran:

\begin{itemize}
    \item Sensores.
    \item Sistemas de adquisición de datos.
    \item Sistema de transmisión de datos.
    \item Sistema de procesamiento de datos.
    \item Sistema de manejo y almacenamiento de datos.
    \item Equipo de análisis y toma de decisiones.
\end{itemize}

\begin{figure}[H]
    \centering
    \includegraphics[width = 0.9\textwidth]{imagenes/cap1_marcoteo/Schematics-of-an-on-line-structural-health-monitoring-system-and-technical-challenges.png}
    \caption{Esquema de un sistema de SHM \citep{lijianfoto2015}.}
    \label{fig:esquema_gral_SHM}
\end{figure}


Autores como \citet{rytter1993vibration} y \citet{farrar2007introduction} han esquematizado la estrategia del SHM categorizando el daño en una estructura por niveles de la siguiente forma:

\begin{enumerate}
    \item Nivel I (detección del daño) ¿Presenta daño el sistema? Es una indicación cualitativa de que puede haber daño presente en la estructura.
    \item Nivel II (localización o ubicación del daño) ¿Dónde está presente el daño? Indica la posible localización del mismo. 
    \item Nivel III (clasificación del daño) ¿Qué tipo de daño está presente? Da información sobre el tipo de daño.
    \item Nivel IV (alcance/grado/extensión del daño) ¿Cuál es el alcance del daño? ¿Qué tan grave es? Da un estimado del alcance.
    \item Nivel V (prognosis del daño) ¿Cuánta vida útil le queda a la estructura? Da un estimado de la seguridad de la estructura.
\end{enumerate}

En la mayoría de los casos, para alcanzar el nivel final es necesario obtener información sobre los niveles previos. Esto indica que a medida que se sube de nivel se tiene un mayor conocimiento sobre el estado de la estructura.

De acuerdo a \citet{chen2018}, los primeros dos niveles, detección y localización, generalmente pueden alcanzarse usando métodos de detección basados en vibración para obtener mediciones sobre la respuesta dinámica de la estructura.

Por su parte, \citet{chen2018} describe el proceso de SHM en general como:

\begin{enumerate}
    \item Observación.
    \item Evaluación.
    \item Calificación.
    \item Gestión.
\end{enumerate}

La estrategia de Monitoreo de Salud Estructural podría resumirse en el siguiente diagrama:

\begin{figure}[H]
    \centering
    \includegraphics[width = \textwidth]{imagenes/cap1_marcoteo/Diagrama SHM timeline.png}
    \caption{Diagrama general del proceso de SHM.}
    \label{fig:diag_SHM}
\end{figure}

\subsection{Criterios de evaluación}

Como se definió anteriormente, el daño estructural puede tener distintas causas y formas. Lo que se sabe con certeza es que una estructura, una vez entra en funcionamiento, análogo a los seres humanos al nacer, estará sujeta a envejecimiento natural y a condiciones adversas. Ahora bien, en el caso del SHM, surge la siguiente pregunta ¿Qué se debe medir para poder detectar este daño?. 

Numerosos autores concluyen que uno de los indicativos de daño de una estructura viene dado por los parámetros modales definidos anteriormente, frecuencia y amortiguamiento. Esta relación entre los parámetros modales y el daño viene dada por la premisa de que todo daño presente en la estructura se reflejará en un cambio en las propiedades dinámicas de la misma. \citet{worden2009modal}, comprobó la relación entre el cambio progresivo en todas las frecuencias naturales de 15 vigas estudiadas a las cuales se les introdujo un daño relacionado con un cambio del Módulo de Young, el cual, como fue mencionado anteriormente, provee un indicativo de la rigidez de un elemento.

Anterioremente en la ecuación \ref{eq:vib_lib} se definió un sistema con un grado de libertad (DOF por sus siglas en ingles), sin embargo, en la realidad las estructuras tienen múltiples grados de libertad, por lo que es conveniente modelarlas de estar forma para obtener resultados más precisos. En el caso de los sistemas \textit{n-dregrees of freedom} (DOF por sus siglas en ingles), la ecuacion de movimiento que describe la dinámica del sistema en vibración libre vendrá dada por:

\begin{equation} \label{eq:ecu_movimiento}
    M\ddot{u} + C\dot{u} + Ku = 0
\end{equation}

Donde M, C y K representan las matrices de masa, amortiguamiento y rigidez de la estructura, respectivamente.

Si se asume un sistema sin amortiguamiento, a fines de estudiar el efecto que tiene sobre la rigidez un cambio en los parámetros modales,  de la ecuación \ref{eq:ecu_movimiento} se obtiene: 

\begin{equation} \label{eq:ecu_movimiento_sindamp}
    M\ddot{u} + Ku = 0
\end{equation}

Si se asume una solución oscilatoria pura, por ser un sistema sin amortiguamiento:

\begin{equation} \label{eq:sol_ecu_motion}
    u = ve^{jwt}
\end{equation}

Al derivar, sustituir y despejar en la ecuación \ref{eq:ecu_movimiento_sindamp} se obtiene:

\begin{equation} \label{eq:eig_problem}
    (K - \lambda M)\phi = 0
\end{equation}

Esta ecuación \ref{eq:eig_problem} representa claramente un problema de autovalores, donde $\lambda$ representa los autovalores asociados a las frecuencias naturales del sistema y $\phi$ representa el autovector de desplazamiento.

Si se introduce un pequeño cambio $\Delta K$ con perturbaciones similares en los otros parámetros:

\begin{equation}
    [(K - \Delta K) - (\lambda - \Delta\lambda)(M - \Delta M)](\phi - \Delta\phi) = 0
\end{equation}

El daño estructural suele venir asociado a un cambio en la rigidez, mas no a cambios en la masa de la estructura, por lo que se asume $\Delta M = 0$, \citep{hearn1991modal}. A su vez, se tiene que $(K - \lambda M)\phi = 0$. \citet{mohamed2014}, \citet{shi1998structural} y \citet{hearn1991modal} desarrollan estas ecuaciones obteniendo la siguiente relación:

\begin{equation} \label{eq:relacion_final}
    \Delta\lambda = \phi^T \Delta K \phi
\end{equation}

De la ecuación \ref{eq:relacion_final} se observa que cambios en los autovalores $\lambda$ que representan las frecuencias naturales, y en los autovectores $\phi$ (formas modales) están directamente relacionados con cambios en la matriz de rigidez (K) del sistema. De aquí surge el interés en monitorear los parámetros modales como indicadores de daño estructural. Es importante recalcar que estos cambios son indicativos de daño global, mas no de la localización del mismo, para lo que se necesitan otras técnicas, \citep{mohamed2014}.

\subsection{Variables de interés}

Tomando en cuenta la relación entre los parámetros modales y el daño presente en una estructura, es preciso definir las variables de interés para el monitoreo de la salud estructural de una estructura. Si bien existen distintas varibales que permiten obtener información valiosa sobre la estructura en estudio, algunas de estas no proporcionan información global del daño, como es el caso de las formas modales y la deflección local \citep{rytter1993vibration}. Sin embargo, estas mediciones proveen indicativos de la ubicación del daño, por lo que pueden constituir parte del sistema de monitoreo en una etapa más avanzada, es decir, una vez el daño fue detectado. A continuación se presentan las más relevantes para el daño global:

    \begin{itemize}
        \item Frecuencias naturales y amortiguamiento: Los parametros modales de la estructura están ligados de forma directa al estado de la misma. El deterioro en una edificiación induce cambios en la rigidez estructural, como se observa claramente en la ecuación \ref{eq:relacion_final}. El daño puede tener efectos distintos en cada modo o cada frecuencia de vibración, por lo que es importante no ubicar los sensores sobre los nodos modales, ya que experimentos han demostrado la ineficacia en las mediciones. Usualmente, el daño se refleja como una disminución en las frecuencias naturales afectadas, aunque se han observado casos de aumento en las frecuencias de vibración en estructuras de concreto pretensado, \citep{rytter1993vibration}.
        
        A su vez, el amortiguamiento varía al introducir daño en la estructura, puesto que su capacidad de disipar energía se ve afectada. Usualmente, los investigadores observan un aumento en el amortiguamiento a medida que el daño aumenta, como se ha demostrado experimentalmente por autores como \citet{hearn1991modal} y \citet{rytter1993vibration}.

        \item Temperatura y humedad: En los sistemas de monitoreo, la detección del daño estructural puede tomar períodos de tiempo considerables, durante los cuales las caracterpisticas sujetas a temperatura y humedad, sufren cambios que afectan la respuesta estructural.
                
        Es evidente que las condiciones climáticas contribuyen con el deterioro de las edificaciones. A pesar de esta conclusión, relacionar las condiciones climáticas con el daño introducido usando mediciones ambientales es difícil. La medición de estas variables suele tomarse en cuenta para poder cuantificar el cambio que producen estas condiciones en los demás indicadores de daño. \citet{rytter1993vibration} observó que la humedad y temperatura afectaban las mediciones de amortiguamiento. Por su parte, \citet{mohamed2014}, observó que las frecuencias naturales de barras y vigas disminuían a medida que aumentaba la temperatura. A su vez, \citet{sohn2007effects} determinó que cuando hay humedad, los puentes de hormigón absorben una cantidad considerable de humedad, lo que aumenta sus masas y altera sus frecuencias naturales.
        
        \item Inclinación y desplazamiento: n practical applications of structural monitoring, the most common is the measurement of linear displacements, which reflect in a very good and direct way the behavior of the structural element / structure or a part of the structure.
        
        Inclinometers are used to measure inclination (tilt) of structural components due to distress in the system. For example, they are often utilised to assess fixity of bridge girders at supports and to monitor longterm movements of bridge piers, abutments and girders.

        Deflection is a very important index for bridge structures, because it not only affects driving comfort, but also reflects the overall response of the bridge. Various factors could contribute to deflection increase during bridge service life, such as concrete creep, steel corrosion, prestress loss, and crack growth. The increase of structural deflection, however, will in turn accelerate the damage accumulation process. Therefore, monitoring bridge deflection is of great significance in the field of structural health monitoring (SHM) to provide early warnings of possible structural changes, damage, or deterioration.

        In several industries during the last few decades, inclinometer sensors have been employed extensively. In fact, in the civil engineering sector, inclinometers were initially used for geotechnical purposes [80]. Improvements in sensor accuracy over time have made it possible to use inclinometers in other areas of civil engineering, such as monitoring the structural health of bridges [79].
        


    \end{itemize}
    

\subsection{Consideraciones y desafíos}

\section{Sensores}

\subsection{Definición y tipos de sensores}

\subsection{Sensores de interés para el Monitoreo de Salud Estructural}

    \begin{itemize}
        \item Acelerómetros e Inclinómetros:
        \item Desplazamiento:
        \item Temperatura:
        \item Humedad:
    \end{itemize}

\subsection{Sensores inteligentes}

\section{Microcontroladores}%

%==================================================================
\chapter{MARCO METODOLÓGICO}\label{CAP:marco_met}
%\markboth{Tu Segundo Capítulo}{Tu Segundo Capítulo}%
En este capítulo se describirá el diseño del sensor inteligente, describiendo detalles de hardware y software del mismo.

\section{Descripción del sistema}

El sistema diseñado integra un conjunto de sensores y módulos que permiten medir variables dinámicas y cuasi-estáticas de un sistema estructural, enviarlas a larga distancia y posteriormente procesarlas y almacenarlas.

Una vez el sistema es encendido, calibra de forma automática todos los sensores, con especial énfasis en eliminar el offset de los 3 ejes del acelerómetro y también calibrar el magnetómetro y giróscopo del acelerómetro de 9 grados de libertad. El sistema ejecutará las tareas de calibración cada 2 registros de datos enviados con éxito.

Luego de la calibración y ajuste de la hora y fecha,  el sistema comienza a medir de forma continua la aceleración triaxial, inclinación, temperatura, humedad y estima la inclinación con sensores electrónicos de bajo consumo para posteriormente enviar los datos de forma inalámbrica a la estación base, que puede estar ubicada a más de 150 metros de distancia. Allí son recibidos, decodificados y pre-procesados para luego ser subidos vía inaálmbrica a un computador que sirve de servidor en donde se almacenaran los datos, siendo controlado el sistema desde esta misma estación base, pudiendo enviar comandos de adquisición de datos de forma remota. A su vez, se desarrolló una interfaz gráfica que permite visualizar los datos obtenidos, realizar peticiones de datos a distancia, observar sus características principales, acceder al histórico de datos recogidos por el sensor en una fecha específica, descargar los datos y ejecutar post-procesamiento a los mismos para evañuar las variables de interés para el monitoreo de la salud estructural.

Puesto que la estación base está conectada a internet, el microcontrolador ubicado en la estación adquiere la fecha y hora actualizada utilizando un servidor del protocolo NTP (\textit{Network Time Protocol}) y sincroniza su RTC (\textit{Real Time Clock}) interno con estos valores. Una vez obtenida la fecha y hora, espera el comando de inicio del sensor inteligente que envía de forma automática una vez se enciende y calibra sus sensores, la cual le indica a la estación base que debe enviar la fecha y hora actual. De esta forma se sincronizan los relojes internos de ambos y permite al sensor inteligente tener la fecha y hora a la cual tomó todo registro, enviando esta información como parte del registro de datos. La fecha y hora se envía bajo el formato \textit{UNIX Epoch}, el cual indica la cantidad de segundos transcurridos desde el 1 de Enero de 1970.
	
En la estación base, el sistema se conectará a internet a través de la red WiFi, enviará los datos recibidos al computador en la estación base, siendo esta herramienta la encargada de pre-procesar los datos recibidos vía inalámbrica y convertirlos a un formato adecuado para poder ser almacenados en el computador y posteriormente procesados usando librerías de análisis numérico. La interfaz de usuario podrá estar disponible para todos los usuarios de la red local, siempre que tengan los servicios necesarios instalados en local.

El comportamiento de los sistemas estructurales a estudiar condiciona los rangos e intervalos de medición de los sensores, con un límite superior cercano a los 3 g en un movimiento telúrico considerable. Por su parte, la temperatura, humedad e inclinación suelen considerarse variables cuasi-estáticas, permitiendo que el sistema mida estas variables con intervalos lo suficientemente largos para poder monitorear cambios considerables en las mismas y posteriormente correlacionarlos a las condiciones estructurales. La frecuencia de muestreo de aceleración es deseable que est

El sistema en su funcionamiento normal está ejecutando las siguientes tareas principales:

\begin{itemize}
    \item Adquisición continua (envío programado a ciertas horas del día o ante eventos importantes): El sensor inteligente mide constantemente las variables de interés y envía periódicamente, a ciertas horas del día programadas con antelación, un registro de datos a la estación base. Al estar midiendo de forma continua, está atento para generar una interrupción o alerta ante algún valor de aceleración o inclinación que esté por encima de algún límite escogido con anterioridad que dependerá en gran medida de la estructura a monitorear y su naturaleza, aunque se escogió por default el valor de 2 m/s como generador de alerta, basado en la \textit{Escala de Mercalli} \citep{mercalli}, la cual indica que este valor de aceleración (que corresponde a 0.20 g) equivale a un sismo fuerte con daño moderado. El sistema envía de forma automática un registro de datos del acontecimiento importante que generó la alerta.
	\item Escuchando petición de trama de datos inmediata (Envía trama ante request/query de estación base): Al recibir desde la estación base el comando de adquisición de datos, ejecutado desde la interfaz de control por el operador, el sistema comienza a tomar un registro de datos de forma inmediata, siempre y cuando el mismo no haya detectado previamente un evento importante que superara los límites establecidos, y lo envía a la estación base permitiendo que el usuario obtenga información del sistema en el instante en el cual se realiza la petición de los datos. Una vez es enviado y recibido con éxito la trama de datos, el sistema regresa a su estado anterior, tomando datos de forma continua y esperando alertas, comandos u horas programadas.

\end{itemize}

\subsection{Diagrama de funcionamiento del sensor inteligente}

\section{Selección de componentes}

Una vez obtenida una base teórica sobre estos temas, se procedió a escoger el hardware y los protocolos de comunicación más adecuados para llevar a cabo el objetivo de diseñar un sensor inteligente para aplicaciones de monitoreo de salud estructural. 
	

\subsection{Protocolo de comunicaciones}

Para el protocolo de comunicación se buscaron protocolos capaces de manejar los datos recogidos por los sensores de forma eficaz y confiable. En este caso se refiere al protocolo utilizado para enviar los datos desde el sensor inteligente hasta la estación base. Sin embargo, se utilizaron distintos protocolos para la comunicación de los sensores con el microcontrolador y a su vez para comunicar la estación base con el receptor de datos.
	
Para el envío de datos a la estación base, preferiblemente el protocolo debía ser capaz de funcionar en rangos de distancia amplios, permitiendo que el sensor inteligente esté ubicado lejos de la estación base en donde serán monitoreadas las variables de interés. En primer lugar se escogieron algunos protocolos de forma preliminar, para luego estudiar a fondo sus características. Estos protocolos y sus características se resumen en la tabla \ref{tab:protocolos}:

\begin{table}[H]
    \centering
    \caption{Comparación entre protocolos de comunicación inalámbrica, \citep{IoTCompare} y \citep{LPWANCompare}}
    \label{tab:protocolos}
    \resizebox{\textwidth}{!}{%
    \begin{tabular}{|c|c|c|c|c|}
    \hline
    \textbf{Protocolo} & \textbf{Frecuencia} & \textbf{Rango} & \textbf{Velocidad} & \multicolumn{1}{l|}{\textbf{Consumo de energía}} \\ \hline
    \textbf{Zigbee} & 784 MHz/2.4 GHz & 100 m - 300 m & 250kbps-500kbps & Bajo \\ \hline
    \textbf{Sigfox} & 868 MHz/915 MHz & 3km - 10km & 100 bps & Bajo \\ \hline
    \textbf{NB-IoT} & LTE & 1km - 10km & 200 kbps & Bajo \\ \hline
    \textbf{WiFi} & 2.4 GHz/5.8 GHz & 100m & 54Mbps/1.3Gbps & Alto \\ \hline
    \textbf{Bluetooth} & 2.4 GHz & 10m - 100m & 720 kbps & Bajo \\ \hline
    \textbf{LoRa} & \begin{tabular}[c]{@{}c@{}}430 MHz/433 MHz\\ /868 MHz/915 MHz\end{tabular} & 15 km-30 km & 0.3kbs hasta 50 kbps & Bajo \\ \hline
    \end{tabular}%
    }
\end{table}

Con base en esta información, se escogió el protocolo LoRa como el más adecuado para el sensor inteligente, debido a su amplio rango y bajo consumo de energía. En general, la vasta mayoría de los módulos están basados en los chips fabricados por Semtech (los precursores del protocolo LoRa) SX126X y SX127X, por tanto se compararon ambas tecnologías:

% Please add the following required packages to your document preamble:
% \usepackage{graphicx}
\begin{table}[H]
    \centering
    \caption{Comparación entre módulos LoRa del fabricante Semtech \citep{datasheetSemtech}.}
    \label{tab:moduloslora}
    \resizebox{\textwidth}{!}{%
    \begin{tabular}{|c|c|c|c|c|c|}
    \hline
    \textbf{Módulo} & \textbf{Modem} & \textbf{Amplificador} & \textbf{Corriente RX} & \multicolumn{1}{l|}{\textbf{Sensibilidad}} & \textbf{Velocidad (bit rate)} \\ \hline
    \textbf{SX1261/62/68} & LoRa y FSK & \begin{tabular}[c]{@{}c@{}}+15 dBm - \\ +22 dBm\end{tabular} & 4.6 mA & -148 dBm & \begin{tabular}[c]{@{}c@{}}62.5 kbps \\ - 300 kbps\end{tabular} \\ \hline
    \textbf{S1272/73} & LoRa & +14 dBm & 10 mA & -137 dBm & 300 kbps \\ \hline
    \textbf{S1276/77/78/79} & LoRa & +14 dBm & 9.9 mA & -148 dBm & 300 kbps \\ \hline
    \end{tabular}%
    }
    \end{table}

\subsection{Sensores}

Para la selección de los sensores a utilizarse, es preciso definir las necesidades de un sistema de adquisición para sistemas estructurales, siendo su comportamiento el que define las características de los instrumentos de medición.

\subsubsection{Aceleración} 

En el caso de la medición de aceleración, el sensor inteligente debe contar con un sensor con las siguientes características:

\begin{itemize}
    \item Bajo nivel de ruido.
    \item Compensación de temperatura.
    \item Ancho de banda dentro del rango deseado en sistemas estructurales.
    \item Buena resolución.
    \item Suficientes grados de libertad.
    \item Compatibilidad con microcontroladores disponibles en el mercado.
    \item Bajo consumo
\end{itemize}

\subsubsection{Temperatura y humedad}

Para la medición de temperatura y humedad, el sensor inteligente debe contar con un sensor con las siguientes características:

\begin{itemize}
    \item Rango de trabajo dentro de las condiciones en las que se encuentre la estructura.
    \item Buena resolución y sensibilidad.
    \item Compatibilidad con microcontroladores.
    \item Bajo consumo.
\end{itemize}

\subsubsection{Inclinación}

Para la medición de inclinación, la cual, como se explica en la sección \ref{subsec:sensorfusion} el sensor inteligente debe contar con un sensor que cuente con las siguientes características:

\begin{itemize}
    \item Acelerómetro, giróscopo y magnetómetro incorporado (\textit{Inertial Measurement Unit}).
    \item Bajo nivel de ruido.
    \item Buena resolución y sensibilidad.
    \item Compatibilidad con microcontroladores.
    \item Bajo consumo.
\end{itemize}

\subsection{Microcontroladores}

En el caso de los microcontroladores, se buscaron microcontroladores capaces de obtener los datos provenientes de los sensores, procesarlos, almacenarlos temporalmente y posteriormente hacer uso del módulo de comunicaciones para su envío, usando este mismo módulo para recibir mensajes o comandos. También se tomaron en cuenta las capacidades de conexión inalámbrica de cada microcontrolador, su documentación y soporte por parte de los fabricantes, y por último su compatibilidad con los distintos frameworks, librerías y entornos de programación disponibles para los sensores y módulos, los cuales disminuyen el tiempo necesario para poner en marcha el funcionamiento del sistema. Se estudiaron las características de distintas placas de desarrollo, para posteriormente escoger el más adecuado. A continuación, en la tabla \ref{tab:microstabla}, se presentan las placas de desarrollo consideradas de forma preliminar y sus características principales:

% Please add the following required packages to your document preamble:
% \usepackage{graphicx}
\begin{table}[H]
    \centering
    \caption{Comparación entre placas de desarrollo basadas en MCU}
    \label{tab:microstabla}
    \resizebox{\textwidth}{!}{%
    \begin{tabular}{|c|c|c|c|c|c|c|}
    \hline
    \textbf{Placa} & \textbf{Procesador} & \textbf{Velocidad de reloj} & \textbf{RAM (kB)} & \multicolumn{1}{l|}{\textbf{ROM (kB)}} & \textbf{GPIO} & \textbf{Conectividad} \\ \hline
    \textbf{Teensy 4.0} & ARM M7 & 600 MHz & 1024 & 2048 & 40 & - \\ \hline
    \textbf{\begin{tabular}[c]{@{}c@{}}Raspberry Pi \\ Pico W\end{tabular}} & Dual ARM-M0 & 133 MHz & 264 & 2048 & 26 & WiFi \\ \hline
    \textbf{STM32 Discovery} & ARM M4 & 168 MHz & 192 & 1024 & 82 & - \\ \hline
    \textbf{STM32 Nucleo} & \begin{tabular}[c]{@{}c@{}}ARM M0 -\\ ARM M4\end{tabular} & \begin{tabular}[c]{@{}c@{}}84 MHz -\\ 180 MHz\end{tabular} & \begin{tabular}[c]{@{}c@{}}96 - \\ 128\end{tabular} & 512 & 50 & - \\ \hline
    \textbf{Espressif ESP32} & Dual Xtensa LX6 & 240 MHz & 520 & 4096 & 34 & WiFi/BT(BLE) \\ \hline
    \textbf{STM32 Blackpill} & ARM M4 & 100 MHz & 128 & 512 & 37 & - \\ \hline
    \end{tabular}%
    }
    \end{table}

\subsection{Diagramas de selección de componentes:}
\subsubsection{Selección del sensor de aceleración}

Con base en la información recopilada de distintos módulos de acelerómetros, se observa en la figura \ref{fig:arañaacl} que el MPU6050 de Invensense se ajusta a las necesidades del acelerómetro necesario para tomar los registros de vibración. Otro módulo del mismo fabricante, el MPU9250 también presenta un buen desempeño. Sin embargo, el MPU6050 tiene un mejor precio y ofrece funcionalidades similares, por lo cual fue el escogido para el prototipo de pruebas.


\begin{figure}[H]
    \centering
    \includegraphics[width = 0.7\textwidth]{imagenes/cap2_marcometod/ArañaACL.png}
    \caption{Diagrama de araña para selección de acelerómetro.}
    \label{fig:arañaacl}
\end{figure}

\subsubsection{Selección del sensor de temperatura y humedad}

Se observa en la figura \ref{fig:arañatemphum} que la mayoría de los sensores de temperatura y humedad, los cuales suelen estar integrados en un mismo módulo, no distan mucho en desempeño entre sí, sin embargo, entre ellos destaca el BME280 del reconocido fabricante Bosch, el cual cuenta con buena resolución además de una excelente documentación y librerías para distintos microcontroladores. El módulo SHT31 muesta potencial por su resolución, en este caso se descarta por la poca disponibilidad del módulo pero es una buena opción para futuras implementaciones. Es por esta razón que se escogió el BME280 para llevar a cabo las mediciones de las variables ambientales en el prototipo de pruebas.

\begin{figure}[H]
    \centering
    \includegraphics[width = 0.7\textwidth]{imagenes/cap2_marcometod/ArañaTempHum.png}
    \caption{Diagrama de araña para selección de sensor de temperatura y humedad.}
    \label{fig:arañatemphum}
\end{figure}

\subsubsection{Selección del acelerómetro para estimación de ángulos}

Utilizando el mismo análisis que se llevó a cabo en la figura \ref{fig:arañaacl}, se modifica para fines de estimación de ángulo tomando en cuenta las premisas de la sección \ref{subsec:sensorfusion} para la fusión de sensores. Por tanto, con base en la figura \ref{fig:arañaimu} se escoje el módulo MPU9250, el cual cumple con la función de ser una IMU de 9 grados de libertas, siendo ideal para la estimación de ángulos en el prototipo.

\begin{figure}[H]
    \centering
    \includegraphics[width = 0.7\textwidth]{imagenes/cap2_marcometod/ArañaIMU.png}
    \caption{Diagrama de araña para selección de unidad de medición inercial.}
    \label{fig:arañaimu}
\end{figure}

\subsubsection{Selección del módulo de comunicaciones}

Se ubicaron módulos de comunicaciones LoRa que fueran compatibles con el microcontrolador escogido, con documentación disponible y cuyas características se ajustaran a las necesidades del proyecto a llevar a cabo. Si bien el protocolo es el que condiciona las características de la gran mayoría de los módulos de comunicación del protocolo en cuestión, se buscó un módulo con facilidad de conexión e intercomunicación con el microcontrolador. Se observa en la figura \ref{fig:arañacomm} que la mayoría de los módulos tienen un desempeño similar, esto se debe a que están basados en distintas versiones de los módulos estudiados en la tabla \ref{tab:moduloslora}. Sin embargo, el condicionante es la disponibilidad y precio de los mismos, siendo el RA-02 de Ai-Thinker el seleccionado en este caso.

\begin{figure}[H]
    \centering
    \includegraphics[width = 0.7\textwidth]{imagenes/cap2_marcometod/ArañaTransmisores.png}
    \caption{Diagrama de araña para selección del módulo de comunicaciones.}
    \label{fig:arañacomm}
\end{figure}

\section{Detalle del diseño}

\subsection{Descripción del hardware}

\subsubsection{Sensor inteligente:}

\subsubsection{Estación base:}

\subsection{Descripción del software}

\subsubsection{Sensor inteligente:}

\subsubsection{Estación base:}

Flujo de NodeRED de parseo de datos

\subsubsection{Aplicación de monitoreo y control:}

\subsection{Diagrama de bloques del sistema}

Con drawio

\subsection{Diagrama de funcionamiento del sistema}

UML con drawio%

%==================================================================
\chapter{PRUEBAS Y RESULTADOS}\label{CAP:pruebas}
%\markboth{Tu Segundo Capítulo}{Tu Segundo Capítulo}%

En el siguiente capítulo se presentan las pruebas y los resultados obtenidos a partir de la metodología descrita en el capítulo anterior. Se presentan las pruebas preliminares de comunicaciones para determinar las características óptimas del canal, las pruebas para la estimación de la inclinación que permitieron escoger el método de estimación apropiado y finalmente las pruebas de funcionamiento del prototipo.

\section{Pruebas de comunicaciones}



Una vez escogido el hardware propuesto en la sección \ref{sec:componentes}, se llevaron a cabo una serie de pruebas  para comprobar el funcionamiento de los módulos de comunicaciones y para evaluar las caracacterísticas más adecuadas para el canal de comunicaciones. 

Como se definió en el apartado \ref{sec:protocololora}, el protocolo LoRa implementado en el módulo Ra-02 de Ai-Thinker requiere fijar el valor de los siguientes parámetros:

\begin{itemize}
    \item Factor de propagación.
    \item Ancho de banda.
    \item Potencia.
    \item Tasa de codificación.
\end{itemize}

La prueba consistió en el envío de un paquete de bytes a una distancia de aproximadamente 115 metros, sin línea de vista y con obstáculos de acero y concreto, como se observa en el mapa de la figura \ref{fig:mapalora}.

\begin{figure}[H]
    \centering
    \includegraphics[width = 0.9\textwidth]{imagenes/cap3_resultados/Pruebas LoRa/MapaLora.png}
    \caption{Vista en mapa de distancia máxima de pruebas usando módulo SX1278.}
    \label{fig:mapalora}
\end{figure}

Se modificaron los parámetros del canal y se midió la tasa de paquetes con errores o paquetes corruptos en el receptor. Se realizaron pruebas con los siguientes parámetros fijos:

\begin{itemize}
    \item Frecuencia de operación: 433 MHz.
    \item Tasa de codificación:
    \item Potencia: 10 dBm.
    \item Longitud del preámbulo: 8 bytes.
    \item Tamaño de la carga útil (payload): 128 bytes.
\end{itemize}

Las pruebas se realizaron haciendo uso del módulo Ra-02 de Ai-Thinker, el cual se conectó a un microcontrolador ESP32. En el microcontrolador se implementaron rutinas de envío y recepción de datos medianto rutinas de interrupción (ISR por sus siglas en inglés).

Estos parámetros son recomendados por el fabricante del módulo y se mantuvieron constantes durante las pruebas. Estos se confirmaron tras pruebas preliminares en las cuales se observó que para \textit{payloads} mayores a 128 bytes el porcentaje de paquetes cuyo CRC era erróneo (data corrupta) aumentaba considerablemente, siendo 128 bytes un valor que disminuía esta proporción. Los parámetros que se modificaron fueron el factor de propagación, el ancho de banda y el período de envío entre paquetes. Los resultados de las pruebas se presentan en la tabla \ref{tab:resultadoslora}.

% Please add the following required packages to your document preamble:
% \usepackage{graphicx}
\begin{table}[H]
    \centering
    \caption{Resultados de pruebas realizadas con módulo de comunicaciones Ra-02.}
    \label{tab:resultadoslora}
    \resizebox{\textwidth}{!}{%
    \begin{tabular}{|c|c|c|c|c|}
    \hline
    \textbf{Configuración} & \textbf{F. de Propagación} & \textbf{Ancho de banda} & \textbf{Período} & \textbf{Tasa de paquetes perdidos} \\ \hline
    1 & 9 & 125 kHz & 500 ms & 105/500 \\ \hline
    2 & \textbf{7} & 250 kHz & 150 ms & 1/500 \\ \hline
    3 & \textbf{8} & 125 kHz & 150 ms & 300/500 \\ \hline
    4 & \textbf{8} & 250 kHz & 500 ms & 2/500 \\ \hline
    5 & \textbf{8} & 250 kHz & 200 ms & Error de CRC \\ \hline
    6 & \textbf{7} & 250 kHz & 250 ms & 2/500 \\ \hline
    7 & \textbf{7} & 250 kHz & 200 ms & 2/500 \\ \hline
    8 & \textbf{7} & 250 kHz & 150 ms & 2/500 \\ \hline
    9 & \textbf{7} & 250 kHz & 100 ms & Error de CRC \\ \hline
    \end{tabular}%
    }
\end{table}

Basados en estos resultados, se escogió la configuración 8 para las pruebas de comunicaciones, ya que es la que presenta la menor tasa de paquetes corruptos a la velocidad más alta de envío de paquetes. Esta configuración se utilizó para las pruebas de estimación de inclinación y para las pruebas de funcionamiento del prototipo.

\section{Pruebas para estimación de inclinación}

Para escoger el método de estimación de ángulos, se compararon los siguientes:

\begin{itemize}
    \item Cálculo trigonométrico a partir de mediciones de acelerómetro.
    \item Filtro de Kalman.
    \item Filtro de Madgwick.
\end{itemize}

 Para observar el comportamiento de cada método, se observaron los resultados en un graficador de datos seriales disponible de forma libre, llamado \textit{TelePlot}, creado por Alexander Brehmer. Este funciona como extensión a Visual Studio Code y permite graficar los datos provenientes del puerto serial escogido a una velocidad de transmisión fija. Se realizaron pruebas con cada método y se compararon los resultados obtenidos. 

 \begin{figure}[H]
    \centering
    \includegraphics[width = 0.8\textwidth]{imagenes/cap3_resultados/Pruebas ACL/Inclinacion/Comparacion entre Metodo1 (ACL) y Metodo 2 (Kalman) ante vibraciones.png}
    \caption{Entorno TelePlot.}
    \label{fig:teleplot}
\end{figure}

Se realizaron pruebas ante vibración ambiental, ante vibraciones forzadas como golpes y ante movimientos sostenidos para ver el cambio en el ángulo ejecutando los 3 algoritmos al mismo tiempo y haciendo uso del acelerómetro de 9 grados de libertad MPU9250 de Invensense. 

Los resultados gráficos de las pruebas ante vibración ambiental se presentan en la figura \ref{fig:pruebasinclinacion}.

 \begin{figure}[H]
    \centering
    \includegraphics[width = 0.8\textwidth]{imagenes/cap3_resultados/Pruebas ACL/Inclinacion/PruebaInclinacion.png}
    \caption{Comparación entre métodos de estimación de inclinación en vibración ambiental.}
    \label{fig:pruebasinclinacion}
\end{figure}

Al golpear y soplar cerca del sensor, se obtuvieron los siguientes resultados:

\begin{figure}[H]
    \centering
    \includegraphics[width = 0.9\textwidth]{imagenes/cap3_resultados/Pruebas ACL/Inclinacion/Comparacion M1 M2 M3 (Maggwick) ante vibraciones.png}
    \caption{Comparación entre métodos de estimación de inclinación ante vibraciones.}
    \label{fig:pruebasinclinacion2}
\end{figure}


Por último, se observó el desempeño de los algoritmos ante movimientos sostenidos que cambiaran el ángulo en el eje de interés:

\begin{figure}[H]
    \centering
    \includegraphics[width = 0.8\textwidth]{imagenes/cap3_resultados/Pruebas ACL/Inclinacion/Comparacion M1 M2 M3 (Maggwick) ante movimiento aleatorios.png}
    \caption{Comparación entre métodos de estimación de inclinación ante movimientos sostenidos.}
    \label{fig:pruebasinclinacion3}
\end{figure}

Se observa claramente en las figuras \ref{fig:pruebasinclinacion}, \ref{fig:pruebasinclinacion2} y \ref{fig:pruebasinclinacion3} que el filtro de Madgwick es el que presenta el mejor desempeño ante vibraciones, golpes y movimientos sostenidos, ya que suaviza las variaciones bruscas en el ángulo. De igual forma, ante vibración ambiental se observa que el filtro de Madgwick es el que presenta un comportamiento más estable, viéndose incluso la resolución en términos de LSB (\textit{Least Significant Bit}) del acelerómetro.

Por lo tanto, se escogió el filtro de Madgwick para la estimación de inclinación en el prototipo.

\section{Pruebas de funcionamiento del prototipo}

Para probar el funcionamiento del prototipo y comparar los resultados obtenidos con un equipo comercial, se llevó a cabo un estudio de vibración sobre una estructura de acero ubicada en el Instituto de Materiales y Modelos Estructurales (IMME) de la Universidad Central de Venezuela. 

La estructura es de tipo aporticada, con una altura de 2,20 metros y una longitud de 2 metros. Cuenta con 4 columnas de acero y distintas vigas con perfil "C" o de canal. El techo de la estructura se encuentra en voladizo, es decir, no está soportado por ninguna columna. La estructura se encuentra en el edificio norte del IMME y se suele utilizar para hacer ensayos de permeabilidad del concreto. Se escogió esta estructura debido a la facilidad para la instrumentación de la misma para realizar las pruebas, siendo esta de poca altura y estando ubicada en un espacio abierto y de fácil acceso por el personal del IMME.

El equipo de medición utilizado para la comparación está basado en la tarjeta de adquisición de datos de \textit{National Instruments} PCI-6221, cuyas características se describen en la tabla \ref{tab:specs6221}, y que puede observarse en la figura  en conjunto con 

%ESPECIFICACIONES DEL DAQ6221

\begin{figure}[H]
    \centering
    \includegraphics[width = 0.8\textwidth]{imagenes/cap3_resultados/Ensayos/NationalInstruments_PCI6221.jpg}
    \caption{Tarjeta de adquisición de datos PCI-6221 de National Instruments.}
    \label{fig:DAQ6221}
\end{figure}

\begin{figure}[H]
    \centering
    \includegraphics[width = 0.8\textwidth]{imagenes/cap3_resultados/Ensayos/National_Instruments_SCB_68.jpg}
    \caption{Tarjeta de conexiones SCB-68 de National Instruments (Artisan Technology).}
    \label{fig:SCB68}
\end{figure}

% Please add the following required packages to your document preamble:
% \usepackage{graphicx}
\begin{table}[H]
    \centering
    \caption{Especificaciones de la tarjeta de adquisición PCI-6221 de National Instruments}
    \label{tab:specs6221}
    \resizebox{\textwidth}{!}{%
    \begin{tabular}{|c|c|}
    \hline
    \textbf{Parámetro} & \textbf{Valor} \\ \hline
    Número de canales & 8 diferenciales o 16 de un solo canal \\ \hline
    Resolución del ADC & 16 bits \\ \hline
    Tasa de muestreo & 250 kS/s \\ \hline
    Rango de entrada & $\pm 0.2 V, \pm 1 V, \pm 5 V, \pm 10 V$ \\ \hline
    CMRR (Rechazo del modo común) & 92dB \\ \hline
    Tamaño del FIFO de entrada & 4095 muestras \\ \hline
    Exactitud estándar (100 muestras, $\Delta T = 10  C^\circ $) & $3100 \mu V$ \\ \hline
    \end{tabular}%
    }
\end{table}

Los sensores utilizados para el ensayo fueron acelerómetros de balance de fuerza de 1 eje modelo \textit{FBA-11} de la marca \textit{Kinemetrics}. Las características de estos sensores se describen en la tabla \ref{tab:specsFBA11} y puede observarse en la figura \ref{fig:FBA11}. El funcionamiento de este tipo de acelerómetros se describe en detalle en la sección \ref{subsec:sensmonitoreo}.

%ESPECIFICACIONES DEL FBA11

\begin{figure}[H]
    \centering
    \includegraphics[width = 0.8\textwidth]{imagenes/cap3_resultados/Ensayos/FBA11.jpg}
    \caption{Acelerómetro FBA-11 de Kinemetrics.}
    \label{fig:FBA11}
\end{figure}

% Please add the following required packages to your document preamble:
% \usepackage{graphicx}
\begin{table}[H]
    \centering
    \caption{Especificaciones del acelerómetro FBA-11 de Kinemetrics}
    \label{tab:specsFBA11}
    \resizebox{\textwidth}{!}{%
    \begin{tabular}{|c|c|}
    \hline
    \textbf{Parámetro} & \textbf{Valor} \\ \hline
    Rango de escala completa & $\pm 1.0 g$ (.1, .25, .5 y 2 g opcionales) \\ \hline
    Frecuencia Natural & 50 Hz \\ \hline
    Amortiguamiento & 70\% \\ \hline
    Salida & $\pm 2.5 V / 1 g$ \\ \hline
    Linealidad & Menos de 1\% \\ \hline
    Ruido (entre 0 - 50 Hz) & $\pm 2.5 \mu V$ \\ \hline
    Rango dinámico & 130dB de 0.01 a 50Hz \\ \hline
    Alimentación & $\pm 12 Vdc$ (2.5 mA por eje) \\ \hline
    \end{tabular}%
    }
    \end{table}

Este equipo es el que se utiliza comúnmente en el IMME para realizar ensayos de vibración en estructuras.

El programa para la adquisición de los datos fue diseñado en LabVIEW 8.20 (Versión 2006), el cual es un software de programación gráfica desarrollado por National Instruments. Este programa se encarga de adquirir los datos de los sensores, procesarlos y graficarlos en tiempo real.

Para fijar los sensores se utilizó yeso, como es común en los ensayos de vibración para acoplar los sensores a la estructura.

Los ensayos se dividieron en 3 partes:
\begin{itemize}
    \item Vibración ambiental.
    \item Vibración forzada por impacto.
    \item Vibración libre por condición inicial.
\end{itemize}

La configuración utilizada se puede observar en las figuras \ref{fig:configuracionensayofrontal} y \ref{fig:configuracionensayoplanta} .

\begin{figure}[H]
    \centering
    \includegraphics[width = 0.8\textwidth]{imagenes/cap3_resultados/Ensayos/CONFIGURACION1.png}
    \caption{Vista frontal de la configuración utilizada en ensayo.}
    \label{fig:configuracionensayofrontal}
\end{figure}

\begin{figure}[H]
    \centering
    \includegraphics[width = 0.8\textwidth]{imagenes/cap3_resultados/Ensayos/CONFIGURACION1PLANTA.png}
    \caption{Vista de planta de la configuración utilizada en ensayo.}
    \label{fig:configuracionensayoplanta}
\end{figure}


Para comparar los resultados obtenidos con el prototipo y con el equipo comercial, se realizaron las pruebas en paralelo, es decir, se colocaron los sensores del prototipo y del equipo comercial en la estructura y se realizaron las pruebas al mismo tiempo. La instrumentación puede verse en las figuras \ref{fig:inst1} y \ref{fig:inst2}. Los resultados obtenidos se compararon en términos de la respuesta en frecuencia y la respuesta en el tiempo.


\begin{figure}[H]
    \centering
    \subfloat[Vista de la estructura instrumentada con los acelerómetros FBA-11 de Kinemetrics]{\includegraphics[width = 0.6\textwidth]{imagenes/cap3_resultados/Ensayos/InstrumentacionConf1FBA.jpg}\label{fig:inst1}}
    \hfill
    \subfloat[Vista de la estructura instrumentada incluyendo el prototipo de pruebas]{\includegraphics[width = 0.6\textwidth]{imagenes/cap3_resultados/Ensayos/InstrumentacionConf1FBASmartSensor.jpg}\label{fig:inst2}}
    \caption{Instrumentación de la estructura}
    \label{fig:inst}
\end{figure}

Durante el ensayo, el sensor inteligente estaba siendo controlado y monitoreado desde la estación base ubicada a pocos metros de la estructura. En esta estación base se encontraba el computador que , a través de una interfaz gráfica, permitía visualizar los datos en y guardarlos en un archivo de texto para su posterior análisis. El diseño de la interfaz gráfica de usuario puede observarse en las figuras \ref{fig:interfaz1} y \ref{fig:interfaz2}. La programación y características principales de esta interfaz fueron descritas en la sección \ref{subsec:softwaredesc}.

\begin{figure}[H]
    \centering
    \includegraphics[width = \textwidth]{imagenes/cap3_resultados/Ensayos/GUI1.jpg}
    \caption{Ventana 1 de la interfaz gráfica diseñada.}
    \label{fig:interfaz1}
\end{figure}

\begin{figure}[H]
    \centering
    \includegraphics[width = \textwidth]{imagenes/cap3_resultados/Ensayos/GUI2.jpg}
    \caption{Ventana 2 de la interfaz gráfica diseñada.}
    \label{fig:interfaz2}
\end{figure}

Para el procesamiento de los datos obtenidos haciendo uso de la tarjeta de adquisición de datos PCI-6221 se implementó un código en MATLAB, donde, de forma similar a la interfaz de la figura \ref{fig:interfaz1}, se grafican los datos en tiempo real y el espectro en frecuencia correspondiente.

Los datos se importaron directamente del archivo \textit{.lvm} generado por LabView y fueron preprocesados con la interfaz de MATLAB para importar datos de archivos externos, como se puede observar en la figura \ref{fig:datosmatlab}.

\begin{figure}[H]
    \centering
    \includegraphics[width = \textwidth]{imagenes/cap3_resultados/Ensayos/datosmatlab.png}
    \caption{Ventana de la interfaz gráfica de MATLAB para importar los datos obtenidos mediante LabVIEW.}
    \label{fig:datosmatlab}
\end{figure}

Las características de instrumentación para ambos sistemas fueron las mostradas en la tabla \ref{tab:comparacionsist}:
% Please add the following required packages to your document preamble:
% \usepackage{graphicx}
\begin{table}[H]
    \centering
    \caption{Comparación entre características de los sistemas de medición utilizados.}
    \label{tab:comparacionsist}
    \resizebox{\textwidth}{!}{%
    \begin{tabular}{|c|c|c|}
    \hline
    \textbf{Parámetro} & \textbf{Sistema basado en PCI6221} & \multicolumn{1}{l|}{\textbf{Sensor inteligente}} \\ \hline
    Frecuencia de muestreo & 200 Hz & 200 Hz \\ \hline
    Número de muestras máx. & 12000 & 1024 \\ \hline
    Alimentación & 120 Vac & 5 Vdc \\ \hline
    Peso aproximado & 25-30 kg & 800 g \\ \hline
    \begin{tabular}[c]{@{}c@{}}Dimensiones aproximadas\\ \end{tabular} & 1,5 m (solo estación base) & 25 cm \\ \hline
    Cableado & Sí & No \\ \hline
    Alerta ante eventos & No & Sí \\ \hline
    \end{tabular}%
    }
    \end{table}

A continuación se presentan y comparan los resultados obtenidos en cada ensayo:

\subsection{Vibración forzada por impacto}

Este ensayo consistió en excitar la estructura haciendo uso de un martillo de goma, haciendo que la misma comience a vibrar. Para este ensayo se contó con la ayuda de personal del IMME para coordinar la toma de datos con el impacto sobre el sistema. Además de buscar la semejanza entre ambos resultados, se buscaba comprobar el efecto que produce un cambio en la masa sobre la frecuencia natural del sistema. Para esto, se añadieron 5 lastres de 17 kg al sistema luego del primer ensayo de vibración por impacto.

En primer lugar, se observa en las figuras \ref{fig:DAQHammer} la respuesta en tiempo y en frecuencia usando el sistema basado en la tarjeta PCI-6221:

\begin{figure}[H]
    \centering
    \subfloat[Aceleración en el tiempo del sistema ante vibración por impacto en dirección Este-Oeste]{\includegraphics[width = \textwidth]{imagenes/cap3_resultados/Ensayos/VibHammer6EsteOesteNIDAQ.jpg}\label{fig:DAQham1}}
    \hfill
    \subfloat[Espectro en frecuencia del sistema ante vibración por impacto en dirección Este-Oeste]{\includegraphics[width = \textwidth]{imagenes/cap3_resultados/Ensayos/VibHammer6EsteOesteEspectroNIDAQ.jpg}\label{fig:DAQham2}}
    \caption{Respuesta del sistema ante vibración por impacto según la tarjeta PCI-6221 de National Instruments}
    \label{fig:DAQHammer}
\end{figure}

El registro obtenido para este mismo impacto haciendo uso del sensor inteligente se puede observar en la figura \ref{fig:impactoGUI}. En la figura \ref{fig:ventana2} se incluye la ventana auxiliar de la interfaz gráfica que permite al operador verificar la inclinación del sensor y si este se encuentra a nivel, así como identificar el ángulo en desnivel para su corrección. También se incluye la gráfica de la densidad espectral de potencia, que en ocasiones permite caracterizar de forma más rápida las frecuencias de vibración, sobre todo en espectros que contienen múltiples frecuencias. La densidad espectral de potencia también es útil para aplicar el método del ancho de banda local, utilizado para estimar el amortiguamiento del sistema.

\begin{figure}[H]
    \centering
    \includegraphics[width = \textwidth]{imagenes/cap3_resultados/Ensayos/VibHammer6EsteOesteSMARTSENSOR.jpg}
    \caption{Registro de vibración por impacto en la dirección Este-Oeste obtenido mediante el sensor inteligente.}
    \label{fig:impactoGUI}
\end{figure}

\begin{figure}[H]
    \centering
    \includegraphics[width = \textwidth]{imagenes/cap3_resultados/Ensayos/VibHammer1NorteSurSMARTSENSORVentana2.jpg}
    \caption{Ventana auxiliar de la GUI para observar la inclinación y la densidad espectral de potencia.}
    \label{fig:ventana2}
\end{figure}

Al comparar los espectros en frecuencia obtenidos en las figuras \ref{fig:DAQham2} y \ref{fig:impactoGUI} se observa que las frecuencias de vibración obtenidas se corresponden entre ambos sistemas de medición, siendo el error entre ambos picos en frecuencia de 0.1 Hz.

Por otro lado, se impactó el sistema sin los lastres en la dirección Norte-Sur, obteniéndose la respuesta observada en la figura \ref{fig:DAQHammerNS} según el sistema basado en la tarjeta PCI-6221:

\begin{figure}[H]
    \centering
    \subfloat[Aceleración en el tiempo del sistema ante vibración por impacto en dirección Norte-Sur]{\includegraphics[width = \textwidth]{imagenes/cap3_resultados/Ensayos/VibHammer1NorteSurNIDAQ.jpg}\label{fig:DAQham1NS}}
    \hfill
    \subfloat[Espectro en frecuencia del sistema ante vibración por impacto en dirección Norte-Sur]{\includegraphics[width = \textwidth]{imagenes/cap3_resultados/Ensayos/VibHammer1NorteSurEspectroNIDAQ.jpg}\label{fig:DAQham2NS}}
    \caption{Respuesta del sistema ante vibración por impacto según la tarjeta PCI-6221 de National Instruments}
    \label{fig:DAQHammerNS}
\end{figure}

Por su parte, el sensor inteligente obtuvo la respuesta del sistema observada en la figura y \ref{fig:impactoGUI_NS}:

\begin{figure}[H]
    \centering
    \includegraphics[width = \textwidth]{imagenes/cap3_resultados/Ensayos/VibHammer1NorteSurSMARTSENSOR.jpg}
    \caption{Registro de vibración por impacto en la dirección Norte-Sur obtenido mediante el sensor inteligente.}
    \label{fig:impactoGUI_NS}
\end{figure}

En primer lugar, se observa la gran similitud entre ambas respuestas, siendo la frecuencia de vibración en la dirección larga de 4.6 Hz en ambos sistemas, con un error de 0.08Hz en este caso. Además, se comprueba que el peso influye en la respuesta en frecuencia, al observarse que la frecuencia de vibración en la dirección larga (que en este caso corresponde al eje y por la configuración de los acelerómetros), disminuye al agregarse los lastres, siendo de 4.6 Hz inicialmente, y cambiando a 3.9 Hz luego de agregar los lastres.

\subsection{Vibración libre por condición inicial}

El ensayo de vibración libre consiste en aplicar una condición inicial a la estructura para posteriormente eliminar esta condición, que puede ser un peso o fuerza aplicada, y permitir que la estructura vibre libremente hasta alcanzar su condición de reposo o de vibración ambiental.

En este caso, se realizaron pruebas aplicando una condición inicial en el sentido norte-sur a la estructura y luego se dejó vibrar libremente al retirar la condición inicial. Análogo al ensayo anterior, se hizo el estudio con y sin lastres para evaluar el efecto del cambio en la respuesta dinámica del sistema al variar la masa del sistema.

En las figuras \ref{fig:DAQlibre1SL} y \ref{fig:DAQlibre1SL} se observa la respuesta del sistema en vibración libre antes de la colocación de los 5 lastres de 17 kg:

\begin{figure}[H]
    \centering
    \subfloat[Aceleración en el tiempo del sistema ante vibración libre sin lastres]{\includegraphics[width = \textwidth]{imagenes/cap3_resultados/Ensayos/AmpVibLibreSinLastresNIDAQ.jpg}\label{fig:DAQlibre1SL}}
    \hfill
    \subfloat[Espectro en frecuencia del sistema ante vibración libre sin lastres]{\includegraphics[width = \textwidth]{imagenes/cap3_resultados/Ensayos/VibLibreSinLastresNIDAQ.jpg}\label{fig:DAQlibre2SL}}
    \caption{Respuesta del sistema ante vibración libre sin lastres según la tarjeta PCI-6221 de National Instruments}
    \label{fig:DAQlibreSL}
\end{figure}

Por su parte, la respuesta obtenida por el sensor inteligente durante este ensayo se observa en la figura 

\begin{figure}[H]
    \centering
    \includegraphics[width = \textwidth]{imagenes/cap3_resultados/Ensayos/VibLibreSinLastresSMARTSENSOR.jpg}
    \caption{Registro de vibración libre sin lastres obtenido mediante el sensor inteligente.}
    \label{fig:libreGUI_SL}
\end{figure}

Al comparar los resultados obtenidos en el espectro en frecuencia usando ambos sistemas, como se ve en las figuras \ref{fig:libreGUI_SL} y \ref{fig:DAQlibreSL}, se observa la similitud entre el espectro en frecuencia del sistema basado en el sensor inteligente diseñado y el sistema comercial disponible en el IMME. En este caso, la diferencia entre la frecuencia natural del sistema en la dirección Norte-Sur obtenida por el sensor inteligente en comparación a la obtenida por la tarjeta PCI-6221 es de 0.09Hz, observándose además que el sensor inteligente mostró un comportamiento estable mientras que el sistema basado en la tarjeta de adquisición presenta problemas de deriva y picos indeseados que agregan componentes frecuenciales ajenas al sistema estructural, demostrándose así la confiabilidad del sensor inteligente.


De igual forma, se ejecutó la misma prueba luego de añadir los lastres, obteniéndose los resultados mostrados en la figura \ref{fig:DAQlibreCL}:
\begin{figure}[H]
    \centering
    \subfloat[Aceleración en el tiempo del sistema ante vibración libre con lastres]{\includegraphics[width = \textwidth]{imagenes/cap3_resultados/Ensayos/AmplitudVibLibreLastresNIDAQ1.jpg}\label{fig:DAQlibre1CL}}
    \hfill
    \subfloat[Espectro en frecuencia del sistema ante vibración libre con lastres]{\includegraphics[width = \textwidth]{imagenes/cap3_resultados/Ensayos/VibLibreLastresNIDAQ1.jpg}\label{fig:DAQlibre2CL}}
    \caption{Respuesta del sistema ante vibración libre con lastres según la tarjeta PCI-6221 de National Instruments}
    \label{fig:DAQlibreCL}
\end{figure}


Mientras que en la figura \ref{fig:libreGUI_CL} se observan los resultados arrojados por el sensor inteligente:

\begin{figure}[H]
    \centering
    \includegraphics[width = \textwidth]{imagenes/cap3_resultados/Ensayos/VibLibreLastresSMARTSENSOR.jpg}
    \caption{Registro de vibración libre con lastres obtenido mediante el sensor inteligente.}
    \label{fig:libreGUI_CL}
\end{figure}

Al observar los resultados obtenidos en las figuras \ref{fig:libreGUI_CL} y \ref{fig:DAQlibreCL}, se pueden apreciar las similitudes entre los espectros en frecuencia de ambos sistemas, donde resaltan claramente la frecuencia de vibración que corresponde a la frecuencia en la dirección larga del sistema. En el caso del sensor inteligente, se obtuvo un valor de 3.90 Hz, mientras que en el sistema basado en la tarjeta PCI-6221 el valor es de 3.87 Hz, siendo el error entre ambos sistemas de apenas 0.03 Hz. Además de confirmarse que los datos obtenidos por el sensor inteligente son confiables y permiten caracterizar el comportamiento dinámico del sistema, se observa como el valor de esta frecuencia disminuyó respecto al valor de 4.6 Hz observado en las figuras \ref{fig:DAQlibreSL} y \ref{fig:libreGUI_SL}, demostrándose la capacidad del sistema de medir cambios en la respuesta natural del sistema producto de cambios en la masa del mismo.

%AMBIENTAL
\subsection{Vibración ambiental}

Este ensayo consistió en tomar registros de datos sin perturbar la estructura. Es decir, siendo esta excitada únicamente por factores naturales como el viento o el paso de personas.

En primer lugar, se observa en las figuras \ref{fig:DAQamb1} y \ref{fig:DAQamb2} la respuesta en tiempo y el espectro en frecuencia del sistema ante vibración ambiental.

\begin{figure}[H]
    \centering
    \subfloat[Aceleración en el tiempo del sistema ante vibración ambiental]{\includegraphics[width = \textwidth]{imagenes/cap3_resultados/Ensayos/AmplitudVibAmb2NIDAQ1.jpg}\label{fig:DAQamb1}}
    \hfill
    \subfloat[Espectro en frecuencia del sistema ante vibración ambiental]{\includegraphics[width = \textwidth]{imagenes/cap3_resultados/Ensayos/VibAmb2EspectroNIDAQ1.jpg}\label{fig:DAQamb2}}
    \caption{Respuesta del sistema ante vibración ambiental según la tarjeta PCI-6221 de National Instruments}
    \label{fig:DAQAmb}
\end{figure}

En la figura \ref{fig:ambientalGUI} se observa la interfaz gráfica diseñada para el sensor inteligente al leer uno de los registros de vibración ambiental obtenido durante el ensayo:

\begin{figure}[H]
    \centering
    \includegraphics[width = \textwidth]{imagenes/cap3_resultados/Ensayos/VibAmb2SmartSensorGUI.jpg}
    \caption{Ventana de la interfaz gráfica diseñada con el registro de vibración ambiental obtenido.}
    \label{fig:ambientalGUI}
\end{figure}

Si bien los resultados obtenidos por vibración ambiental pueden verse por registro directamente en la interfaz, como se observa en la figura \ref{fig:ambientalGUI}, es conveniente concatenar varios archivos debido a la longitud de cada registro. Mientras mayor es el tamaño del registro, mayor resolución tendrá el espectro al tener más datos. Es decir, el $\Delta f$ mejora al aumentar la longitud de los registros.

Al concatenar registros es necesario aventanar cada registro, usando alguna función de aventanamiento como las definidas en la sección \ref{sec:aventanamiento}, para que evitar que se introduzcan al espectro cambios bruscos producto de la concatenación de archivos. En la figura \ref{fig:concatenados3} se observa el resultado de concatenar 3 registros de vibración ambiental que fueron tomados de forma sucesiva mientras se tomó el registro con la tarjeta PCI-6221 registrado en la figura \ref{fig:DAQamb1}. Para esto, se implementó un código en Python capaz de concatenar 3 archivos, generando un registro continuo utilizando la ventana de Hanning.

\begin{figure}[H]
    \centering
    \includegraphics[width = \textwidth]{imagenes/cap3_resultados/Ensayos/AmplitudVibAmb2CONCATENADASMARTSENSOR.jpg}
    \caption{Forma del registro concatenado tras aventanamiento de Hanning utilizando Python.}
    \label{fig:concatenados3}
\end{figure}

Una vez concatenados, se procedió a ejecutar la FFT sobre los datos, obteniéndose el espectro de la figura \ref{fig:espectroconc}:

\begin{figure}[H]
    \centering
    \includegraphics[width = \textwidth]{imagenes/cap3_resultados/Ensayos/VibAmb2EspectroCONC.jpg}
    \caption{Forma del espectro del registro concatenado.}
    \label{fig:espectroconc}
\end{figure}



Al comparar las figuras \ref{fig:espectroconc} y \ref{fig:DAQamb2} se observa que el espectro obtenido mediante el sensor inteligente no contiene los picos observados en el espectro obtenido por la tarjeta PCI-6221. Esto puede deberse al pobre acoplamiento entre el sensor inteligente y la estructura al estar este sobre una placa de prototipos, la cual a su vez se unió al sistema con yeso, evidenciándose en amplitudes bajas. Si bien el registro de la figura \ref{fig:ambientalGUI} muestra el comportamiento esperado por un registro de vibración ambiental en tiempo, para que el sensor sea capaz de captar los leves movimientos de la estructura en vibración ambiental, este debe estar acoplado al sistema, como es el caso de los acelerómetros Kinemetrics. A pesar de obtener estas discrepancias entre ambos espectros, el bajo nivel de ruido presentado por el sensor inteligente en vibraciones ambientales y la facilidad para concatenar registros debido al formato del mismo lo proyecta como un buen candidato para tomar registros de vibraciones ambientales siempre y cuando este se encuentre firmemente acoplado al sistema en estudio.


%==================================================================
\chapter{CONCLUSIONES}\label{CAP:conclu}
%\markboth{Tu Segundo Capítulo}{Tu Segundo Capítulo}%
	
	La revisión bibliográfica mostró los avances realizados por distintos autores en cuanto a la implementación de sistemas de monitoreo de variables estructurales basados en microcontroladores y haciendo uso de canales de comunicación inalámbrica, lo cual permitió concentrar y canalizar el trabajo de investigación de forma efectiva.
	
	Se logró plantear el diseño de un sensor inteligente para aplicaciones de monitoreo de salud estructural con éxito, identificando las variables de interés para este tipo de sistemas e incluso implementando un prototipo de pruebas funcional capaz de llevar a cabo ensayos similares los que se llevan a cabo en el Instituto de Materiales y Modelos Estructurales.
	
	Se diseñaron con éxito los distintos programas necesarios para obtener registros de vibración y mediciones de variables cuasi-estáticas haciendo uso de un microcontrolador. El ESP32 en conjunto con FreeRTOS, el cual permitió controlar de forma efectiva y ordenada las distintas tareas, mostraron ser herramientas capaces de manejar el preprocesamiento y adquisición de los datos de forma exitosa.
	
	Se implementó con éxito una interfaz de monitoreo y control capaz de enviar comandos de control y a la vez visualizar los registros tomados por el sensor inteligente, aplicando herramientas de procesamiento numérico con bajo costo computacional dentro de la misma aplicación, con capacidades de personalización dependiendo del cliente o proyecto y con la posibilidad de  guardar los registros en un formato compatible con la gran mayoría de programas de procesamiento.

	Se comprobó el funcionamiento del diseño propuesto haciendo uso de un prototipo de pruebas, el cual se comparó con un equipo comercial del fabricante National Instruments, obteniendo resultados muy similares, confirmándose la efectividad y veracidad de los datos obtenidos por el sensor inteligente.
	
	Al haber tomado en cuenta la fecha y hora de adquisición de los datos, y por las características del protocolo LoRa y el control que permite tener sobre los dispositivos esclavos asignándoles números de identificación única, se sentaron las bases a la posibilidad de escalar el número de sensores inteligentes de bajo costo en la estructura, aumentando la resolución espacial y permitiendo tener acceso a las gráficas del comportamiento modal de la estructura.
	
	A pesar de las limitaciones de memoria de sistemas basados en microcontroladores, el sensor inteligente cumple sus funciones de monitorear las variables estructurales de interés; permitiendo que el operador tenga acceso a un histórico de datos de la estructura, disminuyendo además los costos y la complejidad de sistemas cableados con fines similares.
	
	En este tipo de aplicaciones es esencial garantizar la integridad de los datos.	A pesar de ser LoRa un protocolo lento en comparación a otros, la seguridad que ofrece al tener muy bajo porcentaje de pérdida de paquetes y el largo alcance, que permite ubicar la estación base en una ubicación segura alejada de la estructura lo posicionan como un protocolo prometedor para aplicaciones de monitoreo de salud estructural. 
	
    Los sistemas de adquisición de datos basados en microcontroladores representan una opción confiable y de bajo costo para implementar soluciones de monitoreo estructural periódico que sustituyen a los sistemas cableados.

%==================================================================
\chapter{RECOMENDACIONES}\label{CAP:recomendaciones}
%\markboth{Tu Segundo Capítulo}{Tu Segundo Capítulo}%
 
\begin{itemize}

    \item Implementar múltiples sensores inteligentes y hacer uso de la fecha y hora exacta de registro para sincronizar los datos provenientes de los múltiples esclavos. 
    
    
    \item Para comprobar que los cambios introducidos en un sistema estructural controlado se corresponden con cambios en la respuesta dinámica del mismo, y que estos cambios son detectados por el sensor inteligente, es conveniente realizar una prueba sobre un modelo a escala de una estructura en un entorno controlado, previamente caracterizada tomando en cuenta su modelo estático y los materiales que la constituyen. Sobre esta estructura se pueden ejecutar ensayos de vibración a medida que se introducen cambios en los elementos estructurales, pudiendo así caracterizar cómo este cambio influye en la respuesta dinámica del sistema y permitiendo llevar un registro de estos cambios en el tiempo.
    
    
    \item Se sugiere ejecutar ensayos periódicos sobre una estructura a escala en donde se puedan controlar las condiciones de temperatura y humedad relativa, además del cambio inducido en el sistema. Este estudio permitiría caracterizar los cambios en la respuesta dinámica producto de las variables ambientales respecto a los que son consecuencia de cambios en la rigidez del sistema, siendo estos últimos los de mayor interés.
    
    
    \item Si bien en este trabajo de investigación se consideraron las variables de interés principales para el monitoreo de salud estructural, se recomienda evaluar la posibilidad de incorporar sensores de variables químicas y electroquímicas que pueden atentar en contra de la integridad de la estructura, como sensores de pH, potencial de corrosión, agentes gaseosos, entre otros. Por la escalabilidad del sistema, se pueden incorporar estos sensores en los sensores inteligentes, permitiendo tener un monitoreo más completo de la estructura.
    
    \item Hacer uso del Digital Motion Processing Unit de los módulos MPU9250 y MPU6050 para ver cómo se comparan respecto a los resultados obtenidos en cuanto a fusión de sensores se refiere.
 
    \item Si bien el sensor MPU6250 permitió comprobar el correcto funcionamiento del diseño planteado, se recomienda hacer uso de un sensor cuyo nivel de ruido permita obtener registros de vibración ambiental de mejor calidad. Algunos acelerómetros de tecnología similar que presentan mejor comportamiento en bajas amplitudes y a baja frecuencia son el MMA8451 del fabricante NXP, el LSM6DSOX de STMicroelectronics y el ADXL355 de Analog Devices. Todos estos cuentan con un comportamiento en ruido menor a $120 \frac{\mu g}{\sqrt{Hz}}$ en el rango de frecuencia de interés.
    
    %126 microg/Hz
    %70 mircog/Hz ST
   %25 microg/Hz AD
    
 
 \item Se sugiere implementar una solución alternativa en estación base a la tecnología MQTT, como por ejemplo HTTP. Las dificultades encontradas que limitaron el tamaño del registro surgen al intentar subir los datos usando MQTT al servidor-computador. El buffer del cliente MQTT utilizado por la librería \textit{PubSubClient} es limitado, y la misma no está diseñada para manejar desbordamiento del buffer, por lo que los datos son truncados de forma inesperada. Se sugiere implementar un manejo de excepción para este caso. 
 
%  \item Implementar la subida de paquetes MQTT uno a uno y luego reconstruir usando NodeRED para generar los arreglos individuales de aceleración.
 
 \item El módulo LoRa tiene pocas pérdidas de datos, pero su velocidad no es la mejor, sin embargo el rango permite ubicar la estación base en un lugar seguro alejado de la estructura. Intentar utilizar otra red diferente de LoRa y enviar datos mientras se están adquiriendo. Las velocidades de Lora no son las mejores para esta topología, la red de 2.4 GHz que utilizan módulos como el nRF24L01 disminuye el rango, más sin embargo, permite alcanzar velocidades de transmisión mucho más alta. Si no se debe esperar a que se tome todo el registro para empezar a enviar, se evita tener que guardar el registro completo de forma temporal. Esto sería conveniente para aplicaciones en donde se requiera una respuesta en tiempo real como ensayos de vibración en campo.
 
 \item Hacer portátil el dispositivo al evaluar las necesidades de energía del sistema para ser alimentado por baterías, posiblemente implementando alguna función de deep-sleep del microcontrolador.
 
 \item Integrar nuevos comandos en estación base como la posibilidad de calibrar a distancia, obtener solo algunos datos y no todo el registro, modificar frecuencia de muestreo y número de datos a tomar.
 
 \item Incorporar al sistema los puertos analógicos del ESP32. Permitiendo incluir sensores de temperatura, humedad, presión, entre otros, cuya interfaz sea analógica.
 
 \item Puesto que no se necesita de WiFi en el sensor inteligente, se puede prescindir de las capacidades de conectividad del microcontrolador, escogiendo uno que tenga menos posibilidades de conectividad, pero que cuente con otras fortalezas, como por ejemplo el Teensy 4.0. Teniendo 1024kb de RAM, a diferencia de los 520kb del ESP32.
 
 \item El módulo de comunicaciones LoRa RYLR896 de REYAX TECHNOLOGY ofrece funcionalidad de control por comandos AT y serial, evitando así el uso de librerías como RadioLib para controlar el módulo. Esto ahorraría espacio en memoria y facilita el código a implementar, además de hacerlo compatible con distintos microcontroladores sin soporte a esa librería.
 
 \item  Evaluar la posibilidad de utilizar variantes del ESP32 (como el ESP32-S3) más potentes que permitirían aumentar las capacidades de cómputo y almacenamiento sin la necesidad de cambiar el código, manteniendo el firmware actual en funcionamiento.
 
 \item Implementar uso de memoria SD en sensor inteligente. Por la dimensión temporal del sistema actual (no en tiempo real) el guardado en SD usando SPI no representaría retrasos en la adquisición de los datos, pues se llevaría a cabo luego de la toma de datos.
 
 \item Evaluar la posibilidad de usar la transformada de ondículas o Wavelet en vez de la FFT para darle una dimensión temporal a los registros.
 
 \item Expandir la memoria SRAM del ESP32 haciendo uso de módulos comerciales como el W25Q128 de 16 MB. Esto permitiría almacenar registros más largos, mejorando la resolución de los mismos y permitiendo llevar a cabo ensayos con constantes de tiempo mayores.
 
 \item Para mejorar la precisión en la hora de toma de datos, se recomienda implementar rutinas de iteración en la sincronización del RTC comparando la hora actual y la hora del sensor inteligente luego de la sincronización inicial hasta disminuir el error. Paralelo a esto se puede utilizar un módulo GPS en la estación base y comparar con hora de NTP adquirida.
\end{itemize}

%==================================================================

\appendix

\renewcommand \thechapter{\Roman{chapter}}
%==================================================================
\chapter{CÓDIGO DEL SENSOR INTELIGENTE}\label{CAP:anexo0}
%\markboth{Tu anexo}{Tu anexo}%
\begin{lstlisting}[language=C++, caption=Tarea de lectura de datos de aceleración triaxial]
  void leerDatosACL(void *pvParameters){
      //Leer los valores del Acelerometro de la IMU
      while(true){
          mpu.getEvent(&a, &g, &tem);
          ACLData aclData; //Estructura a ser llenada con 3 ejes
    
          aclData.AclX = a.acceleration.x - acl_offset[0];
          aclData.AclY = a.acceleration.y - acl_offset[1]; 
          aclData.AclZ = a.acceleration.z - acl_offset[2];
    
          if(F_SAMPLING != 0){
            vTaskDelay(F_SAMPLING/portTICK_PERIOD_MS);
          }
          
          //Evaluo los valores actuales de aceleracion
          if(flag_limite == 0){
            flag_limite = evaluar_limites_acl(aclData.AclX, aclData.AclY, aclData.AclZ);
            flag_time = checktime();
            if(flag_limite != 0){
              flag_acl == true;
              globalTimestamp = getEpochTime();
              //Activando banderas para registro de temperatura y humedad
              flag_inc = 1;
              flag_temp_hum = 1;
    
              //Cambio en el LED de toma de datos
              digitalWrite(LED_EST1, HIGH);
              //Suspende tarea de LED IDLE
              vTaskSuspend(xHandle_blink);
            }
            if(flag_time){
              flag_limite = 1;
              flag_inc = 1;
              flag_temp_hum = 1;
              globalTimestamp = getEpochTime();
    
              //Cambio en el LED de toma de datos
              digitalWrite(LED_EST1, HIGH);
              //Suspende tarea de LED IDLE
              vTaskSuspend(xHandle_blink);
              }
          }
    
          //solo evalua si no se sobrepaso el limite al mismo tiempo
          if(flag_acl == true && flag_limite == 0){
              //Se cambia el valor de flag limite para solo ejecutar esto 1 vez
              flag_limite = 1;
              flag_inc = 1;
              flag_temp_hum = 1;
              globalTimestamp = getEpochTime();
    
              //Cambio en el LED de toma de datos
              digitalWrite(LED_EST1, HIGH);
              //Suspende tarea de LED IDLE
              vTaskSuspend(xHandle_blink);
          }
    
          //Envia los datos a la cola solo si se supero el limite
          if(flag_limite != 0 || flag_acl == true || flag_time == 1)
          {
            if(xQueueSend(aclQueue, &aclData, portMAX_DELAY)){
              vTaskResume(xHandle_crearBuffer); //Se llenan los buffers para enviar los datos
            }
            else{
              continue;
            }
          }
          else{
            continue;
          }
      }
    }
    
\end{lstlisting}

\begin{lstlisting}[language=C++, caption=Tarea de creación de buffer para envío a módulo LoRa]

void crearBuffer(void *pvParameters){
  while(true){

    ACLData datos_acl;

    //Recibo los datos de la cola y los guardo en la estructura creada
    if(xQueueReceive(aclQueue, &datos_acl, portMAX_DELAY)){
      if( k <= NUM_DATOS ){
        if(k == 1) tiempo1 = millis();
        //Lleno el buffer de datos
        struct_buffer_acl.bufferX[k] = datos_acl.AclX;
        struct_buffer_acl.bufferY[k] = datos_acl.AclY;
        struct_buffer_acl.bufferZ[k] = datos_acl.AclZ;
        k++;
      }
      else{
        //Reiniciando para sobreescribir en buffers "nuevos" en la siguiente accion
        k = 0;
        cont2 = 0;
        cont2_inc = 0;

        //Se siguen tomando datos mas no se guardan en los buffers
        flag_limite = 0;
        flag_inc = 0;
        flag_temp_hum = 0;
        flag_time = 0;
        flag_acl = false; //Reinicio booleano de recepcion LoRa para toma de decisiones en proxima peticion

        //Apago led indicativo de toma de datos
        digitalWrite(LED_EST1, LOW);


        //Envio los resultados a la cola
        if(xQueueSend(tramaLoRaQueue, &struct_buffer_acl, portMAX_DELAY)){
           vTaskSuspend(xHandle_leerDatosACL); //suspendo adquisicion hasta que se envie todo
           vTaskSuspend(xHandle_readBMETask);
           vTaskSuspend(xHandle_readMPU9250);

          //Se envian los datos mediante lora
          vTaskResume(xHandle_send_packet);
        }
        else{
          Serial.println("No se envio la cola...");
        }

        vTaskResume(xHandle_blink);
      }
    }
    else{
      Serial.println("No se recibio la cola correctamente...");
    };

    //vPortFree(NULL);
 
    //Se suspende esta tarea para esperar la toma de datos
    vTaskSuspend(xHandle_crearBuffer);

  }
}

\end{lstlisting}

\begin{lstlisting}[language=C++, caption=Función del sensor inteligente para generar arreglos de datos de aceleración en paquetes de 128 bytes]

int generararray(int contador_paquetes)
{

    //Recibiendo cola con datos de aceleracion en estructura de datos 
    //Los datos son un float array dentro del buffer
    if(contador_paquetes == 0){
          const int chunkSize = 64;
          contador_paquetes_interno = 0; //REINICIO CONTADOR INTERNO INDEPENDIENTE DE SI ENTRO POR PRIMERA VEZ

          if(xQueueReceive(tramaLoRaQueue, &trama, portMAX_DELAY))
          {
            //Evita el error por MeditationGuru en Core0
            //Copia los datos por partes en vez de completos, evita problemas de reboot
            for (int i = 0; i < NUM_DATOS; i += chunkSize)
            {
                // Calculate the size of the current chunk
                int currentChunkSize = min(chunkSize, NUM_DATOS - i);

                // Copy and process the chunk
                memcpy(floatArrayX + i, trama.bufferX + i, currentChunkSize * sizeof(float));
                memcpy(floatArrayY + i, trama.bufferY + i, currentChunkSize * sizeof(float));
                memcpy(floatArrayZ + i, trama.bufferZ + i, currentChunkSize * sizeof(float));

                // Process the data in floatArrayX, floatArrayY, and floatArrayZ here
            }
            /*The memcpy function takes three arguments: the destination pointer, the source pointer, and the number of bytes to copy.*/
          }
          selectedFloatArray = floatArrayX;
    }
    else if(contador_paquetes == NUM_DATOS/CHUNK_SIZE) //32/64/128...
    {
          contador_paquetes_interno = 0;
          selectedFloatArray = floatArrayY;
    }
    else if(contador_paquetes == NUM_DATOS/CHUNK_SIZE*2) // (NUM_DATOS/CHUNKSIZE - 1)*2
    {
          contador_paquetes_interno = 0;
          selectedFloatArray = floatArrayZ;
    }

    // Calculate the number of chunks
    int numChunks = sizeof(floatArrayX) / sizeof(float) / CHUNK_SIZE; //El numero de chunks indica el numero de bytes final
    //Print the amount of chunks
    Serial.print("Amount of chunks calculated: ");
    Serial.println(numChunks);
    

    // Check if the size of floatArray is divisible by CHUNK_SIZE
    if (NUM_DATOS / sizeof(float) % CHUNK_SIZE != 0) {
        Serial.println("Error: Size of floatArray is not divisible by CHUNK_SIZE");
    }
    // Create an array to hold the chunks
    float chunks[numChunks][CHUNK_SIZE];

    // Split the array into chunks
    for (int i = 0; i < numChunks; i++) {
      memcpy(chunks[i], &selectedFloatArray[i * CHUNK_SIZE], CHUNK_SIZE * sizeof(float)); //Copiar 128bytes de Float Array (a partir de la posicion especificada por i) en chunks
    }

    if(xQueueSend(arrayQueue, &chunks[contador_paquetes_interno], portMAX_DELAY) == pdTRUE){
      Serial.println("Se envio la cola a la tarea de envio LoRa.");
    }
    else{
      Serial.println("Problema al enviar cola...");
      return 0;
    }

    contador_paquetes_interno++;

    return 1;
}

\end{lstlisting}

\begin{lstlisting}[language=C++, caption=Tarea de envío de datos de aceleración desde sensor inteligente]

void send_packet(void *pvParameters){
  while(1){
    transmitFlag = true;

    //detach interrupt from pin 2
    detachInterrupt(2);

    Serial.print("Contador de paquetes actual antes de entrar en generararray: ");
    Serial.println(contador_paquetes);

    // if(contador_paquetes >= ((NUM_DATOS * 4))*3 / 128){
    //   contador_paquetes = 0;
    // }

    generararray(contador_paquetes); //Genero el array y mando un chunk, dependiendo del contador

    
    if(contador_paquetes < (((NUM_DATOS * 4))*3 / 128)){
         contador_paquetes++; //Aumento el contador para enviar el siguiente paquete en el proximo envio
    }

     float floatarray[CHUNK_SIZE];
     byte data[CHUNK_SIZE * sizeof(float)]; //Inicializacion de byte array

     if(xQueueReceive(arrayQueue, &floatarray, portMAX_DELAY)){
        Serial.println("Se recibio la cola con los datos");
     }
    
     memcpy(data, floatarray, sizeof(floatarray)); //CHUNK_SIZE * sizeof(float)
    
    size_t size_data;

    size_data = sizeof(data);

    //Llamo a la funcion que crea la trama de datos (payload)
    sendmessage_radiolib(size_data, data);
    Serial.println("Sending byte array con datos!");
    Serial.println();

    //Espero X segundos luego de enviar mensaje
    vTaskDelay(interval/portTICK_PERIOD_MS);

    //Suspendo esta tarea hasta que se reciba otro mensaje
    if(contador_paquetes >= (((NUM_DATOS * 4))*3 / 128)){
      
    //Ejecuto funciones para calcular valor promedio de variables cuasiestaticas
      promediofinal_inc();
      promediofinal_temphum();
      send_thi();

      iteraciones_peticiones++;
      Serial.print("Iteraciones: ");
      Serial.println(iteraciones_peticiones);

      if(iteraciones_peticiones == 2){
        Serial.println("Reiniciando micro...");
        esp_restart();
      }

      Serial.println("Reactivando tareas de adquisicion de datos!");
      contador_paquetes = 0;
      vTaskResume(xHandle_leerDatosACL);
      vTaskResume(xHandle_readBMETask);
      vTaskResume(xHandle_readMPU9250);

        Serial.print("Memoria disponible en send task: ");
        Serial.println(ESP.getFreeHeap());
        Serial.println(ESP.getFreePsram());
        Serial.println(uxTaskGetStackHighWaterMark(NULL));

      transmitFlag = false; 
      Serial.print(F("[SX1278] Comienza a escuchar comandos de nuevo... "));
      attachInterrupt(digitalPinToInterrupt(2), setFlag, RISING);
      int state = radio.startReceive();
      if (state == RADIOLIB_ERR_NONE) {
        Serial.println(F("exito!"));
      } 
      else{
        Serial.print(F("fallo, codigo "));
        Serial.println(state);
        //while (true);
      }

      vTaskSuspend(NULL);
    }
  }  
}

\end{lstlisting}

\begin{lstlisting}[language=C++, caption=Tarea de lectura de datos de temperatura y humedad]

void readBMETask(void *parameter) {
  while (true) {
    //Crea una estructura de tipo BMEData
    BMEData data_readtemp;

    //Lee los valores del sensor y los guarda en la estructura
    data_readtemp.humidity = bme.readHumidity();
    data_readtemp.temperature = bme.readTemperature();

    //Envia los datos a la cola dataQueue si bandera esta activada
    if(flag_temp_hum){
        if(xQueueSend(data_temphumQueue, &data_readtemp, portMAX_DELAY)){
          //Serial.println("Se envio correctamente la cola de temperatura");
          vTaskDelay(10 / portTICK_PERIOD_MS);
          vTaskResume(xHandle_receive_temphum); //Reactivo tarea de creacion de buffers y promedios
        }
        else{
          Serial.println("No se envio correctamente la cola de temperatura");
        }
    }
    
    //Delay dependiendo del tiempo de muestreo
    vTaskDelay(T_SAMPLING_TEMPHUM / portTICK_PERIOD_MS); //5Hz
  }
}


\begin{lstlisting}[language=C++, caption= Tarea de lectura de datos del MPU9250]

void readMPU9250(void *pvParameters){
  while (true) {
      //Crea una estructura de tipo IncData
      IncData data_inc;

       //Lee los valores del sensor y los guarda en la estructura
       if(mpu9250.update()){
          data_inc.IncRoll = mpu9250.getRoll();
          data_inc.IncPitch = mpu9250.getPitch();
          data_inc.IncYaw = mpu9250.getYaw();

          //Envia los datos a la cola incQueue solo si la bandera esta activada
          if(flag_inc){
            if(xQueueSend(incQueue, &data_inc, portMAX_DELAY)){
            //Serial.println("Se envio correctamente la cola de inclinacion");
            vTaskResume(xHandle_recInclinacion); //Reactivo tarea de creacion de buffers y promedios
            }
            else{
              Serial.println("No se envio correctamente la cola de inclinacion");
              continue;
            }
          }
          vTaskDelayUntil(&xLastWakeTime, T_SAMPLING_INC / portTICK_PERIOD_MS)
       }
}
\end{lstlisting}

\begin{lstlisting}[language=C++, caption=Tarea de envío de datos de variables ambientales e inclinación desde sensor inteligente]

  int send_thi_radiolib(time_t tstamp, float temp, float hum, float yaw, float pitch, float roll){
  if(transmitFlag){
    Serial.print(F("[SX1278] Transmitiendo T.H.I ... "));

    THIPacket packet; //Creando paquete como estructura Packet

    //Se llena estructura de datos
    packet.messageID = 200;
    packet.senderID = 0x2;
    packet.receiverID = 0x1; 
    packet.timestamp = tstamp;
    packet.temperature = temp;
    packet.humidity = hum;
    packet.yaw = yaw;
    packet.pitch = pitch;
    packet.roll = roll;
    
    //Generacion de byte array
    byte* packetBytes = reinterpret_cast<byte*>(&packet);

    int state = radio.startTransmit(packetBytes, sizeof(packet));

    if (state == RADIOLIB_ERR_NONE) {
      Serial.println(F(" exito!"));
      return 1;

    } else if (state == RADIOLIB_ERR_PACKET_TOO_LONG) {
      Serial.println(F("muy largo!"));
      return 0;

    } else if (state == RADIOLIB_ERR_TX_TIMEOUT) {
      Serial.println(F("timeout!"));
      return 0;

    } else {
      Serial.print(F("fallo, codigo "));
      Serial.println(state);
      return 0;
    }
    return 1;
  }
  else{
      Serial.println("Se estan recibiendo datos");
      return 0;
  }  
}

\end{lstlisting}%

%==================================================================
\chapter{CÓDIGO DE LA ESTACIÓN BASE}\label{CAP:anexo1}
%\markboth{Tu anexo}{Tu anexo}%

\begin{lstlisting}[language=C++, caption=Tarea de recepción de datos de aceleración vía Lora en estación base]
    void receive_task(void *pvParameter)
    {
      while (true)
      {
        if (transmitFlag != true)
        {
          // reset flag
          receivedFlag = false;
    
          PacketUnion packetUnion;
    
          int numBytes = radio.getPacketLength();
          byte byteArr[numBytes];
    
          if (numBytes == 22)
          {
            Serial.println("Se recibio una actualizacion de SmartSensor.");
            data_o_comando = 1;
          }
          else if (numBytes == 131)
          {
            Serial.println("Se recibieron datos del SS.");
            data_o_comando = 0;
          }
          else if (numBytes == 28)
          {
            Serial.println("Se recibieron datos de temp y humedad.");
            data_o_comando = 2;
          }
          else
          {
            Serial.println("Se recibio algo que no es un comando ni una actualizacion de RTC.");
            datacorrupta = true;
          }
    
          Serial.print("Packet length: ");
          Serial.println(numBytes);
          int state = radio.readData(byteArr, numBytes);
    
          if (state == RADIOLIB_ERR_NONE || state == 0 && datacorrupta == false)
          {
            // packet was successfully received
            Serial.println(F("[SX1278] Paquete recibido!"));
    
            // Serial.print("Valor actual de datacomando> ");
            // Serial.println(data_o_comando);
    
            contador++;
    
            switch(data_o_comando)
            {
              case 0:
                  memcpy(&packetUnion.packet1, byteArr, sizeof(packetUnion.packet1));
    
                  Serial.print("[SX1278] Message ID: ");
                  Serial.println(packetUnion.packet1.messageID);
                  Serial.print("[SX1278] Sender ID: ");
                  Serial.println(packetUnion.packet1.senderID);
                  Serial.print("[SX1278] Receiver ID: ");
                  Serial.println(packetUnion.packet1.receiverID);
    
                  //Muestra payload
                  Serial.print(F("[SX1278] Payload:\t\t"));
                  //print the byte array
                  for (int i = 0; i < sizeof(packetUnion.packet1.payload); i+=4) {
                    float value = *((float*)(packetUnion.packet1.payload + i));
                    Serial.print(value);
                    Serial.print(F(" "));
                  }
                  Serial.println("");
    
                  leer_datos(sizeof(packetUnion.packet1.payload), packetUnion.packet1.messageID, packetUnion.packet1.senderID, packetUnion.packet1.receiverID, packetUnion.packet1.payload);
    
                  //ERA 95
                  if(packetUnion.packet1.messageID == (((NUM_DATOS * 4))*3 / 128) - 1){
                    if(xQueueSend(xQueueBufferACL, &buffer_prueba, portMAX_DELAY)){
                      Serial.println("Se envio la estructura bufferprueba a la cola xQueueBufferACL");
                      //vTaskResume(xHandle_send_mqtt);
                    }
                    else{
                      Serial.println("No se pudo enviar la estructura bufferprueba a la cola xQueueBufferACL");
                    }
                  }
    
                  break;
    
              case 1:
                  memcpy(&packetUnion.stringPacket, byteArr, numBytes);
                  Serial.println("El SS esta activo...");
    
                  Serial.print("[SX1278] Message ID: ");
                  Serial.println(packetUnion.stringPacket.messageID);
                  Serial.print("[SX1278] Sender ID: ");
                  Serial.println(packetUnion.stringPacket.senderID);
                  Serial.print("[SX1278] Receiver ID: ");
                  Serial.println(packetUnion.stringPacket.receiverID);
    
                  if(int(packetUnion.stringPacket.messageID) == 255){
                    Serial.println("Listo para enviar RTC...");
                    transmitFlag = true; //Activo bandera de envio
                    vTaskResume(xHandle_send_RTC_task);
                  }
                  break;
              case 2:
                  memcpy(&packetUnion.thipacket, byteArr, numBytes);
    
                  Serial.print("[SX1278] Message ID: ");
                  Serial.println(packetUnion.thipacket.messageID);
                  Serial.print("[SX1278] Sender ID: ");
                  Serial.println(packetUnion.thipacket.senderID);
                  Serial.print("[SX1278] Receiver ID: ");
                  Serial.println(packetUnion.thipacket.receiverID);
    
                  Serial.print(F("[SX1278] Payload:\t\t"));
                  //Print the temperature and humidity
                  Serial.print(F("Temperature: "));
                  Serial.print(packetUnion.thipacket.temperature);
                  Serial.print(F(" Humidity: "));
                  Serial.println(packetUnion.thipacket.humidity);
                  //Print the angle values
                  Serial.print(F("Yaw: "));
                  Serial.print(packetUnion.thipacket.yaw);
                  Serial.print(F(" Pitch: "));
                  Serial.print(packetUnion.thipacket.pitch);
                  Serial.print(F(" Roll: "));
                  Serial.println(packetUnion.thipacket.roll);
    
    
                  if(packetUnion.thipacket.messageID == 200){
                    if(xQueueSend(xQueueTempHumInc, &packetUnion.thipacket, portMAX_DELAY)){
                      Serial.println(xPortGetFreeHeapSize());
                      Serial.println("Se envio la estructura THI a la cola");
                      vTaskResume(xHandle_send_mqtt_thi);
                    }
                    else{
                      Serial.println("No se pudo enviar la estructura bufferprueba a la cola xQueueBufferACL");
                    }
                  }
                  break;
            }
    
    
          }
          else if (state == RADIOLIB_ERR_CRC_MISMATCH)
          {
            //Paquete recibido pero esta corrupto
            Serial.println(F("[SX1278] CRC error!"));
    
            // EJECUTAR CASO ESPECIAL DE DATA CORRUPTA PARA AUMENTAR CURRENTPOS Y GUARDAR 0s EN BUFFER DE INTERES
            if (numBytes == 131)
            {
              fillBuffer(bufferactual, NULL, sizeof(packetUnion.packet1.payload), true);
            }
            else
            {
              Serial.println("[SX1278] La data recibida esta corrupta y no es para este receptor");
            }
    
            contador_errores++;
          }
          else if (state == -1)
          {
            // some other error occurred
            Serial.print(F("[SX1278] Fallo, codigo "));
            Serial.println(state);
            contador_errores++;
          }
          else if (datacorrupta)
          {
            // some other error occurred
            Serial.print(F("[SX1278] Data corrupta..."));
            Serial.println(state);
            datacorrupta = false; ///Reinicio bandera
            contador_errores++;
          }
    
          Serial.print(F("[SX1278] Contador: "));
          Serial.println(contador);
          Serial.print(F("[SX1278] Contador error: "));
          Serial.println(contador_errores);
    
          Serial.println("");
        }
      else
      {
        Serial.println("Se estan enviando datos...");
      }
      vTaskSuspend(NULL);
      }
    }
    
    \end{lstlisting}

    
\begin{lstlisting}[language=C++, caption=Tarea para envío de actualización de RTC desde estación base]

    void send_RTC_task(void *pvParameters)
    {
      while (1)
      {
        if (transmitFlag == true)
        {
          Serial.print(F("[SX1278] Transmisitendo actualizacion de RTC... "));
    
          detachInterrupt(2);
    
          TimePacket packet;
    
          //Estructura de tiempo
          timestruct time_packet;
    
          struct tm timeinfo = rtc.getTimeStruct();
    
          time_packet.year = timeinfo.tm_year + 1900;
          time_packet.month = timeinfo.tm_mon + 1;
          time_packet.day = timeinfo.tm_mday;
          time_packet.hour = timeinfo.tm_hour;
          time_packet.minute = timeinfo.tm_min;
          time_packet.second = timeinfo.tm_sec;
    
          //Convierte estructura a byte array
          byte *byteArrTime = (byte *)&time_packet;
    
          //Muestra el payload
          Serial.print(F("[SX1278] Payload:\t\t"));
          //print the byte array
          // Print the byte array
          for (int i = 0; i < sizeof(time_packet); i++) {
              Serial.print(byteArrTime[i], HEX);
              Serial.print(F(" "));
          }
    
    
          Serial.println("");
    
          Serial.println(sizeof(time_packet));
    
          packet.messageID = 255;
          packet.senderID = 1;                                        
          packet.receiverID = 2;                                      
          memcpy(packet.payload, byteArrTime, sizeof(packet.payload));
    
          //Convierte estructura packet a byte array incluyendo encabezado
          byte *packetBytes = reinterpret_cast<byte *>(&packet);
    
          // Delay antes de transmitir para permitir que SS se configure en modo listening
          delay(500);
    
          int state = radio.startTransmit(packetBytes, sizeof(packet));
    
          if (state == RADIOLIB_ERR_NONE)
          {
            //Paquete transmitido con exito
            Serial.println(F(" exito!"));
          }
          else if (state == RADIOLIB_ERR_PACKET_TOO_LONG)
          {
            //Paquete mayor a 256 bytes
            Serial.println(F("muy largo!"));
          }
          else if (state == RADIOLIB_ERR_TX_TIMEOUT)
          {
            //timeout
            Serial.println(F("timeout!"));
          }
          else
          {
            //Otro error
            Serial.print(F("fallo, codigo "));
            Serial.println(state);
          }
    
          // Delay para permitir que se termine de enviar el paquete, no poner en modo receptor de inmediato
          delay(500);
    
          transmitFlag = false;
    
          Serial.print(F("[SX1278] Comienza a escuchar paquetes otra vez... \n"));
    
          int state2 = radio.startReceive();
    
          attachInterrupt(digitalPinToInterrupt(2), setFlag, RISING); // Reinicia ISR
    
          if (state2 == RADIOLIB_ERR_NONE)
          {
            Serial.println(F("Exito!"));
          }
          else
          {
            Serial.print(F("fallo, codigo "));
            Serial.println(state2);
            // while (true);
          }
    
          vTaskSuspend(NULL);
        }
        else
        {
          Serial.println("Se estan recibiendo datos en este momento...");
        }
      }
    }
    
    \end{lstlisting}

\begin{lstlisting}[language=C++, caption=Tarea de envio de datos de aceleración vía MQTT]

    
//Se activa una vez se reciben todos los paquetes Lora...
void send_mqtt(void *pvParameters){
    while(true){
        
        if(xQueueReceive(xQueueBufferACL, &bufferaceleracion, portMAX_DELAY)){
            Serial.println("Recibido de la cola BufferACL");
        } else {
            Serial.println("Error recibiendo de la cola BufferACL");
        }

        delay(500);

        sendAxis("esp32/x", "x", bufferaceleracion.bufferX, ARRAY_SIZE);
        sendAxis("esp32/y", "y", bufferaceleracion.bufferY, ARRAY_SIZE);
        sendAxis("esp32/z", "z", bufferaceleracion.bufferZ, ARRAY_SIZE);

        // Wait for some time before publishing again
        vTaskResume(xHandle_keepalive_task);
        vTaskSuspend(NULL);
    }
}

\end{lstlisting}


\begin{lstlisting}[language=C++, caption=Tarea de conversión a tipo JSON de un eje de aceleración]

void sendAxis(const char* topic, char* axis, const float* data, size_t dataSize) {
    DynamicJsonDocument doc(15000);

    //Crea arreglo JSON
    JsonArray array = doc.createNestedArray(axis);

    //Genera el arreglo a partir del float array
    for (size_t i = 0; i < dataSize; i++) {
        // Limit the float to 2 decimal places
        float value = round(data[i] * 100.0) / 100.0;
        array.add(value);
    }

    //Convierte el documento a string
    String json;
    serializeJson(doc, json);


    //Publica al topico escogido
    if(mqttClient.publish(topic, json.c_str()))
        {
            Serial.println("Mensaje publicado en topico MQTT");
        } else {
            Serial.println("Error publicando mensaje en topico MQTT");
        }
}

\end{lstlisting}


\begin{lstlisting}[language=C++, caption=Tarea de envío de datos de variables cuasi-estáticas vía MQTT]

    void send_mqtt_thi(void *pvParameter){
        while(1){
          vTaskSuspend(xHandle_keepalive_task);
      
          THIPacket thipacket;
      
          //Recibe arreglos de la cola
          if(xQueueReceive(xQueueTempHumInc, &thipacket, portMAX_DELAY)){
              Serial.println("Recibido de la cola TempHumInc");
          } else {
              Serial.println("Error recibiendo de la cola TempHumInc");
          }
      
          delay(200);
      
          sendTHI("esp32/temp", thipacket.temperature);
          sendTHI("esp32/hum", thipacket.humidity);
          sendTHI("esp32/inc_y", thipacket.yaw);
          sendTHI("esp32/inc_p", thipacket.pitch);
          sendTHI("esp32/inc_r", thipacket.roll);
          sendTHI("esp32/timestamp", (float)thipacket.timestamp);
      
          vTaskResume(xHandle_send_mqtt);
          vTaskSuspend(NULL);
        }
      }
      
\end{lstlisting}
    
\begin{lstlisting}[language=C++, caption=Tarea de callback ante recepción de datos MQTT]

    void messageReceived(char* topic, byte* payload, unsigned int length) {
        Serial.print("Mensaje recibido en topico: ");
        Serial.println(topic);
      
        // Step 2: Convert payload to string
        String message;
        for (int i = 0; i < length; i++) {
          message += (char)payload[i];
        }
      
        // Step 3: Check if the message is "ON"
        if (message == "ON") {
          // Step 4: Call the software ISR
          Serial.println("Peticion de datos via MQTT... Enviando paquete Lora si no se estan enviando datos");
          if(!transmitFlag){
            ISR_MQTT_Request();
          }
        }
      }    

    \end{lstlisting}

%\backmatter
%==================================================================
\chapter{CÓDIGO DE LA INTERFAZ GRÁFICA}\label{CAP:anexo2}
%\markboth{Tu anexo}{Tu anexo}%

\begin{lstlisting}[language=Python, caption=Código para interfaz gráfica de control y monitoreo]

import customtkinter as ctk
import tkinter as tk
import tkinter.filedialog as fd
import pandas as pd
import numpy as np
import matplotlib.pyplot as plt
from matplotlib.backends.backend_tkagg import FigureCanvasTkAgg
import os
import shutil
#import tkinter.filedialog
import paho.mqtt.client as mqtt
from tkinter import PhotoImage
import filtros
from datetime import datetime,timedelta
from tkinter import ttk
from PIL import Image
from scipy.signal import butter, filtfilt
from scipy.integrate import cumtrapz

#Funcion callback cuando se recibe un mensaje MQTT con el payload received
def on_message(client, userdata, message):
    #Chequea si el payload es "received"
    if message.payload.decode() == "received":
        #Crea el label con el texto y lo muestra en pantalla
        label.configure(text="Nuevo registro disponible!")

#Funcion para enviar comando via MQTT al topico esp32/command 
def send_mqtt_message():
    #Publica el mensaje al topico
    client.publish("esp32/command", "ON")

    #Desactiva el boton temporalmente
    control_button.configure(state="disabled")

    #Reactiva el boton luego de 5 segundos
    tab1.after(5000, lambda: control_button.configure(state="normal"))

#CLIENTE MQTT
client = mqtt.Client(mqtt.CallbackAPIVersion.VERSION2)
#Se conecta al broker
client.connect("127.0.0.1", 1883, 60) #El broker esta en localhost
#Configura la funcion callback a llamar cuando se recibe mensaje
client.on_message = on_message
#Se suscribe al topico de alerta o notificacion de recepcion
client.subscribe("esp32/alerta_rx")
#Inicia el loop que mantiene la conexion de MQTT y gestiona los mensajes
client.loop_start()

ctk.set_appearance_mode("Light")   
ctk.set_default_color_theme("green")  
 
#VENTANA PRINCIPAL
window = ctk.CTk()
window.title("INTERFAZ DE MONITOREO DE CONTROL - SISTEMA ESP32 JOSE TOVAR")


#Crea 2 tabs
notebook = ctk.CTkTabview(window, anchor="nw")

# Frames de cada tab
tab1 = notebook.add('Tab 1')
tab2 = notebook.add('Tab 2')

#columnas
tab1.columnconfigure(0, weight=1)
tab1.columnconfigure(1, weight=1)
tab1.columnconfigure(2, weight=1)


image = Image.open("C:\\Users\\jatov\\Documents\\Universidad\\TEG\\CosasTEG_JT\\FFT-Python\\Codigos-Python\\IMME.jpg")
background_image = ctk.CTkImage(image, size=(100, 100))
# Create a label with the image
image_label = ctk.CTkLabel(tab1, image=background_image, text="")
# Place the label in the grid
image_label.grid(column=0, row=0, sticky='n')


image2 = Image.open("C:\\Users\\jatov\\Documents\\Universidad\\TEG\\CosasTEG_JT\\FFT-Python\\Codigos-Python\\angulos.png")
pruebaang = ctk.CTkImage(image2, size=(130, 130))
# Create a label with the image
image_label = ctk.CTkLabel(tab2, image=pruebaang, text="")
# Place the label in the grid
image_label.grid(column=0, row=2, sticky='n')


#SECCION DE CONTROL
control_frame = ctk.CTkFrame(tab1)
control_frame.grid(column=0, row=0, sticky='n', pady=170, padx=10)
#Crea label para seccion de control
control_label = ctk.CTkLabel(control_frame, text="Seccion de Control", font=("Arial", 14, "bold"))
control_label.pack()
#Crea boton para seccion de control
control_button = ctk.CTkButton(control_frame, text="Obtener registro", command=send_mqtt_message)
control_button.pack(pady=10)

#LISTA DE ARCHIVOS
folder_path = "C:\\Users\\jatov\\Documents\\Universidad\\TEG\\Pruebas_DatosAceleracion\\DatosACL_P1"
file_names = os.listdir(folder_path)

#crea el frame para los archivos
frameArchivos = ctk.CTkFrame(tab1)
frameArchivos.grid(column=1, row=3)

label = ctk.CTkLabel(frameArchivos, text="No hay registros nuevos")
label.grid(column=0, row=1, padx = 5)

selected_file_name = ctk.StringVar(tab1)
selected_file_name.set("Seleccione un registro")
#MENU DROPDOWN PARA ESCOGER ARCHIVO
option_menu = ctk.CTkOptionMenu(master=frameArchivos, variable=selected_file_name, values=file_names)
option_menu.grid(column=0, row=2)

# Funcion para actualizar lista de archivos
def update_file_list(option_menu):
    #Consigue la lista de archivos disponibles
    file_names = os.listdir(folder_path)

    # Sort the files by modification time
    file_names.sort(key=lambda x: os.path.getmtime(os.path.join(folder_path, x)))
    
    # Destroy the existing CTkOptionMenu widget
    option_menu.destroy()
    
    # Create a new CTkOptionMenu widget with the updated options
    option_menu = ctk.CTkOptionMenu(master=frameArchivos, variable=selected_file_name, values=file_names)
    option_menu.grid(column=1, row=1)

    #Actualiza cada 10 segundos
    tab1.after(10000, lambda: update_file_list(option_menu)) 

    # #Configura un timer para actualizar la lista cada 5 segundos
    # tab1.after(5000, update_file_list)

#FIGURA DE ACELERACION
fig, ax = plt.subplots()
fig.suptitle('REGISTRO DE ACELERACION EN TIEMPO', fontsize=10, fontweight='bold')
#fig.set_facecolor('lightgray')
#fig.set_edgecolor('black')
fig.set_alpha(0.5)
canvas = FigureCanvasTkAgg(fig, master=tab1)
widget = canvas.get_tk_widget()
widget.grid(column=1, row=0)
widget.config(bd = 2, relief = "groove")

#FIGURA PARA FFT
fig_fft, ax_fft = plt.subplots()
fig_fft.suptitle('ESPECTRO EN FRECUENCIA DEL REGISTRO', fontsize=10, fontweight='bold')
#fig.set_facecolor('lightgray')
canvas_fft = FigureCanvasTkAgg(fig_fft, master=tab1)
widget_fft = canvas_fft.get_tk_widget()
widget_fft.grid(column=2, row=0)
widget_fft.config(bd = 2, relief = "groove") #'flat', 'raised', 'sunken', 'ridge', 'groove' and 'solid'.

#FIGURA DE PSD
figpsd, ax_psd = plt.subplots()
figpsd.suptitle('DENSIDAD ESPECTRAL DE POTENCIA', fontsize=10, fontweight='bold')
#figpsd.set_facecolor('lightgray')
#figpsd.set_edgecolor('black')
figpsd.set_alpha(0.5)
canvas_psd = FigureCanvasTkAgg(figpsd, master=tab2)
widget_psd = canvas_psd.get_tk_widget()
widget_psd.grid(column=1, row=0)
widget_psd.config(bd = 2, relief = "groove")

#FIGURA DE BARRA
figbar, ax_bar = plt.subplots()
figbar.suptitle('GRAFICA DE BARRAS PARA VER INCLINACION EN SENSOR INTELIGENTE', fontsize=10, fontweight='bold')
#figpsd.set_facecolor('lightgray')
#figpsd.set_edgecolor('black')
figbar.set_alpha(0.5)
canvas_bar = FigureCanvasTkAgg(figbar, master=tab2)
widget_bar = canvas_bar.get_tk_widget()
widget_bar.grid(column=0, row=0)
widget_bar.config(bd = 2, relief = "groove")

label_nivel = ctk.CTkLabel(master=tab2, text="No se ha seleccionado un archivo.", font=("Arial", 14, "bold"))
label_nivel.grid(column=0, row=1)



#FRAME DE ABAJO A LA DERECHA
frameVentanas = ctk.CTkFrame(tab1)
frameVentanas.grid(column=2, row=2)
windowing_label = ctk.CTkLabel(master=frameVentanas, text="Aventanamiento")
windowing_label.grid(column=1, row=1, padx=5,pady= 5)  # Adjust the grid position as needed

#AVENTANAMIENTO
windowing_functions = ['Ninguno','Hanning', 'Hamming', 'Blackman']
#Guarda aventanamiento seleccionado
selected_windowing_function = ctk.StringVar()
#Configura el aventanamiento default
selected_windowing_function.set(windowing_functions[0])

#Crea el menu dropdown de aventanamientos
windowing_menu = ctk.CTkOptionMenu(master=frameVentanas, variable=selected_windowing_function, values=windowing_functions)
windowing_menu.grid(column=2, row=1, padx=5, pady= 5)  # Adjust the grid position as needed

#FUNCION DE GUARDADO EN CSV
def save_file():
    #Pregunta al usuario el archivo a escoger
    dest_filename = ctk.filedialog.asksaveasfilename(defaultextension=".csv")

    #Si el usuario no cancela el dialogo
    if dest_filename:
        #Copia el archivo actual en la direccion especificada
        shutil.copyfile(folder_path + "\\" + selected_file_name.get(), dest_filename)

#Crea el boton para guardar CSV en pantalla
save_button = ctk.CTkButton(tab1, text="Guardar CSV", command=save_file)
save_button.grid(column=2, row=3)

#VALORES ACTUALES DE TEMPERATURA Y HUMEDAD
#Crea frame de temperatura y humedad
frameTH = ctk.CTkFrame(tab1, bg_color="lightgray")
frameTH.grid(column=2, row=1, pady = 5)
#Crea los labels de temperatura y humedad con color
current_value_label_x = ctk.CTkLabel(frameTH, text="Variables ambientales:", font=("Arial", 14, "bold"))
current_value_label_x.grid(column=1, row=1, padx=5)
current_value_label_x = ctk.CTkLabel(frameTH, text="Humedad relativa:", font=("Arial", 14, "bold"), corner_radius=50, fg_color="darkgray")
current_value_label_x.grid(column=2, row=1, padx=10)
current_value_label_y = ctk.CTkLabel(frameTH, text="Temperatura:", font=("Arial", 14, "bold"), corner_radius=50, fg_color="darkgray")
current_value_label_y.grid(column=3, row=1, padx=10)

#VALORES ACTUALES DE INCLINACION
#Crea el frame donde se mostraran las inclinaciones
frameInc = ctk.CTkFrame(tab1, bg_color="lightgray")
frameInc.grid(column=1, row=1, pady = 5)
#Labels de valor actual de inclinacion
current_value_label_inc = ctk.CTkLabel(frameInc, text="InclinaciOn:", font=("Arial", 14, "bold"))
current_value_label_inc.grid(column=1, row=1, padx=5)

current_value_label_roll = ctk.CTkLabel(frameInc, text="Omega:", font=("Arial", 14, "bold"), corner_radius=50, fg_color="darkgray")
current_value_label_roll.grid(column=2, row=1, padx=10)
current_value_label_pitch = ctk.CTkLabel(frameInc, text="Phi:", font=("Arial", 14, "bold"), corner_radius=50, fg_color="darkgray")
current_value_label_pitch.grid(column=3, row=1, padx=10)
current_value_label_yaw = ctk.CTkLabel(frameInc, text="Kappa:", font=("Arial", 14, "bold"), corner_radius=50, fg_color="darkgray")
current_value_label_yaw.grid(column=4, row=1, padx=10)

#FECHA Y HORA DE REGISTRO
time_label = ctk.CTkLabel(tab1, text="No se ha escogido ningun registro")
time_label.grid(column=1, row=4)

#FUNCION PRINCIPAL DE ACTUALIZACION DE PLOTS
def update_plots():
    #Obtiene el nombre del archivo seleccionado
    file_name = selected_file_name.get()

    #Borra los contenidos previos del label para no solapar
    label.configure(text = "No hay registros nuevos")

    #Lee el archivo CSV y asigna nombres a las columnas por orden
    df = pd.read_csv(os.path.join( folder_path, file_name), header=None, names=['x', 'y', 'z', 'temp', 'hum', 'yaw', 'pitch', 'roll', 'time'])

    #Extrae las columnas x,y,z
    x = df['x']
    y = df['y']
    z = df['z']
    temp = df['temp'].iloc[0] #Extrae solo el valor de la primera fila
    hum = df['hum'].iloc[0]
    yaw = df['yaw'].iloc[0]
    pitch = df['pitch'].iloc[0]
    roll = df['roll'].iloc[0]
    time = df['time'].iloc[0]

    if(pitch < -1 or pitch > 1):
        label_nivel.configure(text="El sensor no esta nivel en el eje y (Phi).", font=("Arial", 14, "bold"))
    elif(roll < -1 or roll > 1):
        label_nivel.configure(text="El sensor no esta a nivel en el eje x (Omega).", font=("Arial", 14, "bold"))
    else:
        label_nivel.configure(text="El sensor esta a nivel.", font=("Arial", 14, "bold"))
        

    #Lista de angulos
    angles = [pitch, roll]
    labels = ['Phi', 'Omega']

    #Convierte de UNIX epoch a datetime
    time = datetime.fromtimestamp(df['time'].iloc[0]) + timedelta(hours=4) #Diferencia de 4 horas por GMT

    #Actualiza la etiqueta de tiempo
    time_label.configure(text="Fecha y Hora del registro: " + str(time))

    #Actualiza los labels de valores actuales de temperatura, humedad e inclinacion
    current_value_label_x.configure(text="Humedad relativa: " + str(hum) + "%")
    current_value_label_y.configure(text="Temperatura: " + str(temp))
    current_value_label_yaw.configure(text="Kappa: " + str(yaw))
    current_value_label_pitch.configure(text="Omega: " + str(roll))
    current_value_label_roll.configure(text="Phi: " + str(pitch))

    # Aplica el aventanamiento seleccionado a los datos de aceleracion
    if selected_windowing_function.get() == 'Hanning':
        window = np.hanning(len(x))
    elif selected_windowing_function.get() == 'Hamming':
        window = np.hamming(len(x))
    elif selected_windowing_function.get() == 'Blackman':
        window = np.blackman(len(x))
    elif selected_windowing_function.get() == 'Ninguno':
        window = 1 # No windowing

    windowed_data_x = x * window
    windowed_data_y = y * window
    windowed_data_z = z * window
    
    #Aplica FFT en datos aventanados con la funcion escogida
    fft_1 = np.fft.fft(windowed_data_x)
    fft_2 = np.fft.fft(windowed_data_y)
    fft_3 = np.fft.fft(windowed_data_z)

    # Calculate the PSD
    xfil_psd = np.abs(fft_1)**2
    yfil_psd = np.abs(fft_2)**2
    zfil_psd = np.abs(fft_3)**2

    #Obtiene las frecuencias positivas de la FFT
    freq = np.fft.fftfreq(len(x), d=1/200)  # d is the inverse of the sampling rate
    positive_freq = freq[:len(freq)//2]

    #Limpia la grafica anterior
    ax.clear()
    ax_fft.clear()
    ax_psd.clear()
    ax_bar.clear()

    # Genera eje de tiempo
    start_time = datetime.fromtimestamp(df['time'].iloc[0])

    timearr = [start_time + timedelta(seconds=i/200) for i in range(len(x))]

    #Grafica los nuevos datos

    #ACELERACION
    ax.plot(timearr, x, label='ACL X')
    ax.plot(timearr, y, label='ACL Y')
    ax.plot(timearr, z, label='ACL Z')
    ax.set_xlabel('Tiempo (s)')
    ax.set_ylabel('Aceleracion')
    ax.grid(True)

    #FFT SIN FILTRAR

    xfil = fft_1
    yfil = fft_2
    zfil = fft_3

    #FFT filtrada
    # xfil = filtros.butter_lowpass_filter(fft_1, 40, 200, 5)
    # yfil = filtros.butter_lowpass_filter(fft_2, 40, 200, 5)
    # zfil = filtros.butter_lowpass_filter(fft_3, 40, 200, 5)

    #FFT
    ax_fft.plot(positive_freq, np.abs(xfil[:len(positive_freq)]), label='Espectro Eje X')
    ax_fft.plot(positive_freq, np.abs(yfil[:len(positive_freq)]), label='Espectro Eje Y')
    ax_fft.plot(positive_freq, np.abs(zfil[:len(positive_freq)]), label='Espectro Eje Z')

    ax_fft.set_xlabel('Frecuencia (Hz)')
    ax_fft.set_ylabel('Amplitud')
    ax_fft.grid(True)

    #Grafica la densidad espectral de potencia
    ax_psd.plot(positive_freq, np.abs(xfil_psd[:len(positive_freq)]), label='Densidad Espectral X')
    ax_psd.plot(positive_freq, np.abs(yfil_psd[:len(positive_freq)]), label='Densidad Espectral Y')
    ax_psd.plot(positive_freq, np.abs(zfil_psd[:len(positive_freq)]), label='Densidad Espectral Z')
    #ax_psd.set_yscale('log')
    ax_psd.set_xlabel('Frequency (Hz)')
    ax_psd.set_ylabel('Power Spectral Density')
    ax_psd.grid(True)

    #Grafico de barras para inclinacion
    ax_bar.axhline(1, color='r', linestyle='--')  
    ax_bar.axhline(-1, color='r', linestyle='--') 
    ax_bar.bar(labels, angles)
    ax_bar.set_title('Diagrama de barras para inclinacion en sensor inteligente')
    ax_bar.set_xlabel('Angulo')
    ax_bar.set_ylabel('Grados')


    #Obtiene las frecuencias maximas de cada eje
    max_freq1 = positive_freq[np.argmax(np.abs(xfil[:len(positive_freq)]))]
    max_freq2 = positive_freq[np.argmax(np.abs(yfil[:len(positive_freq)]))]
    max_freq3 = positive_freq[np.argmax(np.abs(zfil[:len(positive_freq)]))]

    median_freq = positive_freq[len(positive_freq)//2]

    #Muestra las frecuencias maximas en la grafica
    ax_fft.annotate(f'Fmax_X: {max_freq1:.2f}', xy=(max_freq1, np.max(np.abs(xfil[:len(positive_freq)]))), xytext=(-10 if max_freq1 > median_freq else 10,30), textcoords='offset points', arrowprops=dict(arrowstyle='->'))
    ax_fft.annotate(f'Fmax_Y: {max_freq2:.2f}', xy=(max_freq2, np.max(np.abs(yfil[:len(positive_freq)]))), xytext=(-10 if max_freq2 > median_freq else 10,10), textcoords='offset points', arrowprops=dict(arrowstyle='->'))
    ax_fft.annotate(f'Fmax_Z: {max_freq3:.2f}', xy=(max_freq3, np.max(np.abs(zfil[:len(positive_freq)]))), xytext=(-10 if max_freq3 > median_freq else 10,-10), textcoords='offset points', arrowprops=dict(arrowstyle='->'))

    #Frecuencias maximas en PSD
    ax_psd.annotate(f'Fmax_X: {max_freq1:.2f}', xy=(max_freq1, np.max(np.abs(xfil_psd[:len(positive_freq)]))), xytext=(-10 if max_freq1 > median_freq else 10,30), textcoords='offset points', arrowprops=dict(arrowstyle='->'))
    ax_psd.annotate(f'Fmax_Y: {max_freq2:.2f}', xy=(max_freq2, np.max(np.abs(yfil_psd[:len(positive_freq)]))), xytext=(-10 if max_freq2 > median_freq else 10,10), textcoords='offset points', arrowprops=dict(arrowstyle='->'))
    ax_psd.annotate(f'Fmax_Z: {max_freq3:.2f}', xy=(max_freq3, np.max(np.abs(zfil_psd[:len(positive_freq)]))), xytext=(-10 if max_freq3 > median_freq else 10,-10), textcoords='offset points', arrowprops=dict(arrowstyle='->'))

    
    ax.legend()
    ax_fft.legend()

    #Redibuja las graficas
    canvas.draw()
    canvas_fft.draw()
    canvas_bar.draw()
    canvas_psd.draw()

#Crea el boton para actualizar grafica
button = ctk.CTkButton(tab1, text="Actualizar grafica", command=update_plots)
button.grid(column=1, row=2)

#Ejecuta la funcion de actualizacion de archivos disponibles
update_file_list(option_menu)

notebook.pack(expand=True, fill='both')

#Loop principal de la ventana
window.mainloop()

\end{lstlisting}

%



%\backmatter

%==================================================================
\newpage
%\markboth{Referencias}{Referencias}%
%\addcontentsline{toc}{chapter}{Referencias}%

% References here (outcomment the appropriate case)
% CASE 1: BiBTeX used to constantly update the references (while the paper is being written).
%\bibliographystyle{IEEEtran}%{IEEEtranS}%%% outcomment this and next line in Case 1 siam
\bibliographystyle{apacite}%
\renewcommand{\bibname}{REFERENCIAS}
\let\oldbibsection\bibsection
\bibliography{biblioteca} % if more than one, comma separated and without extension bib


% CASE 2: BiBTeX used to generate EIETdeG.bbl (to be further fine tuned)
%\documentclass[letterpaper,titlepage,12pt,oneside,spanish,final]{report_eie}

%PARA QUE APAREZCAN LAS SUBSUBSECTIONS EN EL INDICE
%\setcounter{tocdepth}{3}

%\documentclass[letterpaper,titlepage,12pt,twoside,openright,spanish,final]{report_eie}

%%%%%%%%%%%%%%%%%%%%%%%%%%%% LIBRERIAS %%%%%%%%%%%%%%%%%%%%%%%%%%%%%%%%%%%%%%%%%%%%

\usepackage[spanish]{babel}
\usepackage[utf8]{inputenc}
%\usepackage[latin1]{inputenc}
\usepackage[T1]{fontenc}  %Estilo de fuente times new roman

\usepackage{subfig}

\usepackage{amssymb}
\usepackage{amsfonts}
\usepackage{amsmath}
\usepackage{latexsym}
\usepackage[letterpaper]{geometry}

\usepackage{float}
\usepackage{makeidx}
\usepackage{color}

\usepackage{tocbibind}
\usepackage{acronym}
%OJO CAMBIE A CAPTION ENVEZ DE CAPTION2
\usepackage{caption} %EL CENTER CENTRA TODOS LOS 
\captionsetup[figure]{format=plain,justification=centering}%CAPTIONS DE TABLAS Y FIGURAS OJO
\usepackage{epsfig}
\usepackage{graphicx}
%\usepackage{slashbox}
\usepackage{setspace}
\usepackage{multicol}
\usepackage{longtable}
%\usepackage{doublespace}
\usepackage{array}
\newcolumntype{P}[1]{>{\centering\arraybackslash}p{#1}}


\usepackage{fancyhdr}
%\usepackage{fancyheadings}

\usepackage{booktabs}

%NUMEROS DE LINEA
% \usepackage[pagewise]{lineno}
% \linenumbers

%========= Define el estilo de referencias ===============
%\usepackage[round,authoryear]{natbib}%\usepackage[square,numbers]{natbib}%
%\usepackage[comma,authoryear]{natbib} esto está abajo

%========= Define el estilo de referencias APA ===============
\usepackage[natbibapa]{apacite}%natbibapa
%\usepackage[square, numbers]{natbib}
%\usepackage[numbers]{natbib} %PARA QUE SEAN NUMEROS EN CORCHETE
%\usepackage[apaciteclassic]{apacite}%natbibapa
%\usepackage[compact]{titlesec} %modificar espaciado

\usepackage{url}
\usepackage{hyperref}
%\usepackage[dvips,colorlinks=true,urlcolor=red,citecolor=black,anchorcolor=black,linkcolor=black]{hyperref}

%%%%%%%%%%%%%%%%CODIGO%%%%%%%%%%%%%%%%

\usepackage{listings}
\usepackage{xcolor}

\definecolor{codegreen}{rgb}{0,0.6,0}
\definecolor{codegray}{rgb}{0.5,0.5,0.5}
\definecolor{codepurple}{rgb}{0.58,0,0.82}
\definecolor{backcolour}{rgb}{0.95,0.95,0.92}

\lstdefinestyle{mystyle}{
    backgroundcolor=\color{backcolour},   
    commentstyle=\color{codegreen},
    keywordstyle=\color{blue},
    numberstyle=\tiny\color{codegray},
    stringstyle=\color{codepurple},
    basicstyle=\ttfamily\footnotesize,
    breakatwhitespace=false,         
    breaklines=true,                 
    captionpos=b,                    
    keepspaces=true,                 
    numbers=left,                    
    numbersep=5pt,                  
    showspaces=false,                
    showstringspaces=false,
    showtabs=false,                  
    tabsize=2
}

\lstset{style=mystyle}

\def\lstlistingname{Código}


%%%%%%%%%%%%%%%%%%%%%%%%%%%%%%%%%%%%%%%%%%%%%%%%%%%%%%%%%%%%%%%%%%
%       Definición del Documento PDF, (PDFLaTeX)        %
%%%%%%%%%%%%%%%%%%%%%%%%%%%%%%%%%%%%%%%%%%%%%%%%%%%%%%%%%%%%%%%%%%

\hypersetup{pdfauthor=Nombre}

\hypersetup{pdftitle=Título}%

\hypersetup{pdfkeywords=Palabras clave}

\pdfstringdef{\Produce}{Escuela de Ingeniería Eléctrica, Facultad de Ingeniería, UCV}%

\pdfstringdef{\area}{Área del trabajo}

\hypersetup{pdfproducer=\Produce}

\hypersetup{pdfsubject=\area}

\hypersetup{bookmarksnumbered=true}

%%%%%%%%%%%%%%%%%%%%%%%%%%%%%%%%%%%%%%%%%%%%%%%%%%%%%%%%%%%%%%%%%%

%\setcounter{MaxMatrixCols}{10}

%===================== Re-definición de Ambientes =================
\newtheorem{theorem}{Teorema}
\newtheorem{acknowledgement}[theorem]{Acknowledgement}
\newtheorem{algoritmo}[theorem]{Algorithm}
\newtheorem{supuestos}[theorem]{Supuestos}
\newtheorem{hipotesis}[theorem]{Hipótesis}
\newtheorem{axiom}[theorem]{Axiom}
\newtheorem{case}[theorem]{Case}
\newtheorem{claim}[theorem]{Claim}
\newtheorem{conclusion}[theorem]{Conclusión}
\newtheorem{condition}{Condición}
\newtheorem{conjecture}{Conjecture}
\newtheorem{corollary}{Corollary}
\newtheorem{criterion}{Criterion}
\newtheorem{definition}{Definición}  %{Definition}
\newtheorem{example}[theorem]{Ejemplo}%{Example}
\newtheorem{exercise}[theorem]{Exercise}
\newtheorem{lemma}{Lemma}
\newtheorem{notation}[theorem]{Notation}
\newtheorem{problem}{Problem}
\newtheorem{property}{Property}
\newtheorem{proposition}{Proposition}
\newtheorem{remark}[theorem]{Remark}
\newtheorem{solution}{Solution}
\newtheorem{summary}[theorem]{Summary}
\newenvironment{proof}[1][Proof]{\noindent\textbf{#1.} }{\ \rule{0.5em}{0.5em}}%

\numberwithin{equation}{chapter}%
\numberwithin{figure}{chapter}%
\numberwithin{table}{chapter}%
\numberwithin{definition}{chapter}%
\numberwithin{lemma}{chapter}%
\numberwithin{theorem}{chapter}%
\numberwithin{corollary}{chapter}%
\numberwithin{condition}{chapter}%
\numberwithin{criterion}{chapter}%
 \numberwithin{problem}{chapter}%
\numberwithin{property}{chapter}%
\numberwithin{proposition}{chapter}%
\numberwithin{solution}{chapter}%
\numberwithin{conjecture}{chapter}%

%==================== Separación en sílabas ========================
\hyphenpenalty=6800%
\input{silabear.tex}

%==================== Diseño de Página =============================
%\pagestyle{headings}
%\setlength{\headheight}{0.2cm}
\setlength{\textwidth}{14.52cm}%
%\pagestyle{fancy}
\renewcommand{\chaptermark}[1]{\markboth{#1}{}}
%\renewcommand{\sectionmark}[1]{\markright{\thesection\ #1}}
%\rhead[\fancyplain{}{\bfseries\thepage}]{\fancyplain{}{\bfseries\rightmark}}%\thepage
%\lhead[\fancyplain{}{\bfseries\leftmark}]{\fancyplain{}{\bfseries}} \cfoot{}%

%\fancyhead[R]{}

\rfoot[\fancyplain{}{\textit{E. Brea}}] {\fancyplain{}{}}
\lfoot[\fancyplain{}{}] {\fancyplain{}{\textit{}}}    %%%%%%%%%%%%%%%%%%% OJO ACA %%%%%%%%%%
\cfoot[\fancyplain{}{}] {\fancyplain{}{\bfseries\thepage}}
%\setlength{\footrulewidth}{0.0pt}%
%\setlength{\headrulewidth}{0.1pt}%

%===================================================================

%================== Diseño de Párrafo y delimitador ================
\renewcommand{\baselinestretch}{1.5}% Espaciado entre linea
\geometry{left=4cm,right=3cm,top=3cm,bottom=3cm}
\frenchspacing %
%\raggedright % Sólo para justificar el texto a la izquierda
%\renewcommand{\captionlabeldelim}{.}%
\setlength{\parindent}{0.7cm}% Espacio de la sangría
\setlength{\parskip}{14pt plus 1pt minus 1pt}% Separación entre párrafos

%\setlength{\parskip}{1ex plus 0.5ex minus 0.2ex}%

%===================================================================

%==========================  Español venezolano =====================
%%Personalización de caption
\addto\captionsspanish{%
  \def\prefacename{Prefacio}%
  \def\refname{REFERENCIAS}%
  \def\abstractname{Resumen}%
  \def\bibname{REFERENCIAS}%{Bibliografía}%
  \def\chaptername{CAPÍTULO}%
  \def\appendixname{Apéndice}%{Anexo}
  \def\contentsname{ÍNDICE GENERAL}
  \def\listfigurename{LISTA DE FIGURAS}%Índice de Figuras\hspace*{10em}
  \def\listfigurenameTofC{LISTA DE FIGURAS}%Índice de Figuras
  \def\listtablename{LISTA DE TABLAS}%Índice de Tablas
  \def\indexname{Índice alfabético}%
  \def\figurename{Figura}%
  \def\tablename{Tabla}%
  \def\partname{Parte}%
  \def\enclname{Adjunto}%
  \def\ccname{Copia a}%
  \def\headtoname{A}%
  \def\pagename{Página}%
  \def\seename{véase}%
  \def\alsoname{véase también}%
  \def\proofname{Demostración}%
  \def\glossaryname{Glosario}
  }%

%==================================================================

%\setcounter{secnumdepth}{1}
%\setcounter{page}{4}
%\addtocounter{page}{4}%

\pagenumbering{roman}

\makeindex

%%%%%%%%%%%%%%%%%%%%%%%%%%%%%%%%%%%%%%%%%%%%%%%%%%%%%%%%%%%%%%%%%

\begin{document}
%\frontmatter

%%%%%%%%%%%%%%%%%%%%%%%%%%%%%%%%%%%%%%%%%%%%%%%%%%%

%               Primera Página
%================================== Portada ========================
\input{portada.tex}

%======================= Constancia de Aprobación ===================
%\newpage
\begin{figure}
        \begin{center}
        %\centering
        %\includegraphics[height=23cm]{aprobacion.eps}

        \vspace{0.5mm}
        \label{Fig.aprobacion}
        \end{center}
        \end{figure}
\thispagestyle{empty}

%======================= Mención Honorífica =========================
\newpage
%\thispagestyle{empty}

\begin{figure}
        \begin{center}
        %\centering
        %\includegraphics[height=24cm]{mencion.eps}
        \vspace{0.5mm}
        \label{Fig.mencion}
        \end{center}
\end{figure}
\thispagestyle{empty}

%======================= Página de Dedicatoria ======================
\newpage%
\newenvironment{dedication}%
{\cleardoublepage \thispagestyle{empty} \vspace*{\stretch{1}}%
\begin{center} \em} {\end{center} \vspace*{\stretch{3}} }%
\begin{dedication}%
Dedicado a Gregoria del Valle Briceño Aray y José Edmundo Tovar Silva.
\end{dedication}%

%=============================== RECONOCIMIENTOS Y AGRADECIMIENTOS ===================================
\chapter*{RECONOCIMIENTOS Y AGRADECIMIENTOS}
%\markboth{Reconocimientos}{Reconocimientos}%
\addcontentsline{toc}{chapter}{RECONOCIMIENTOS Y AGRADECIMIENTOS}%
%\setlength{\parskip}{0.2cm}%
%\input{agradecimientos.tex}%

%======================= Página de Resumen (Abstract) ==========================
\newpage
\renewcommand*{\abstract}{\begin{center}\end{center}}
%\begin{abstract}
\begin{spacing}{1}
\begin{center}%

\textbf{Tovar B., José A.}

\begin{large}
DISEÑO DE UN SENSOR INTELIGENTE PARA APLICACIONES DE MONITOREO DE SALUD ESTRUCTURAL
\end{large}
\end{center}

\noindent%
\textbf{Tutor Académico: Prof. Jose Romero. Tesis.
Caracas, Universidad Central de Venezuela. Facultad de Ingeniería.
Escuela de Ingeniería Eléctrica. Ingeniero Electricista. Opción: Electrónica y Control. Institución: IMME Año 2024,
xvii, 153 h. + anexos}

\noindent
\textbf{Palabras Claves:} Salud estructural, Sensor inteligente,  Respuesta dinámica, Frecuencia natural, Microcontrolador, Acelerómetro, LoRa, ESP32, MQTT, Espectro en frecuencia. \\[1ex]

\noindent \textbf{Resumen.-} En el siguiente trabajo se plantea el diseño de un sensor inteligente para ser utilizado en aplicaciones de salud estructural. En primer lugar, se llevó a cabo la investigación documental necesaria para identificar las variables de interés al evaluar la respuesta dinámica de los sistemas estructurales y su relación con la instrumentación electrónica. Se escogió el hardware necesario para la implementación de un prototipo de pruebas capaz de verificar el funcionamiento del diseño. Se escogieron sensores de tecnología MEMS como el MPU6250, MPU9250 y BME280, en conjunto con el microcontrolador ESP32 y el módulo Ra-02 de Ai-Thinker para las comunicaciones inalámbricas. Se diseñó el software necesario para controlar el sistema haciendo uso de FreeRTOS y se implementaron con éxito las tareas tanto en el sensor inteligente como en la estación base para obtener los registros de vibración y de las variables cuasi-estáticas (temperatura, humedad relativa e inclinación). Se programó una interfaz gráfica de monitoreo y control para observar los registros y enviar comandos al sensor inteligente. Las pruebas realizadas en el IMME demostraron el funcionamiento satisfactorio del sensor inteligente al comparar los registros obtenidos con el equipo de vibración comercial basado en la tarjeta PCI-6221 de National Instruments, obteniendo resultados muy similares que permitieron caracterizar el comportamiento dinámico de una estructura de acero.

\end{spacing}

%\underline{RESUMEN}
%
\thispagestyle{empty}%
%\input{resumen.tex}%
%\end{abstract}
%====================== Páginas de Contenidos =====================
\renewcommand{\baselinestretch}{1.5}% Espaciado entre linea
\addtocounter{page}{3}%
\setlength{\parskip}{3pt}% Separación entre párrafos

\tableofcontents%

\listoffigures%

\listoftables%


%==================================================================
\chapter*{LISTA DE ACRÓNIMOS}%
%\markboth{Lista de Acrónimos}{Lista de Acrónimos}%
\addcontentsline{toc}{chapter}{LISTA DE ACRÓNIMOS}%


\begin{acronym}
\acro{SHM}{Structural Health Monitoring}
\acro{ADC}{Analog-to-Digital Converter}
\acro{NASA}{National Aeronautics and Space Administration}
\acro{SMIS}{Shuttle Modal Inspection System}
\acro{DOF}{Degrees of Freedom}
\acro{VLSI}{Very Large Scale Integration}
\acro{RAM}{Random Access Memory}
\acro{ROM}{Read-Only Memory}
\acro{DAC}{Digital-to-Analog Converter}
\acro{SoC}{System on a Chip}
\acro{MCU}{Microcontroller Unit}
\acro{SBC}{Single Board Computer}
\acro{UART}{Universal Asynchronous Receiver/Transmitter}
\acro{I2C}{Inter-Integrated Circuit}
\acro{SDA}{Serial Data Line}
\acro{SCL}{Serial Clock Line}
\acro{SPI}{Serial Peripheral Interface}
\acro{USB}{Universal Serial Bus}
\acro{RTOS}{Real-Time Operating System}
\acro{SMP}{Symmetric Multiprocessing}
\acro{MIT}{Massachusetts Institute of Technology}
\acro{FIFO}{First In, First Out}
\acro{MEMS}{Micro-Electro-Mechanical Systems}
\acro{DSP}{Digital Signal Processing}
\acro{IEEE}{Institute of Electrical and Electronics Engineers}
\acro{DAQ}{Data Acquisition}
\acro{WiFi}{Wireless Fidelity}
\acro{MQTT}{Message Queuing Telemetry Transport}
\acro{IoT}{Internet of Things}
\acro{M2M}{Machine to Machine}
\acro{CSS}{Chirp Spread Spectrum}
\acro{CRC}{Cyclic Redundancy Check}
\acro{FFT}{Fast Fourier Transform}
\acro{NTP}{Network Time Protocol}
\acro{RTC}{Real-Time Clock}
\acro{IMU}{Inertial Measurement Unit}
\acro{MARG}{Magnetic, Angular Rate, and Gravity}
\acro{JSON}{JavaScript Object Notation}
\acro{CSV}{Comma-Separated Values}
\acro{GUI}{Graphical User Interface}
\acro{ISR}{Interrupt Service Routine}
\end{acronym}
%


%==================================================================
\chapter*{INTRODUCCIÓN}\label{CAP:intro}
\setlength{\parskip}{14pt}% Separación entre párrafos
\addcontentsline{toc}{chapter}{INTRODUCCIÓN}%
%\markboth{Introducción}{Introducción}%

\pagenumbering{arabic}%
La seguridad de las infraestructuras es un tema de gran importancia en la actualidad, especialmente cuando se trata de estructuras como edificios o puentes.

Los primeros indicios del monitoreo del estado de las infraestructuras data de nuestros comienzos como especie sedentaria.  En la antigüedad, los especialistas utilizaban técnicas de inspección visual y auditiva para detectar posibles problemas en las estructuras, como grietas o ruidos inusuales. Con el tiempo, se desarrollaron técnicas más avanzadas para el monitoreo de estructuras, como la utilización de medidores de deformación, inclinación, sensores de vibración, entre otros.

La integración de la instrumentación con el análisis estructural comenzó a desarrollarse en la década de 1960 con el advenimiento de la informática y la disponibilidad de computadoras capaces de realizar cálculos estructurales complejos. En esa época, se comenzaron a utilizar sistemas de adquisición de datos para recopilar información sobre el comportamiento de las estructuras en tiempo real y utilizarla para calibrar y validar los modelos estructurales.

Actualmente, las normas sismo-resistentes apuntan a estructuras que sean capaces de mantener su integridad ante un evento de cierta magnitud. Además, el monitoreo continuo de ciertos indicadores en la estructura permiten determinar un índice de la salud estructural y ajustar el modelo a las condiciones actuales de la misma para evaluar el cumplimiento de la normativa sismorresistente. Para el monitoreo a largo plazo, el resultado de este proceso es información actualizada periódicamente sobre la capacidad de la estructura para desempeñar su función prevista a la luz del inevitable envejecimiento y degradación resultantes de los entornos operativos.

Según \citep{balageas2010structural}, el monitoreo de la salud estructural (SHM) tiene por objeto proporcionar, en cada momento de la vida de una estructura, un diagnóstico del estado de los materiales constitutivos, de las diferentes
partes, y del conjunto de estas partes que constituyen la estructura en su totalidad. El estado de la estructura debe permanecer en el ámbito especificado en el diseño, aunque este puede verse alterado por el envejecimiento normal debido al uso, por la acción del medio ambiente y por sucesos accidentales. Gracias a la dimensión temporal de la supervisión, que permite tener en cuenta toda la base de datos histórica de la estructura, y con la ayuda del monitoreo del funcionamiento. También puede proporcionar un pronóstico (evolución de los daños, vida residual, entre otros).

Si consideramos solo la primera función, el diagnóstico, podríamos estimar que el monitoreo de la salud estructural es una forma nueva y mejorada de realizar una evaluación no destructiva. Esto es parcialmente cierto, pero SHM es mucho más. Implica la integración de sensores, posiblemente materiales inteligentes, transmisión de datos, potencia computacional y capacidad de procesamiento en el interior de las estructuras. Permite reconsiderar el diseño de la estructura y la gestión completa de la propia estructura y de la estructura considerada como parte de sistemas más amplios.


%Incluir conjunto de elementos que influyen en el deterioro de la estructura. Envejecimiento, mantenimiento

En este sentido, el monitoreo de las estructuras se ha convertido en una herramienta esencial para garantizar la seguridad de las personas durante la vida útil de la misma, incluyendo la ocurrencia de eventos de cierta magnitud. Además, el monitoreo de las estructuras puede ayudar a mejorar la eficacia de las normas sismorresistentes, ya que permite validar y mejorar los modelos estructurales utilizados en la normativa.

Según \citep{nagayama2007structural}, dado que las edificaciones suelen ser grandes y complejas, la información
de unos pocos sensores es inadecuada para evaluar con precisión el estado estructural. El
comportamiento dinámico de estas estructuras es complejo tanto a escala espacial como temporal. Además,
los daños y/o el deterioro es intrínsecamente un fenómeno local. Por lo tanto, para comprender el
comportamiento dinámico, el movimiento de las estructuras debe ser supervisado por sensores
con una frecuencia de muestreo suficiente para captar las características dinámicas más destacadas. Esta información combinada con el registro del comportamiento estático de la estructura permiten tener una visión más amplia del estado actual de la estructura.


El primer paso, además de un mantenimiento adecuado, para garantizar la seguridad de estas estructuras, es contar con sistemas de monitoreo que permitan detectar posibles daños o fallas en su funcionamiento y tomar medidas preventivas. Por tanto, los sistemas de adquisición de datos y monitoreo son herramientas esenciales en la prevención de accidentes y daños.

A su vez, según \citep{nagayama2007structural}, un dispositivo inteligente, es decir, con capacidad de procesamiento de datos en el caso de los sensores, es una característica esencial que permite incrementar el potencial de los sensores al ser estos inalámbricos. Los sensores inteligentes pueden procesar localmente los datos medidos y trasmitir solo la información importante a través de comunicaciones inalámbricas. Cuando estos son configurados como una red, se extienden las capacidades de los mismos.

Los sensores inteligentes, con sus capacidades de cómputo y de comunicación integradas, ofrecen nuevas oportunidades para la SHM. Sin necesidad de cables de alimentación o comunicación, los costes de instalación pueden reducirse drásticamente. Los sensores inteligentes ayudarán a que el monitoreo de las estructuras con un denso conjunto de sensores sea económicamente práctico. Se espera que los sensores inteligentes instalados en masa sean fuentes de información muy valiosa para la SHM.

En este trabajo de grado se abordará el diseño para una futura implementación de un sistema de adquisición de datos de bajo costo basado dispositivos programables con capacidad de interconexión para el monitoreo y procesamiento de variables como aceleración, inclinación, humedad y temperatura en estructuras críticas, con el objetivo de prevenir daños y accidentes.



De acuerdo a Brea  la transformada de Laplace debe estudiarse como
una función definida en el campo de los números complejos
\cite{brea5}.

Otro modo de referencial es \citep{brea5}

El resto del reporte consta de: en el Capítulo \ref{CAP:referencial} se
describe...

En el trabajo se emplea el enfoque de \cite{brigham1}

De acuerdo a la ecuación
%

%==================================================================
\chapter{MARCO REFERENCIAL}\label{CAP:referencial}
\section{Planteamiento del problema}

\section{Justificación}

\section{Objetivos}

\subsection{Objetivo general}

\subsection{Objetivos específicos}

\section{Antecedentes}

%

%==================================================================
\chapter{MARCO TEÓRICO}\label{CAP:marco_teor}
%\markboth{Tu Primer Capítulo}{Tu Primer Capítulo}%

En este capítulo se definirán los conceptos  o fundamentos de instrumentación estructural, sensores inteligentes y adquisición de datos, necesarios para llevar a cabo esta investigación.

\section{Estructuras civiles}

\subsection{Características generales}

Una estructura se refiere a un sistema de partes o elementos que se interconectan para cumplir una función es específico. En el caso de la ingeniería civil, suelen ser miembros que se utilizan para soportar una carga. Algunos ejemplos importantes son los edificios, los puentes y las torres; y en otras ramas de la ingeniería, son importantes las corazas de barcos y aviones, los sistemas mecánicos y las estructuras que soportan las líneas de transmisión eléctrica \citep{hibbeler1997structural}.

\subsection{Tipos de estructuras}

Según \citet{hibbeler1997structural}, cada sistema está formado por uno o varios de los cuatro tipos básicos de estructuras: 

\begin{itemize}
    \item Celosías.
    \item Cables y arcos.
    \item Armazones.
    \item Estructuras de superficie.
\end{itemize}

En general, estos elementos suelen soportar cargas, pueden ser estacionarios y también estar restringidos. Sus diferencias suelen basarse en la cantidad de fuerzas a las que están sujetos estos elementos en un instante dado.

La combinación de estos elementos y los materiales que los componen es lo que se denomina un sistema estructural. Estos sistemas, aunque sean pasados por alto, son utilizados diariamente por industrias y personas, siendo elementos claves en el desarrollo y progreso de la civilización actual.

\subsection{Comportamiento de las estructuras civiles}

La gran mayoría de los sistemas cuentan con una respuesta dinámica y estática. Ambas respuestas permiten conocer el comportaiento completo del sistema en estudio ante distintas entradas o en diferentes situaciones. Al estudiar el comportamiento estructural se encuentra una extensa literatura tanto para el estudio dinámico como para el régimen estático, recopilándose lo siguiente:

\begin{itemize}
    \item{Respuesta estática}: En la ingeniería civil toda estructura se diseña para que se encuentre en reposo cuando actúan sobre esta fuerzas externas, es decir, la estructura en conjunto debe cumplir con las condiciones de equilibrio, siendo la fuerza y el momento resultanto sobre esta igual a cero en todo momento. Para describir estas condiciones de equilibrio se cuentan con herramientas matemáticas que proporcionan las condiciones necesarias para su cumplimiento. Estas ecuaciones permiten la resolución estática de la estructura, la cual permite determinar el valor de todas las incógnitas estáticas de interés \citep{basset2014analisis}.
    
    \indent Cuando las fuerzas que actúan sobre la estructura pueden calcularse a partir de las ecuaciones de equilibrio, se tiene una estructura en equilibrio y se denonima estructura estáticamente determinada. En caso de tenerse más fuerzas desconocidas que ecuaciones de equilibrio se habla de una estructura estáticamente indeterminada.

        \begin{itemize}
            \item Rigidez: Uno de los parámetros más importantes dentro de la respuesta estática es la rigidez. Esta se define como la propiedad que tiene un elemento estructural de soportar la deformación o deflección al estar bajo la acción de una fuerza o carga. Una medida de la rigidez viene dada por el Módulo de Young; esta es una constante del material y es independiente de la cantidad de material.
        \end{itemize}

    \item{Respuesta dinámica}: La dinámica estructural se encarga de estudiar el efecto que tienen cargas dinámicas sobre el sistema. La respuesta ante estos eventos, como pueden ser sismos, vientos, equipos mecánicos, paso de vehículos o personas, se denomina respuesta dinámica \citep{hurtado2000}. Además, la respuesta dinámica permite caracterizar algunos parámetros de gran interés para estudiar su comportamiento conocidos como parámetros modales. Estos parámetros surgen al estudiar las ecuaciones diferenciales que describen el movimiento de la estructura, partiendo de un modelo idealizado simple de masa concentrada como el de la Figura \ref{fig:masa_estructural}.
    
    \begin{figure}[H]
        \centering
        \includegraphics[width = 0.25\textwidth]{imagenes/cap1_marcoteo/modelo_masa_simple.png}
        \caption{Modelo de masa concentrada de 1 grado de libertad \citet{hurtado2000}.}
        \label{fig:masa_estructural}
    \end{figure}

    La dinámica de este modelo puede describirse utilizando la ecuación diferencial de movimiento:

    \begin{equation} \label{eq:vib_lib}
        m\ddot{u} + f_R(t) = p(t)
    \end{equation}

    La ecuación \ref{eq:vib_lib} se conoce como ecuación de vibración libre sin amortiguamiento. Donde $p(t)$ representa las cargas dinámicas y $f_R(t)$ la fuerza de restitución propia de un material elástico.  Esta ecuación es una ecuación diferencial de coeficientes constantes, que consta de una solución homogénea más una solución particular. La solución homogénea será la respuesta de la estructura a la vibración libre, es decir, si la masa de la Figura \ref{fig:masa_estructural} se deja oscilar libremente.

    Se sabe que una ecuación de este tipo tendrá una solución como:

    \begin{equation} \label{eq:sol_equ_dif}
        u = A.sin\omega t + B.cos\omega t
    \end{equation}

    La ecuación \ref{eq:sol_equ_dif} contiene información relevante para la caracterización dinámica de la estructura. Esta caracterización parte del estudio de los parámetros modales de la misma.

    Entre estos parámetros modales se encuentran: 
        \begin{itemize}
            \item Frecuencia natural: Toda estructura física tiene asociada una frecuencia de vibración natural. Las máquinas, los puentes, los edificios; todas estas estructuras vibran u oscilan al ser perturbadas o removidas de su estado de reposo inicial. Es una propiedad es intrínseca del sistema y depende de su masa, rigidez y amortiguamiento. Todas tienen al menos una frecuencia natural y es posible que tengan múltiples frecuencias de resonancia \citep{irvine2000introduction}. 
            
                Se suele calcular la frecuencia natural de resonancia de un sistema libre usando:

                \begin{equation}
                    f =  \frac{1}{\sqrt{\frac{k}{m}}}
                \end{equation}

            \item Amortiguamiento: Toda estructura comienza a oscilar una vez es removida de su estado de reposo o equilibrio, sin embargo, ese movimiento no es perpetuo. El amortiguamiento se define como la capacidad de disipación de energía que posee la estructura bajo excitaciones externas. Las soluciones a la ecuación \ref{eq:vib_lib}, al añadir el amortiguamiento de tipo viscoso, arrojan 3 posibles casos:
                \begin{enumerate}
                    \item Sistema críticamente amortiguado: El sistema no vibra.
                    \item Subamortiguado o amortiguado subcrítico: Caso más común por la naturaleza de los materiales utilizados en las estructuras. La respuesta del sistema decae con el tiempo de forma exponencial, como se puede ver en la Figura \ref{fig:resp_subamorti}. 
                    
                    \begin{figure}[H]
                        \centering
                        \includegraphics[width = 0.7\textwidth]{imagenes/cap1_marcoteo/respuesta_sist_subamorti.png}
                        \caption{Respuesta ante vibración libre en sistema Subamortiguado \citep{hurtado2000}.}
                        \label{fig:resp_subamorti}
                    \end{figure}

                    \item Sobreamortiguado: Nunca se encuentra esta respuesta en sistemas estructurales por los materiales utilizados.
                \end{enumerate}
            
        \end{itemize}        
\end{itemize}

\subsection{Respuesta en frecuencia}

    Incluir?

\subsection{Daño en estructuras}

El daño a una estructura civil o mecánica puede definirse como todo cambio en las propiedades materiales o geométricas del material que llegan a afectar de forma adversa la confiabilidad y el desempeño actual o futuro del sistema. Por tanto, el daño es una comparación entre el sistema en cuestión en 2 instantes de tiempo distintos \citep{farrar2007introduction}. Estos efectos adversos pueden ser, en el caso estructural, desplazamientos, estrés indeseado en un elemento o vibraciones estructurales indeseadas \citep{chen2018}.

Toda estructura civil, como puentes y edificios, acumulan daño de forma continua a medida que están en servicio y transcurre su vida útil. Este daño puede manifestarse como fracturas, fatiga, socavaciones o desprendimiento del concreto. El daño que no sea detectado puede conducir a una falla estrcutural que a su vez ocasione pérdidas humanas. Por tanto, es imperativo y necesario detectar el daño en una estrcutrura tan pronto como sea posible, \citep{chen2018}.

Entre algunos de los factores que influyen del deterioro de una estructura se encuentran:
    
        \begin{itemize}
            \item Proceso de degradación natural de los materiales.
            \item Corrosión del acero de refuerzo.
            \item Evento sísmico, incendios o condiciones de guerra.
            \item Carga por encima del límite de diseño.
        \end{itemize}
    
Las escalas de tiempo y de extensión del daño son diversas. Por ejemplo, el deterioro por el paso del tiempo bajo ciertas condiciones climáticas es muy lento comparado al daño causado por un evento catastrófico.

\subsection{Principios de la Sismoresistencia}

Una edificación sismorresistente es aquella que está diseñada y construida para soportar las fuerzas causadas por eventos sísmicos. Sin embargo, incluso las edificaciones diseñadas y construidas según las normas sismorresistentes pueden sufrir daños en caso de un terremoto muy fuerte, sin embargo, las normas establecen los requisitos mínimos para proteger la vida de
las personas que ocupan la edificación

Algunas de las características de una estructura sismoresistente son:

        \begin{itemize}
            \item Forma regular.
            \item Bajo peso.
            \item Mayor rigidez.
            \item Buena estabilidad.
            \item Suelo firme y buena cimentación.
            \item Materiales competentes.
            \item Capacidad de disipación de energía.
            \item Fijación de acabados e instalaciones.
        \end{itemize}

En Venezuela las estructuras deben cumplir con la Norma Venezolana COVENIN 1756:2001 (Edificaciones Sismorresistentes).

Se ha observado que al estudiar el comportamiento de las estructuras luego de un evento sísmico, es evidente que cuando se toman en cuenta las normas de diseño sismorresistente dispuestas en la ley y la construcción es debidamente supervisada, los daños estructurales resultan ser considerablemente menores que en las edificaciones en las cuales no se cumplen los requerimientos mínimos indispensables estipulados en la norma, \citep{blanco2012criterios}.

\subsubsection{Importancia de la instrumentación} La instrumentación estructural permite medir y monitorear las acciones y respuestas estructurales ante distintos eventos. Esto proporciona datos en tiempo real sobre el comportamiento dinámico y estático de la estructura, como deformaciones, aceleraciones y desplazamientos, que son fundamentales para evaluar y verificar si la estructura cumple con los criterios de diseño sismoresistente establecidos en la norma.

La instrumentación estructural ayuda a validar los modelos y suposiciones utilizados en el diseño estructural inicial. Al comparar los datos recopilados por la instrumentación durante un evento sísimco con las predicciones del modelo, es posible verificar si la estructura se comporta de acuerdo con las expectativas y si cumple con los criterios de seguridad establecidos en la normas.

Además, el monitoreo continuo de la estructura permite conocer el estado actual de la misma, tema que representa la idea principal del Monitoreo de Salud Estructural, permitiendo a los ingenieros evaluar si se sigue cumpliendo con la norma para luego tomar decisiones y actuar en pro de la seguridad de la edificación.


\section{Salud estructural}

\subsection{Definición}


El proceso de implementar una estrategia de identificación de daño para estructuras civiles, mecánicas o aeroespaciales se conoce como Monitoreo de Salud Estructural (SHM por sus siglas en inglés). Esta estrategia requiere medir las condiciones y el ambiente en el que opera la estructura, además de la respuesta de la misma durante un período de tiempo tomando muestras periódicamente espaciadas, \citep{farrar2007introduction}.


La estrategia del SHM requiere de equipos multidisciplinarios de ingeniería, ya que necesita de una red de sensores que midan las variables de interés, el procesamiento y análisis de los datos obtenidos y posteriormente una prognosis del daño para una eventual toma de decisiones. El objetivo del SHM es proveer, en toda la vida útil de la estructura, un diagnóstico del estado de sus materiales constitutivos, de los diferentes elementos que la componen y de la estructura en sí como el conjunto de todas estas partes. Esto para garantizar que la misma se comporte dentro de los parámetros iniciales de diseño, aunque estos cambien por la acción natural del tiempo, el ambiente y accidentes, \citep{balageas2010structural}.

El resultado de este proceso es información actualizada sobre el estado de la estructura y sobre su capacidad actual para seguir desempeñado la función para la cual fue diseñada.

Según \citet{enckell2006structural}, el SHM se ha convertido en una herramienta muy conocida y utilizada en ingeniería estructural en los últimos años en diferentes países.

\subsection{Reseña histórica}

Las técnicas de detección de daño basadas en vibración tienen sus primeras aplicaciones desde hace cientos de años. En la antigüedad, los constructores golpeaban las estructuras para encontrar espacios vacíos o grietas en elementos de arcilla. La utilidad de estas inspecciones tan simples indicaban que la sofisticación de estos métodos podía proveer información muy valiosa sobre el elemento de interés, sin embargo, esto requiere de instrumentos y herramientas matemáticas que se han desarrollado con el pasar de los años. El auge en el uso de SHM en años recientes es consecuencia de la evolución y miniaturización del hardware computacional actual.


El uso más exitoso del SHM ha sido el monitoreo de la condición de máquinas rotativas, las cuales actualmente han adoptado un enfoque de indetificación de daño sin basarse en un modelo de forma casi exclusiva, \citep{farrar2007introduction}.

En los años 70 la industria petrolera consideró el uso de técnicas basadas en vibración para identificar daños en plataformas costa-afuera, este enfoque se diferenció de las máquinas rotativas al estudiar un sistema en donde la ubicación del daño es desconocida y difícil de instrumentar.

En esa misma época, la comunidad aeroespacial y la \textit{National Eeronautics Space Agency} (NASA), comenzaron a estudiar esta técnica de identificación de daño en los comienzos de la era de lanzamientos espaciales. Este trabajo continúa hoy en día y el \textit{Shuttle Modal Inspection System} (SMIS) se desarrolló para identificar fatiga en distintos componentes de cohetes espaciales reusables, los cuales representan el futuro de esta industria.


Usualmente, los enfoques de estas industrias se basan en comparar modelos analíticos de estructuras sin daño con las mediciones de estructuras con daño, observando principalmente las propiedades modales de las mismas. Se ha observado que cambios en la rigidez en ambos modelos han permitido localizar y cuantificar el daño, \citep{farrar2007introduction}.

Inicialmente, las técnicas no destructivas fueron introducidas en la ingeniería civil a mediados de los años 40, \citep{mohamed2014}. La necesidad principal surgió en determinar propiedades del concreto fresco \textit{in-situ}. Estas técnicas, que buscaban evaluar la homogeneidad y la resistencia del concreto eran en su mayoría pruebas con martillo y pruebas de \textit{pull-out}. A medida que las estructuras envejecieron, los ingenieros necesitaban idear maneras de medir o estimar las propiedades mecánicas de los elementos que consituyen las estructuras, además de detectar daños que no eran fáciles de observar por la envergadura de las estructuras civiles que se han desarrollado en los últimos 150 años. Es ahí, en los años 70, donde surgen nuevas estrategias no destructivas tales como:

    \begin{itemize}
        \item Emisión acústica.
        \item Métodos de ultrasonido y radar.
        \item Termografía.
        \item Métodos basados en vibración
    \end{itemize}

La comunidad de ingeniería civil ha estudiado la identificación de daño basada en vibración en puentes y edificios desde comienzos de los años 80. Las propiedades modales han sido estudiadas por diferentes autores y son las principales características que se analizan al identificar daño. El auge del SHM es tal, que algunos países asiáticos han implementado regulaciones en donde las compañías constructoras deben verificar la salud estructural de los puentes periódicamente. Estas regulaciones han provocado que la investigación e inversión en esta área siga aumentando de forma considerable, \citep{chen2018}. 

\subsection{Línea de trabajo del Monitoreo de Salud Estructural}

Los sistemas de SHM consisten de varios elementos que permiten a los ingenieros tener información sobre el estado de una estructura, entre esos elementos se encuentran:

\begin{itemize}
    \item Sensores.
    \item Sistemas de adquisición de datos.
    \item Sistema de transmisión de datos.
    \item Sistema de procesamiento de datos.
    \item Sistema de manejo y almacenamiento de datos.
    \item Equipo de análisis y toma de decisiones.
\end{itemize}

\begin{figure}[H]
    \centering
    \includegraphics[width = 0.9\textwidth]{imagenes/cap1_marcoteo/Schematics-of-an-on-line-structural-health-monitoring-system-and-technical-challenges.png}
    \caption{Esquema de un sistema de SHM \citep{lijianfoto2015}.}
    \label{fig:esquema_gral_SHM}
\end{figure}


Autores como \citet{rytter1993vibration} y \citet{farrar2007introduction} han esquematizado la estrategia del SHM categorizando el daño en una estructura por niveles de la siguiente forma:

\begin{enumerate}
    \item Nivel I (detección del daño) ¿Presenta daño el sistema? Es una indicación cualitativa de que puede haber daño presente en la estructura.
    \item Nivel II (localización o ubicación del daño) ¿Dónde está presente el daño? Indica la posible localización del mismo. 
    \item Nivel III (clasificación del daño) ¿Qué tipo de daño está presente? Da información sobre el tipo de daño.
    \item Nivel IV (alcance/grado/extensión del daño) ¿Cuál es el alcance del daño? ¿Qué tan grave es? Da un estimado del alcance.
    \item Nivel V (prognosis del daño) ¿Cuánta vida útil le queda a la estructura? Da un estimado de la seguridad de la estructura.
\end{enumerate}

En la mayoría de los casos, para alcanzar el nivel final es necesario obtener información sobre los niveles previos. Esto indica que a medida que se sube de nivel se tiene un mayor conocimiento sobre el estado de la estructura.

De acuerdo a \citet{chen2018}, los primeros dos niveles, detección y localización, generalmente pueden alcanzarse usando métodos de detección basados en vibración para obtener mediciones sobre la respuesta dinámica de la estructura.

Por su parte, \citet{chen2018} describe el proceso de SHM en general como:

\begin{enumerate}
    \item Observación.
    \item Evaluación.
    \item Calificación.
    \item Gestión.
\end{enumerate}

La estrategia de Monitoreo de Salud Estructural podría resumirse en el siguiente diagrama:

\begin{figure}[H]
    \centering
    \includegraphics[width = \textwidth]{imagenes/cap1_marcoteo/Diagrama SHM timeline.png}
    \caption{Diagrama general del proceso de SHM.}
    \label{fig:diag_SHM}
\end{figure}

\subsection{Criterios de evaluación}

Como se definió anteriormente, el daño estructural puede tener distintas causas y formas. Lo que se sabe con certeza es que una estructura, una vez entra en funcionamiento, análogo a los seres humanos al nacer, estará sujeta a envejecimiento natural y a condiciones adversas. Ahora bien, en el caso del SHM, surge la siguiente pregunta ¿Qué se debe medir para poder detectar este daño?. 

Numerosos autores concluyen que uno de los indicativos de daño de una estructura viene dado por los parámetros modales definidos anteriormente, frecuencia y amortiguamiento. Esta relación entre los parámetros modales y el daño viene dada por la premisa de que todo daño presente en la estructura se reflejará en un cambio en las propiedades dinámicas de la misma. \citet{worden2009modal}, comprobó la relación entre el cambio progresivo en todas las frecuencias naturales de 15 vigas estudiadas a las cuales se les introdujo un daño relacionado con un cambio del Módulo de Young, el cual, como fue mencionado anteriormente, provee un indicativo de la rigidez de un elemento.

Anterioremente en la ecuación \ref{eq:vib_lib} se definió un sistema con un grado de libertad (DOF por sus siglas en ingles), sin embargo, en la realidad las estructuras tienen múltiples grados de libertad, por lo que es conveniente modelarlas de estar forma para obtener resultados más precisos. En el caso de los sistemas \textit{n-dregrees of freedom} (DOF por sus siglas en ingles), la ecuacion de movimiento que describe la dinámica del sistema en vibración libre vendrá dada por:

\begin{equation} \label{eq:ecu_movimiento}
    M\ddot{u} + C\dot{u} + Ku = 0
\end{equation}

Donde M, C y K representan las matrices de masa, amortiguamiento y rigidez de la estructura, respectivamente.

Si se asume un sistema sin amortiguamiento, a fines de estudiar el efecto que tiene sobre la rigidez un cambio en los parámetros modales,  de la ecuación \ref{eq:ecu_movimiento} se obtiene: 

\begin{equation} \label{eq:ecu_movimiento_sindamp}
    M\ddot{u} + Ku = 0
\end{equation}

Si se asume una solución oscilatoria pura, por ser un sistema sin amortiguamiento:

\begin{equation} \label{eq:sol_ecu_motion}
    u = ve^{jwt}
\end{equation}

Al derivar, sustituir y despejar en la ecuación \ref{eq:ecu_movimiento_sindamp} se obtiene:

\begin{equation} \label{eq:eig_problem}
    (K - \lambda M)\phi = 0
\end{equation}

Esta ecuación \ref{eq:eig_problem} representa claramente un problema de autovalores, donde $\lambda$ representa los autovalores asociados a las frecuencias naturales del sistema y $\phi$ representa el autovector de desplazamiento.

Si se introduce un pequeño cambio $\Delta K$ con perturbaciones similares en los otros parámetros:

\begin{equation}
    [(K - \Delta K) - (\lambda - \Delta\lambda)(M - \Delta M)](\phi - \Delta\phi) = 0
\end{equation}

El daño estructural suele venir asociado a un cambio en la rigidez, mas no a cambios en la masa de la estructura, por lo que se asume $\Delta M = 0$, \citep{hearn1991modal}. A su vez, se tiene que $(K - \lambda M)\phi = 0$. \citet{mohamed2014}, \citet{shi1998structural} y \citet{hearn1991modal} desarrollan estas ecuaciones obteniendo la siguiente relación:

\begin{equation} \label{eq:relacion_final}
    \Delta\lambda = \phi^T \Delta K \phi
\end{equation}

De la ecuación \ref{eq:relacion_final} se observa que cambios en los autovalores $\lambda$ que representan las frecuencias naturales, y en los autovectores $\phi$ (formas modales) están directamente relacionados con cambios en la matriz de rigidez (K) del sistema. De aquí surge el interés en monitorear los parámetros modales como indicadores de daño estructural. Es importante recalcar que estos cambios son indicativos de daño global, mas no de la localización del mismo, para lo que se necesitan otras técnicas, \citep{mohamed2014}.

\subsection{Variables de interés}

Tomando en cuenta la relación entre los parámetros modales y el daño presente en una estructura, es preciso definir las variables de interés para el monitoreo de la salud estructural de una estructura. Si bien existen distintas varibales que permiten obtener información valiosa sobre la estructura en estudio, algunas de estas no proporcionan información global del daño, como es el caso de las formas modales y la deflección local \citep{rytter1993vibration}. Sin embargo, estas mediciones proveen indicativos de la ubicación del daño, por lo que pueden constituir parte del sistema de monitoreo en una etapa más avanzada, es decir, una vez el daño fue detectado. A continuación se presentan las más relevantes para el daño global:

    \begin{itemize}
        \item Frecuencias naturales y amortiguamiento: Los parametros modales de la estructura están ligados de forma directa al estado de la misma. El deterioro en una edificiación induce cambios en la rigidez estructural, como se observa claramente en la ecuación \ref{eq:relacion_final}. El daño puede tener efectos distintos en cada modo o cada frecuencia de vibración, por lo que es importante no ubicar los sensores sobre los nodos modales, ya que experimentos han demostrado la ineficacia en las mediciones. Usualmente, el daño se refleja como una disminución en las frecuencias naturales afectadas, aunque se han observado casos de aumento en las frecuencias de vibración en estructuras de concreto pretensado, \citep{rytter1993vibration}.
        
        A su vez, el amortiguamiento varía al introducir daño en la estructura, puesto que su capacidad de disipar energía se ve afectada. Usualmente, los investigadores observan un aumento en el amortiguamiento a medida que el daño aumenta, como se ha demostrado experimentalmente por autores como \citet{hearn1991modal} y \citet{rytter1993vibration}.

        \item Temperatura y humedad: En los sistemas de monitoreo, la detección del daño estructural puede tomar períodos de tiempo considerables, durante los cuales las caracterpisticas sujetas a temperatura y humedad, sufren cambios que afectan la respuesta estructural.
                
        Es evidente que las condiciones climáticas contribuyen con el deterioro de las edificaciones. A pesar de esta conclusión, relacionar las condiciones climáticas con el daño introducido usando mediciones ambientales es difícil. La medición de estas variables suele tomarse en cuenta para poder cuantificar el cambio que producen estas condiciones en los demás indicadores de daño. \citet{rytter1993vibration} observó que la humedad y temperatura afectaban las mediciones de amortiguamiento. Por su parte, \citet{mohamed2014}, observó que las frecuencias naturales de barras y vigas disminuían a medida que aumentaba la temperatura. A su vez, \citet{sohn2007effects} determinó que cuando hay humedad, los puentes de hormigón absorben una cantidad considerable de humedad, lo que aumenta sus masas y altera sus frecuencias naturales.
        
        \item Inclinación y desplazamiento: n practical applications of structural monitoring, the most common is the measurement of linear displacements, which reflect in a very good and direct way the behavior of the structural element / structure or a part of the structure.
        
        Inclinometers are used to measure inclination (tilt) of structural components due to distress in the system. For example, they are often utilised to assess fixity of bridge girders at supports and to monitor longterm movements of bridge piers, abutments and girders.

        Deflection is a very important index for bridge structures, because it not only affects driving comfort, but also reflects the overall response of the bridge. Various factors could contribute to deflection increase during bridge service life, such as concrete creep, steel corrosion, prestress loss, and crack growth. The increase of structural deflection, however, will in turn accelerate the damage accumulation process. Therefore, monitoring bridge deflection is of great significance in the field of structural health monitoring (SHM) to provide early warnings of possible structural changes, damage, or deterioration.

        In several industries during the last few decades, inclinometer sensors have been employed extensively. In fact, in the civil engineering sector, inclinometers were initially used for geotechnical purposes [80]. Improvements in sensor accuracy over time have made it possible to use inclinometers in other areas of civil engineering, such as monitoring the structural health of bridges [79].
        


    \end{itemize}
    

\subsection{Consideraciones y desafíos}

\section{Sensores}

\subsection{Definición y tipos de sensores}

\subsection{Sensores de interés para el Monitoreo de Salud Estructural}

    \begin{itemize}
        \item Acelerómetros e Inclinómetros:
        \item Desplazamiento:
        \item Temperatura:
        \item Humedad:
    \end{itemize}

\subsection{Sensores inteligentes}

\section{Microcontroladores}%

%==================================================================
\chapter{MARCO METODOLÓGICO}\label{CAP:marco_met}
%\markboth{Tu Segundo Capítulo}{Tu Segundo Capítulo}%
En este capítulo se describirá el diseño del sensor inteligente, describiendo detalles de hardware y software del mismo.

\section{Descripción del sistema}

El sistema diseñado integra un conjunto de sensores y módulos que permiten medir variables dinámicas y cuasi-estáticas de un sistema estructural, enviarlas a larga distancia y posteriormente procesarlas y almacenarlas.

Una vez el sistema es encendido, calibra de forma automática todos los sensores, con especial énfasis en eliminar el offset de los 3 ejes del acelerómetro y también calibrar el magnetómetro y giróscopo del acelerómetro de 9 grados de libertad. El sistema ejecutará las tareas de calibración cada 2 registros de datos enviados con éxito.

Luego de la calibración y ajuste de la hora y fecha,  el sistema comienza a medir de forma continua la aceleración triaxial, inclinación, temperatura, humedad y estima la inclinación con sensores electrónicos de bajo consumo para posteriormente enviar los datos de forma inalámbrica a la estación base, que puede estar ubicada a más de 150 metros de distancia. Allí son recibidos, decodificados y pre-procesados para luego ser subidos vía inaálmbrica a un computador que sirve de servidor en donde se almacenaran los datos, siendo controlado el sistema desde esta misma estación base, pudiendo enviar comandos de adquisición de datos de forma remota. A su vez, se desarrolló una interfaz gráfica que permite visualizar los datos obtenidos, realizar peticiones de datos a distancia, observar sus características principales, acceder al histórico de datos recogidos por el sensor en una fecha específica, descargar los datos y ejecutar post-procesamiento a los mismos para evañuar las variables de interés para el monitoreo de la salud estructural.

Puesto que la estación base está conectada a internet, el microcontrolador ubicado en la estación adquiere la fecha y hora actualizada utilizando un servidor del protocolo NTP (\textit{Network Time Protocol}) y sincroniza su RTC (\textit{Real Time Clock}) interno con estos valores. Una vez obtenida la fecha y hora, espera el comando de inicio del sensor inteligente que envía de forma automática una vez se enciende y calibra sus sensores, la cual le indica a la estación base que debe enviar la fecha y hora actual. De esta forma se sincronizan los relojes internos de ambos y permite al sensor inteligente tener la fecha y hora a la cual tomó todo registro, enviando esta información como parte del registro de datos. La fecha y hora se envía bajo el formato \textit{UNIX Epoch}, el cual indica la cantidad de segundos transcurridos desde el 1 de Enero de 1970.
	
En la estación base, el sistema se conectará a internet a través de la red WiFi, enviará los datos recibidos al computador en la estación base, siendo esta herramienta la encargada de pre-procesar los datos recibidos vía inalámbrica y convertirlos a un formato adecuado para poder ser almacenados en el computador y posteriormente procesados usando librerías de análisis numérico. La interfaz de usuario podrá estar disponible para todos los usuarios de la red local, siempre que tengan los servicios necesarios instalados en local.

El comportamiento de los sistemas estructurales a estudiar condiciona los rangos e intervalos de medición de los sensores, con un límite superior cercano a los 3 g en un movimiento telúrico considerable. Por su parte, la temperatura, humedad e inclinación suelen considerarse variables cuasi-estáticas, permitiendo que el sistema mida estas variables con intervalos lo suficientemente largos para poder monitorear cambios considerables en las mismas y posteriormente correlacionarlos a las condiciones estructurales. La frecuencia de muestreo de aceleración es deseable que est

El sistema en su funcionamiento normal está ejecutando las siguientes tareas principales:

\begin{itemize}
    \item Adquisición continua (envío programado a ciertas horas del día o ante eventos importantes): El sensor inteligente mide constantemente las variables de interés y envía periódicamente, a ciertas horas del día programadas con antelación, un registro de datos a la estación base. Al estar midiendo de forma continua, está atento para generar una interrupción o alerta ante algún valor de aceleración o inclinación que esté por encima de algún límite escogido con anterioridad que dependerá en gran medida de la estructura a monitorear y su naturaleza, aunque se escogió por default el valor de 2 m/s como generador de alerta, basado en la \textit{Escala de Mercalli} \citep{mercalli}, la cual indica que este valor de aceleración (que corresponde a 0.20 g) equivale a un sismo fuerte con daño moderado. El sistema envía de forma automática un registro de datos del acontecimiento importante que generó la alerta.
	\item Escuchando petición de trama de datos inmediata (Envía trama ante request/query de estación base): Al recibir desde la estación base el comando de adquisición de datos, ejecutado desde la interfaz de control por el operador, el sistema comienza a tomar un registro de datos de forma inmediata, siempre y cuando el mismo no haya detectado previamente un evento importante que superara los límites establecidos, y lo envía a la estación base permitiendo que el usuario obtenga información del sistema en el instante en el cual se realiza la petición de los datos. Una vez es enviado y recibido con éxito la trama de datos, el sistema regresa a su estado anterior, tomando datos de forma continua y esperando alertas, comandos u horas programadas.

\end{itemize}

\subsection{Diagrama de funcionamiento del sensor inteligente}

\section{Selección de componentes}

Una vez obtenida una base teórica sobre estos temas, se procedió a escoger el hardware y los protocolos de comunicación más adecuados para llevar a cabo el objetivo de diseñar un sensor inteligente para aplicaciones de monitoreo de salud estructural. 
	

\subsection{Protocolo de comunicaciones}

Para el protocolo de comunicación se buscaron protocolos capaces de manejar los datos recogidos por los sensores de forma eficaz y confiable. En este caso se refiere al protocolo utilizado para enviar los datos desde el sensor inteligente hasta la estación base. Sin embargo, se utilizaron distintos protocolos para la comunicación de los sensores con el microcontrolador y a su vez para comunicar la estación base con el receptor de datos.
	
Para el envío de datos a la estación base, preferiblemente el protocolo debía ser capaz de funcionar en rangos de distancia amplios, permitiendo que el sensor inteligente esté ubicado lejos de la estación base en donde serán monitoreadas las variables de interés. En primer lugar se escogieron algunos protocolos de forma preliminar, para luego estudiar a fondo sus características. Estos protocolos y sus características se resumen en la tabla \ref{tab:protocolos}:

\begin{table}[H]
    \centering
    \caption{Comparación entre protocolos de comunicación inalámbrica, \citep{IoTCompare} y \citep{LPWANCompare}}
    \label{tab:protocolos}
    \resizebox{\textwidth}{!}{%
    \begin{tabular}{|c|c|c|c|c|}
    \hline
    \textbf{Protocolo} & \textbf{Frecuencia} & \textbf{Rango} & \textbf{Velocidad} & \multicolumn{1}{l|}{\textbf{Consumo de energía}} \\ \hline
    \textbf{Zigbee} & 784 MHz/2.4 GHz & 100 m - 300 m & 250kbps-500kbps & Bajo \\ \hline
    \textbf{Sigfox} & 868 MHz/915 MHz & 3km - 10km & 100 bps & Bajo \\ \hline
    \textbf{NB-IoT} & LTE & 1km - 10km & 200 kbps & Bajo \\ \hline
    \textbf{WiFi} & 2.4 GHz/5.8 GHz & 100m & 54Mbps/1.3Gbps & Alto \\ \hline
    \textbf{Bluetooth} & 2.4 GHz & 10m - 100m & 720 kbps & Bajo \\ \hline
    \textbf{LoRa} & \begin{tabular}[c]{@{}c@{}}430 MHz/433 MHz\\ /868 MHz/915 MHz\end{tabular} & 15 km-30 km & 0.3kbs hasta 50 kbps & Bajo \\ \hline
    \end{tabular}%
    }
\end{table}

Con base en esta información, se escogió el protocolo LoRa como el más adecuado para el sensor inteligente, debido a su amplio rango y bajo consumo de energía. En general, la vasta mayoría de los módulos están basados en los chips fabricados por Semtech (los precursores del protocolo LoRa) SX126X y SX127X, por tanto se compararon ambas tecnologías:

% Please add the following required packages to your document preamble:
% \usepackage{graphicx}
\begin{table}[H]
    \centering
    \caption{Comparación entre módulos LoRa del fabricante Semtech \citep{datasheetSemtech}.}
    \label{tab:moduloslora}
    \resizebox{\textwidth}{!}{%
    \begin{tabular}{|c|c|c|c|c|c|}
    \hline
    \textbf{Módulo} & \textbf{Modem} & \textbf{Amplificador} & \textbf{Corriente RX} & \multicolumn{1}{l|}{\textbf{Sensibilidad}} & \textbf{Velocidad (bit rate)} \\ \hline
    \textbf{SX1261/62/68} & LoRa y FSK & \begin{tabular}[c]{@{}c@{}}+15 dBm - \\ +22 dBm\end{tabular} & 4.6 mA & -148 dBm & \begin{tabular}[c]{@{}c@{}}62.5 kbps \\ - 300 kbps\end{tabular} \\ \hline
    \textbf{S1272/73} & LoRa & +14 dBm & 10 mA & -137 dBm & 300 kbps \\ \hline
    \textbf{S1276/77/78/79} & LoRa & +14 dBm & 9.9 mA & -148 dBm & 300 kbps \\ \hline
    \end{tabular}%
    }
    \end{table}

\subsection{Sensores}

Para la selección de los sensores a utilizarse, es preciso definir las necesidades de un sistema de adquisición para sistemas estructurales, siendo su comportamiento el que define las características de los instrumentos de medición.

\subsubsection{Aceleración} 

En el caso de la medición de aceleración, el sensor inteligente debe contar con un sensor con las siguientes características:

\begin{itemize}
    \item Bajo nivel de ruido.
    \item Compensación de temperatura.
    \item Ancho de banda dentro del rango deseado en sistemas estructurales.
    \item Buena resolución.
    \item Suficientes grados de libertad.
    \item Compatibilidad con microcontroladores disponibles en el mercado.
    \item Bajo consumo
\end{itemize}

\subsubsection{Temperatura y humedad}

Para la medición de temperatura y humedad, el sensor inteligente debe contar con un sensor con las siguientes características:

\begin{itemize}
    \item Rango de trabajo dentro de las condiciones en las que se encuentre la estructura.
    \item Buena resolución y sensibilidad.
    \item Compatibilidad con microcontroladores.
    \item Bajo consumo.
\end{itemize}

\subsubsection{Inclinación}

Para la medición de inclinación, la cual, como se explica en la sección \ref{subsec:sensorfusion} el sensor inteligente debe contar con un sensor que cuente con las siguientes características:

\begin{itemize}
    \item Acelerómetro, giróscopo y magnetómetro incorporado (\textit{Inertial Measurement Unit}).
    \item Bajo nivel de ruido.
    \item Buena resolución y sensibilidad.
    \item Compatibilidad con microcontroladores.
    \item Bajo consumo.
\end{itemize}

\subsection{Microcontroladores}

En el caso de los microcontroladores, se buscaron microcontroladores capaces de obtener los datos provenientes de los sensores, procesarlos, almacenarlos temporalmente y posteriormente hacer uso del módulo de comunicaciones para su envío, usando este mismo módulo para recibir mensajes o comandos. También se tomaron en cuenta las capacidades de conexión inalámbrica de cada microcontrolador, su documentación y soporte por parte de los fabricantes, y por último su compatibilidad con los distintos frameworks, librerías y entornos de programación disponibles para los sensores y módulos, los cuales disminuyen el tiempo necesario para poner en marcha el funcionamiento del sistema. Se estudiaron las características de distintas placas de desarrollo, para posteriormente escoger el más adecuado. A continuación, en la tabla \ref{tab:microstabla}, se presentan las placas de desarrollo consideradas de forma preliminar y sus características principales:

% Please add the following required packages to your document preamble:
% \usepackage{graphicx}
\begin{table}[H]
    \centering
    \caption{Comparación entre placas de desarrollo basadas en MCU}
    \label{tab:microstabla}
    \resizebox{\textwidth}{!}{%
    \begin{tabular}{|c|c|c|c|c|c|c|}
    \hline
    \textbf{Placa} & \textbf{Procesador} & \textbf{Velocidad de reloj} & \textbf{RAM (kB)} & \multicolumn{1}{l|}{\textbf{ROM (kB)}} & \textbf{GPIO} & \textbf{Conectividad} \\ \hline
    \textbf{Teensy 4.0} & ARM M7 & 600 MHz & 1024 & 2048 & 40 & - \\ \hline
    \textbf{\begin{tabular}[c]{@{}c@{}}Raspberry Pi \\ Pico W\end{tabular}} & Dual ARM-M0 & 133 MHz & 264 & 2048 & 26 & WiFi \\ \hline
    \textbf{STM32 Discovery} & ARM M4 & 168 MHz & 192 & 1024 & 82 & - \\ \hline
    \textbf{STM32 Nucleo} & \begin{tabular}[c]{@{}c@{}}ARM M0 -\\ ARM M4\end{tabular} & \begin{tabular}[c]{@{}c@{}}84 MHz -\\ 180 MHz\end{tabular} & \begin{tabular}[c]{@{}c@{}}96 - \\ 128\end{tabular} & 512 & 50 & - \\ \hline
    \textbf{Espressif ESP32} & Dual Xtensa LX6 & 240 MHz & 520 & 4096 & 34 & WiFi/BT(BLE) \\ \hline
    \textbf{STM32 Blackpill} & ARM M4 & 100 MHz & 128 & 512 & 37 & - \\ \hline
    \end{tabular}%
    }
    \end{table}

\subsection{Diagramas de selección de componentes:}
\subsubsection{Selección del sensor de aceleración}

Con base en la información recopilada de distintos módulos de acelerómetros, se observa en la figura \ref{fig:arañaacl} que el MPU6050 de Invensense se ajusta a las necesidades del acelerómetro necesario para tomar los registros de vibración. Otro módulo del mismo fabricante, el MPU9250 también presenta un buen desempeño. Sin embargo, el MPU6050 tiene un mejor precio y ofrece funcionalidades similares, por lo cual fue el escogido para el prototipo de pruebas.


\begin{figure}[H]
    \centering
    \includegraphics[width = 0.7\textwidth]{imagenes/cap2_marcometod/ArañaACL.png}
    \caption{Diagrama de araña para selección de acelerómetro.}
    \label{fig:arañaacl}
\end{figure}

\subsubsection{Selección del sensor de temperatura y humedad}

Se observa en la figura \ref{fig:arañatemphum} que la mayoría de los sensores de temperatura y humedad, los cuales suelen estar integrados en un mismo módulo, no distan mucho en desempeño entre sí, sin embargo, entre ellos destaca el BME280 del reconocido fabricante Bosch, el cual cuenta con buena resolución además de una excelente documentación y librerías para distintos microcontroladores. El módulo SHT31 muesta potencial por su resolución, en este caso se descarta por la poca disponibilidad del módulo pero es una buena opción para futuras implementaciones. Es por esta razón que se escogió el BME280 para llevar a cabo las mediciones de las variables ambientales en el prototipo de pruebas.

\begin{figure}[H]
    \centering
    \includegraphics[width = 0.7\textwidth]{imagenes/cap2_marcometod/ArañaTempHum.png}
    \caption{Diagrama de araña para selección de sensor de temperatura y humedad.}
    \label{fig:arañatemphum}
\end{figure}

\subsubsection{Selección del acelerómetro para estimación de ángulos}

Utilizando el mismo análisis que se llevó a cabo en la figura \ref{fig:arañaacl}, se modifica para fines de estimación de ángulo tomando en cuenta las premisas de la sección \ref{subsec:sensorfusion} para la fusión de sensores. Por tanto, con base en la figura \ref{fig:arañaimu} se escoje el módulo MPU9250, el cual cumple con la función de ser una IMU de 9 grados de libertas, siendo ideal para la estimación de ángulos en el prototipo.

\begin{figure}[H]
    \centering
    \includegraphics[width = 0.7\textwidth]{imagenes/cap2_marcometod/ArañaIMU.png}
    \caption{Diagrama de araña para selección de unidad de medición inercial.}
    \label{fig:arañaimu}
\end{figure}

\subsubsection{Selección del módulo de comunicaciones}

Se ubicaron módulos de comunicaciones LoRa que fueran compatibles con el microcontrolador escogido, con documentación disponible y cuyas características se ajustaran a las necesidades del proyecto a llevar a cabo. Si bien el protocolo es el que condiciona las características de la gran mayoría de los módulos de comunicación del protocolo en cuestión, se buscó un módulo con facilidad de conexión e intercomunicación con el microcontrolador. Se observa en la figura \ref{fig:arañacomm} que la mayoría de los módulos tienen un desempeño similar, esto se debe a que están basados en distintas versiones de los módulos estudiados en la tabla \ref{tab:moduloslora}. Sin embargo, el condicionante es la disponibilidad y precio de los mismos, siendo el RA-02 de Ai-Thinker el seleccionado en este caso.

\begin{figure}[H]
    \centering
    \includegraphics[width = 0.7\textwidth]{imagenes/cap2_marcometod/ArañaTransmisores.png}
    \caption{Diagrama de araña para selección del módulo de comunicaciones.}
    \label{fig:arañacomm}
\end{figure}

\section{Detalle del diseño}

\subsection{Descripción del hardware}

\subsubsection{Sensor inteligente:}

\subsubsection{Estación base:}

\subsection{Descripción del software}

\subsubsection{Sensor inteligente:}

\subsubsection{Estación base:}

Flujo de NodeRED de parseo de datos

\subsubsection{Aplicación de monitoreo y control:}

\subsection{Diagrama de bloques del sistema}

Con drawio

\subsection{Diagrama de funcionamiento del sistema}

UML con drawio%

%==================================================================
\chapter{PRUEBAS Y RESULTADOS}\label{CAP:pruebas}
%\markboth{Tu Segundo Capítulo}{Tu Segundo Capítulo}%

En el siguiente capítulo se presentan las pruebas y los resultados obtenidos a partir de la metodología descrita en el capítulo anterior. Se presentan las pruebas preliminares de comunicaciones para determinar las características óptimas del canal, las pruebas para la estimación de la inclinación que permitieron escoger el método de estimación apropiado y finalmente las pruebas de funcionamiento del prototipo.

\section{Pruebas de comunicaciones}



Una vez escogido el hardware propuesto en la sección \ref{sec:componentes}, se llevaron a cabo una serie de pruebas  para comprobar el funcionamiento de los módulos de comunicaciones y para evaluar las caracacterísticas más adecuadas para el canal de comunicaciones. 

Como se definió en el apartado \ref{sec:protocololora}, el protocolo LoRa implementado en el módulo Ra-02 de Ai-Thinker requiere fijar el valor de los siguientes parámetros:

\begin{itemize}
    \item Factor de propagación.
    \item Ancho de banda.
    \item Potencia.
    \item Tasa de codificación.
\end{itemize}

La prueba consistió en el envío de un paquete de bytes a una distancia de aproximadamente 115 metros, sin línea de vista y con obstáculos de acero y concreto, como se observa en el mapa de la figura \ref{fig:mapalora}.

\begin{figure}[H]
    \centering
    \includegraphics[width = 0.9\textwidth]{imagenes/cap3_resultados/Pruebas LoRa/MapaLora.png}
    \caption{Vista en mapa de distancia máxima de pruebas usando módulo SX1278.}
    \label{fig:mapalora}
\end{figure}

Se modificaron los parámetros del canal y se midió la tasa de paquetes con errores o paquetes corruptos en el receptor. Se realizaron pruebas con los siguientes parámetros fijos:

\begin{itemize}
    \item Frecuencia de operación: 433 MHz.
    \item Tasa de codificación:
    \item Potencia: 10 dBm.
    \item Longitud del preámbulo: 8 bytes.
    \item Tamaño de la carga útil (payload): 128 bytes.
\end{itemize}

Las pruebas se realizaron haciendo uso del módulo Ra-02 de Ai-Thinker, el cual se conectó a un microcontrolador ESP32. En el microcontrolador se implementaron rutinas de envío y recepción de datos medianto rutinas de interrupción (ISR por sus siglas en inglés).

Estos parámetros son recomendados por el fabricante del módulo y se mantuvieron constantes durante las pruebas. Estos se confirmaron tras pruebas preliminares en las cuales se observó que para \textit{payloads} mayores a 128 bytes el porcentaje de paquetes cuyo CRC era erróneo (data corrupta) aumentaba considerablemente, siendo 128 bytes un valor que disminuía esta proporción. Los parámetros que se modificaron fueron el factor de propagación, el ancho de banda y el período de envío entre paquetes. Los resultados de las pruebas se presentan en la tabla \ref{tab:resultadoslora}.

% Please add the following required packages to your document preamble:
% \usepackage{graphicx}
\begin{table}[H]
    \centering
    \caption{Resultados de pruebas realizadas con módulo de comunicaciones Ra-02.}
    \label{tab:resultadoslora}
    \resizebox{\textwidth}{!}{%
    \begin{tabular}{|c|c|c|c|c|}
    \hline
    \textbf{Configuración} & \textbf{F. de Propagación} & \textbf{Ancho de banda} & \textbf{Período} & \textbf{Tasa de paquetes perdidos} \\ \hline
    1 & 9 & 125 kHz & 500 ms & 105/500 \\ \hline
    2 & \textbf{7} & 250 kHz & 150 ms & 1/500 \\ \hline
    3 & \textbf{8} & 125 kHz & 150 ms & 300/500 \\ \hline
    4 & \textbf{8} & 250 kHz & 500 ms & 2/500 \\ \hline
    5 & \textbf{8} & 250 kHz & 200 ms & Error de CRC \\ \hline
    6 & \textbf{7} & 250 kHz & 250 ms & 2/500 \\ \hline
    7 & \textbf{7} & 250 kHz & 200 ms & 2/500 \\ \hline
    8 & \textbf{7} & 250 kHz & 150 ms & 2/500 \\ \hline
    9 & \textbf{7} & 250 kHz & 100 ms & Error de CRC \\ \hline
    \end{tabular}%
    }
\end{table}

Basados en estos resultados, se escogió la configuración 8 para las pruebas de comunicaciones, ya que es la que presenta la menor tasa de paquetes corruptos a la velocidad más alta de envío de paquetes. Esta configuración se utilizó para las pruebas de estimación de inclinación y para las pruebas de funcionamiento del prototipo.

\section{Pruebas para estimación de inclinación}

Para escoger el método de estimación de ángulos, se compararon los siguientes:

\begin{itemize}
    \item Cálculo trigonométrico a partir de mediciones de acelerómetro.
    \item Filtro de Kalman.
    \item Filtro de Madgwick.
\end{itemize}

 Para observar el comportamiento de cada método, se observaron los resultados en un graficador de datos seriales disponible de forma libre, llamado \textit{TelePlot}, creado por Alexander Brehmer. Este funciona como extensión a Visual Studio Code y permite graficar los datos provenientes del puerto serial escogido a una velocidad de transmisión fija. Se realizaron pruebas con cada método y se compararon los resultados obtenidos. 

 \begin{figure}[H]
    \centering
    \includegraphics[width = 0.8\textwidth]{imagenes/cap3_resultados/Pruebas ACL/Inclinacion/Comparacion entre Metodo1 (ACL) y Metodo 2 (Kalman) ante vibraciones.png}
    \caption{Entorno TelePlot.}
    \label{fig:teleplot}
\end{figure}

Se realizaron pruebas ante vibración ambiental, ante vibraciones forzadas como golpes y ante movimientos sostenidos para ver el cambio en el ángulo ejecutando los 3 algoritmos al mismo tiempo y haciendo uso del acelerómetro de 9 grados de libertad MPU9250 de Invensense. 

Los resultados gráficos de las pruebas ante vibración ambiental se presentan en la figura \ref{fig:pruebasinclinacion}.

 \begin{figure}[H]
    \centering
    \includegraphics[width = 0.8\textwidth]{imagenes/cap3_resultados/Pruebas ACL/Inclinacion/PruebaInclinacion.png}
    \caption{Comparación entre métodos de estimación de inclinación en vibración ambiental.}
    \label{fig:pruebasinclinacion}
\end{figure}

Al golpear y soplar cerca del sensor, se obtuvieron los siguientes resultados:

\begin{figure}[H]
    \centering
    \includegraphics[width = 0.9\textwidth]{imagenes/cap3_resultados/Pruebas ACL/Inclinacion/Comparacion M1 M2 M3 (Maggwick) ante vibraciones.png}
    \caption{Comparación entre métodos de estimación de inclinación ante vibraciones.}
    \label{fig:pruebasinclinacion2}
\end{figure}


Por último, se observó el desempeño de los algoritmos ante movimientos sostenidos que cambiaran el ángulo en el eje de interés:

\begin{figure}[H]
    \centering
    \includegraphics[width = 0.8\textwidth]{imagenes/cap3_resultados/Pruebas ACL/Inclinacion/Comparacion M1 M2 M3 (Maggwick) ante movimiento aleatorios.png}
    \caption{Comparación entre métodos de estimación de inclinación ante movimientos sostenidos.}
    \label{fig:pruebasinclinacion3}
\end{figure}

Se observa claramente en las figuras \ref{fig:pruebasinclinacion}, \ref{fig:pruebasinclinacion2} y \ref{fig:pruebasinclinacion3} que el filtro de Madgwick es el que presenta el mejor desempeño ante vibraciones, golpes y movimientos sostenidos, ya que suaviza las variaciones bruscas en el ángulo. De igual forma, ante vibración ambiental se observa que el filtro de Madgwick es el que presenta un comportamiento más estable, viéndose incluso la resolución en términos de LSB (\textit{Least Significant Bit}) del acelerómetro.

Por lo tanto, se escogió el filtro de Madgwick para la estimación de inclinación en el prototipo.

\section{Pruebas de funcionamiento del prototipo}

Para probar el funcionamiento del prototipo y comparar los resultados obtenidos con un equipo comercial, se llevó a cabo un estudio de vibración sobre una estructura de acero ubicada en el Instituto de Materiales y Modelos Estructurales (IMME) de la Universidad Central de Venezuela. 

La estructura es de tipo aporticada, con una altura de 2,20 metros y una longitud de 2 metros. Cuenta con 4 columnas de acero y distintas vigas con perfil "C" o de canal. El techo de la estructura se encuentra en voladizo, es decir, no está soportado por ninguna columna. La estructura se encuentra en el edificio norte del IMME y se suele utilizar para hacer ensayos de permeabilidad del concreto. Se escogió esta estructura debido a la facilidad para la instrumentación de la misma para realizar las pruebas, siendo esta de poca altura y estando ubicada en un espacio abierto y de fácil acceso por el personal del IMME.

El equipo de medición utilizado para la comparación está basado en la tarjeta de adquisición de datos de \textit{National Instruments} PCI-6221, cuyas características se describen en la tabla \ref{tab:specs6221}, y que puede observarse en la figura  en conjunto con 

%ESPECIFICACIONES DEL DAQ6221

\begin{figure}[H]
    \centering
    \includegraphics[width = 0.8\textwidth]{imagenes/cap3_resultados/Ensayos/NationalInstruments_PCI6221.jpg}
    \caption{Tarjeta de adquisición de datos PCI-6221 de National Instruments.}
    \label{fig:DAQ6221}
\end{figure}

\begin{figure}[H]
    \centering
    \includegraphics[width = 0.8\textwidth]{imagenes/cap3_resultados/Ensayos/National_Instruments_SCB_68.jpg}
    \caption{Tarjeta de conexiones SCB-68 de National Instruments (Artisan Technology).}
    \label{fig:SCB68}
\end{figure}

% Please add the following required packages to your document preamble:
% \usepackage{graphicx}
\begin{table}[H]
    \centering
    \caption{Especificaciones de la tarjeta de adquisición PCI-6221 de National Instruments}
    \label{tab:specs6221}
    \resizebox{\textwidth}{!}{%
    \begin{tabular}{|c|c|}
    \hline
    \textbf{Parámetro} & \textbf{Valor} \\ \hline
    Número de canales & 8 diferenciales o 16 de un solo canal \\ \hline
    Resolución del ADC & 16 bits \\ \hline
    Tasa de muestreo & 250 kS/s \\ \hline
    Rango de entrada & $\pm 0.2 V, \pm 1 V, \pm 5 V, \pm 10 V$ \\ \hline
    CMRR (Rechazo del modo común) & 92dB \\ \hline
    Tamaño del FIFO de entrada & 4095 muestras \\ \hline
    Exactitud estándar (100 muestras, $\Delta T = 10  C^\circ $) & $3100 \mu V$ \\ \hline
    \end{tabular}%
    }
\end{table}

Los sensores utilizados para el ensayo fueron acelerómetros de balance de fuerza de 1 eje modelo \textit{FBA-11} de la marca \textit{Kinemetrics}. Las características de estos sensores se describen en la tabla \ref{tab:specsFBA11} y puede observarse en la figura \ref{fig:FBA11}. El funcionamiento de este tipo de acelerómetros se describe en detalle en la sección \ref{subsec:sensmonitoreo}.

%ESPECIFICACIONES DEL FBA11

\begin{figure}[H]
    \centering
    \includegraphics[width = 0.8\textwidth]{imagenes/cap3_resultados/Ensayos/FBA11.jpg}
    \caption{Acelerómetro FBA-11 de Kinemetrics.}
    \label{fig:FBA11}
\end{figure}

% Please add the following required packages to your document preamble:
% \usepackage{graphicx}
\begin{table}[H]
    \centering
    \caption{Especificaciones del acelerómetro FBA-11 de Kinemetrics}
    \label{tab:specsFBA11}
    \resizebox{\textwidth}{!}{%
    \begin{tabular}{|c|c|}
    \hline
    \textbf{Parámetro} & \textbf{Valor} \\ \hline
    Rango de escala completa & $\pm 1.0 g$ (.1, .25, .5 y 2 g opcionales) \\ \hline
    Frecuencia Natural & 50 Hz \\ \hline
    Amortiguamiento & 70\% \\ \hline
    Salida & $\pm 2.5 V / 1 g$ \\ \hline
    Linealidad & Menos de 1\% \\ \hline
    Ruido (entre 0 - 50 Hz) & $\pm 2.5 \mu V$ \\ \hline
    Rango dinámico & 130dB de 0.01 a 50Hz \\ \hline
    Alimentación & $\pm 12 Vdc$ (2.5 mA por eje) \\ \hline
    \end{tabular}%
    }
    \end{table}

Este equipo es el que se utiliza comúnmente en el IMME para realizar ensayos de vibración en estructuras.

El programa para la adquisición de los datos fue diseñado en LabVIEW 8.20 (Versión 2006), el cual es un software de programación gráfica desarrollado por National Instruments. Este programa se encarga de adquirir los datos de los sensores, procesarlos y graficarlos en tiempo real.

Para fijar los sensores se utilizó yeso, como es común en los ensayos de vibración para acoplar los sensores a la estructura.

Los ensayos se dividieron en 3 partes:
\begin{itemize}
    \item Vibración ambiental.
    \item Vibración forzada por impacto.
    \item Vibración libre por condición inicial.
\end{itemize}

La configuración utilizada se puede observar en las figuras \ref{fig:configuracionensayofrontal} y \ref{fig:configuracionensayoplanta} .

\begin{figure}[H]
    \centering
    \includegraphics[width = 0.8\textwidth]{imagenes/cap3_resultados/Ensayos/CONFIGURACION1.png}
    \caption{Vista frontal de la configuración utilizada en ensayo.}
    \label{fig:configuracionensayofrontal}
\end{figure}

\begin{figure}[H]
    \centering
    \includegraphics[width = 0.8\textwidth]{imagenes/cap3_resultados/Ensayos/CONFIGURACION1PLANTA.png}
    \caption{Vista de planta de la configuración utilizada en ensayo.}
    \label{fig:configuracionensayoplanta}
\end{figure}


Para comparar los resultados obtenidos con el prototipo y con el equipo comercial, se realizaron las pruebas en paralelo, es decir, se colocaron los sensores del prototipo y del equipo comercial en la estructura y se realizaron las pruebas al mismo tiempo. La instrumentación puede verse en las figuras \ref{fig:inst1} y \ref{fig:inst2}. Los resultados obtenidos se compararon en términos de la respuesta en frecuencia y la respuesta en el tiempo.


\begin{figure}[H]
    \centering
    \subfloat[Vista de la estructura instrumentada con los acelerómetros FBA-11 de Kinemetrics]{\includegraphics[width = 0.6\textwidth]{imagenes/cap3_resultados/Ensayos/InstrumentacionConf1FBA.jpg}\label{fig:inst1}}
    \hfill
    \subfloat[Vista de la estructura instrumentada incluyendo el prototipo de pruebas]{\includegraphics[width = 0.6\textwidth]{imagenes/cap3_resultados/Ensayos/InstrumentacionConf1FBASmartSensor.jpg}\label{fig:inst2}}
    \caption{Instrumentación de la estructura}
    \label{fig:inst}
\end{figure}

Durante el ensayo, el sensor inteligente estaba siendo controlado y monitoreado desde la estación base ubicada a pocos metros de la estructura. En esta estación base se encontraba el computador que , a través de una interfaz gráfica, permitía visualizar los datos en y guardarlos en un archivo de texto para su posterior análisis. El diseño de la interfaz gráfica de usuario puede observarse en las figuras \ref{fig:interfaz1} y \ref{fig:interfaz2}. La programación y características principales de esta interfaz fueron descritas en la sección \ref{subsec:softwaredesc}.

\begin{figure}[H]
    \centering
    \includegraphics[width = \textwidth]{imagenes/cap3_resultados/Ensayos/GUI1.jpg}
    \caption{Ventana 1 de la interfaz gráfica diseñada.}
    \label{fig:interfaz1}
\end{figure}

\begin{figure}[H]
    \centering
    \includegraphics[width = \textwidth]{imagenes/cap3_resultados/Ensayos/GUI2.jpg}
    \caption{Ventana 2 de la interfaz gráfica diseñada.}
    \label{fig:interfaz2}
\end{figure}

Para el procesamiento de los datos obtenidos haciendo uso de la tarjeta de adquisición de datos PCI-6221 se implementó un código en MATLAB, donde, de forma similar a la interfaz de la figura \ref{fig:interfaz1}, se grafican los datos en tiempo real y el espectro en frecuencia correspondiente.

Los datos se importaron directamente del archivo \textit{.lvm} generado por LabView y fueron preprocesados con la interfaz de MATLAB para importar datos de archivos externos, como se puede observar en la figura \ref{fig:datosmatlab}.

\begin{figure}[H]
    \centering
    \includegraphics[width = \textwidth]{imagenes/cap3_resultados/Ensayos/datosmatlab.png}
    \caption{Ventana de la interfaz gráfica de MATLAB para importar los datos obtenidos mediante LabVIEW.}
    \label{fig:datosmatlab}
\end{figure}

Las características de instrumentación para ambos sistemas fueron las mostradas en la tabla \ref{tab:comparacionsist}:
% Please add the following required packages to your document preamble:
% \usepackage{graphicx}
\begin{table}[H]
    \centering
    \caption{Comparación entre características de los sistemas de medición utilizados.}
    \label{tab:comparacionsist}
    \resizebox{\textwidth}{!}{%
    \begin{tabular}{|c|c|c|}
    \hline
    \textbf{Parámetro} & \textbf{Sistema basado en PCI6221} & \multicolumn{1}{l|}{\textbf{Sensor inteligente}} \\ \hline
    Frecuencia de muestreo & 200 Hz & 200 Hz \\ \hline
    Número de muestras máx. & 12000 & 1024 \\ \hline
    Alimentación & 120 Vac & 5 Vdc \\ \hline
    Peso aproximado & 25-30 kg & 800 g \\ \hline
    \begin{tabular}[c]{@{}c@{}}Dimensiones aproximadas\\ \end{tabular} & 1,5 m (solo estación base) & 25 cm \\ \hline
    Cableado & Sí & No \\ \hline
    Alerta ante eventos & No & Sí \\ \hline
    \end{tabular}%
    }
    \end{table}

A continuación se presentan y comparan los resultados obtenidos en cada ensayo:

\subsection{Vibración forzada por impacto}

Este ensayo consistió en excitar la estructura haciendo uso de un martillo de goma, haciendo que la misma comience a vibrar. Para este ensayo se contó con la ayuda de personal del IMME para coordinar la toma de datos con el impacto sobre el sistema. Además de buscar la semejanza entre ambos resultados, se buscaba comprobar el efecto que produce un cambio en la masa sobre la frecuencia natural del sistema. Para esto, se añadieron 5 lastres de 17 kg al sistema luego del primer ensayo de vibración por impacto.

En primer lugar, se observa en las figuras \ref{fig:DAQHammer} la respuesta en tiempo y en frecuencia usando el sistema basado en la tarjeta PCI-6221:

\begin{figure}[H]
    \centering
    \subfloat[Aceleración en el tiempo del sistema ante vibración por impacto en dirección Este-Oeste]{\includegraphics[width = \textwidth]{imagenes/cap3_resultados/Ensayos/VibHammer6EsteOesteNIDAQ.jpg}\label{fig:DAQham1}}
    \hfill
    \subfloat[Espectro en frecuencia del sistema ante vibración por impacto en dirección Este-Oeste]{\includegraphics[width = \textwidth]{imagenes/cap3_resultados/Ensayos/VibHammer6EsteOesteEspectroNIDAQ.jpg}\label{fig:DAQham2}}
    \caption{Respuesta del sistema ante vibración por impacto según la tarjeta PCI-6221 de National Instruments}
    \label{fig:DAQHammer}
\end{figure}

El registro obtenido para este mismo impacto haciendo uso del sensor inteligente se puede observar en la figura \ref{fig:impactoGUI}. En la figura \ref{fig:ventana2} se incluye la ventana auxiliar de la interfaz gráfica que permite al operador verificar la inclinación del sensor y si este se encuentra a nivel, así como identificar el ángulo en desnivel para su corrección. También se incluye la gráfica de la densidad espectral de potencia, que en ocasiones permite caracterizar de forma más rápida las frecuencias de vibración, sobre todo en espectros que contienen múltiples frecuencias. La densidad espectral de potencia también es útil para aplicar el método del ancho de banda local, utilizado para estimar el amortiguamiento del sistema.

\begin{figure}[H]
    \centering
    \includegraphics[width = \textwidth]{imagenes/cap3_resultados/Ensayos/VibHammer6EsteOesteSMARTSENSOR.jpg}
    \caption{Registro de vibración por impacto en la dirección Este-Oeste obtenido mediante el sensor inteligente.}
    \label{fig:impactoGUI}
\end{figure}

\begin{figure}[H]
    \centering
    \includegraphics[width = \textwidth]{imagenes/cap3_resultados/Ensayos/VibHammer1NorteSurSMARTSENSORVentana2.jpg}
    \caption{Ventana auxiliar de la GUI para observar la inclinación y la densidad espectral de potencia.}
    \label{fig:ventana2}
\end{figure}

Al comparar los espectros en frecuencia obtenidos en las figuras \ref{fig:DAQham2} y \ref{fig:impactoGUI} se observa que las frecuencias de vibración obtenidas se corresponden entre ambos sistemas de medición, siendo el error entre ambos picos en frecuencia de 0.1 Hz.

Por otro lado, se impactó el sistema sin los lastres en la dirección Norte-Sur, obteniéndose la respuesta observada en la figura \ref{fig:DAQHammerNS} según el sistema basado en la tarjeta PCI-6221:

\begin{figure}[H]
    \centering
    \subfloat[Aceleración en el tiempo del sistema ante vibración por impacto en dirección Norte-Sur]{\includegraphics[width = \textwidth]{imagenes/cap3_resultados/Ensayos/VibHammer1NorteSurNIDAQ.jpg}\label{fig:DAQham1NS}}
    \hfill
    \subfloat[Espectro en frecuencia del sistema ante vibración por impacto en dirección Norte-Sur]{\includegraphics[width = \textwidth]{imagenes/cap3_resultados/Ensayos/VibHammer1NorteSurEspectroNIDAQ.jpg}\label{fig:DAQham2NS}}
    \caption{Respuesta del sistema ante vibración por impacto según la tarjeta PCI-6221 de National Instruments}
    \label{fig:DAQHammerNS}
\end{figure}

Por su parte, el sensor inteligente obtuvo la respuesta del sistema observada en la figura y \ref{fig:impactoGUI_NS}:

\begin{figure}[H]
    \centering
    \includegraphics[width = \textwidth]{imagenes/cap3_resultados/Ensayos/VibHammer1NorteSurSMARTSENSOR.jpg}
    \caption{Registro de vibración por impacto en la dirección Norte-Sur obtenido mediante el sensor inteligente.}
    \label{fig:impactoGUI_NS}
\end{figure}

En primer lugar, se observa la gran similitud entre ambas respuestas, siendo la frecuencia de vibración en la dirección larga de 4.6 Hz en ambos sistemas, con un error de 0.08Hz en este caso. Además, se comprueba que el peso influye en la respuesta en frecuencia, al observarse que la frecuencia de vibración en la dirección larga (que en este caso corresponde al eje y por la configuración de los acelerómetros), disminuye al agregarse los lastres, siendo de 4.6 Hz inicialmente, y cambiando a 3.9 Hz luego de agregar los lastres.

\subsection{Vibración libre por condición inicial}

El ensayo de vibración libre consiste en aplicar una condición inicial a la estructura para posteriormente eliminar esta condición, que puede ser un peso o fuerza aplicada, y permitir que la estructura vibre libremente hasta alcanzar su condición de reposo o de vibración ambiental.

En este caso, se realizaron pruebas aplicando una condición inicial en el sentido norte-sur a la estructura y luego se dejó vibrar libremente al retirar la condición inicial. Análogo al ensayo anterior, se hizo el estudio con y sin lastres para evaluar el efecto del cambio en la respuesta dinámica del sistema al variar la masa del sistema.

En las figuras \ref{fig:DAQlibre1SL} y \ref{fig:DAQlibre1SL} se observa la respuesta del sistema en vibración libre antes de la colocación de los 5 lastres de 17 kg:

\begin{figure}[H]
    \centering
    \subfloat[Aceleración en el tiempo del sistema ante vibración libre sin lastres]{\includegraphics[width = \textwidth]{imagenes/cap3_resultados/Ensayos/AmpVibLibreSinLastresNIDAQ.jpg}\label{fig:DAQlibre1SL}}
    \hfill
    \subfloat[Espectro en frecuencia del sistema ante vibración libre sin lastres]{\includegraphics[width = \textwidth]{imagenes/cap3_resultados/Ensayos/VibLibreSinLastresNIDAQ.jpg}\label{fig:DAQlibre2SL}}
    \caption{Respuesta del sistema ante vibración libre sin lastres según la tarjeta PCI-6221 de National Instruments}
    \label{fig:DAQlibreSL}
\end{figure}

Por su parte, la respuesta obtenida por el sensor inteligente durante este ensayo se observa en la figura 

\begin{figure}[H]
    \centering
    \includegraphics[width = \textwidth]{imagenes/cap3_resultados/Ensayos/VibLibreSinLastresSMARTSENSOR.jpg}
    \caption{Registro de vibración libre sin lastres obtenido mediante el sensor inteligente.}
    \label{fig:libreGUI_SL}
\end{figure}

Al comparar los resultados obtenidos en el espectro en frecuencia usando ambos sistemas, como se ve en las figuras \ref{fig:libreGUI_SL} y \ref{fig:DAQlibreSL}, se observa la similitud entre el espectro en frecuencia del sistema basado en el sensor inteligente diseñado y el sistema comercial disponible en el IMME. En este caso, la diferencia entre la frecuencia natural del sistema en la dirección Norte-Sur obtenida por el sensor inteligente en comparación a la obtenida por la tarjeta PCI-6221 es de 0.09Hz, observándose además que el sensor inteligente mostró un comportamiento estable mientras que el sistema basado en la tarjeta de adquisición presenta problemas de deriva y picos indeseados que agregan componentes frecuenciales ajenas al sistema estructural, demostrándose así la confiabilidad del sensor inteligente.


De igual forma, se ejecutó la misma prueba luego de añadir los lastres, obteniéndose los resultados mostrados en la figura \ref{fig:DAQlibreCL}:
\begin{figure}[H]
    \centering
    \subfloat[Aceleración en el tiempo del sistema ante vibración libre con lastres]{\includegraphics[width = \textwidth]{imagenes/cap3_resultados/Ensayos/AmplitudVibLibreLastresNIDAQ1.jpg}\label{fig:DAQlibre1CL}}
    \hfill
    \subfloat[Espectro en frecuencia del sistema ante vibración libre con lastres]{\includegraphics[width = \textwidth]{imagenes/cap3_resultados/Ensayos/VibLibreLastresNIDAQ1.jpg}\label{fig:DAQlibre2CL}}
    \caption{Respuesta del sistema ante vibración libre con lastres según la tarjeta PCI-6221 de National Instruments}
    \label{fig:DAQlibreCL}
\end{figure}


Mientras que en la figura \ref{fig:libreGUI_CL} se observan los resultados arrojados por el sensor inteligente:

\begin{figure}[H]
    \centering
    \includegraphics[width = \textwidth]{imagenes/cap3_resultados/Ensayos/VibLibreLastresSMARTSENSOR.jpg}
    \caption{Registro de vibración libre con lastres obtenido mediante el sensor inteligente.}
    \label{fig:libreGUI_CL}
\end{figure}

Al observar los resultados obtenidos en las figuras \ref{fig:libreGUI_CL} y \ref{fig:DAQlibreCL}, se pueden apreciar las similitudes entre los espectros en frecuencia de ambos sistemas, donde resaltan claramente la frecuencia de vibración que corresponde a la frecuencia en la dirección larga del sistema. En el caso del sensor inteligente, se obtuvo un valor de 3.90 Hz, mientras que en el sistema basado en la tarjeta PCI-6221 el valor es de 3.87 Hz, siendo el error entre ambos sistemas de apenas 0.03 Hz. Además de confirmarse que los datos obtenidos por el sensor inteligente son confiables y permiten caracterizar el comportamiento dinámico del sistema, se observa como el valor de esta frecuencia disminuyó respecto al valor de 4.6 Hz observado en las figuras \ref{fig:DAQlibreSL} y \ref{fig:libreGUI_SL}, demostrándose la capacidad del sistema de medir cambios en la respuesta natural del sistema producto de cambios en la masa del mismo.

%AMBIENTAL
\subsection{Vibración ambiental}

Este ensayo consistió en tomar registros de datos sin perturbar la estructura. Es decir, siendo esta excitada únicamente por factores naturales como el viento o el paso de personas.

En primer lugar, se observa en las figuras \ref{fig:DAQamb1} y \ref{fig:DAQamb2} la respuesta en tiempo y el espectro en frecuencia del sistema ante vibración ambiental.

\begin{figure}[H]
    \centering
    \subfloat[Aceleración en el tiempo del sistema ante vibración ambiental]{\includegraphics[width = \textwidth]{imagenes/cap3_resultados/Ensayos/AmplitudVibAmb2NIDAQ1.jpg}\label{fig:DAQamb1}}
    \hfill
    \subfloat[Espectro en frecuencia del sistema ante vibración ambiental]{\includegraphics[width = \textwidth]{imagenes/cap3_resultados/Ensayos/VibAmb2EspectroNIDAQ1.jpg}\label{fig:DAQamb2}}
    \caption{Respuesta del sistema ante vibración ambiental según la tarjeta PCI-6221 de National Instruments}
    \label{fig:DAQAmb}
\end{figure}

En la figura \ref{fig:ambientalGUI} se observa la interfaz gráfica diseñada para el sensor inteligente al leer uno de los registros de vibración ambiental obtenido durante el ensayo:

\begin{figure}[H]
    \centering
    \includegraphics[width = \textwidth]{imagenes/cap3_resultados/Ensayos/VibAmb2SmartSensorGUI.jpg}
    \caption{Ventana de la interfaz gráfica diseñada con el registro de vibración ambiental obtenido.}
    \label{fig:ambientalGUI}
\end{figure}

Si bien los resultados obtenidos por vibración ambiental pueden verse por registro directamente en la interfaz, como se observa en la figura \ref{fig:ambientalGUI}, es conveniente concatenar varios archivos debido a la longitud de cada registro. Mientras mayor es el tamaño del registro, mayor resolución tendrá el espectro al tener más datos. Es decir, el $\Delta f$ mejora al aumentar la longitud de los registros.

Al concatenar registros es necesario aventanar cada registro, usando alguna función de aventanamiento como las definidas en la sección \ref{sec:aventanamiento}, para que evitar que se introduzcan al espectro cambios bruscos producto de la concatenación de archivos. En la figura \ref{fig:concatenados3} se observa el resultado de concatenar 3 registros de vibración ambiental que fueron tomados de forma sucesiva mientras se tomó el registro con la tarjeta PCI-6221 registrado en la figura \ref{fig:DAQamb1}. Para esto, se implementó un código en Python capaz de concatenar 3 archivos, generando un registro continuo utilizando la ventana de Hanning.

\begin{figure}[H]
    \centering
    \includegraphics[width = \textwidth]{imagenes/cap3_resultados/Ensayos/AmplitudVibAmb2CONCATENADASMARTSENSOR.jpg}
    \caption{Forma del registro concatenado tras aventanamiento de Hanning utilizando Python.}
    \label{fig:concatenados3}
\end{figure}

Una vez concatenados, se procedió a ejecutar la FFT sobre los datos, obteniéndose el espectro de la figura \ref{fig:espectroconc}:

\begin{figure}[H]
    \centering
    \includegraphics[width = \textwidth]{imagenes/cap3_resultados/Ensayos/VibAmb2EspectroCONC.jpg}
    \caption{Forma del espectro del registro concatenado.}
    \label{fig:espectroconc}
\end{figure}



Al comparar las figuras \ref{fig:espectroconc} y \ref{fig:DAQamb2} se observa que el espectro obtenido mediante el sensor inteligente no contiene los picos observados en el espectro obtenido por la tarjeta PCI-6221. Esto puede deberse al pobre acoplamiento entre el sensor inteligente y la estructura al estar este sobre una placa de prototipos, la cual a su vez se unió al sistema con yeso, evidenciándose en amplitudes bajas. Si bien el registro de la figura \ref{fig:ambientalGUI} muestra el comportamiento esperado por un registro de vibración ambiental en tiempo, para que el sensor sea capaz de captar los leves movimientos de la estructura en vibración ambiental, este debe estar acoplado al sistema, como es el caso de los acelerómetros Kinemetrics. A pesar de obtener estas discrepancias entre ambos espectros, el bajo nivel de ruido presentado por el sensor inteligente en vibraciones ambientales y la facilidad para concatenar registros debido al formato del mismo lo proyecta como un buen candidato para tomar registros de vibraciones ambientales siempre y cuando este se encuentre firmemente acoplado al sistema en estudio.


%==================================================================
\chapter{CONCLUSIONES}\label{CAP:conclu}
%\markboth{Tu Segundo Capítulo}{Tu Segundo Capítulo}%
	
	La revisión bibliográfica mostró los avances realizados por distintos autores en cuanto a la implementación de sistemas de monitoreo de variables estructurales basados en microcontroladores y haciendo uso de canales de comunicación inalámbrica, lo cual permitió concentrar y canalizar el trabajo de investigación de forma efectiva.
	
	Se logró plantear el diseño de un sensor inteligente para aplicaciones de monitoreo de salud estructural con éxito, identificando las variables de interés para este tipo de sistemas e incluso implementando un prototipo de pruebas funcional capaz de llevar a cabo ensayos similares los que se llevan a cabo en el Instituto de Materiales y Modelos Estructurales.
	
	Se diseñaron con éxito los distintos programas necesarios para obtener registros de vibración y mediciones de variables cuasi-estáticas haciendo uso de un microcontrolador. El ESP32 en conjunto con FreeRTOS, el cual permitió controlar de forma efectiva y ordenada las distintas tareas, mostraron ser herramientas capaces de manejar el preprocesamiento y adquisición de los datos de forma exitosa.
	
	Se implementó con éxito una interfaz de monitoreo y control capaz de enviar comandos de control y a la vez visualizar los registros tomados por el sensor inteligente, aplicando herramientas de procesamiento numérico con bajo costo computacional dentro de la misma aplicación, con capacidades de personalización dependiendo del cliente o proyecto y con la posibilidad de  guardar los registros en un formato compatible con la gran mayoría de programas de procesamiento.

	Se comprobó el funcionamiento del diseño propuesto haciendo uso de un prototipo de pruebas, el cual se comparó con un equipo comercial del fabricante National Instruments, obteniendo resultados muy similares, confirmándose la efectividad y veracidad de los datos obtenidos por el sensor inteligente.
	
	Al haber tomado en cuenta la fecha y hora de adquisición de los datos, y por las características del protocolo LoRa y el control que permite tener sobre los dispositivos esclavos asignándoles números de identificación única, se sentaron las bases a la posibilidad de escalar el número de sensores inteligentes de bajo costo en la estructura, aumentando la resolución espacial y permitiendo tener acceso a las gráficas del comportamiento modal de la estructura.
	
	A pesar de las limitaciones de memoria de sistemas basados en microcontroladores, el sensor inteligente cumple sus funciones de monitorear las variables estructurales de interés; permitiendo que el operador tenga acceso a un histórico de datos de la estructura, disminuyendo además los costos y la complejidad de sistemas cableados con fines similares.
	
	En este tipo de aplicaciones es esencial garantizar la integridad de los datos.	A pesar de ser LoRa un protocolo lento en comparación a otros, la seguridad que ofrece al tener muy bajo porcentaje de pérdida de paquetes y el largo alcance, que permite ubicar la estación base en una ubicación segura alejada de la estructura lo posicionan como un protocolo prometedor para aplicaciones de monitoreo de salud estructural. 
	
    Los sistemas de adquisición de datos basados en microcontroladores representan una opción confiable y de bajo costo para implementar soluciones de monitoreo estructural periódico que sustituyen a los sistemas cableados.

%==================================================================
\chapter{RECOMENDACIONES}\label{CAP:recomendaciones}
%\markboth{Tu Segundo Capítulo}{Tu Segundo Capítulo}%
 
\begin{itemize}

    \item Implementar múltiples sensores inteligentes y hacer uso de la fecha y hora exacta de registro para sincronizar los datos provenientes de los múltiples esclavos. 
    
    
    \item Para comprobar que los cambios introducidos en un sistema estructural controlado se corresponden con cambios en la respuesta dinámica del mismo, y que estos cambios son detectados por el sensor inteligente, es conveniente realizar una prueba sobre un modelo a escala de una estructura en un entorno controlado, previamente caracterizada tomando en cuenta su modelo estático y los materiales que la constituyen. Sobre esta estructura se pueden ejecutar ensayos de vibración a medida que se introducen cambios en los elementos estructurales, pudiendo así caracterizar cómo este cambio influye en la respuesta dinámica del sistema y permitiendo llevar un registro de estos cambios en el tiempo.
    
    
    \item Se sugiere ejecutar ensayos periódicos sobre una estructura a escala en donde se puedan controlar las condiciones de temperatura y humedad relativa, además del cambio inducido en el sistema. Este estudio permitiría caracterizar los cambios en la respuesta dinámica producto de las variables ambientales respecto a los que son consecuencia de cambios en la rigidez del sistema, siendo estos últimos los de mayor interés.
    
    
    \item Si bien en este trabajo de investigación se consideraron las variables de interés principales para el monitoreo de salud estructural, se recomienda evaluar la posibilidad de incorporar sensores de variables químicas y electroquímicas que pueden atentar en contra de la integridad de la estructura, como sensores de pH, potencial de corrosión, agentes gaseosos, entre otros. Por la escalabilidad del sistema, se pueden incorporar estos sensores en los sensores inteligentes, permitiendo tener un monitoreo más completo de la estructura.
    
    \item Hacer uso del Digital Motion Processing Unit de los módulos MPU9250 y MPU6050 para ver cómo se comparan respecto a los resultados obtenidos en cuanto a fusión de sensores se refiere.
 
    \item Si bien el sensor MPU6250 permitió comprobar el correcto funcionamiento del diseño planteado, se recomienda hacer uso de un sensor cuyo nivel de ruido permita obtener registros de vibración ambiental de mejor calidad. Algunos acelerómetros de tecnología similar que presentan mejor comportamiento en bajas amplitudes y a baja frecuencia son el MMA8451 del fabricante NXP, el LSM6DSOX de STMicroelectronics y el ADXL355 de Analog Devices. Todos estos cuentan con un comportamiento en ruido menor a $120 \frac{\mu g}{\sqrt{Hz}}$ en el rango de frecuencia de interés.
    
    %126 microg/Hz
    %70 mircog/Hz ST
   %25 microg/Hz AD
    
 
 \item Se sugiere implementar una solución alternativa en estación base a la tecnología MQTT, como por ejemplo HTTP. Las dificultades encontradas que limitaron el tamaño del registro surgen al intentar subir los datos usando MQTT al servidor-computador. El buffer del cliente MQTT utilizado por la librería \textit{PubSubClient} es limitado, y la misma no está diseñada para manejar desbordamiento del buffer, por lo que los datos son truncados de forma inesperada. Se sugiere implementar un manejo de excepción para este caso. 
 
%  \item Implementar la subida de paquetes MQTT uno a uno y luego reconstruir usando NodeRED para generar los arreglos individuales de aceleración.
 
 \item El módulo LoRa tiene pocas pérdidas de datos, pero su velocidad no es la mejor, sin embargo el rango permite ubicar la estación base en un lugar seguro alejado de la estructura. Intentar utilizar otra red diferente de LoRa y enviar datos mientras se están adquiriendo. Las velocidades de Lora no son las mejores para esta topología, la red de 2.4 GHz que utilizan módulos como el nRF24L01 disminuye el rango, más sin embargo, permite alcanzar velocidades de transmisión mucho más alta. Si no se debe esperar a que se tome todo el registro para empezar a enviar, se evita tener que guardar el registro completo de forma temporal. Esto sería conveniente para aplicaciones en donde se requiera una respuesta en tiempo real como ensayos de vibración en campo.
 
 \item Hacer portátil el dispositivo al evaluar las necesidades de energía del sistema para ser alimentado por baterías, posiblemente implementando alguna función de deep-sleep del microcontrolador.
 
 \item Integrar nuevos comandos en estación base como la posibilidad de calibrar a distancia, obtener solo algunos datos y no todo el registro, modificar frecuencia de muestreo y número de datos a tomar.
 
 \item Incorporar al sistema los puertos analógicos del ESP32. Permitiendo incluir sensores de temperatura, humedad, presión, entre otros, cuya interfaz sea analógica.
 
 \item Puesto que no se necesita de WiFi en el sensor inteligente, se puede prescindir de las capacidades de conectividad del microcontrolador, escogiendo uno que tenga menos posibilidades de conectividad, pero que cuente con otras fortalezas, como por ejemplo el Teensy 4.0. Teniendo 1024kb de RAM, a diferencia de los 520kb del ESP32.
 
 \item El módulo de comunicaciones LoRa RYLR896 de REYAX TECHNOLOGY ofrece funcionalidad de control por comandos AT y serial, evitando así el uso de librerías como RadioLib para controlar el módulo. Esto ahorraría espacio en memoria y facilita el código a implementar, además de hacerlo compatible con distintos microcontroladores sin soporte a esa librería.
 
 \item  Evaluar la posibilidad de utilizar variantes del ESP32 (como el ESP32-S3) más potentes que permitirían aumentar las capacidades de cómputo y almacenamiento sin la necesidad de cambiar el código, manteniendo el firmware actual en funcionamiento.
 
 \item Implementar uso de memoria SD en sensor inteligente. Por la dimensión temporal del sistema actual (no en tiempo real) el guardado en SD usando SPI no representaría retrasos en la adquisición de los datos, pues se llevaría a cabo luego de la toma de datos.
 
 \item Evaluar la posibilidad de usar la transformada de ondículas o Wavelet en vez de la FFT para darle una dimensión temporal a los registros.
 
 \item Expandir la memoria SRAM del ESP32 haciendo uso de módulos comerciales como el W25Q128 de 16 MB. Esto permitiría almacenar registros más largos, mejorando la resolución de los mismos y permitiendo llevar a cabo ensayos con constantes de tiempo mayores.
 
 \item Para mejorar la precisión en la hora de toma de datos, se recomienda implementar rutinas de iteración en la sincronización del RTC comparando la hora actual y la hora del sensor inteligente luego de la sincronización inicial hasta disminuir el error. Paralelo a esto se puede utilizar un módulo GPS en la estación base y comparar con hora de NTP adquirida.
\end{itemize}

%==================================================================

\appendix

\renewcommand \thechapter{\Roman{chapter}}
%==================================================================
\chapter{CÓDIGO DEL SENSOR INTELIGENTE}\label{CAP:anexo0}
%\markboth{Tu anexo}{Tu anexo}%
\begin{lstlisting}[language=C++, caption=Tarea de lectura de datos de aceleración triaxial]
  void leerDatosACL(void *pvParameters){
      //Leer los valores del Acelerometro de la IMU
      while(true){
          mpu.getEvent(&a, &g, &tem);
          ACLData aclData; //Estructura a ser llenada con 3 ejes
    
          aclData.AclX = a.acceleration.x - acl_offset[0];
          aclData.AclY = a.acceleration.y - acl_offset[1]; 
          aclData.AclZ = a.acceleration.z - acl_offset[2];
    
          if(F_SAMPLING != 0){
            vTaskDelay(F_SAMPLING/portTICK_PERIOD_MS);
          }
          
          //Evaluo los valores actuales de aceleracion
          if(flag_limite == 0){
            flag_limite = evaluar_limites_acl(aclData.AclX, aclData.AclY, aclData.AclZ);
            flag_time = checktime();
            if(flag_limite != 0){
              flag_acl == true;
              globalTimestamp = getEpochTime();
              //Activando banderas para registro de temperatura y humedad
              flag_inc = 1;
              flag_temp_hum = 1;
    
              //Cambio en el LED de toma de datos
              digitalWrite(LED_EST1, HIGH);
              //Suspende tarea de LED IDLE
              vTaskSuspend(xHandle_blink);
            }
            if(flag_time){
              flag_limite = 1;
              flag_inc = 1;
              flag_temp_hum = 1;
              globalTimestamp = getEpochTime();
    
              //Cambio en el LED de toma de datos
              digitalWrite(LED_EST1, HIGH);
              //Suspende tarea de LED IDLE
              vTaskSuspend(xHandle_blink);
              }
          }
    
          //solo evalua si no se sobrepaso el limite al mismo tiempo
          if(flag_acl == true && flag_limite == 0){
              //Se cambia el valor de flag limite para solo ejecutar esto 1 vez
              flag_limite = 1;
              flag_inc = 1;
              flag_temp_hum = 1;
              globalTimestamp = getEpochTime();
    
              //Cambio en el LED de toma de datos
              digitalWrite(LED_EST1, HIGH);
              //Suspende tarea de LED IDLE
              vTaskSuspend(xHandle_blink);
          }
    
          //Envia los datos a la cola solo si se supero el limite
          if(flag_limite != 0 || flag_acl == true || flag_time == 1)
          {
            if(xQueueSend(aclQueue, &aclData, portMAX_DELAY)){
              vTaskResume(xHandle_crearBuffer); //Se llenan los buffers para enviar los datos
            }
            else{
              continue;
            }
          }
          else{
            continue;
          }
      }
    }
    
\end{lstlisting}

\begin{lstlisting}[language=C++, caption=Tarea de creación de buffer para envío a módulo LoRa]

void crearBuffer(void *pvParameters){
  while(true){

    ACLData datos_acl;

    //Recibo los datos de la cola y los guardo en la estructura creada
    if(xQueueReceive(aclQueue, &datos_acl, portMAX_DELAY)){
      if( k <= NUM_DATOS ){
        if(k == 1) tiempo1 = millis();
        //Lleno el buffer de datos
        struct_buffer_acl.bufferX[k] = datos_acl.AclX;
        struct_buffer_acl.bufferY[k] = datos_acl.AclY;
        struct_buffer_acl.bufferZ[k] = datos_acl.AclZ;
        k++;
      }
      else{
        //Reiniciando para sobreescribir en buffers "nuevos" en la siguiente accion
        k = 0;
        cont2 = 0;
        cont2_inc = 0;

        //Se siguen tomando datos mas no se guardan en los buffers
        flag_limite = 0;
        flag_inc = 0;
        flag_temp_hum = 0;
        flag_time = 0;
        flag_acl = false; //Reinicio booleano de recepcion LoRa para toma de decisiones en proxima peticion

        //Apago led indicativo de toma de datos
        digitalWrite(LED_EST1, LOW);


        //Envio los resultados a la cola
        if(xQueueSend(tramaLoRaQueue, &struct_buffer_acl, portMAX_DELAY)){
           vTaskSuspend(xHandle_leerDatosACL); //suspendo adquisicion hasta que se envie todo
           vTaskSuspend(xHandle_readBMETask);
           vTaskSuspend(xHandle_readMPU9250);

          //Se envian los datos mediante lora
          vTaskResume(xHandle_send_packet);
        }
        else{
          Serial.println("No se envio la cola...");
        }

        vTaskResume(xHandle_blink);
      }
    }
    else{
      Serial.println("No se recibio la cola correctamente...");
    };

    //vPortFree(NULL);
 
    //Se suspende esta tarea para esperar la toma de datos
    vTaskSuspend(xHandle_crearBuffer);

  }
}

\end{lstlisting}

\begin{lstlisting}[language=C++, caption=Función del sensor inteligente para generar arreglos de datos de aceleración en paquetes de 128 bytes]

int generararray(int contador_paquetes)
{

    //Recibiendo cola con datos de aceleracion en estructura de datos 
    //Los datos son un float array dentro del buffer
    if(contador_paquetes == 0){
          const int chunkSize = 64;
          contador_paquetes_interno = 0; //REINICIO CONTADOR INTERNO INDEPENDIENTE DE SI ENTRO POR PRIMERA VEZ

          if(xQueueReceive(tramaLoRaQueue, &trama, portMAX_DELAY))
          {
            //Evita el error por MeditationGuru en Core0
            //Copia los datos por partes en vez de completos, evita problemas de reboot
            for (int i = 0; i < NUM_DATOS; i += chunkSize)
            {
                // Calculate the size of the current chunk
                int currentChunkSize = min(chunkSize, NUM_DATOS - i);

                // Copy and process the chunk
                memcpy(floatArrayX + i, trama.bufferX + i, currentChunkSize * sizeof(float));
                memcpy(floatArrayY + i, trama.bufferY + i, currentChunkSize * sizeof(float));
                memcpy(floatArrayZ + i, trama.bufferZ + i, currentChunkSize * sizeof(float));

                // Process the data in floatArrayX, floatArrayY, and floatArrayZ here
            }
            /*The memcpy function takes three arguments: the destination pointer, the source pointer, and the number of bytes to copy.*/
          }
          selectedFloatArray = floatArrayX;
    }
    else if(contador_paquetes == NUM_DATOS/CHUNK_SIZE) //32/64/128...
    {
          contador_paquetes_interno = 0;
          selectedFloatArray = floatArrayY;
    }
    else if(contador_paquetes == NUM_DATOS/CHUNK_SIZE*2) // (NUM_DATOS/CHUNKSIZE - 1)*2
    {
          contador_paquetes_interno = 0;
          selectedFloatArray = floatArrayZ;
    }

    // Calculate the number of chunks
    int numChunks = sizeof(floatArrayX) / sizeof(float) / CHUNK_SIZE; //El numero de chunks indica el numero de bytes final
    //Print the amount of chunks
    Serial.print("Amount of chunks calculated: ");
    Serial.println(numChunks);
    

    // Check if the size of floatArray is divisible by CHUNK_SIZE
    if (NUM_DATOS / sizeof(float) % CHUNK_SIZE != 0) {
        Serial.println("Error: Size of floatArray is not divisible by CHUNK_SIZE");
    }
    // Create an array to hold the chunks
    float chunks[numChunks][CHUNK_SIZE];

    // Split the array into chunks
    for (int i = 0; i < numChunks; i++) {
      memcpy(chunks[i], &selectedFloatArray[i * CHUNK_SIZE], CHUNK_SIZE * sizeof(float)); //Copiar 128bytes de Float Array (a partir de la posicion especificada por i) en chunks
    }

    if(xQueueSend(arrayQueue, &chunks[contador_paquetes_interno], portMAX_DELAY) == pdTRUE){
      Serial.println("Se envio la cola a la tarea de envio LoRa.");
    }
    else{
      Serial.println("Problema al enviar cola...");
      return 0;
    }

    contador_paquetes_interno++;

    return 1;
}

\end{lstlisting}

\begin{lstlisting}[language=C++, caption=Tarea de envío de datos de aceleración desde sensor inteligente]

void send_packet(void *pvParameters){
  while(1){
    transmitFlag = true;

    //detach interrupt from pin 2
    detachInterrupt(2);

    Serial.print("Contador de paquetes actual antes de entrar en generararray: ");
    Serial.println(contador_paquetes);

    // if(contador_paquetes >= ((NUM_DATOS * 4))*3 / 128){
    //   contador_paquetes = 0;
    // }

    generararray(contador_paquetes); //Genero el array y mando un chunk, dependiendo del contador

    
    if(contador_paquetes < (((NUM_DATOS * 4))*3 / 128)){
         contador_paquetes++; //Aumento el contador para enviar el siguiente paquete en el proximo envio
    }

     float floatarray[CHUNK_SIZE];
     byte data[CHUNK_SIZE * sizeof(float)]; //Inicializacion de byte array

     if(xQueueReceive(arrayQueue, &floatarray, portMAX_DELAY)){
        Serial.println("Se recibio la cola con los datos");
     }
    
     memcpy(data, floatarray, sizeof(floatarray)); //CHUNK_SIZE * sizeof(float)
    
    size_t size_data;

    size_data = sizeof(data);

    //Llamo a la funcion que crea la trama de datos (payload)
    sendmessage_radiolib(size_data, data);
    Serial.println("Sending byte array con datos!");
    Serial.println();

    //Espero X segundos luego de enviar mensaje
    vTaskDelay(interval/portTICK_PERIOD_MS);

    //Suspendo esta tarea hasta que se reciba otro mensaje
    if(contador_paquetes >= (((NUM_DATOS * 4))*3 / 128)){
      
    //Ejecuto funciones para calcular valor promedio de variables cuasiestaticas
      promediofinal_inc();
      promediofinal_temphum();
      send_thi();

      iteraciones_peticiones++;
      Serial.print("Iteraciones: ");
      Serial.println(iteraciones_peticiones);

      if(iteraciones_peticiones == 2){
        Serial.println("Reiniciando micro...");
        esp_restart();
      }

      Serial.println("Reactivando tareas de adquisicion de datos!");
      contador_paquetes = 0;
      vTaskResume(xHandle_leerDatosACL);
      vTaskResume(xHandle_readBMETask);
      vTaskResume(xHandle_readMPU9250);

        Serial.print("Memoria disponible en send task: ");
        Serial.println(ESP.getFreeHeap());
        Serial.println(ESP.getFreePsram());
        Serial.println(uxTaskGetStackHighWaterMark(NULL));

      transmitFlag = false; 
      Serial.print(F("[SX1278] Comienza a escuchar comandos de nuevo... "));
      attachInterrupt(digitalPinToInterrupt(2), setFlag, RISING);
      int state = radio.startReceive();
      if (state == RADIOLIB_ERR_NONE) {
        Serial.println(F("exito!"));
      } 
      else{
        Serial.print(F("fallo, codigo "));
        Serial.println(state);
        //while (true);
      }

      vTaskSuspend(NULL);
    }
  }  
}

\end{lstlisting}

\begin{lstlisting}[language=C++, caption=Tarea de lectura de datos de temperatura y humedad]

void readBMETask(void *parameter) {
  while (true) {
    //Crea una estructura de tipo BMEData
    BMEData data_readtemp;

    //Lee los valores del sensor y los guarda en la estructura
    data_readtemp.humidity = bme.readHumidity();
    data_readtemp.temperature = bme.readTemperature();

    //Envia los datos a la cola dataQueue si bandera esta activada
    if(flag_temp_hum){
        if(xQueueSend(data_temphumQueue, &data_readtemp, portMAX_DELAY)){
          //Serial.println("Se envio correctamente la cola de temperatura");
          vTaskDelay(10 / portTICK_PERIOD_MS);
          vTaskResume(xHandle_receive_temphum); //Reactivo tarea de creacion de buffers y promedios
        }
        else{
          Serial.println("No se envio correctamente la cola de temperatura");
        }
    }
    
    //Delay dependiendo del tiempo de muestreo
    vTaskDelay(T_SAMPLING_TEMPHUM / portTICK_PERIOD_MS); //5Hz
  }
}


\begin{lstlisting}[language=C++, caption= Tarea de lectura de datos del MPU9250]

void readMPU9250(void *pvParameters){
  while (true) {
      //Crea una estructura de tipo IncData
      IncData data_inc;

       //Lee los valores del sensor y los guarda en la estructura
       if(mpu9250.update()){
          data_inc.IncRoll = mpu9250.getRoll();
          data_inc.IncPitch = mpu9250.getPitch();
          data_inc.IncYaw = mpu9250.getYaw();

          //Envia los datos a la cola incQueue solo si la bandera esta activada
          if(flag_inc){
            if(xQueueSend(incQueue, &data_inc, portMAX_DELAY)){
            //Serial.println("Se envio correctamente la cola de inclinacion");
            vTaskResume(xHandle_recInclinacion); //Reactivo tarea de creacion de buffers y promedios
            }
            else{
              Serial.println("No se envio correctamente la cola de inclinacion");
              continue;
            }
          }
          vTaskDelayUntil(&xLastWakeTime, T_SAMPLING_INC / portTICK_PERIOD_MS)
       }
}
\end{lstlisting}

\begin{lstlisting}[language=C++, caption=Tarea de envío de datos de variables ambientales e inclinación desde sensor inteligente]

  int send_thi_radiolib(time_t tstamp, float temp, float hum, float yaw, float pitch, float roll){
  if(transmitFlag){
    Serial.print(F("[SX1278] Transmitiendo T.H.I ... "));

    THIPacket packet; //Creando paquete como estructura Packet

    //Se llena estructura de datos
    packet.messageID = 200;
    packet.senderID = 0x2;
    packet.receiverID = 0x1; 
    packet.timestamp = tstamp;
    packet.temperature = temp;
    packet.humidity = hum;
    packet.yaw = yaw;
    packet.pitch = pitch;
    packet.roll = roll;
    
    //Generacion de byte array
    byte* packetBytes = reinterpret_cast<byte*>(&packet);

    int state = radio.startTransmit(packetBytes, sizeof(packet));

    if (state == RADIOLIB_ERR_NONE) {
      Serial.println(F(" exito!"));
      return 1;

    } else if (state == RADIOLIB_ERR_PACKET_TOO_LONG) {
      Serial.println(F("muy largo!"));
      return 0;

    } else if (state == RADIOLIB_ERR_TX_TIMEOUT) {
      Serial.println(F("timeout!"));
      return 0;

    } else {
      Serial.print(F("fallo, codigo "));
      Serial.println(state);
      return 0;
    }
    return 1;
  }
  else{
      Serial.println("Se estan recibiendo datos");
      return 0;
  }  
}

\end{lstlisting}%

%==================================================================
\chapter{CÓDIGO DE LA ESTACIÓN BASE}\label{CAP:anexo1}
%\markboth{Tu anexo}{Tu anexo}%

\begin{lstlisting}[language=C++, caption=Tarea de recepción de datos de aceleración vía Lora en estación base]
    void receive_task(void *pvParameter)
    {
      while (true)
      {
        if (transmitFlag != true)
        {
          // reset flag
          receivedFlag = false;
    
          PacketUnion packetUnion;
    
          int numBytes = radio.getPacketLength();
          byte byteArr[numBytes];
    
          if (numBytes == 22)
          {
            Serial.println("Se recibio una actualizacion de SmartSensor.");
            data_o_comando = 1;
          }
          else if (numBytes == 131)
          {
            Serial.println("Se recibieron datos del SS.");
            data_o_comando = 0;
          }
          else if (numBytes == 28)
          {
            Serial.println("Se recibieron datos de temp y humedad.");
            data_o_comando = 2;
          }
          else
          {
            Serial.println("Se recibio algo que no es un comando ni una actualizacion de RTC.");
            datacorrupta = true;
          }
    
          Serial.print("Packet length: ");
          Serial.println(numBytes);
          int state = radio.readData(byteArr, numBytes);
    
          if (state == RADIOLIB_ERR_NONE || state == 0 && datacorrupta == false)
          {
            // packet was successfully received
            Serial.println(F("[SX1278] Paquete recibido!"));
    
            // Serial.print("Valor actual de datacomando> ");
            // Serial.println(data_o_comando);
    
            contador++;
    
            switch(data_o_comando)
            {
              case 0:
                  memcpy(&packetUnion.packet1, byteArr, sizeof(packetUnion.packet1));
    
                  Serial.print("[SX1278] Message ID: ");
                  Serial.println(packetUnion.packet1.messageID);
                  Serial.print("[SX1278] Sender ID: ");
                  Serial.println(packetUnion.packet1.senderID);
                  Serial.print("[SX1278] Receiver ID: ");
                  Serial.println(packetUnion.packet1.receiverID);
    
                  //Muestra payload
                  Serial.print(F("[SX1278] Payload:\t\t"));
                  //print the byte array
                  for (int i = 0; i < sizeof(packetUnion.packet1.payload); i+=4) {
                    float value = *((float*)(packetUnion.packet1.payload + i));
                    Serial.print(value);
                    Serial.print(F(" "));
                  }
                  Serial.println("");
    
                  leer_datos(sizeof(packetUnion.packet1.payload), packetUnion.packet1.messageID, packetUnion.packet1.senderID, packetUnion.packet1.receiverID, packetUnion.packet1.payload);
    
                  //ERA 95
                  if(packetUnion.packet1.messageID == (((NUM_DATOS * 4))*3 / 128) - 1){
                    if(xQueueSend(xQueueBufferACL, &buffer_prueba, portMAX_DELAY)){
                      Serial.println("Se envio la estructura bufferprueba a la cola xQueueBufferACL");
                      //vTaskResume(xHandle_send_mqtt);
                    }
                    else{
                      Serial.println("No se pudo enviar la estructura bufferprueba a la cola xQueueBufferACL");
                    }
                  }
    
                  break;
    
              case 1:
                  memcpy(&packetUnion.stringPacket, byteArr, numBytes);
                  Serial.println("El SS esta activo...");
    
                  Serial.print("[SX1278] Message ID: ");
                  Serial.println(packetUnion.stringPacket.messageID);
                  Serial.print("[SX1278] Sender ID: ");
                  Serial.println(packetUnion.stringPacket.senderID);
                  Serial.print("[SX1278] Receiver ID: ");
                  Serial.println(packetUnion.stringPacket.receiverID);
    
                  if(int(packetUnion.stringPacket.messageID) == 255){
                    Serial.println("Listo para enviar RTC...");
                    transmitFlag = true; //Activo bandera de envio
                    vTaskResume(xHandle_send_RTC_task);
                  }
                  break;
              case 2:
                  memcpy(&packetUnion.thipacket, byteArr, numBytes);
    
                  Serial.print("[SX1278] Message ID: ");
                  Serial.println(packetUnion.thipacket.messageID);
                  Serial.print("[SX1278] Sender ID: ");
                  Serial.println(packetUnion.thipacket.senderID);
                  Serial.print("[SX1278] Receiver ID: ");
                  Serial.println(packetUnion.thipacket.receiverID);
    
                  Serial.print(F("[SX1278] Payload:\t\t"));
                  //Print the temperature and humidity
                  Serial.print(F("Temperature: "));
                  Serial.print(packetUnion.thipacket.temperature);
                  Serial.print(F(" Humidity: "));
                  Serial.println(packetUnion.thipacket.humidity);
                  //Print the angle values
                  Serial.print(F("Yaw: "));
                  Serial.print(packetUnion.thipacket.yaw);
                  Serial.print(F(" Pitch: "));
                  Serial.print(packetUnion.thipacket.pitch);
                  Serial.print(F(" Roll: "));
                  Serial.println(packetUnion.thipacket.roll);
    
    
                  if(packetUnion.thipacket.messageID == 200){
                    if(xQueueSend(xQueueTempHumInc, &packetUnion.thipacket, portMAX_DELAY)){
                      Serial.println(xPortGetFreeHeapSize());
                      Serial.println("Se envio la estructura THI a la cola");
                      vTaskResume(xHandle_send_mqtt_thi);
                    }
                    else{
                      Serial.println("No se pudo enviar la estructura bufferprueba a la cola xQueueBufferACL");
                    }
                  }
                  break;
            }
    
    
          }
          else if (state == RADIOLIB_ERR_CRC_MISMATCH)
          {
            //Paquete recibido pero esta corrupto
            Serial.println(F("[SX1278] CRC error!"));
    
            // EJECUTAR CASO ESPECIAL DE DATA CORRUPTA PARA AUMENTAR CURRENTPOS Y GUARDAR 0s EN BUFFER DE INTERES
            if (numBytes == 131)
            {
              fillBuffer(bufferactual, NULL, sizeof(packetUnion.packet1.payload), true);
            }
            else
            {
              Serial.println("[SX1278] La data recibida esta corrupta y no es para este receptor");
            }
    
            contador_errores++;
          }
          else if (state == -1)
          {
            // some other error occurred
            Serial.print(F("[SX1278] Fallo, codigo "));
            Serial.println(state);
            contador_errores++;
          }
          else if (datacorrupta)
          {
            // some other error occurred
            Serial.print(F("[SX1278] Data corrupta..."));
            Serial.println(state);
            datacorrupta = false; ///Reinicio bandera
            contador_errores++;
          }
    
          Serial.print(F("[SX1278] Contador: "));
          Serial.println(contador);
          Serial.print(F("[SX1278] Contador error: "));
          Serial.println(contador_errores);
    
          Serial.println("");
        }
      else
      {
        Serial.println("Se estan enviando datos...");
      }
      vTaskSuspend(NULL);
      }
    }
    
    \end{lstlisting}

    
\begin{lstlisting}[language=C++, caption=Tarea para envío de actualización de RTC desde estación base]

    void send_RTC_task(void *pvParameters)
    {
      while (1)
      {
        if (transmitFlag == true)
        {
          Serial.print(F("[SX1278] Transmisitendo actualizacion de RTC... "));
    
          detachInterrupt(2);
    
          TimePacket packet;
    
          //Estructura de tiempo
          timestruct time_packet;
    
          struct tm timeinfo = rtc.getTimeStruct();
    
          time_packet.year = timeinfo.tm_year + 1900;
          time_packet.month = timeinfo.tm_mon + 1;
          time_packet.day = timeinfo.tm_mday;
          time_packet.hour = timeinfo.tm_hour;
          time_packet.minute = timeinfo.tm_min;
          time_packet.second = timeinfo.tm_sec;
    
          //Convierte estructura a byte array
          byte *byteArrTime = (byte *)&time_packet;
    
          //Muestra el payload
          Serial.print(F("[SX1278] Payload:\t\t"));
          //print the byte array
          // Print the byte array
          for (int i = 0; i < sizeof(time_packet); i++) {
              Serial.print(byteArrTime[i], HEX);
              Serial.print(F(" "));
          }
    
    
          Serial.println("");
    
          Serial.println(sizeof(time_packet));
    
          packet.messageID = 255;
          packet.senderID = 1;                                        
          packet.receiverID = 2;                                      
          memcpy(packet.payload, byteArrTime, sizeof(packet.payload));
    
          //Convierte estructura packet a byte array incluyendo encabezado
          byte *packetBytes = reinterpret_cast<byte *>(&packet);
    
          // Delay antes de transmitir para permitir que SS se configure en modo listening
          delay(500);
    
          int state = radio.startTransmit(packetBytes, sizeof(packet));
    
          if (state == RADIOLIB_ERR_NONE)
          {
            //Paquete transmitido con exito
            Serial.println(F(" exito!"));
          }
          else if (state == RADIOLIB_ERR_PACKET_TOO_LONG)
          {
            //Paquete mayor a 256 bytes
            Serial.println(F("muy largo!"));
          }
          else if (state == RADIOLIB_ERR_TX_TIMEOUT)
          {
            //timeout
            Serial.println(F("timeout!"));
          }
          else
          {
            //Otro error
            Serial.print(F("fallo, codigo "));
            Serial.println(state);
          }
    
          // Delay para permitir que se termine de enviar el paquete, no poner en modo receptor de inmediato
          delay(500);
    
          transmitFlag = false;
    
          Serial.print(F("[SX1278] Comienza a escuchar paquetes otra vez... \n"));
    
          int state2 = radio.startReceive();
    
          attachInterrupt(digitalPinToInterrupt(2), setFlag, RISING); // Reinicia ISR
    
          if (state2 == RADIOLIB_ERR_NONE)
          {
            Serial.println(F("Exito!"));
          }
          else
          {
            Serial.print(F("fallo, codigo "));
            Serial.println(state2);
            // while (true);
          }
    
          vTaskSuspend(NULL);
        }
        else
        {
          Serial.println("Se estan recibiendo datos en este momento...");
        }
      }
    }
    
    \end{lstlisting}

\begin{lstlisting}[language=C++, caption=Tarea de envio de datos de aceleración vía MQTT]

    
//Se activa una vez se reciben todos los paquetes Lora...
void send_mqtt(void *pvParameters){
    while(true){
        
        if(xQueueReceive(xQueueBufferACL, &bufferaceleracion, portMAX_DELAY)){
            Serial.println("Recibido de la cola BufferACL");
        } else {
            Serial.println("Error recibiendo de la cola BufferACL");
        }

        delay(500);

        sendAxis("esp32/x", "x", bufferaceleracion.bufferX, ARRAY_SIZE);
        sendAxis("esp32/y", "y", bufferaceleracion.bufferY, ARRAY_SIZE);
        sendAxis("esp32/z", "z", bufferaceleracion.bufferZ, ARRAY_SIZE);

        // Wait for some time before publishing again
        vTaskResume(xHandle_keepalive_task);
        vTaskSuspend(NULL);
    }
}

\end{lstlisting}


\begin{lstlisting}[language=C++, caption=Tarea de conversión a tipo JSON de un eje de aceleración]

void sendAxis(const char* topic, char* axis, const float* data, size_t dataSize) {
    DynamicJsonDocument doc(15000);

    //Crea arreglo JSON
    JsonArray array = doc.createNestedArray(axis);

    //Genera el arreglo a partir del float array
    for (size_t i = 0; i < dataSize; i++) {
        // Limit the float to 2 decimal places
        float value = round(data[i] * 100.0) / 100.0;
        array.add(value);
    }

    //Convierte el documento a string
    String json;
    serializeJson(doc, json);


    //Publica al topico escogido
    if(mqttClient.publish(topic, json.c_str()))
        {
            Serial.println("Mensaje publicado en topico MQTT");
        } else {
            Serial.println("Error publicando mensaje en topico MQTT");
        }
}

\end{lstlisting}


\begin{lstlisting}[language=C++, caption=Tarea de envío de datos de variables cuasi-estáticas vía MQTT]

    void send_mqtt_thi(void *pvParameter){
        while(1){
          vTaskSuspend(xHandle_keepalive_task);
      
          THIPacket thipacket;
      
          //Recibe arreglos de la cola
          if(xQueueReceive(xQueueTempHumInc, &thipacket, portMAX_DELAY)){
              Serial.println("Recibido de la cola TempHumInc");
          } else {
              Serial.println("Error recibiendo de la cola TempHumInc");
          }
      
          delay(200);
      
          sendTHI("esp32/temp", thipacket.temperature);
          sendTHI("esp32/hum", thipacket.humidity);
          sendTHI("esp32/inc_y", thipacket.yaw);
          sendTHI("esp32/inc_p", thipacket.pitch);
          sendTHI("esp32/inc_r", thipacket.roll);
          sendTHI("esp32/timestamp", (float)thipacket.timestamp);
      
          vTaskResume(xHandle_send_mqtt);
          vTaskSuspend(NULL);
        }
      }
      
\end{lstlisting}
    
\begin{lstlisting}[language=C++, caption=Tarea de callback ante recepción de datos MQTT]

    void messageReceived(char* topic, byte* payload, unsigned int length) {
        Serial.print("Mensaje recibido en topico: ");
        Serial.println(topic);
      
        // Step 2: Convert payload to string
        String message;
        for (int i = 0; i < length; i++) {
          message += (char)payload[i];
        }
      
        // Step 3: Check if the message is "ON"
        if (message == "ON") {
          // Step 4: Call the software ISR
          Serial.println("Peticion de datos via MQTT... Enviando paquete Lora si no se estan enviando datos");
          if(!transmitFlag){
            ISR_MQTT_Request();
          }
        }
      }    

    \end{lstlisting}

%\backmatter
%==================================================================
\chapter{CÓDIGO DE LA INTERFAZ GRÁFICA}\label{CAP:anexo2}
%\markboth{Tu anexo}{Tu anexo}%

\begin{lstlisting}[language=Python, caption=Código para interfaz gráfica de control y monitoreo]

import customtkinter as ctk
import tkinter as tk
import tkinter.filedialog as fd
import pandas as pd
import numpy as np
import matplotlib.pyplot as plt
from matplotlib.backends.backend_tkagg import FigureCanvasTkAgg
import os
import shutil
#import tkinter.filedialog
import paho.mqtt.client as mqtt
from tkinter import PhotoImage
import filtros
from datetime import datetime,timedelta
from tkinter import ttk
from PIL import Image
from scipy.signal import butter, filtfilt
from scipy.integrate import cumtrapz

#Funcion callback cuando se recibe un mensaje MQTT con el payload received
def on_message(client, userdata, message):
    #Chequea si el payload es "received"
    if message.payload.decode() == "received":
        #Crea el label con el texto y lo muestra en pantalla
        label.configure(text="Nuevo registro disponible!")

#Funcion para enviar comando via MQTT al topico esp32/command 
def send_mqtt_message():
    #Publica el mensaje al topico
    client.publish("esp32/command", "ON")

    #Desactiva el boton temporalmente
    control_button.configure(state="disabled")

    #Reactiva el boton luego de 5 segundos
    tab1.after(5000, lambda: control_button.configure(state="normal"))

#CLIENTE MQTT
client = mqtt.Client(mqtt.CallbackAPIVersion.VERSION2)
#Se conecta al broker
client.connect("127.0.0.1", 1883, 60) #El broker esta en localhost
#Configura la funcion callback a llamar cuando se recibe mensaje
client.on_message = on_message
#Se suscribe al topico de alerta o notificacion de recepcion
client.subscribe("esp32/alerta_rx")
#Inicia el loop que mantiene la conexion de MQTT y gestiona los mensajes
client.loop_start()

ctk.set_appearance_mode("Light")   
ctk.set_default_color_theme("green")  
 
#VENTANA PRINCIPAL
window = ctk.CTk()
window.title("INTERFAZ DE MONITOREO DE CONTROL - SISTEMA ESP32 JOSE TOVAR")


#Crea 2 tabs
notebook = ctk.CTkTabview(window, anchor="nw")

# Frames de cada tab
tab1 = notebook.add('Tab 1')
tab2 = notebook.add('Tab 2')

#columnas
tab1.columnconfigure(0, weight=1)
tab1.columnconfigure(1, weight=1)
tab1.columnconfigure(2, weight=1)


image = Image.open("C:\\Users\\jatov\\Documents\\Universidad\\TEG\\CosasTEG_JT\\FFT-Python\\Codigos-Python\\IMME.jpg")
background_image = ctk.CTkImage(image, size=(100, 100))
# Create a label with the image
image_label = ctk.CTkLabel(tab1, image=background_image, text="")
# Place the label in the grid
image_label.grid(column=0, row=0, sticky='n')


image2 = Image.open("C:\\Users\\jatov\\Documents\\Universidad\\TEG\\CosasTEG_JT\\FFT-Python\\Codigos-Python\\angulos.png")
pruebaang = ctk.CTkImage(image2, size=(130, 130))
# Create a label with the image
image_label = ctk.CTkLabel(tab2, image=pruebaang, text="")
# Place the label in the grid
image_label.grid(column=0, row=2, sticky='n')


#SECCION DE CONTROL
control_frame = ctk.CTkFrame(tab1)
control_frame.grid(column=0, row=0, sticky='n', pady=170, padx=10)
#Crea label para seccion de control
control_label = ctk.CTkLabel(control_frame, text="Seccion de Control", font=("Arial", 14, "bold"))
control_label.pack()
#Crea boton para seccion de control
control_button = ctk.CTkButton(control_frame, text="Obtener registro", command=send_mqtt_message)
control_button.pack(pady=10)

#LISTA DE ARCHIVOS
folder_path = "C:\\Users\\jatov\\Documents\\Universidad\\TEG\\Pruebas_DatosAceleracion\\DatosACL_P1"
file_names = os.listdir(folder_path)

#crea el frame para los archivos
frameArchivos = ctk.CTkFrame(tab1)
frameArchivos.grid(column=1, row=3)

label = ctk.CTkLabel(frameArchivos, text="No hay registros nuevos")
label.grid(column=0, row=1, padx = 5)

selected_file_name = ctk.StringVar(tab1)
selected_file_name.set("Seleccione un registro")
#MENU DROPDOWN PARA ESCOGER ARCHIVO
option_menu = ctk.CTkOptionMenu(master=frameArchivos, variable=selected_file_name, values=file_names)
option_menu.grid(column=0, row=2)

# Funcion para actualizar lista de archivos
def update_file_list(option_menu):
    #Consigue la lista de archivos disponibles
    file_names = os.listdir(folder_path)

    # Sort the files by modification time
    file_names.sort(key=lambda x: os.path.getmtime(os.path.join(folder_path, x)))
    
    # Destroy the existing CTkOptionMenu widget
    option_menu.destroy()
    
    # Create a new CTkOptionMenu widget with the updated options
    option_menu = ctk.CTkOptionMenu(master=frameArchivos, variable=selected_file_name, values=file_names)
    option_menu.grid(column=1, row=1)

    #Actualiza cada 10 segundos
    tab1.after(10000, lambda: update_file_list(option_menu)) 

    # #Configura un timer para actualizar la lista cada 5 segundos
    # tab1.after(5000, update_file_list)

#FIGURA DE ACELERACION
fig, ax = plt.subplots()
fig.suptitle('REGISTRO DE ACELERACION EN TIEMPO', fontsize=10, fontweight='bold')
#fig.set_facecolor('lightgray')
#fig.set_edgecolor('black')
fig.set_alpha(0.5)
canvas = FigureCanvasTkAgg(fig, master=tab1)
widget = canvas.get_tk_widget()
widget.grid(column=1, row=0)
widget.config(bd = 2, relief = "groove")

#FIGURA PARA FFT
fig_fft, ax_fft = plt.subplots()
fig_fft.suptitle('ESPECTRO EN FRECUENCIA DEL REGISTRO', fontsize=10, fontweight='bold')
#fig.set_facecolor('lightgray')
canvas_fft = FigureCanvasTkAgg(fig_fft, master=tab1)
widget_fft = canvas_fft.get_tk_widget()
widget_fft.grid(column=2, row=0)
widget_fft.config(bd = 2, relief = "groove") #'flat', 'raised', 'sunken', 'ridge', 'groove' and 'solid'.

#FIGURA DE PSD
figpsd, ax_psd = plt.subplots()
figpsd.suptitle('DENSIDAD ESPECTRAL DE POTENCIA', fontsize=10, fontweight='bold')
#figpsd.set_facecolor('lightgray')
#figpsd.set_edgecolor('black')
figpsd.set_alpha(0.5)
canvas_psd = FigureCanvasTkAgg(figpsd, master=tab2)
widget_psd = canvas_psd.get_tk_widget()
widget_psd.grid(column=1, row=0)
widget_psd.config(bd = 2, relief = "groove")

#FIGURA DE BARRA
figbar, ax_bar = plt.subplots()
figbar.suptitle('GRAFICA DE BARRAS PARA VER INCLINACION EN SENSOR INTELIGENTE', fontsize=10, fontweight='bold')
#figpsd.set_facecolor('lightgray')
#figpsd.set_edgecolor('black')
figbar.set_alpha(0.5)
canvas_bar = FigureCanvasTkAgg(figbar, master=tab2)
widget_bar = canvas_bar.get_tk_widget()
widget_bar.grid(column=0, row=0)
widget_bar.config(bd = 2, relief = "groove")

label_nivel = ctk.CTkLabel(master=tab2, text="No se ha seleccionado un archivo.", font=("Arial", 14, "bold"))
label_nivel.grid(column=0, row=1)



#FRAME DE ABAJO A LA DERECHA
frameVentanas = ctk.CTkFrame(tab1)
frameVentanas.grid(column=2, row=2)
windowing_label = ctk.CTkLabel(master=frameVentanas, text="Aventanamiento")
windowing_label.grid(column=1, row=1, padx=5,pady= 5)  # Adjust the grid position as needed

#AVENTANAMIENTO
windowing_functions = ['Ninguno','Hanning', 'Hamming', 'Blackman']
#Guarda aventanamiento seleccionado
selected_windowing_function = ctk.StringVar()
#Configura el aventanamiento default
selected_windowing_function.set(windowing_functions[0])

#Crea el menu dropdown de aventanamientos
windowing_menu = ctk.CTkOptionMenu(master=frameVentanas, variable=selected_windowing_function, values=windowing_functions)
windowing_menu.grid(column=2, row=1, padx=5, pady= 5)  # Adjust the grid position as needed

#FUNCION DE GUARDADO EN CSV
def save_file():
    #Pregunta al usuario el archivo a escoger
    dest_filename = ctk.filedialog.asksaveasfilename(defaultextension=".csv")

    #Si el usuario no cancela el dialogo
    if dest_filename:
        #Copia el archivo actual en la direccion especificada
        shutil.copyfile(folder_path + "\\" + selected_file_name.get(), dest_filename)

#Crea el boton para guardar CSV en pantalla
save_button = ctk.CTkButton(tab1, text="Guardar CSV", command=save_file)
save_button.grid(column=2, row=3)

#VALORES ACTUALES DE TEMPERATURA Y HUMEDAD
#Crea frame de temperatura y humedad
frameTH = ctk.CTkFrame(tab1, bg_color="lightgray")
frameTH.grid(column=2, row=1, pady = 5)
#Crea los labels de temperatura y humedad con color
current_value_label_x = ctk.CTkLabel(frameTH, text="Variables ambientales:", font=("Arial", 14, "bold"))
current_value_label_x.grid(column=1, row=1, padx=5)
current_value_label_x = ctk.CTkLabel(frameTH, text="Humedad relativa:", font=("Arial", 14, "bold"), corner_radius=50, fg_color="darkgray")
current_value_label_x.grid(column=2, row=1, padx=10)
current_value_label_y = ctk.CTkLabel(frameTH, text="Temperatura:", font=("Arial", 14, "bold"), corner_radius=50, fg_color="darkgray")
current_value_label_y.grid(column=3, row=1, padx=10)

#VALORES ACTUALES DE INCLINACION
#Crea el frame donde se mostraran las inclinaciones
frameInc = ctk.CTkFrame(tab1, bg_color="lightgray")
frameInc.grid(column=1, row=1, pady = 5)
#Labels de valor actual de inclinacion
current_value_label_inc = ctk.CTkLabel(frameInc, text="InclinaciOn:", font=("Arial", 14, "bold"))
current_value_label_inc.grid(column=1, row=1, padx=5)

current_value_label_roll = ctk.CTkLabel(frameInc, text="Omega:", font=("Arial", 14, "bold"), corner_radius=50, fg_color="darkgray")
current_value_label_roll.grid(column=2, row=1, padx=10)
current_value_label_pitch = ctk.CTkLabel(frameInc, text="Phi:", font=("Arial", 14, "bold"), corner_radius=50, fg_color="darkgray")
current_value_label_pitch.grid(column=3, row=1, padx=10)
current_value_label_yaw = ctk.CTkLabel(frameInc, text="Kappa:", font=("Arial", 14, "bold"), corner_radius=50, fg_color="darkgray")
current_value_label_yaw.grid(column=4, row=1, padx=10)

#FECHA Y HORA DE REGISTRO
time_label = ctk.CTkLabel(tab1, text="No se ha escogido ningun registro")
time_label.grid(column=1, row=4)

#FUNCION PRINCIPAL DE ACTUALIZACION DE PLOTS
def update_plots():
    #Obtiene el nombre del archivo seleccionado
    file_name = selected_file_name.get()

    #Borra los contenidos previos del label para no solapar
    label.configure(text = "No hay registros nuevos")

    #Lee el archivo CSV y asigna nombres a las columnas por orden
    df = pd.read_csv(os.path.join( folder_path, file_name), header=None, names=['x', 'y', 'z', 'temp', 'hum', 'yaw', 'pitch', 'roll', 'time'])

    #Extrae las columnas x,y,z
    x = df['x']
    y = df['y']
    z = df['z']
    temp = df['temp'].iloc[0] #Extrae solo el valor de la primera fila
    hum = df['hum'].iloc[0]
    yaw = df['yaw'].iloc[0]
    pitch = df['pitch'].iloc[0]
    roll = df['roll'].iloc[0]
    time = df['time'].iloc[0]

    if(pitch < -1 or pitch > 1):
        label_nivel.configure(text="El sensor no esta nivel en el eje y (Phi).", font=("Arial", 14, "bold"))
    elif(roll < -1 or roll > 1):
        label_nivel.configure(text="El sensor no esta a nivel en el eje x (Omega).", font=("Arial", 14, "bold"))
    else:
        label_nivel.configure(text="El sensor esta a nivel.", font=("Arial", 14, "bold"))
        

    #Lista de angulos
    angles = [pitch, roll]
    labels = ['Phi', 'Omega']

    #Convierte de UNIX epoch a datetime
    time = datetime.fromtimestamp(df['time'].iloc[0]) + timedelta(hours=4) #Diferencia de 4 horas por GMT

    #Actualiza la etiqueta de tiempo
    time_label.configure(text="Fecha y Hora del registro: " + str(time))

    #Actualiza los labels de valores actuales de temperatura, humedad e inclinacion
    current_value_label_x.configure(text="Humedad relativa: " + str(hum) + "%")
    current_value_label_y.configure(text="Temperatura: " + str(temp))
    current_value_label_yaw.configure(text="Kappa: " + str(yaw))
    current_value_label_pitch.configure(text="Omega: " + str(roll))
    current_value_label_roll.configure(text="Phi: " + str(pitch))

    # Aplica el aventanamiento seleccionado a los datos de aceleracion
    if selected_windowing_function.get() == 'Hanning':
        window = np.hanning(len(x))
    elif selected_windowing_function.get() == 'Hamming':
        window = np.hamming(len(x))
    elif selected_windowing_function.get() == 'Blackman':
        window = np.blackman(len(x))
    elif selected_windowing_function.get() == 'Ninguno':
        window = 1 # No windowing

    windowed_data_x = x * window
    windowed_data_y = y * window
    windowed_data_z = z * window
    
    #Aplica FFT en datos aventanados con la funcion escogida
    fft_1 = np.fft.fft(windowed_data_x)
    fft_2 = np.fft.fft(windowed_data_y)
    fft_3 = np.fft.fft(windowed_data_z)

    # Calculate the PSD
    xfil_psd = np.abs(fft_1)**2
    yfil_psd = np.abs(fft_2)**2
    zfil_psd = np.abs(fft_3)**2

    #Obtiene las frecuencias positivas de la FFT
    freq = np.fft.fftfreq(len(x), d=1/200)  # d is the inverse of the sampling rate
    positive_freq = freq[:len(freq)//2]

    #Limpia la grafica anterior
    ax.clear()
    ax_fft.clear()
    ax_psd.clear()
    ax_bar.clear()

    # Genera eje de tiempo
    start_time = datetime.fromtimestamp(df['time'].iloc[0])

    timearr = [start_time + timedelta(seconds=i/200) for i in range(len(x))]

    #Grafica los nuevos datos

    #ACELERACION
    ax.plot(timearr, x, label='ACL X')
    ax.plot(timearr, y, label='ACL Y')
    ax.plot(timearr, z, label='ACL Z')
    ax.set_xlabel('Tiempo (s)')
    ax.set_ylabel('Aceleracion')
    ax.grid(True)

    #FFT SIN FILTRAR

    xfil = fft_1
    yfil = fft_2
    zfil = fft_3

    #FFT filtrada
    # xfil = filtros.butter_lowpass_filter(fft_1, 40, 200, 5)
    # yfil = filtros.butter_lowpass_filter(fft_2, 40, 200, 5)
    # zfil = filtros.butter_lowpass_filter(fft_3, 40, 200, 5)

    #FFT
    ax_fft.plot(positive_freq, np.abs(xfil[:len(positive_freq)]), label='Espectro Eje X')
    ax_fft.plot(positive_freq, np.abs(yfil[:len(positive_freq)]), label='Espectro Eje Y')
    ax_fft.plot(positive_freq, np.abs(zfil[:len(positive_freq)]), label='Espectro Eje Z')

    ax_fft.set_xlabel('Frecuencia (Hz)')
    ax_fft.set_ylabel('Amplitud')
    ax_fft.grid(True)

    #Grafica la densidad espectral de potencia
    ax_psd.plot(positive_freq, np.abs(xfil_psd[:len(positive_freq)]), label='Densidad Espectral X')
    ax_psd.plot(positive_freq, np.abs(yfil_psd[:len(positive_freq)]), label='Densidad Espectral Y')
    ax_psd.plot(positive_freq, np.abs(zfil_psd[:len(positive_freq)]), label='Densidad Espectral Z')
    #ax_psd.set_yscale('log')
    ax_psd.set_xlabel('Frequency (Hz)')
    ax_psd.set_ylabel('Power Spectral Density')
    ax_psd.grid(True)

    #Grafico de barras para inclinacion
    ax_bar.axhline(1, color='r', linestyle='--')  
    ax_bar.axhline(-1, color='r', linestyle='--') 
    ax_bar.bar(labels, angles)
    ax_bar.set_title('Diagrama de barras para inclinacion en sensor inteligente')
    ax_bar.set_xlabel('Angulo')
    ax_bar.set_ylabel('Grados')


    #Obtiene las frecuencias maximas de cada eje
    max_freq1 = positive_freq[np.argmax(np.abs(xfil[:len(positive_freq)]))]
    max_freq2 = positive_freq[np.argmax(np.abs(yfil[:len(positive_freq)]))]
    max_freq3 = positive_freq[np.argmax(np.abs(zfil[:len(positive_freq)]))]

    median_freq = positive_freq[len(positive_freq)//2]

    #Muestra las frecuencias maximas en la grafica
    ax_fft.annotate(f'Fmax_X: {max_freq1:.2f}', xy=(max_freq1, np.max(np.abs(xfil[:len(positive_freq)]))), xytext=(-10 if max_freq1 > median_freq else 10,30), textcoords='offset points', arrowprops=dict(arrowstyle='->'))
    ax_fft.annotate(f'Fmax_Y: {max_freq2:.2f}', xy=(max_freq2, np.max(np.abs(yfil[:len(positive_freq)]))), xytext=(-10 if max_freq2 > median_freq else 10,10), textcoords='offset points', arrowprops=dict(arrowstyle='->'))
    ax_fft.annotate(f'Fmax_Z: {max_freq3:.2f}', xy=(max_freq3, np.max(np.abs(zfil[:len(positive_freq)]))), xytext=(-10 if max_freq3 > median_freq else 10,-10), textcoords='offset points', arrowprops=dict(arrowstyle='->'))

    #Frecuencias maximas en PSD
    ax_psd.annotate(f'Fmax_X: {max_freq1:.2f}', xy=(max_freq1, np.max(np.abs(xfil_psd[:len(positive_freq)]))), xytext=(-10 if max_freq1 > median_freq else 10,30), textcoords='offset points', arrowprops=dict(arrowstyle='->'))
    ax_psd.annotate(f'Fmax_Y: {max_freq2:.2f}', xy=(max_freq2, np.max(np.abs(yfil_psd[:len(positive_freq)]))), xytext=(-10 if max_freq2 > median_freq else 10,10), textcoords='offset points', arrowprops=dict(arrowstyle='->'))
    ax_psd.annotate(f'Fmax_Z: {max_freq3:.2f}', xy=(max_freq3, np.max(np.abs(zfil_psd[:len(positive_freq)]))), xytext=(-10 if max_freq3 > median_freq else 10,-10), textcoords='offset points', arrowprops=dict(arrowstyle='->'))

    
    ax.legend()
    ax_fft.legend()

    #Redibuja las graficas
    canvas.draw()
    canvas_fft.draw()
    canvas_bar.draw()
    canvas_psd.draw()

#Crea el boton para actualizar grafica
button = ctk.CTkButton(tab1, text="Actualizar grafica", command=update_plots)
button.grid(column=1, row=2)

#Ejecuta la funcion de actualizacion de archivos disponibles
update_file_list(option_menu)

notebook.pack(expand=True, fill='both')

#Loop principal de la ventana
window.mainloop()

\end{lstlisting}

%



%\backmatter

%==================================================================
\newpage
%\markboth{Referencias}{Referencias}%
%\addcontentsline{toc}{chapter}{Referencias}%

% References here (outcomment the appropriate case)
% CASE 1: BiBTeX used to constantly update the references (while the paper is being written).
%\bibliographystyle{IEEEtran}%{IEEEtranS}%%% outcomment this and next line in Case 1 siam
\bibliographystyle{apacite}%
\renewcommand{\bibname}{REFERENCIAS}
\let\oldbibsection\bibsection
\bibliography{biblioteca} % if more than one, comma separated and without extension bib


% CASE 2: BiBTeX used to generate EIETdeG.bbl (to be further fine tuned)
%\documentclass[letterpaper,titlepage,12pt,oneside,spanish,final]{report_eie}

%PARA QUE APAREZCAN LAS SUBSUBSECTIONS EN EL INDICE
%\setcounter{tocdepth}{3}

%\documentclass[letterpaper,titlepage,12pt,twoside,openright,spanish,final]{report_eie}

\input{setup.tex}

\begin{document}
%\frontmatter

%%%%%%%%%%%%%%%%%%%%%%%%%%%%%%%%%%%%%%%%%%%%%%%%%%%

%               Primera Página
%================================== Portada ========================
\input{portada.tex}

%======================= Constancia de Aprobación ===================
%\newpage
\begin{figure}
        \begin{center}
        %\centering
        %\includegraphics[height=23cm]{aprobacion.eps}

        \vspace{0.5mm}
        \label{Fig.aprobacion}
        \end{center}
        \end{figure}
\thispagestyle{empty}

%======================= Mención Honorífica =========================
\newpage
%\thispagestyle{empty}

\begin{figure}
        \begin{center}
        %\centering
        %\includegraphics[height=24cm]{mencion.eps}
        \vspace{0.5mm}
        \label{Fig.mencion}
        \end{center}
\end{figure}
\thispagestyle{empty}

%======================= Página de Dedicatoria ======================
\newpage%
\newenvironment{dedication}%
{\cleardoublepage \thispagestyle{empty} \vspace*{\stretch{1}}%
\begin{center} \em} {\end{center} \vspace*{\stretch{3}} }%
\begin{dedication}%
Dedicado a Gregoria del Valle Briceño Aray y José Edmundo Tovar Silva.
\end{dedication}%

%=============================== RECONOCIMIENTOS Y AGRADECIMIENTOS ===================================
\chapter*{RECONOCIMIENTOS Y AGRADECIMIENTOS}
%\markboth{Reconocimientos}{Reconocimientos}%
\addcontentsline{toc}{chapter}{RECONOCIMIENTOS Y AGRADECIMIENTOS}%
%\setlength{\parskip}{0.2cm}%
%\input{agradecimientos.tex}%

%======================= Página de Resumen (Abstract) ==========================
\newpage
\renewcommand*{\abstract}{\begin{center}\end{center}}
%\begin{abstract}
\begin{spacing}{1}
\begin{center}%

\textbf{Tovar B., José A.}

\begin{large}
DISEÑO DE UN SENSOR INTELIGENTE PARA APLICACIONES DE MONITOREO DE SALUD ESTRUCTURAL
\end{large}
\end{center}

\noindent%
\textbf{Tutor Académico: Prof. Jose Romero. Tesis.
Caracas, Universidad Central de Venezuela. Facultad de Ingeniería.
Escuela de Ingeniería Eléctrica. Ingeniero Electricista. Opción: Electrónica y Control. Institución: IMME Año 2024,
xvii, 153 h. + anexos}

\noindent
\textbf{Palabras Claves:} Salud estructural, Sensor inteligente,  Respuesta dinámica, Frecuencia natural, Microcontrolador, Acelerómetro, LoRa, ESP32, MQTT, Espectro en frecuencia. \\[1ex]

\noindent \textbf{Resumen.-} En el siguiente trabajo se plantea el diseño de un sensor inteligente para ser utilizado en aplicaciones de salud estructural. En primer lugar, se llevó a cabo la investigación documental necesaria para identificar las variables de interés al evaluar la respuesta dinámica de los sistemas estructurales y su relación con la instrumentación electrónica. Se escogió el hardware necesario para la implementación de un prototipo de pruebas capaz de verificar el funcionamiento del diseño. Se escogieron sensores de tecnología MEMS como el MPU6250, MPU9250 y BME280, en conjunto con el microcontrolador ESP32 y el módulo Ra-02 de Ai-Thinker para las comunicaciones inalámbricas. Se diseñó el software necesario para controlar el sistema haciendo uso de FreeRTOS y se implementaron con éxito las tareas tanto en el sensor inteligente como en la estación base para obtener los registros de vibración y de las variables cuasi-estáticas (temperatura, humedad relativa e inclinación). Se programó una interfaz gráfica de monitoreo y control para observar los registros y enviar comandos al sensor inteligente. Las pruebas realizadas en el IMME demostraron el funcionamiento satisfactorio del sensor inteligente al comparar los registros obtenidos con el equipo de vibración comercial basado en la tarjeta PCI-6221 de National Instruments, obteniendo resultados muy similares que permitieron caracterizar el comportamiento dinámico de una estructura de acero.

\end{spacing}

%\underline{RESUMEN}
%
\thispagestyle{empty}%
%\input{resumen.tex}%
%\end{abstract}
%====================== Páginas de Contenidos =====================
\renewcommand{\baselinestretch}{1.5}% Espaciado entre linea
\addtocounter{page}{3}%
\setlength{\parskip}{3pt}% Separación entre párrafos

\tableofcontents%

\listoffigures%

\listoftables%


%==================================================================
\chapter*{LISTA DE ACRÓNIMOS}%
%\markboth{Lista de Acrónimos}{Lista de Acrónimos}%
\addcontentsline{toc}{chapter}{LISTA DE ACRÓNIMOS}%
\input{acroni.tex}%


%==================================================================
\chapter*{INTRODUCCIÓN}\label{CAP:intro}
\setlength{\parskip}{14pt}% Separación entre párrafos
\addcontentsline{toc}{chapter}{INTRODUCCIÓN}%
%\markboth{Introducción}{Introducción}%

\pagenumbering{arabic}%
\input{cap0_intro.tex}%

%==================================================================
\chapter{MARCO REFERENCIAL}\label{CAP:referencial}
\input{cap1_marcoreferencial.tex}%

%==================================================================
\chapter{MARCO TEÓRICO}\label{CAP:marco_teor}
%\markboth{Tu Primer Capítulo}{Tu Primer Capítulo}%
\input{cap2_marcoteorico.tex}%

%==================================================================
\chapter{MARCO METODOLÓGICO}\label{CAP:marco_met}
%\markboth{Tu Segundo Capítulo}{Tu Segundo Capítulo}%
\input{cap3_marco_meto.tex}%

%==================================================================
\chapter{PRUEBAS Y RESULTADOS}\label{CAP:pruebas}
%\markboth{Tu Segundo Capítulo}{Tu Segundo Capítulo}%
\input{cap4_pruebas_resul.tex}

%==================================================================
\chapter{CONCLUSIONES}\label{CAP:conclu}
%\markboth{Tu Segundo Capítulo}{Tu Segundo Capítulo}%
\input{cap5_conclu.tex}

%==================================================================
\chapter{RECOMENDACIONES}\label{CAP:recomendaciones}
%\markboth{Tu Segundo Capítulo}{Tu Segundo Capítulo}%
\input{cap6_recom.tex}

%==================================================================

\appendix

\renewcommand \thechapter{\Roman{chapter}}
%==================================================================
\chapter{CÓDIGO DEL SENSOR INTELIGENTE}\label{CAP:anexo0}
%\markboth{Tu anexo}{Tu anexo}%
\input{anexos1.tex}%

%==================================================================
\chapter{CÓDIGO DE LA ESTACIÓN BASE}\label{CAP:anexo1}
%\markboth{Tu anexo}{Tu anexo}%
\input{anexos2.tex}

%\backmatter
%==================================================================
\chapter{CÓDIGO DE LA INTERFAZ GRÁFICA}\label{CAP:anexo2}
%\markboth{Tu anexo}{Tu anexo}%
\input{anexos3.tex}%



%\backmatter

%==================================================================
\newpage
%\markboth{Referencias}{Referencias}%
%\addcontentsline{toc}{chapter}{Referencias}%

% References here (outcomment the appropriate case)
% CASE 1: BiBTeX used to constantly update the references (while the paper is being written).
%\bibliographystyle{IEEEtran}%{IEEEtranS}%%% outcomment this and next line in Case 1 siam
\bibliographystyle{apacite}%
\renewcommand{\bibname}{REFERENCIAS}
\let\oldbibsection\bibsection
\bibliography{biblioteca} % if more than one, comma separated and without extension bib


% CASE 2: BiBTeX used to generate EIETdeG.bbl (to be further fine tuned)
%\input{EIETdeG.bbl} % outcomment this line in Case


%==================================================================
%\markboth{index}{index}%
%\addcontentsline{toc}{chapter}{Índice alfabético}%
\printindex%

\end{document}
 % outcomment this line in Case


%==================================================================
%\markboth{index}{index}%
%\addcontentsline{toc}{chapter}{Índice alfabético}%
\printindex%

\end{document}
 % outcomment this line in Case


%==================================================================
%\markboth{index}{index}%
%\addcontentsline{toc}{chapter}{Índice alfabético}%
\printindex%

\end{document}
 % outcomment this line in Case


%==================================================================
%\markboth{index}{index}%
%\addcontentsline{toc}{chapter}{Índice alfabético}%
\printindex%

\end{document}
