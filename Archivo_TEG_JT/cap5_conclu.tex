	
	La revisión bibliográfica mostró los avances realizados por distintos autores en cuanto a la implementación de sistemas de monitoreo de variables estructurales basados en microcontroladores y haciendo uso de canales de comunicación inalámbrica, lo cual permitió concentrar y canalizar el trabajo de investigación de forma efectiva.
	
	Se logró plantear el diseño de un sensor inteligente para aplicaciones de monitoreo de salud estructural con éxito, identificando las variables de interés para este tipo de sistemas e incluso implementando un prototipo de pruebas funcional capaz de llevar a cabo ensayos similares los que se llevan a cabo en el Instituto de Materiales y Modelos Estructurales.
	
	Se diseñaron con éxito los distintos programas necesarios para obtener registros de vibración y mediciones de variables cuasi-estáticas haciendo uso de un microcontrolador. El ESP32 en conjunto con FreeRTOS, el cual permitió controlar de forma efectiva y ordenada las distintas tareas, mostraron ser herramientas capaces de manejar el preprocesamiento y adquisición de los datos de forma exitosa.
	
	Se implementó con éxito una interfaz de monitoreo y control capaz de enviar comandos de control y a la vez visualizar los registros tomados por el sensor inteligente, aplicando herramientas de procesamiento numérico con bajo costo computacional dentro de la misma aplicación, con capacidades de personalización dependiendo del cliente o proyecto y con la posibilidad de  guardar los registros en un formato compatible con la gran mayoría de programas de procesamiento.

	Se comprobó el funcionamiento del diseño propuesto haciendo uso de un prototipo de pruebas, el cual se comparó con un equipo comercial del fabricante National Instruments, obteniendo resultados muy similares, confirmándose la efectividad y veracidad de los datos obtenidos por el sensor inteligente.
	
	Al haber tomado en cuenta la fecha y hora de adquisición de los datos, y por las características del protocolo LoRa y el control que permite tener sobre los dispositivos esclavos asignándoles números de identificación única, se sentaron las bases a la posibilidad de escalar el número de sensores inteligentes de bajo costo en la estructura, aumentando la resolución espacial y permitiendo tener acceso a las gráficas del comportamiento modal de la estructura.
	
	A pesar de las limitaciones de memoria de sistemas basados en microcontroladores, el sensor inteligente cumple sus funciones de monitorear las variables estructurales de interés; permitiendo que el operador tenga acceso a un histórico de datos de la estructura, disminuyendo además los costos y la complejidad de sistemas cableados con fines similares.
	
	En este tipo de aplicaciones es esencial garantizar la integridad de los datos.	A pesar de ser LoRa un protocolo lento en comparación a otros, la seguridad que ofrece al tener muy bajo porcentaje de pérdida de paquetes y el largo alcance, que permite ubicar la estación base en una ubicación segura alejada de la estructura lo posicionan como un protocolo prometedor para aplicaciones de monitoreo de salud estructural. 
	
    Los sistemas de adquisición de datos basados en microcontroladores representan una opción confiable y de bajo costo para implementar soluciones de monitoreo estructural periódico que sustituyen a los sistemas cableados.