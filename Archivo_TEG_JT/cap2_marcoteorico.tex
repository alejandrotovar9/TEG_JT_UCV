
En este capítulo se definirán los conceptos  o fundamentos de instrumentación estructural y adquisición de datos, necesarios para llevar a cabo esta investigación.

\section{Estructuras civiles:}

\subsection{Características generales:}

\subsection{Tipos de estructuras:}

\subsection{Comportamiento de las estructuras civiles:}

\begin{itemize}
    \item{Respuesta estática:}
        \begin{itemize}
            \item Rigidez
        \end{itemize}

    \item{Respuesta dinámica:}
        \begin{itemize}
            \item Frecuencia natural:
            \item Amortiguamiento
            \item Respuesta en frecuencia:
        \end{itemize}
\end{itemize}


\subsection{Daño en estructuras:}

El daño a una estructura civil o mecánica puede definirse como todo cambio en las propiedades materiales o geométricas del material que llegan a afectar de forma adversa la confiabilidad y el desempeño actual o futuro del sistema. Por tanto, el daño es una comparación entre el sistema en cuestión en 2 instantes de tiempo distintos.

En el caso de una estructura civil, un desempeño inadecuado puede traer como consecuencias la pérdida de vidas humanas y destrucción del patrimonio material.

Entre algunos de los factores que influyen del deterioro de una estructura se encuentran:
    
        \begin{itemize}
            \item Proceso de degradación natural de los materiales.
            \item Corrosión.
            \item Evento sísmico, incendio, condiciones de guerra.
            \item Carga por encima del límite de diseño.
        \end{itemize}
    
Las escalas de tiempo y de extensión del daño son diversas. Por ejemplo, el deterioro por el paso del tiempo bajo ciertas condiciones climáticas es muy lento comparado al daño causado por un evento catastrófico.

